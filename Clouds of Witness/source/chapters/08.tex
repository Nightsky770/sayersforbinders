%!TeX root=../cloudstop.tex


\chapter{Mr Parker Takes Notes}

\epigraph{A man was taken to the Zoo and shown the giraffe.  After gazing at it a little in silence: »I don't believe it,« he said.}{}


\lettrine[lines=4]{P}{arker's} first impulse was to doubt his own sanity; his next, to doubt Lady Mary's. Then, as the clouds rolled away from his brain, he decided that she was merely not speaking the truth.

\zz
»Come, Lady Mary,« he said encouragingly, but with an accent of reprimand as to an over-imaginative child, »you can't expect us to believe that, you know.«

»But you must,« said the girl gravely; »it's a fact. I shot him. I did, really. I didn't exactly mean to do it; it was a—well, a sort of accident.«

Mr Parker got up and paced about the room.

»You have put me in a terrible position, Lady Mary,« he said. »You see, I'm a police-officer. I never imagined\longdash«

»It doesn't matter,« said Lady Mary. »Of course you'll have to arrest me, or detain me, or whatever you call it. That's what I came for. I'm quite ready to go quietly—that's the right expression, isn't it? I'd like to explain about it, though, first. Of course I ought to have done it long ago, but I'm afraid I lost my head. I didn't realize that Gerald would get blamed. I hoped they'd bring it in suicide. Do I make a statement to you now? Or do I do it at the police-station?«

Parker groaned.

»They won't—they won't punish me so badly if it was an accident, will they?« There was a quiver in the voice.

»No, of course not—of course not. But if only you had spoken earlier!  No,« said Parker, stopping suddenly short in his distracted pacing and sitting down beside her. »It's impossible—absurd.« He caught the girl's hand suddenly in his own. »Nothing will convince me,« he said.  »It's absurd. It's not like you.«

»But an accident\longdash«

»I don't mean that—you know I don't mean that. But that you should keep silence\longdash«

»I was afraid. I'm telling you now.«

»No, no, no,« cried the detective. »You're lying to me. Nobly, I know; but it's not worth it. No man could be worth it. Let him go, I implore you. Tell the truth. Don't shield this man. If he murdered Denis Cathcart\longdash«

»\textit{No!}« The girl sprang to her feet, wrenching her hand away. »There was no other man. How dare you say it or think it! I killed Denis Cathcart, I tell you, and you \textit{shall} believe it. I swear to you that there was no other man.«

Parker pulled himself together.

»Sit down, please. Lady Mary, you are determined to make this statement?«

»Yes.«

»Knowing that I have no choice but to act upon it?«

»If you will not hear it I shall go straight to the police.«

Parker pulled out his note-book. »Go on,« he said.

With no other sign of emotion than a nervous fidgeting with her gloves, Lady Mary began her confession in a clear, hard voice, as though she were reciting it by heart.

»On the evening of Wednesday, October 13\textsuperscript{th}, I went upstairs at half-past nine. I sat up writing a letter. At a quarter past ten I heard my brother and Denis quarrelling in the passage. I heard my brother call Denis a cheat, and tell him that he was never to speak to me again. I heard Denis run out. I listened for some time, but did not hear him return. At half-past eleven I became alarmed. I changed my dress and went out to try and find Denis and bring him in. I feared he might do something desperate. After some time I found him in the shrubbery. I begged him to come in. He refused, and he told me about my brother's accusation and the quarrel. I was very much horrified, of course. He said where was the good of denying anything, as Gerald was determined to ruin him, and asked me to go away and marry him and live abroad. I said I was surprised that he should suggest such a thing in the circumstances. We both became very angry. I said »Come in now.  Tomorrow you can leave by the first train.« He seemed almost crazy. He pulled out a pistol and said that he'd come to the end of things, that his life was ruined, that we were a lot of hypocrites, and that I had never cared for him, or I shouldn't have minded what he'd done. Anyway, he said, if I wouldn't come with him it was all over, and he might as well be hanged for a sheep as a lamb—he'd shoot me and himself. I think he was quite out of his mind. He pulled out a revolver; I caught his hand; we struggled; I got the muzzle right up against his chest, and—either I pulled the trigger or it went off of itself—I'm not clear which. It was all in such a whirl.«

She paused. Parker's pen took down the words, and his face showed growing concern. Lady Mary went on:

»He wasn't quite dead. I helped him up. We struggled back nearly to the house. He fell once\longdash«

»Why,« asked Parker, »did you not leave him and run into the house to fetch help?«

Lady Mary hesitated.

»It didn't occur to me. It was a nightmare. I could only think of getting him along. I think—\textit{I think I wanted him to die}.«

There was a dreadful pause.

»He did die. He died at the door. I went into the conservatory and sat down. I sat for hours and tried to think. I hated him for being a cheat and a scoundrel. I'd been taken in, you see—made a fool of by a common sharper. I was glad he was dead. I must have sat there for hours without a coherent thought. It wasn't till my brother came along that I realized what I'd done, and that I might be suspected of murdering him. I was simply terrified. I made up my mind all in a moment that I'd pretend I knew nothing—that I'd heard a shot and come down. You know what I did.«

»Why, Lady Mary,« said Parker, in a perfectly toneless voice, »why did you say to your brother »Good God, Gerald, you've killed him«?«

Another hesitant pause.

»I never said that. I said, »Good God, Gerald, he's killed, then.« I never meant to suggest anything but suicide.«

»You admitted to those words at the inquest?«

»Yes\longdash« Her hands knotted the gloves into all manner of shapes. »By that time I had decided on a burglar story, you see.«

The telephone bell rang, and Parker went to the instrument. A voice came thinly over the wire:

»Is that 110 Piccadilly? This is Charing Cross Hospital. A man was brought in tonight who says he is Lord Peter Wimsey. He was shot in the shoulder, and struck his head in falling. He has only just recovered consciousness. He was brought in at 9:15. No, he will probably do very well now. Yes, come round by all means.«

»Peter has been shot,« said Parker. »Will you come round with me to Charing Cross Hospital? They say he is in no danger; still\longdash«

»Oh, quick!« cried Lady Mary.

Gathering up Mr Bunter as they hurried through the hall, detective and self-accused rushed hurriedly out into Pall Mall, and, picking up a belated taxi at Hyde Park Corner, drove madly away through the deserted streets.
