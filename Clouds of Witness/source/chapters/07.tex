%!TeX root=../cloudstop.tex



\chapter{The Club and the Bullet}

\epigraph{He is dead, and by my hand. It were better that I were dead myself, for the guilty wretch I am.}{\textit{Adventures Of Sexton Blake}}


\lettrine[lines=4]{H}{our} after hour Mr Parker sat waiting for his friend's return. Again and again he went over the Riddlesdale Case, checking his notes here, amplifying them there, involving his tired brain in speculations of the most fantastic kind. He wandered about the room, taking down here and there a book from the shelves, strumming a few unskilful bars upon the piano, glancing through the weeklies, fidgeting restlessly.  At length he selected a volume from the criminological section of the bookshelves, and forced himself to read with attention that most fascinating and dramatic of poison trials\allowbreak---\allowbreak the Seddon Case. Gradually the mystery gripped him, as it invariably did, and it was with a start of astonishment that he looked up at a long and vigorous whirring of the doorbell, to find that it was already long past midnight.

His first thought was that Wimsey must have left his latchkey behind, and he was preparing a facetious greeting when the door opened\allowbreak---\allowbreak exactly as in the beginning of a Sherlock Holmes story\allowbreak---\allowbreak to admit a tall and beautiful young woman, in an extreme state of nervous agitation, with a halo of golden hair, violet-blue eyes, and disordered apparel all complete; for as she threw back her heavy travelling-coat he observed that she wore evening dress, with light green silk stockings and heavy brogue shoes thickly covered with mud.

»His lordship has not yet returned, my lady,« said Mr Bunter, »but Mr  Parker is here waiting for him, and we are expecting him at any minute now. Will your ladyship take anything?«

»No, no,« said the vision hastily, »nothing, thanks. I'll wait. Good evening, Mr Parker. Where's Peter?«

»He has been called out, Lady Mary,« said Parker. »I can't think why he isn't back yet. Do sit down.«

»Where did he go?«

»To Scotland Yard\allowbreak---\allowbreak but that was about six o'clock. I can't imagine\longdash«

Lady Mary made a gesture of despair.

»I knew it. Oh, Mr Parker, what am I to do?«

Mr Parker was speechless.

»I \textit{must} see Peter,« cried Lady Mary. »It's a matter of life and death. Can't you send for him?«

»But I don't know where he is,« said Parker. »Please, Lady Mary\longdash«

»He's doing something dreadful\allowbreak---\allowbreak he's all \textit{wrong},« cried the young woman, wringing her hands with desperate vehemence. »I must see him\allowbreak---\allowbreak tell him\allowbreak---\allowbreak Oh! did anybody ever get into such dreadful trouble! I\allowbreak---\allowbreak oh!\longdash«

Here the lady laughed loudly and burst into tears.

»Lady Mary\allowbreak---\allowbreak I beg you\allowbreak---\allowbreak please don't,« cried Mr Parker anxiously, with a strong feeling that he was being incompetent and rather ridiculous.  »Please sit down. Drink a glass of wine. You'll be ill if you cry like that. If it is crying,« he added dubiously to himself. »It \textit{sounds} like hiccups. Bunter!«

Mr Bunter was not far off. In fact, he was just outside the door with a small tray. With a respectful »Allow me, sir,« he stepped forward to the writhing Lady Mary and presented a small phial to her nose. The effect was startling. The patient gave two or three fearful whoops, and sat up, erect and furious.

»How \textit{dare} you, Bunter!« said Lady Mary. »Go away at once!«

»Your ladyship had better take a drop of brandy,« said Mr Bunter, replacing the stopper in the smelling-bottle, but not before Parker had caught the pungent reek of ammonia. »This is the 1800 Napoleon brandy my lady. Please don't snort so, if I may make the suggestion.  His lordship would be greatly distressed to think that any of it should be wasted. Did your ladyship dine on the way up? No? Most unwise, my lady, to undertake a long journey on a vacant interior. I will take the liberty of sending in an omelette for your ladyship. Perhaps you would like a little snack of something yourself, sir, as it is getting late?«

»Anything you like,« said Mr Parker, waving him off hurriedly. »Now, Lady Mary, you're feeling better, aren't you? Let me help you off with your coat.«

No more of an exciting nature was said until the omelette was disposed of, and Lady Mary comfortably settled on the Chesterfield. She had by now recovered her poise. Looking at her, Parker noticed how her recent illness (however produced) had left its mark upon her. Her complexion had nothing of the brilliance which he remembered; she looked strained and white, with purple hollows under her eyes.

»I am sorry I was so foolish just now, Mr Parker,« she said, looking into his eyes with a charming frankness and confidence, »but I was dreadfully distressed, and I came up from Riddlesdale so hurriedly.«

»Not at all,« said Parker meaninglessly. »Is there anything I can do in your brother's absence?«

»I suppose you and Peter do everything together?«

»I think I may say that neither of us knows anything about this investigation which he has not communicated to the other.«

»If I tell you, it's the same thing?«

»Exactly the same thing. If you can bring yourself to honour me with your confidence\longdash«

»Wait a minute, Mr Parker. I'm in a difficult position. I don't quite know what I ought\allowbreak---\allowbreak Can you tell me just how far you've got\allowbreak---\allowbreak what you have discovered?«

Mr Parker was a little taken aback. Although the face of Lady Mary had been haunting his imagination ever since the inquest, and although the agitation of his feelings had risen to boiling-point during this romantic interview, the official instinct of caution had not wholly deserted him. Holding, as he did, proofs of Lady Mary's complicity in the crime, whatever it was, he was not so far gone as to fling all his cards on the table.

»I'm afraid,« he said, »that I can't quite tell you that. You see, so much of what we've got is only suspicion as yet. I might accidentally do great mischief to an innocent person.«

»Ah! You definitely suspect somebody, then?«

»\textit{In}definitely would be a better word for it,« said Mr Parker with a smile. »But if you have anything to tell us which may throw light on the matter, I beg you to speak. We may be suspecting a totally wrong person.«

»I shouldn't be surprised,« said Lady Mary, with a sharp, nervous little laugh. Her hand strayed to the table and began pleating the orange envelope into folds. »What do you want to know?« she asked suddenly, with a change of tone. Parker was conscious of a new hardness in her manner\allowbreak---\allowbreak a something braced and rigid.

He opened his note-book, and as he began his questioning his nervousness left him; the official reasserted himself.

»You were in Paris last February?«

Lady Mary assented.

»Do you recollect going with Captain Cathcart\allowbreak---\allowbreak oh! by the way, you speak French, I presume?«

»Yes, very fluently.«

»As well as your brother\allowbreak---\allowbreak practically without accent?«

»Quite as well. We always had French governesses as children, and mother was very particular about it.«

»I see. Well, now, do you remember going with Captain Cathcart on February 6\textsuperscript{th} to a jeweller's in the Rue de la Paix and buying, or his buying for you, a tortoiseshell comb set with diamonds and a diamond-and-platinum cat with emerald eyes?«

He saw a lurking awareness come into the girl's eyes.

»Is that the cat you have been making inquiries about in Riddlesdale?« she demanded.

It being never worth while to deny the obvious, Parker replied, »Yes.«

»It was found in the shrubbery, wasn't it?«

»Had you lost it? Or was it Cathcart's?«

»If I said it was his\longdash«

»I should be ready to believe you. \textit{Was} it his?«

»No«---a long breath---»it was mine.«

»When did you lose it?«

»That night.«

»Where?«

»I suppose in the shrubbery. Wherever you found it. I didn't miss it till later.«

»Is it the one you bought in Paris?«

»Yes.«

»Why did you say before that it was not yours?«

»I was afraid.«

»And now?«

»I am going to speak the truth.«

Parker looked at her again. She met his eye frankly, but there was a tenseness in her manner which showed that it had cost her something to make her mind up.

»Very well,« said Parker, »we shall all be glad of that, for I think there were one or two points at the inquest on which you didn't tell the truth, weren't there?«

»Yes.«

»Do believe,« said Parker, »that I am sorry to have to ask these questions. The terrible position in which your brother is placed\longdash«

»In which I helped to place him.«

»I don't say that.«

»I do. I helped to put him in jail. Don't say I didn't, because I did.«

»Well,« said Parker, »don't worry. There's plenty of time to put it all right again. Shall I go on?«

»Yes.«

»Well, now, Lady Mary, it wasn't true about hearing that shot at three o'clock, was it?«

»No.«

»Did you hear the shot at all?«

»Yes.«

»When?«

»At 11:50.«

»What was it, then, Lady Mary, you hid behind the plants in the conservatory?«

»I hid nothing there.«

»And in the oak chest on the landing?«

»My skirt.«

»You went out\allowbreak---\allowbreak why?---to meet Cathcart?«

»Yes.«

»Who was the other man?«

»What other man?«

»The other man who was in the shrubbery. A tall, fair man dressed in a Burberry?«

»There was no other man.«

»Oh, pardon me, Lady Mary. We saw his footmarks all the way up from the shrubbery to the conservatory.«

»It must have been some tramp. I know nothing about him.«

»But we have proof that he was there\allowbreak---\allowbreak of what he did, and how he escaped. For heaven's sake, and your brother's sake, Lady Mary, tell us the truth\allowbreak---\allowbreak for that man in the Burberry was the man who shot Cathcart.«

»No,« said the girl, with a white face, »that is impossible.«

»Why impossible?«

»I shot Denis Cathcart myself.«

\noindent\hfil\rule{0.5\textwidth}{.4pt}\hfil

»So that's how the matter stands, you see, Lord Peter,« said the Chief of Scotland Yard, rising from his desk with a friendly gesture of dismissal. »The man was undoubtedly seen at Marylebone on the Friday morning, and, though we have unfortunately lost him again for the moment, I have no doubt whatever that we shall lay hands on him before long. The delay has been due to the unfortunate illness of the porter Morrison, whose evidence has been so material. But we are wasting no time now.«

»I'm sure I may leave it to you with every confidence, Sir Andrew,« replied Wimsey, cordially shaking hands. »I'm diggin' away too; between us we ought to get somethin'---you in your small corner and I in mine, as the hymn says\allowbreak---\allowbreak or is it a hymn? I remember readin' it in a book about missionaries when I was small. Did you want to be a missionary in your youth? I did. I think most kids do some time or another, which is odd, seein' how unsatisfactory most of us turn out.«

»Meanwhile,« said Sir Andrew Mackenzie, »if you run across the man yourself, let us know. I would never deny your extraordinary good fortune, or it may be good judgment, in running across the criminals we may be wanting.«

»If I catch the bloke,« said Lord Peter, »I'll come and shriek under your windows till you let me in, if it's the middle of the night and you in your little night-shirt. And talking of night-shirts reminds me that we hope to see you down at Denver one of these days, as soon as this business is over. Mother sends kind regards, of course.«

»Thanks very much,« replied Sir Andrew. »I hope you feel that all is going well. I had Parker in here this morning to report, and he seemed a little dissatisfied.«

»He's been doing a lot of ungrateful routine work,« said Wimsey, »and being altogether the fine, sound man he always is. He's been a damn good friend to me, Sir Andrew, and it's a real privilege to be allowed to work with him. Well, so long, Chief.«

He found that his interview with Sir Andrew Mackenzie had taken up a couple of hours, and that it was nearly eight o'clock. He was just trying to make up his mind where to dine when he was accosted by a cheerful young woman with bobbed red hair, dressed in a short checked skirt, brilliant jumper, corduroy jacket, and a rakish green velvet tam-o'-shanter.

»Surely,« said the young woman, extending a shapely, ungloved hand, »it's Lord Peter Wimsey. How're you? And how's Mary?«

»B'Jove!« said Wimsey gallantly, »it's Miss Tarrant. How perfectly rippin' to see you again. Absolutely delightful. Thanks, Mary ain't as fit as she might be\allowbreak---\allowbreak worryin' about this murder business, y'know.  You've heard that we're what the poor so kindly and tactfully call »in trouble,« I expect, what?«

»Yes, of course,« replied Miss Tarrant eagerly, \enquote{and, of course, as a good Socialist, I can't help rejoicing rather when a peer gets taken up, because it does make him look so silly, you know, and the House of Lords is silly, isn't it? But, really, I'd rather it was anybody else's brother. Mary and I were such great friends, you know, and, of course, \textit{you} do investigate things, don't you, not just live on your estates in the country and shoot birds? So I suppose that makes a difference.}

»That's very kind of you,« said Peter. »If you can prevail upon yourself to overlook the misfortune of my birth and my other deficiencies, p'raps you would honour me by comin' along and havin' a bit of dinner somewhere, what?«

»Oh, I'd have \textit{loved} to,« cried Miss Tarrant, with enormous energy, »but I've promised to be at the club tonight. There's a meeting at nine. Mr Coke\allowbreak---\allowbreak the Labour leader, you know\allowbreak---\allowbreak is going to make a speech about converting the Army and Navy to Communism. We expect to be raided, and there's going to be a grand hunt for spies before we begin.  But look here, do come along and dine with me there, and, if you like, I'll try to smuggle you in to the meeting, and you'll be seized and turned out. I suppose I oughtn't to have told you anything about it, because you ought to be a deadly enemy, but I can't really believe you're dangerous.«

»I'm just an ordinary capitalist, I expect,« said Lord Peter, »highly obnoxious.«

»Well, come to dinner, anyhow. I \textit{do} so want to hear all the news.«

Peter reflected that the dinner at the Soviet Club would be worse than execrable, and was just preparing an excuse when it occurred to him that Miss Tarrant might be able to tell him a good many of the things that he didn't know, and really ought to know, about his own sister.  Accordingly, he altered his polite refusal into a polite acceptance, and, plunging after Miss Tarrant, was led at a reckless pace and by a series of grimy short cuts into Gerrard Street, where an orange door, flanked by windows with magenta curtains, sufficiently indicated the Soviet Club.

The Soviet Club, being founded to accommodate free thinking rather than high living, had that curious amateur air which pervades all worldly institutions planned by unworldly people. Exactly why it made Lord Peter instantly think of mission teas he could not say, unless it was that all the members looked as though they cherished a purpose in life, and that the staff seemed rather sketchily trained and strongly in evidence. Wimsey reminded himself that in so democratic an institution one could hardly expect the assistants to assume that air of superiority which marks the servants in a West End club. For one thing, they would not be such capitalists. In the dining-room below the resemblance to a mission tea was increased by the exceedingly heated atmosphere, the babel of conversation, and the curious inequalities of the cutlery. Miss Tarrant secured seats at a rather crumby table near the serving-hatch, and Peter wedged himself in with some difficulty next to a very large, curly-haired man in a velvet coat, who was earnestly conversing with a thin, eager young woman in a Russian blouse, Venetian beads, a Hungarian shawl and a Spanish comb, looking like a personification of the United Front of the »Internationale.«

Lord Peter endeavoured to please his hostess by a question about the great Mr Coke, but was checked by an agitated »Hush!«

»\textit{Please} don't shout about it,« said Miss Tarrant, leaning across till her auburn mop positively tickled his eyebrows. »It's \textit{so} secret.«

»I'm awfully sorry,« said Wimsey apologetically. »I say, d'you know you're dipping those jolly little beads of yours in the soup?«

»Oh, am I?« cried Miss Tarrant, withdrawing hastily. »Oh, thank you so much. Especially as the colour runs. I hope it isn't arsenic or anything.« Then, leaning forward again, she whispered hoarsely:

»The girl next me is Erica Heath-Warburton\allowbreak---\allowbreak the writer, you know.«

Wimsey looked with a new respect at the lady in the Russian blouse.  Few books were capable of calling up a blush to his cheek, but he remembered that one of Miss Heath-Warburton's had done it. The authoress was just saying impressively to her companion:

»---ever know a sincere emotion to express itself in a subordinate clause?«

»Joyce has freed us from the superstition of syntax,« agreed the curly man.

»Scenes which make emotional history,« said Miss Heath-Warburton, »should ideally be expressed in a series of animal squeals.«

»The D. H. Lawrence formula,« said the other.

»Or even Dada,« said the authoress.

»We need a new notation,« said the curly-haired man, putting both elbows on the table and knocking Wimsey's bread on to the floor. »Have you heard Robert Snoates recite his own verse to the tom-tom and the penny whistle?«

Lord Peter with difficulty detached his attention from this fascinating discussion to find that Miss Tarrant was saying something about Mary.

»One misses your sister very much,« she said. »Her wonderful enthusiasm. She spoke so well at meetings. She had such a \textit{real} sympathy with the worker.«

»It seems astonishing to me,« said Wimsey, »seeing Mary's never had to do a stroke of work in her life.«

»Oh,« cried Miss Tarrant, »but she \textit{did} work. She worked for us.  Wonderfully! She was secretary to our Propaganda Society for nearly six months. And then she worked so hard for Mr Goyles. To say nothing of her nursing in the war. Of course, I don't approve of England's attitude in the war, but nobody would say the work wasn't hard.«

»Who is Mr Goyles?«

»Oh, one of our leading speakers\allowbreak---\allowbreak quite young, but the Government are really afraid of him. I expect he'll be here tonight. He has been lecturing in the North, but I believe he's back now.«
»I say, do look out,« said Peter. »Your beads are in your plate again.«

»Are they? Well, perhaps they'll flavour the mutton. I'm afraid the cooking isn't very good here, but the subscription's so small, you see.  I wonder Mary never told you about Mr Goyles. They were so \textit{very} friendly, you know, some time ago. Everybody thought she was going to marry him\allowbreak---\allowbreak but it seemed to fall through. And then your sister left town. Do you know about it?«

»That was the fellow, was it? Yes\allowbreak---\allowbreak well, my people didn't altogether see it, you know. Thought Mr Goyles wasn't quite the son-in-law they'd take to. Family row and so on. Wasn't there myself; besides, Mary'd never listen to \textit{me}. Still, that's what I gathered.«

»Another instance of the absurd, old-fashioned tyranny of parents,« said Miss Tarrant warmly. »You wouldn't think it could still be possible\allowbreak---\allowbreak in post-war times.«

»I don't know,« said Wimsey, »that you could exactly call it that. Not parents exactly. My mother's a remarkable woman. I don't think she interfered. Fact, I fancy she wanted to ask Mr Goyles to Denver. But my brother put his foot down.«

»Oh, well, what can you expect?« said Miss Tarrant scornfully. »But I don't see what business it was of his.«

»Oh, none,« agreed Wimsey. »Only, owin' to my late father's circumscribed ideas of what was owin' to women, my brother has the handlin' of Mary's money till she marries with his consent. I don't say it's a good plan\allowbreak---\allowbreak I think it's a rotten plan. But there it is.«

»Monstrous!« said Miss Tarrant, shaking her head so angrily that she looked like shock-headed Peter. »Barbarous! Simply feudal, you know.  But, after all, what's money?«

»Nothing, of course,« said Peter. »But if you've been brought up to havin' it it's a bit awkward to drop it suddenly. Like baths, you know.«

»I can't understand how it could have made any difference to Mary,« persisted Miss Tarrant mournfully. »She liked being a worker. We once tried living in a workman's cottage for eight weeks, five of us, on eighteen shillings a week. It was a \textit{marvellous} experience\allowbreak---\allowbreak on the very \textit{edge} of the New Forest.«

»In the winter?«

»Well, no\allowbreak---\allowbreak we thought we'd better not \textit{begin} with winter. But we had nine wet days, and the kitchen chimney smoked all the time. You see, the wood came out of the forest, so it was all damp.«

»I see. It must have been uncommonly interestin'.«

»It was an experience I shall \textit{never} forget,« said Miss Tarrant. »One felt so \textit{close} to the earth and the primitive things. If only we could abolish industrialism. I'm afraid, though, we shall never get it put right without a »bloody revolution,« you know. It's very terrible, of course, but salutary and inevitable. Shall we have coffee? We shall have to carry it upstairs ourselves, if you don't mind. The maids don't bring it up after dinner.«

Miss Tarrant settled her bill and returned, thrusting a cup of coffee into his hand. It had already overflowed into the saucer, and as he groped his way round a screen and up a steep and twisted staircase it overflowed quite an amount more.

Emerging from the basement, they almost ran into a young man with fair hair who was hunting for letters in a dark little row of pigeon-holes.  Finding nothing, he retreated into the lounge. Miss Tarrant uttered an exclamation of pleasure.

»Why, there \textit{is} Mr Goyles,« she cried.

Wimsey glanced across, and at the sight of the tall, slightly stooping figure with the untidy fair hair and the gloved right hand he gave an irrepressible little gasp.

»Won't you introduce me?« he said.

»I'll fetch him,« said Miss Tarrant. She made off across the lounge and addressed the young agitator, who started, looked across at Wimsey, shook his head, appeared to apologize, gave a hurried glance at his watch, and darted out by the entrance. Wimsey sprang forward in pursuit.

»Extraordinary,« cried Miss Tarrant, with a blank face. »He says he has an appointment\allowbreak---\allowbreak but he can't surely be missing the\longdash«

»Excuse me,« said Peter. He dashed out, in time to perceive a dark figure retreating across the street. He gave chase. The man took to his heels, and seemed to plunge into the dark little alley which leads into the Charing Cross Road. Hurrying in pursuit, Wimsey was almost blinded by a sudden flash and smoke nearly in his face. A crashing blow on the left shoulder and a deafening report whirled his surroundings away. He staggered violently, and collapsed on to a second-hand brass bedstead.
