%!TeX root=../cloudstop.tex


\chapter{The Green-Eyed Cat}

\epigraph{And here's to the hound\\With his nose unto the ground}{\textit{Drink, Puppy, Drink}}

\lettrine[lines=4]{S}{ome} people hold that breakfast is the best meal of the day. Others, less robust, hold that it is the worst, and that, of all breakfasts in the week, Sunday morning breakfast is incomparably the worst.

The party gathered about the breakfast-table at Riddlesdale Lodge held, if one might judge from their faces, no brief for that day miscalled of sweet refection and holy love. The only member of it who seemed neither angry nor embarrassed was the Hon. Freddy Arbuthnot, and he was silent, engaged in trying to take the whole skeleton out of a bloater at once.  The very presence of that undistinguished fish upon the Duchess's breakfast-table indicated a disorganized household.

The Duchess of Denver was pouring out coffee. This was one of her uncomfortable habits. Persons arriving late for breakfast were thereby made painfully aware of their sloth. She was a long-necked, long-backed woman, who disciplined her hair and her children. She was never embarrassed, and her anger, though never permitted to be visible, made itself felt the more.

Colonel and Mrs Marchbanks sat side by side. They had nothing beautiful about them but a stolid mutual affection. Mrs Marchbanks was not angry, but she was embarrassed in the presence of the Duchess, because she could not feel sorry for her. When you felt sorry for people you called them »poor old dear« or »poor dear old man.« Since, obviously, you could not call the Duchess poor old dear, you were not being properly sorry for her. This distressed Mrs Marchbanks. The Colonel was both embarrassed and angry\allowbreak---\allowbreak embarrassed because, 'pon my soul, it was very difficult to know what to talk about in a house where your host had been arrested for murder; angry in a dim way, like an injured animal, because unpleasant things like this had no business to break in on the shooting-season.

Mrs Pettigrew-Robinson was not only angry, she was outraged. As a girl she had adopted the motto stamped upon the school notepaper: \textit{Quœcunque honesta}. She had always thought it \textit{wrong} to let your mind \textit{dwell} on anything that was not really nice. In middle life she still made a point of ignoring those newspaper paragraphs which bore such headlines as: »\textsc{Assault upon a School teacher at Cricklewood}«; »\textsc{Death in a Pint of Stout}«; »\textsc{\textsterling 75 for a Kiss}«; or »\textsc{She called him Hubbykins}«. She said she could not see what \textit{good} it did you to know about such things. She regretted having consented to visit Riddlesdale Lodge in the absence of the Duchess. She had never liked Lady Mary; she considered her a very objectionable specimen of the modern independent young woman; besides, there had been that very undignified incident connected with a Bolshevist while Lady Mary was nursing in London during the war. Nor had Mrs Pettigrew-Robinson at all cared for Captain Denis Cathcart. She did not like a young man to be handsome in that obvious kind of way. But, of course, since Mr  Pettigrew-Robinson had wanted to come to Riddlesdale, it was her place to be with him. She was not to blame for the unfortunate result.

Mr Pettigrew-Robinson was angry, quite simply, because the detective from Scotland Yard had not accepted his help in searching the house and grounds for footprints. As an older man of some experience in these matters (Mr Pettigrew-Robinson was a county magistrate) he had gone out of his way to place himself at the man's disposal. Not only had the man been short with him, but he had rudely ordered him out of the conservatory, where he (Mr Pettigrew-Robinson) had been reconstructing the affair from the point of view of Lady Mary.

All these angers and embarrassments might have caused less pain to the company had they not been aggravated by the presence of the detective himself, a quiet young man in a tweed suit, eating curry at one end of the table next to Mr Murbles, the solicitor. This person had arrived from London on Friday, had corrected the local police, and strongly dissented from the opinion of Inspector Craikes. He had suppressed at the inquest information which, if openly given, might have precluded the arrest of the Duke. He had officiously detained the whole unhappy party, on the grounds that he wanted to re-examine everybody, and was thus keeping them miserably cooped up together over a horrible Sunday; and he had put the coping-stone on his offenses by turning out to be an intimate friend of Lord Peter Wimsey's, and having, in consequence, to be accommodated with a bed in the gamekeeper's cottage and breakfast at the Lodge.

Mr Murbles, who was elderly and had a delicate digestion, had traveled up in a hurry on Thursday night. He had found the inquest very improperly conducted and his client altogether impracticable. He had spent all his time trying to get hold of Sir Impey Biggs, K.C., who had vanished for the weekend, leaving no address. He was eating a little dry toast, and was inclined to like the detective, who called him »Sir«, and passed him the butter.

»Is anybody thinking of going to church?« asked the Duchess.

»Theodore and I should like to go,« said Mrs Pettigrew-Robinson, »if it is not too much trouble; or we could walk. It is not so \textit{very} far.«

»It's two and a half miles, good,« said Colonel Marchbanks.

Mr Pettigrew-Robinson looked at him gratefully.

»Of course you will come in the car,« said the Duchess. »I am going myself.«

»Are you, though?« said the Hon. Freddy. »I say, won't you get a bit stared at, what?«

»Really, Freddy,« said the Duchess, »does that matter?«

»Well,« said the Hon. Freddy, »I mean to say, these bounders about here are all Socialists and Methodists....«

»If they are Methodists,« said Mrs Pettigrew-Robinson, »they will not be at church.«

»Won't they?« retorted the Hon. Freddy. »You bet they will if there's anything to see. Why, it'll be better'n a funeral to 'em.«

»Surely,« said Mrs Pettigrew-Robinson, »one has a \textit{duty} in the matter, whatever our private feelings may be\allowbreak---\allowbreak especially at the present day, when people are so terribly \textit{slack}.«

She glanced at the Hon. Freddy.

»Oh, don't you mind me, Mrs P.,« said that youth amiably. »All \textit{I} say is, if these blighters make things unpleasant, don't blame me.«

»Whoever thought of blaming you, Freddy?« said the Duchess.

»Manner of speaking,« said the Hon. Freddy.

»What do you think, Mr Murbles?« inquired her ladyship.

»I feel,« said the lawyer, carefully stirring his coffee, »that, while your intention is a very admirable one, and does you very great credit, my dear lady, yet Mr Arbuthnot is right in saying it may involve you in some\allowbreak---\allowbreak er---unpleasant publicity. Er\allowbreak---\allowbreak I have always been a sincere Christian myself, but I cannot feel that our religion demands that we should make ourselves conspicuous\allowbreak---\allowbreak er---in such very painful circumstances.«

Mr Parker reminded himself of a dictum of Lord Melbourne.

»Well, after all,« said Mrs Marchbanks, »as Helen so rightly says, does it matter? Nobody's really got anything to be ashamed of. There has been a stupid mistake, of course, but I don't see why anybody who wants to shouldn't go to church.«

»Certainly not, certainly not, my dear,« said the Colonel heartily. »We might look in ourselves, eh, dear? Take a walk that way I mean, and come out before the sermon. I think it's a good thing. Shows \textit{we} don't believe old Denver's done anything wrong, anyhow.«

»You forget, dear,« said his wife, »I've promised to stay at home with Mary, poor girl.«

»Of course, of course\allowbreak---\allowbreak stupid of me,« said the Colonel. »How is she?«

»She was very restless last night, poor child,« said the Duchess.  »Perhaps she will get a little sleep this morning. It has been a shock to her.«

»One which may prove a blessing in disguise,« said Mrs  Pettigrew-Robinson.

»My dear!« said her husband.

»Wonder when we shall hear from Sir Impey,« said Colonel Marchbanks hurriedly.

»Yes, indeed,« moaned Mr Murbles. »I am counting on his influence with the Duke.«

»Of course,« said Mrs Pettigrew-Robinson, »he must speak out\allowbreak---\allowbreak for everybody's sake. He must say what he was doing out of doors at that time. Or, if he does not, it must be discovered. Dear me! That's what these detectives are for, aren't they?«

»That is their ungrateful task,« said Mr Parker suddenly. He had said nothing for a long time, and everybody jumped.

»There,« said Mrs Marchbanks, »I expect you'll clear it all up in no time, Mr Parker. Perhaps you've got the real mur\allowbreak---\allowbreak the culprit up your sleeve all the time.«

»Not quite,« said Mr Parker, »but I'll do my best to get him.  Besides,« he added, with a grin, »I'll probably have some help on the job.«

»From whom?« inquired Mr Pettigrew-Robinson.

»Her grace's brother-in-law.«

»Peter?« said the Duchess. »Mr Parker must be amused at the family amateur,« she added.

»Not at all,« said Parker. »Wimsey would be one of the finest detectives in England if he wasn't lazy. Only we can't get hold of him.«

»I've wired to Ajaccio\allowbreak---\allowbreak poste restante,« said Mr Murbles, »but I don't know when he's likely to call there. He said nothing about when he was coming back to England.«

»He's a rummy old bird,« said the Hon. Freddy tactlessly, »but he oughter be here, what? What I mean to say is, if anything happens to old Denver, don't you see, he's the head of the family, ain't he\allowbreak---\allowbreak till little Pickled Gherkins comes of age.«

In the frightful silence which followed this remark, the sound of a walking-stick being clattered into an umbrella-stand was distinctly
audible.

»Who's that, I wonder,« said the Duchess.

The door waltzed open.

»Mornin', dear old things,« said the newcomer cheerfully. »How are you all? Hullo, Helen! Colonel, you owe me half a crown since last September year. Mornin', Mrs Marchbanks, Mornin', Mrs P. Well, Mr  Murbles, how d'you like this bili-beastly weather? Don't trouble to get up, Freddy; I'd simply hate to inconvenience you. Parker, old man, what a damned reliable old bird you are! Always on the spot, like that patent ointment thing. I say, have you all finished? I meant to get up earlier, but I was snorin' so Bunter hadn't the heart to wake me. I nearly blew in last night, only we didn't arrive till 2 \textsc{a.m.} and I thought you wouldn't half bless me if I did. Eh, what, Colonel? Airplane \textit{Victoria} from Paris to London\allowbreak---\allowbreak North-Eastern to Northallerton\allowbreak---\allowbreak damn bad roads the rest of the way, and a puncture just below Riddlesdale. Damn bad bed at the »Lord in Glory«; thought I'd blow in for the last sausage here, if I was lucky. What? Sunday morning in an English family and no sausages? God bless my soul, what's the world coming to, eh, Colonel? I say, Helen, old Gerald's been an' gone an' done it this time, what? You've no business to leave him on his own, you know; he always gets into mischief. What's that? Curry? Thanks, old man. Here, I say, you needn't be so stingy about it; I've been traveling for three days on end. Freddy, pass the toast. Beg pardon, Mrs Marchbanks? Oh, rather, yes; Corsica was perfectly amazin'---all black-eyed fellows with knives in their belts and jolly fine-looking girls. Old Bunter had a regular affair with the innkeeper's daughter in one place. D'you know, he's an awfully susceptible old beggar. You'd never think it, would you? Jove! I am hungry. I say, Helen, I meant to get you some fetchin' crêpe-de-Chine undies from Paris, but I saw that old Parker was gettin' ahead of me over the bloodstains, so we packed up our things and buzzed off.«

Mrs Pettigrew-Robinson rose.

»Theodore,« she said, »I think we ought to be getting ready for church.«

»I will order the car,« said the Duchess. »Peter, of course I'm exceedingly glad to see you. Your leaving no address was most inconvenient. Ring for anything you want. It is a pity you didn't arrive in time to see Gerald.«

»Oh, that's all right,« said Lord Peter cheerfully; »I'll look him up in quod. Y'know, it's rather a good idea to keep one's crimes in the family; one has so many more facilities. I'm sorry for poor old Polly, though. How is she?«

»She must not be disturbed today,« said the Duchess with decision.

»Not a bit of it,« said Lord Peter; »she'll keep. Today Parker and I hold high revel. Today he shows me all the bloody footprints\allowbreak---\allowbreak it's all right, Helen, that's not swearin', that's an adjective of quality. I hope they aren't all washed away, are they, old thing?«

»No,« said Parker, »I've got most of them under flower-pots.«

»Then pass the bread and squish,« said Lord Peter, »and tell me all about it.«

The departure of the church-going element had induced a more humanitarian atmosphere. Mrs Marchbanks stumped off upstairs to tell Mary that Peter had come, and the Colonel lit a large cigar. The Hon.  Freddy rose, stretched himself, pulled a leather armchair to the fireside, and sat down with his feet on the brass fender, while Parker marched round and poured himself out another cup of coffee.

»I suppose you've seen the papers,« he said.

»Oh, yes, I read up the inquest,« said Lord Peter. »Y'know, if you'll excuse my saying so, I think you rather mucked it between you.«

»It was disgraceful,« said Mr Murbles, »disgraceful. The Coroner behaved most improperly. He had no business to give such a summing-up.  With a jury of ignorant country fellows, what could one expect? And the details that were allowed to come out! If I could have got here earlier\longdash«

»I'm afraid that was partly my fault, Wimsey,« said Parker penitently. »Craikes rather resents me. The Superintendent at Stapley sent to us over his head, and when the message came through I ran along to the Chief and asked for the job, because I thought if there should be any misconception or difficulty, you see, you'd just as soon I tackled it as anybody else. I had a few little arrangements to make about a forgery I've been looking into, and, what with one thing and another, I didn't get off till the night express. By the time I turned up on Friday, Craikes and the Coroner were already as thick as thieves, had fixed the inquest for that morning\allowbreak---\allowbreak which was ridiculous\allowbreak---\allowbreak and arranged to produce their blessed evidence as dramatically as possible. I only had time to skim over the ground (disfigured, I'm sorry to say, by the prints of Craikes and his local ruffians), and really had nothing for the jury.«

»Cheer up,« said Wimsey. »I'm not blaming you. Besides, it all lends excitement to the chase.«

»Fact is,« said the Hon. Freddy, »that we ain't popular with respectable Coroners. Giddy aristocrats and immoral Frenchmen. I say, Peter, sorry you've missed Miss Lydia Cathcart. You'd have loved her. She's gone back to Golders Green and taken the body with her.«

»Oh, well,« said Wimsey. »I don't suppose there was anything abstruse about the body.«

»No,« said Parker, »the medical evidence was all right as far as it went. He was shot through the lungs, and that's all.«

»Though, mind you,« said the Hon. Freddy, »he didn't shoot himself. I didn't say anything, not wishin' to upset old Denver's story, but, you know, all that stuff about his bein' so upset and go-to-blazes in his manner was all my whiskers.«

»How do you know?« said Peter.

»Why, my dear man, Cathcart'n I toddled up to bed together. I was rather fed up, havin' dropped a lot on some shares, besides missin' everything I shot at in the mornin', an' lost a bet I made with the Colonel about the number of toes on the kitchen cat, an' I said to Cathcart it was a hell of a damn-fool world, or words to that effect. »Not a bit of it,« he said; »it's a damn good world. I'm goin' to ask Mary for a date tomorrow, an' then we'll go and live in Paris, where they understand sex.« I said somethin' or other vague, and he went off whistlin'.«

Parker looked grave. Colonel Marchbanks cleared his throat.

»Well, well,« he said, »there's no accounting for a man like Cathcart, no accounting at all. Brought up in France, you know. Not at all like a straight-forward Englishman. Always up and down, up and down! Very sad, poor fellow. Well, well, Peter, hope you and Mr Parker will find out something about it. We mustn't have poor old Denver cooped up in jail like this, you know. Awfully unpleasant for him, poor chap, and with the birds so good this year. Well, I expect you'll be making a tour of inspection, eh, Mr Parker? What do you say to shoving the balls about a bit, Freddy?«

»Right you are,« said the Hon. Freddy; »you'll have to give me a hundred, though, Colonel.«

»Nonsense, nonsense,« said that veteran, in high good humor; »you play an excellent game.«

Mr Murbles having withdrawn, Wimsey and Parker faced each other over the remains of the breakfast.

»Peter,« said the detective, »I don't know if I've done the right thing by coming. If you feel\longdash«

»Look here, old man,« said his friend earnestly, »let's cut out the considerations of delicacy. We're goin' to work this case like any other. If anything unpleasant turns up, I'd rather you saw it than anybody else. It's an uncommonly pretty little case, on its merits, and I'm goin' to put some damn good work into it.«

»If you're sure it's all right\longdash«

»My dear man, if you hadn't been here I'd have sent for you. Now let's get to business. Of course, I'm settin' off with the assumption that old Gerald didn't do it.«

»I'm sure he didn't,« agreed Parker.

»No, no,« said Wimsey, »that isn't your line. Nothing rash about you\allowbreak---\allowbreak nothing trustful. You are expected to throw cold water on my hopes and doubt all my conclusions.«

»Right ho!« said Parker. »Where would you like to begin?«

Peter considered. »I think we'll start from Cathcart's bedroom,« he said.

The bedroom was of moderate size, with a single window overlooking the front door. The bed was on the right-hand side, the dressing-table before the window. On the left was the fireplace, with an armchair before it, and a small writing-table.

»Everything's as it was,« said Parker. »Craikes had that much sense.«

»Yes,« said Lord Peter. »Very well. Gerald says that when he charged Cathcart with bein' a scamp, Cathcart jumped up, nearly knockin' the table over. That's the writin'-table, then, so Cathcart was sittin' in the armchair. Yes, he was\allowbreak---\allowbreak and he pushed it back violently and rumpled up the carpet. See! So far, so good. Now what was he doin' there? He wasn't readin', because there's no book about, and we know that he rushed straight out of the room and never came back. Very good. Was he writin'? No; virgin sheet of blottin'-paper\longdash«

»He might have been writing in pencil,« suggested Parker.

»That's true, old Kill-Joy, so he might. Well, if he was he shoved the paper into his pocket when Gerald came in, because it isn't here; but he didn't, because it wasn't found on his body; so he wasn't writing.«

»Unless he threw the paper away somewhere else,« said Parker. »I haven't been all over the grounds, you know, and at the smallest computation\allowbreak---\allowbreak if we accept the shot heard by Hardraw at 11:50 as the shot\allowbreak---\allowbreak there's an hour and a half unaccounted for.«

»Very well. Let's say there is nothing to show he was writing. Will that do? Well, then\longdash«

Lord Peter drew out a lens and scrutinized the surface of the armchair carefully before sitting down in it.

»Nothing helpful there,« he said. »To proceed, Cathcart sat where I am sitting. He wasn't writing; he\allowbreak---\allowbreak You're sure this room hasn't been touched?«

»Certain.«

»Then he wasn't smoking.«

»Why not? He might have chucked the stub of a cigar or cigarette into the fire when Denver came in.«

»Not a cigarette,« said Peter, »or we should find traces somewhere\allowbreak---\allowbreak on the floor or in the grate. That light ash blows about so. But a cigar\allowbreak---\allowbreak well, he might have smoked a cigar without leaving a sign, I suppose. But I hope he didn't.«

»Why?«

»Because, old son, I'd rather Gerald's account had some element of truth in it. A nervy man doesn't sit down to the delicate enjoyment of a cigar before bed, and cherish the ash with such scrupulous care. On the other hand, if Freddy's right, and Cathcart was feelin' unusually sleek and pleased with life, that's just the sort of thing he would do.«

»Do you think Mr Arbuthnot would have invented all that, as a matter of fact?« said Parker thoughtfully. »He doesn't strike me that way. He'd have to be imaginative and spiteful to make it up, and I really don't think he's either.«

»I know,« said Lord Peter. »I've known old Freddy all my life, and he wouldn't hurt a fly. Besides, he simply hasn't the wits to make up any sort of a story. But what bothers me is that Gerald most certainly hasn't the wits either to invent that Adelphi drama between him and Cathcart.«

»On the other hand,« said Parker, »if we allow for a moment that he shot Cathcart, he had an incentive to invent it. He would be trying to get his head out of the\allowbreak---\allowbreak I mean, when anything important is at stake it's wonderful how it sharpens one's wits. And the story being so far-fetched does rather suggest an unpracticed storyteller.«

»True, O King. Well, you've sat on all my discoveries so far. Never mind. My head is bloody but unbowed. Cathcart was sitting here\longdash«

»So your brother said.«

»Curse you, I say he was; at least, somebody was; he's left the impression of his sit-me-down-upon on the cushion.«

»That might have been earlier in the day.«

»Rot. They were out all day. You needn't overdo this Sadducee attitude, Charles. I say Cathcart was sitting here, and\allowbreak---\allowbreak Hullo! Hullo!«

He leaned forward and stared into the grate.

»There's some burnt paper here, Charles.«

»I know. I was frightfully excited about that yesterday, but I found it was just the same in several of the rooms. They often let the bedroom fires go out when everybody's out during the day, and relight them about an hour before dinner. There's only the cook, housemaid, and Fleming here, you see, and they've got a lot to do with such a large party.«

Lord Peter was picking the charred fragments over.

»I can find nothing to contradict your suggestion,« he sadly said, »and this fragment of the \textit{Morning Post} rather confirms it. Then we can only suppose that Cathcart sat here in a brown study, doing nothing at all. That doesn't get us much further, I'm afraid.« He got up and went to the dressing-table.

»I like these tortoiseshell sets,« he said, »and the perfume is »Baiser du Soir«---very nice too. New to me. I must draw Bunter's attention to it. A charming manicure set, isn't it? You know, I like being clean and neat and all that, but Cathcart was the kind of man who always impressed you as bein' just a little too well turned out. Poor devil! And he'll be buried at Golders Green after all. I only saw him once or twice, you know. He impressed me as knowin' about everything there was to know. I was rather surprised at Mary takin' to him, but, then, I know really awfully little about Mary. You see, she's five years younger than me. When the war broke out she'd just left school and gone to a place in Paris, and I joined up, and she came back and did nursing and social work, so I only saw her occasionally. At that time she was rather taken up with new schemes for puttin' the world to rights and hadn't a lot to say to me. And she got hold of some pacifist fellow who was a bit of a stumer, I fancy. Then I was ill, you know, and then I got the chuck from Barbara and didn't feel much like botherin' about other people's heart-to-hearts, and then I got mixed up in the Attenbury diamond case\allowbreak---\allowbreak and the result is I know uncommonly little about my own sister. But it looks as though her taste in men had altered. I know my mother said Cathcart had charm; that means he was attractive to women, I suppose. No man can see what makes that in another man, but mother is usually right. What's become of this fellow's papers?«

»He left very little here,« replied Parker. »There's a check-book on Cox's Charing Cross branch, but it's a new one and not very helpful. Apparently he only kept a small current account with them for convenience when he was in England. The checks are mostly to self, with an occasional hotel or tailor.«

»Any pass-book?«

»I think all his important papers are in Paris. He has a flat there, near the river somewhere. We're in communication with the Paris police. He had a room at the Albany. I've told them to lock it up till I get there. I thought of running up to town tomorrow.«

»Yes, you'd better. Any pocket-book?«

»Yes; here you are. About \textsterling 30 in various notes, a wine-merchant's card, and a bill for a pair of riding-breeches.«

»No correspondence?«

»Not a line.«

»No,« said Wimsey, »he was the kind, I imagine, that didn't keep letters. Much too good an instinct of self-preservation.«

»Yes. I asked the servants about his letters, as a matter of fact. They said he got a good number, but never left them about. They couldn't tell me much about the ones he wrote, because all the outgoing letters are dropped into the post-bag, which is carried down to the post-office as it is and opened there, or handed over to the postman when\allowbreak---\allowbreak or if\allowbreak---\allowbreak he calls. The general impression was that he didn't write much. The housemaid said she never found anything to speak of in the waste-paper basket.«

»Well, that's uncommonly helpful. Wait a moment. Here's his fountain-pen. Very handsome\allowbreak---\allowbreak Onoto with complete gold casing. Dear me! Entirely empty. Well, I don't know that one can deduce anything from that, exactly. I don't see any pencil about, by the way. I'm inclined to think you're wrong in supposing that he was writing letters.«

»I didn't suppose anything,« said Parker mildly. »I daresay you're right.«

Lord Peter left the dressing-table, looked through the contents of the wardrobe, and turned over the two or three books on the pedestal beside the bed.

»\foreignlanguage{french}{\textit{La Rôtisserie de la Reine Pédauque, L'Anneau d'Améthyste, South Wind}} (our young friend works out very true to type), \foreignlanguage{french}{\textit{Chronique d'un Cadet de Coutras}} (tut-tut, Charles!), \textit{Manon Lescaut}. H'm! Is there anything else in this room I ought to look at?«

»I don't think so. Where'd you like to go now?«

»We'll follow 'em down. Wait a jiff. Who are in the other rooms? Oh, yes. Here's Gerald's room. Helen's at church. In we go. Of course, this has been dusted and cleaned up, and generally ruined for purposes of observation?«

»I'm afraid so. I could hardly keep the Duchess out of her bedroom.«

»No. Here's the window Gerald shouted out of. H'm! Nothing in the grate here, naturally\allowbreak---\allowbreak the fire's been lit since. I say, I wonder where Gerald did put that letter to\allowbreak---\allowbreak Freeborn's, I mean.«

»Nobody's been able to get a word out of him about it,« said Parker. »Old Mr Murbles had a fearful time with him. The Duke insists simply that he destroyed it. Mr Murbles says that's absurd. So it is. If he was going to bring that sort of accusation against his sister's fiancé he'd want \textit{some} evidence of a method in his madness, wouldn't he? Or was he one of those Roman brothers who say simply: »As the head of the family I forbid the banns and that's enough«?«

»Gerald,« said Wimsey, »is a good, clean, decent, thoroughbred public schoolboy, and a shocking ass. But I don't think he's so medieval as that.«

»But if he has the letter, why not produce it?«

»Why, indeed? Letters from old college friends in Egypt aren't, as a rule, compromising.«

»You don't suppose,« suggested Parker tentatively, »that this Mr Freeborn referred in his letter to any old\allowbreak---\allowbreak er\allowbreak---\allowbreak entanglement which your brother wouldn't wish the Duchess to know about?«

Lord Peter paused, while absently examining a row of boots.

»That's an idea,« he said. »There were occasions\allowbreak---\allowbreak mild ones, but Helen would make the most of them.« He whistled thoughtfully. »Still, when it comes to the gallows\longdash«

»Do you suppose, Wimsey, that your brother really contemplates the gallows?« asked Parker.

»I think Murbles put it to him pretty straight,« said Lord Peter.

»Quite so. But does he actually realize\allowbreak---\allowbreak imaginatively---that it is possible to hang an English peer for murder on circumstantial evidence?«

Lord Peter considered this.

»Imagination isn't Gerald's strong point,« he admitted. »I suppose they do hang peers? They can't be beheaded on Tower Hill or anything?«

»I'll look it up,« said Parker; »but they certainly hanged Earl Ferrers in 1760.«

»Did they, though?« said Lord Peter. »Ah, well, as the old pagan said of the Gospels, after all, it was a long time ago, and we'll hope it wasn't true.«

»It's true enough,« said Parker; »and he was dissected and anatomized afterwards. But that part of the treatment is obsolete.«

»We'll tell Gerald about it,« said Lord Peter, »and persuade him to take the matter seriously. Which are the boots he wore Wednesday night?«

»These,« said Parker, »but the fool's cleaned them.«

»Yes,« said Lord Peter bitterly. »M'm! a good heavy lace-up boot\allowbreak---\allowbreak the sort that sends the blood to the head.«

»He wore leggings, too,« said Parker; »these.«

»Rather elaborate preparations for a stroll in the garden. But, as you were just going to say, the night was wet. I must ask Helen if Gerald ever suffered from insomnia.«

»I did. She said she thought not as a rule, but that he occasionally had toothache, which made him restless.«

»It wouldn't send one out of doors on a cold night, though. Well, let's get downstairs.«

They passed through the billiard-room, where the Colonel was making a sensational break, and into the small conservatory which led from it.

Lord Peter looked gloomily round at the chrysanthemums and boxes of bulbs.

»These damned flowers look jolly healthy,« he said. »Do you mean you've been letting the gardener swarm in here every day to water 'em?«

»Yes,« said Parker apologetically, »I did. But he's had strict orders only to walk on these mats.«

»Good,« said Lord Peter. »Take 'em up, then, and let's get to work.«

With his lens to his eye he crawled cautiously over the floor.

»They all came through this way, I suppose,« he said.

»Yes,« said Parker. »I've identified most of the marks. People went in and out. Here's the Duke. He comes in from outside. He trips over the body.« (Parker had opened the outer door and lifted some matting, to show a trampled patch of gravel, discolored with blood.) »He kneels by the body. Here are his knees and toes. Afterwards he goes into the house, through the conservatory, leaving a good impression in black mud and gravel just inside the door.«

Lord Peter squatted carefully over the marks.

»It's lucky the gravel's so soft here,« he said.

»Yes. It's just a patch. The gardener tells me it gets very trampled and messy just here owing to his coming to fill cans from the water-trough. They fill the trough up from the well every so often, and then carry the water away in cans. It got extra bad this year, and they put down fresh gravel a few weeks ago.«

»Pity they didn't extend their labors all down the path while they were about it,« grunted Lord Peter, who was balancing himself precariously on a small piece of sacking. »Well, that bears out old Gerald so far. Here's an elephant been over this bit of box border. Who's that?«

»Oh, that's a constable. I put him at eighteen stone. He's nothing. And this rubber sole with a patch on it is Craikes. He's all over the place. This squelchy-looking thing is Mr Arbuthnot in bedroom slippers, and the galoshes are Mr Pettigrew-Robinson. We can dismiss all those. But now here, just coming over the threshold, is a woman's foot in a strong shoe. I make that out to be Lady Mary's. Here it is again, just at the edge of the well. She came out to examine the body.«

»Quite so,« said Peter; »and then she came in again, with a few grains of red gravel on her shoes. Well, that's all right. Hullo!«

On the outer side of the conservatory were some shelves for small plants, and, beneath these, a damp and dismal bed of earth, occupied, in a sprawling and lackadaisical fashion, by stringy cactus plants and a sporadic growth of maidenhair fern, and masked by a row of large chrysanthemums in pots.

»What've you got?« inquired Parker, seeing his friend peering into this green retreat.

Lord Peter withdrew his long nose from between two pots and said: »Who put what down here?«

Parker hastened to the place. There, among the cacti, was certainly the clear mark of some oblong object, with corners, that had been stood out of sight on the earth behind the pots.

»It's a good thing Gerald's gardener ain't one of those conscientious blighters that can't even let a cactus alone for the winter,« said Lord Peter, »or he'd've tenderly lifted these little drooping heads\allowbreak---\allowbreak Oh! damn and blast the beastly plant for a crimson porcupine! You measure it.«

Parker measured it.

»Two and a half feet by six inches,« he said. »And fairly heavy, for it's sunk in and broken the plants about. Was it a bar of anything?«

»I fancy not,« said Lord Peter. »The impression is deeper on the farther side. I think it was something bulky set up on edge, and leaned against the glass. If you asked for my private opinion I should guess that it was a suit-case.«

»A suit-case!« exclaimed Parker. »Why a suit-case?«

»Why indeed? I think we may assume that it didn't stay here very long. It would have been exceedingly visible in the daytime. But somebody might very well have shoved it in here if they were caught with it\allowbreak---\allowbreak say at three o'clock in the morning\allowbreak---\allowbreak and didn't want it to be seen.«

»Then when did they take it away?«

»Almost immediately, I should say. Before daylight, anyhow, or even Inspector Craikes could hardly have failed to see it.«

»It's not the doctor's bag, I suppose?«

»No\allowbreak---\allowbreak unless the doctor's a fool. Why put a bag inconveniently in a damp and dirty place out of the way when every law of sense and convenience would urge him to pop it down handy by the body? No. Unless Craikes or the gardener has been leaving things about, it was thrust away there on Wednesday night by Gerald, by Cathcart\allowbreak---\allowbreak or, I suppose, by Mary. Nobody else could be supposed to have anything to hide.«

»Yes,« said Parker, »one person.«

»Who's that?«

»The Person Unknown.«

»Who's he?«

For answer Mr Parker proudly stepped to a row of wooden frames, carefully covered with matting. Stripping this away, with the air of a bishop unveiling a memorial, he disclosed a V-shaped line of footprints.

»These,« said Parker, »belong to nobody\allowbreak---\allowbreak to nobody I've ever seen or heard of, I mean.«

»Hurray!« said Peter. »
\begin{verse}
Then downwards from the steep hill's edge\\
They tracked the footmarks small
\end{verse}
(only they're largish).«

»No such luck,« said Parker. »It's more a case of:
\begin{verse}
They followed from the earthy bank\\
Those footsteps one by one,\\
Into the middle of the plank;\\
And farther there were none!\end{verse}«

»Great poet, Wordsworth,« said Lord Peter; »how often I've had that feeling. Now let's see. These footmarks\allowbreak---\allowbreak a man's № 10 with worn-down heels and a patch on the left inner side\allowbreak---\allowbreak advance from the hard bit of the path which shows no footmarks; they come to the body\allowbreak---\allowbreak here, where that pool of blood is. I say, that's rather odd, don't you think? No? Perhaps not. There are no footmarks under the body? Can't say, it's such a mess. Well, the Unknown gets so far\allowbreak---\allowbreak here's a footmark deeply pressed in. Was he just going to throw Cathcart into the well? He hears a sound; he starts; he turns; he runs on tiptoe\allowbreak---\allowbreak into the shrubbery, by Jove!«

»Yes,« said Parker, »and the tracks come out on one of the grass paths in the wood, and there's an end of them.«

»H'm! Well, we'll follow them later. Now where did they come from?«

Together the two friends followed the path away from the house. The gravel, except for the little patch before the conservatory, was old and hard, and afforded but little trace, particularly as the last few days had been rainy. Parker, however, was able to assure Wimsey that there had been definite traces of dragging and bloodstains.

»What sort of bloodstains? Smears?«

»Yes, smears mostly. There were pebbles displaced, too, all the way\allowbreak---\allowbreak and now here is something odd.«

It was the clear impression of the palm of a man's hand heavily pressed into the earth of a herbaceous border, the fingers pointing towards the house. On the path the gravel had been scraped up in two long furrows. There was blood on the grass border between the path and the bed, and the edge of the grass was broken and trampled.

»I don't like that,« said Lord Peter.

»Ugly, isn't it?« agreed Parker.

»Poor devil!« said Peter. »He made a determined effort to hang on here. That explains the blood by the conservatory door. But what kind of a devil drags a corpse that isn't quite dead?«

A few yards farther the path ran into the main drive. This was bordered with trees, widening into a thicket. At the point of intersection of the two paths were some further indistinct marks, and in another twenty yards or so they turned aside into the thicket. A large tree had fallen at some time and made a little clearing, in the midst of which a tarpaulin had been carefully spread out and pegged down. The air was heavy with the smell of fungus and fallen leaves.

»Scene of the tragedy,« said Parker briefly, rolling back the tarpaulin.

Lord Peter gazed down sadly. Muffled in an overcoat and a thick grey scarf, he looked, with his long, narrow face, like a melancholy adjutant stork. The writhing body of the fallen man had scraped up the dead leaves and left a depression in the sodden ground. At one place the darker earth showed where a great pool of blood had soaked into it, and the yellow leaves of a Spanish poplar were rusted with no autumnal stain.

»That's where they found the handkerchief and revolver,« said Parker. »I looked for finger-marks, but the rain and mud had messed everything up.«

Wimsey took out his lens, lay down, and conducted a personal tour of the whole space slowly on his stomach, Parker moving mutely after him.

»He paced up and down for some time,« said Lord Peter. »He wasn't smoking. He was turning something over in his mind, or waiting for somebody. What's this? Aha! Here's our № 10 foot again, coming in through the trees on the farther side. No signs of a struggle. That's odd! Cathcart was shot close up, wasn't he?«

»Yes; it singed his shirt-front.«

»Quite so. Why did he stand still to be shot at?«

»I imagine,« said Parker, »that if he had an appointment with № 10 Boots it was somebody he knew, who could get close to him without arousing suspicion.«

»Then the interview was a friendly one\allowbreak---\allowbreak on Cathcart's side, anyhow. But the revolver's a difficulty. How did № 10 get hold of Gerald's revolver?«

»The conservatory door was open,« said Parker dubiously.

»Nobody knew about that except Gerald and Fleming,« retorted Lord Peter. »Besides, do you mean to tell me that № 10 walked in here, went to the study, fetched the revolver, walked back here, and shot Cathcart? It seems a clumsy method. If he wanted to do any shooting, why didn't he come armed in the first place?«

»It seems more probable that Cathcart brought the revolver,« said Parker.

»Then why no signs of a struggle?«

»Perhaps Cathcart shot himself,« said Parker.

»Then why should № 10 drag him into a conspicuous position and then run away?«

»Wait a minute,« said Parker. »How's this? № 10 has an appointment with Cathcart\allowbreak---\allowbreak to blackmail him, let's say. He somehow gets word of his intention to him between 9:45 and 10:15. That would account for the alteration in Cathcart's manner, and allow both Mr Arbuthnot and the Duke to be telling the truth. Cathcart rushes violently out after his row with your brother. He comes down here to keep his appointment. He paces up and down waiting for № 10. № 10 arrives and parleys with Cathcart. Cathcart offers him money. № 10 stands out for more. Cathcart says he really hasn't got it. № 10 says in that case he blows the gaff. Cathcart retorts, »In that case you can go to the devil. I'm going there myself.« Cathcart, who has previously got hold of the revolver, shoots himself. № 10 is seized with remorse. He sees that Cathcart isn't quite dead. He picks him up and part drags, part carries him to the house. He is smaller than Cathcart and not very strong, and finds it a hard job. They have just got to the conservatory door when Cathcart has a final hemorrhage and gives up the ghost. № 10 suddenly becomes aware that his position in somebody else's grounds, alone with a corpse at 3 \textsc{a.m.}, wants some explaining. He drops Cathcart\allowbreak---\allowbreak and bolts. Enter the Duke of Denver and falls over the body. Tableau.«

»That's good,« said Lord Peter; »that's very good. But when do you suppose it happened? Gerald found the body at 3 \textsc{a.m.}; the doctor was here at 4:30, and said Cathcart had been dead several hours. Very well. Now, how about that shot my sister heard at three o'clock?«

»Look here, old man,« said Parker, »I don't want to appear rude to your sister. May I put it like this? I suggest that that shot at 3 \textsc{a.m.} was poachers.«

»Poachers by all means,« said Lord Peter. »Well, really, Parker, I think that hangs together. Let's adopt that explanation provisionally. The first thing to do is now to find № 10, since he can bear witness that Cathcart committed suicide; and that, as far as my brother is concerned, is the only thing that matters a rap. But for the satisfaction of my own curiosity I'd like to know: What was № 10 blackmailing Cathcart about? Who hid a suit-case in the conservatory? And what was Gerald doing in the garden at 3 \textsc{a.m.}?«

»Well,« said Parker, »suppose we begin by tracing where № 10 came from.«

»Hi, hi!« cried Wimsey, as they returned to the trail. »Here's something\allowbreak---\allowbreak here's real treasure-trove, Parker!«

From amid the mud and the fallen leaves he retrieved a tiny, glittering object\allowbreak---\allowbreak a flash of white and green between his fingertips.

It was a little charm such as women hang upon a bracelet\allowbreak---\allowbreak a diminutive diamond cat with eyes of bright emerald.