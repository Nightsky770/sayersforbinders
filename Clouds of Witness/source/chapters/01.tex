%!TeX root=../cloudstop.tex



\chapter{Of His Malice Aforethought}

\epigraph{O, who hath done this deed?}{\textit{Othello}}


\lettrine[lines=4]{L}{ord} Peter Wimsey stretched himself luxuriously be\-tween the sheets provided by the Hôtel Meurice. After his exertions in the unravelling of the Battersea Mystery, he had followed Sir Julian Freke's advice and taken a holiday. He had felt suddenly weary of breakfasting every morning before his view over the Green Park; he had realized that the picking up of first editions at sales afforded insufficient exercise for a man of thirty-three; the very crimes of London were over-sophisticated. He had abandoned his flat and his friends and fled to the wilds of Corsica. For the last three months he had forsworn letters, newspapers, and telegrams. He had tramped about the mountains, admiring from a cautious distance the wild beauty of Corsican peasant-women, and studying the vendetta in its natural haunt. In such conditions murder seemed not only reasonable, but lovable. Bunter, his confidential man and assistant sleuth, had nobly sacrificed his civilized habits, had let his master go dirty and even unshaven, and had turned his faithful camera from the recording of fingerprints to that of craggy scenery. It had been very refreshing.

Now, however, the call of the blood was upon Lord~Peter. They had returned late last night in a vile train to Paris, and had picked up their luggage. The autumn light, filtering through the curtains, touched caressingly the silver-topped bottles on the dressing-table, outlined an electric lamp-shade and the shape of the telephone. A noise of running water near by proclaimed that Bunter had turned on the bath (h. \& c.) and was laying out scented soap, bath-salts, the huge bath-sponge, for which there had been no scope in Corsica, and the delightful flesh-brush with the long handle, which rasped you so agreeably all down the spine. »Contrast,« philosophized Lord~Peter sleepily, »is life. Corsica—Paris—then London\dots Good morning, Bunter.«

»Good morning, my lord. Fine morning, my lord. Your lordship's bath-water is ready.«

»Thanks,« said Lord~Peter. He blinked at the sunlight.

It was a glorious bath. He wondered, as he soaked in it, how he could have existed in Corsica. He wallowed happily and sang a few bars of a song. In a soporific interval he heard the valet de chambre bringing in coffee and rolls. Coffee and rolls! He heaved himself out with a splash, towelled himself luxuriously, enveloped his long-mortified body in a silken bath-robe, and wandered back.

To his immense surprise he perceived Mr~Bunter calmly replacing all the fittings in his dressing-case. Another astonished glance showed him the bags—scarcely opened the previous night—repacked, relabelled, and standing ready for a journey.

»I say, Bunter, what's up?« said his lordship. »We're stayin' here a fortnight y'know.«

»Excuse me, my lord,« said Mr~Bunter, deferentially, »but, having seen \textit{The Times} (delivered here every morning by air, my lord; and very expeditious I'm sure, all things considered), I made no doubt your lordship would be wishing to go to Riddlesdale at once.«

»Riddlesdale!« exclaimed Peter. »What's the matter? Anything wrong with my brother?«
%\enlargethispage{\baselineskip}

For answer Mr~Bunter handed him the paper, folded open at the heading:
\begin{center}
\textsc{Riddlesdale Inquest\\
Duke of Denver Arrested\\
On Murder Charge}
\end{center}


Lord~Peter stared as if hypnotized.

»I thought your lordship wouldn't wish to miss anything,« said Mr~Bunter, »so I took the liberty\longdash«

Lord~Peter pulled himself together.

»When's the next train?« he asked.

»I beg your lordship's pardon—I thought your lordship would wish to take the quickest route. I took it on myself to book two seats in the airplane \textit{Victoria}. She starts at 11:30.«

Lord~Peter looked at his watch.

»Ten o'clock,« he said. »Very well. You did quite right. Dear me! Poor old Gerald arrested for murder. Uncommonly worryin' for him, poor chap.  Always hated my bein' mixed up with police-courts. Now he's there himself. Lord~Peter Wimsey in the witness-box—very distressin' to feelin's of a brother. Duke of Denver in the dock—worse still. Dear me! Well, I suppose one must have breakfast.«

»Yes, my lord. Full account of the inquest in the paper, my lord.«

»Yes. Who's on the case, by the way?«

»Mr~Parker, my lord.«

»Parker? That's good. Splendid old Parker! Wonder how he managed to get put on to it. How do things look, Bunter?«

»If I may say so, my lord, I fancy the investigations will prove very interesting. There are several extremely suggestive points in the evidence, my lord.«

»From a criminological point of view I daresay it is interesting,« replied his lordship, sitting down cheerfully to his \foreignlanguage{french}{\textit{café au lait}}, »but it's deuced awkward for my brother, all the same, havin' no turn for criminology, what?«

»Ah, well!« said Mr~Bunter, »they say, my lord, there's nothing like having a personal interest.«

\begin{quote}
\indent The inquest was held today at Riddlesdale, in the North Riding of Yorkshire, on the body of Captain Denis Cathcart, which was found at three o'clock on Thursday morning lying just outside the conservatory door of the Duke of Denver's shooting-box, Riddlesdale Lodge. Evidence was given to show that deceased had quarrelled with the Duke of Denver on the preceding evening, and was subsequently shot in a small thicket adjoining the house. A pistol belonging to the Duke was found near the scene of the crime. A verdict of murder was returned against the Duke of Denver. Lady~Mary Wimsey, sister of the Duke, who was engaged to be married to the deceased, collapsed after giving evidence, and is now lying seriously ill at the Lodge. The Duchess of Denver hastened from town yesterday and was present at the inquest. Full report on p. 12.  
\end{quote}

»Poor old Gerald!« thought Lord~Peter, as he turned to page 12; »and poor old Mary! I wonder if she really was fond of the fellow. Mother always said not, but Mary never would let on about herself.«

The full report began by describing the little village of Riddlesdale, where the Duke of Denver had recently taken a small shooting-box for the season. When the tragedy occurred the Duke had been staying there with a party of guests. In the Duchess's absence Lady~Mary Wimsey had acted as hostess. The other guests were Colonel and Mrs~Marchbanks, the Hon. Frederick Arbuthnot, Mr~and Mrs~Pettigrew-Robinson, and the dead man, Denis Cathcart.

The first witness was the Duke of Denver, who claimed to have discovered the body. He gave evidence that he was coming into the house by the conservatory door at three o'clock in the morning of Thursday, October 14th, when his foot struck against something. He had switched on his electric torch and seen the body of Denis Cathcart at his feet.  He had at once turned it over, and seen that Cathcart had been shot in the chest. He was quite dead. As Denver was bending over the body, he heard a cry in the conservatory, and, looking up, saw Lady~Mary Wimsey gazing out horror-struck. She came out by the conservatory door, and exclaimed at once, »O God, Gerald, you've killed him!« (Sensation).\footnote{This report, though substantially the same as that read by Lord~Peter in \textit{The Times}, has been corrected, amplified and annotated from the shorthand report made at the time by Mr~Parker.}

\begin{dialogue}

\speak{The Coroner} Were you surprised by that remark?

\speak{Duke of D.} Well, I was so shocked and surprised at the whole thing.  I think I said to her, »Don't look,« and she said, »Oh, it's Denis!  Whatever can have happened? Has there been an accident?« I stayed with the body, and sent her up to rouse the house.

\speak{The Coroner} Did you expect to see Lady~Mary Wimsey in the conservatory?

\speak{Duke of D.} Really, as I say, I was so astonished all round, don't you know, I didn't think about it.

\speak{The Coroner} Do you remember how she was dressed?

\speak{Duke of D.} I don't think she was in her pyjamas. \direct{Laughter.} I think she had a coat on.

\speak{The Coroner} I understand that Lady~Mary Wimsey was engaged to be married to the deceased?

\speak{Duke of D.} Yes.

\speak{The Coroner} He was well known to you?

\speak{Duke of D.} He was the son of an old friend of my father's; his parents are dead. I believe he lived chiefly abroad. I ran across him during the war, and in 1919 he came to stay at Denver. He became engaged to my sister at the beginning of this year.

\speak{The Coroner} With your consent, and with that of the family?

\speak{Duke of D.} Oh, yes, certainly.

\speak{The Coroner} What kind of man was Captain Cathcart?

\speak{Duke of D.} Well—he was a Sahib and all that. I don't know what he did before he joined in 1914. I think he lived on his income; his father was well off. Crack shot, good at games, and so on. I never heard anything against him—till that evening.

\speak{The Coroner} What was that?

\speak{Duke of D.} Well—the fact is—it was deuced queer. He—If anybody but Tommy Freeborn had said it I should never have believed it. \direct{Sensation.}

\speak{The Coroner} I'm afraid I must ask your grace of what exactly you had to accuse the deceased.

\speak{Duke of D.} Well, I didn't—I don't—exactly accuse him. An old friend of mine made a suggestion. Of course I thought it must be all a mistake, so I went to Cathcart, and, to my amazement, he practically admitted it! Then we both got angry, and he told me to go to the devil, and rushed out of the house. \direct{Renewed sensation.}

\speak{The Coroner} When did this quarrel occur?

\speak{Duke of D.} On Wednesday night. That was the last I saw of him. \direct{Unparalleled sensation.}

\speak{The Coroner} Please, please, we cannot have this disturbance. Now, will your grace kindly give me, as far as you can remember it, the exact history of this quarrel?

\speak{Duke of D.} Well, it was like this. We'd had a long day on the moors and had dinner early, and about half-past nine we began to feel like turning in. My sister and Mrs~Pettigrew-Robinson toddled on up, and we were havin' a last peg in the billiard-room when Fleming—that's my man—came in with the letters. They come rather any old time in the evening, you know, we being two and a half miles from the village.  No—I wasn't in the billiard-room at the time—I was lockin' up the gun-room. The letter was from an old friend of mine I hadn't seen for years—Tom Freeborn—used to know him at the House—

\speak{The Coroner} Whose house?

\speak{Duke of D.} Oh, Christ Church, Oxford. He wrote to say he'd seen the announcement of my sister's engagement in Egypt.

\speak{The Coroner} In Egypt?

\speak{Duke of D.} I mean, \textit{he} was in Egypt—Tom Freeborn, you see—that's why he hadn't written before. He engineers. He went out there after the war was over, you see, and, bein' somewhere up near the sources of the Nile, he doesn't get the papers regularly. He said, would I 'scuse him for interferin' in a very delicate matter, and all that, but did I know who Cathcart was? Said he'd met him in Paris during the war, and he lived by cheatin' at cards—said he could swear to it, with details of a row there'd been in some French place or other. Said he knew I'd want to chaw his head off—Freeborn's, I mean—for buttin' in, but he'd seen the man's photo in the paper, an' he thought I ought to know.

\speak{The Coroner} Did this letter surprise you?

\speak{Duke of D.} Couldn't believe it at first. If it hadn't been old Tom Freeborn I'd have put the thing in the fire straight off, and, even as it was, I didn't quite know what to think. I mean, it wasn't as if it had happened in England, you know. I mean to say, Frenchmen get so excited about nothing. Only there was Freeborn, and he isn't the kind of man that makes mistakes.

\speak{The Coroner} What did you do?

\speak{Duke of D.} Well, the more I looked at it the less I liked it, you know. Still, I couldn't quite leave it like that, so I thought the best way was to go straight to Cathcart. They'd all gone up while I was sittin' thinkin' about it, so I went up and knocked at Cathcart's door.  He said, »What's that?« or »Who the devil's that?« or somethin' of the sort, and I went in, »Look here,« I said, »can I just have a word with you?« »Well, cut it short, then,« he said. I was surprised—he wasn't usually rude. »Well,« I said, »fact is, I've had a letter I don't much like the look of, and I thought the best thing to do was to bring it straight away to you an' have the whole thing cleared up. It's from a man—a very decent sort—old college friend, who says he's met you in Paris.« »Paris!« he said, in a most uncommonly unpleasant way. »Paris!  What the hell do you want to come talkin' to me about Paris for?« »Well,« I said, »don't talk like that, because it's misleadin' under the circumstances.« »What are you drivin' at?« says Cathcart. »Spit it out and go to bed, for God's sake.« I said, »Right oh! I will. It's a man called Freeborn, who says he knew you in Paris and that you made money cheatin' at cards.« I thought he'd break out at that, but all he said was, »What about it?« »What about it?« I said. »Well, of course, it's not the sort of thing I'm goin' to believe like that, right bang-slap off, without any proofs.« Then he said a funny thing. He said, »Beliefs don't matter—it's what one \textit{knows} about people.« »Do you mean to say you don't deny it?« I said. »It's no good my denying it,« he said; »you must make up your own mind. Nobody could \textit{dis}prove it.« And then he suddenly jumped up, nearly knocking the table over, and said, »I don't care what you think or what you do, if you'll only get out. For God's sake leave me alone!« »Look here,« I said, »you needn't take it that way. I don't say I do believe it—in fact,« I said, »I'm sure there must be some mistake; only, you bein' engaged to Mary,« I said, »I couldn't just let it go at that without looking into it, could I?« »Oh!« says Cathcart, »if that's what's worrying you, it needn't. That's off.« I said, »What?« He said, »Our engagement.« »Off?« I said. »But I was talking to Mary about it only yesterday.« »I haven't told her yet,« he said. »Well,« I said, »I think that's damned cool.  Who the hell do you think you are, to come here and jilt my sister?« Well, I said quite a lot, first and last. »You can get out,« I said; »I've no use for swine like you.« »I will,« he said, and he pushed past me an' slammed downstairs and out of the front door, an' banged it after him.

\speak{The Coroner} What did you do?

\speak{Duke of D.} I ran into my bedroom, which has a window over the conservatory, and shouted out to him not to be a silly fool. It was pourin' with rain and beastly cold. He didn't come back, so I told Fleming to leave the conservatory door open—in case he thought better of it—and went to bed.

\speak{The Coroner} What explanation can you suggest for Cathcart's behaviour?

\speak{Duke of D.} None. I was simply staggered. But I think he must somehow have got wind of the letter, and knew the game was up.

\speak{The Coroner} Did you mention the matter to anybody else?

\speak{Duke of D.} No. It wasn't pleasant, and I thought I'd better leave it till the morning.

\speak{The Coroner} So you did nothing further in the matter?

\speak{Duke of D.} No. I didn't want to go out huntin' for the fellow. I was too angry. Besides, I thought he'd change his mind before long—it was a brute of a night and he'd only a dinner-jacket.

\speak{The Coroner} Then you just went quietly to bed and never saw deceased again?

\speak{Duke of D.} Not till I fell over him outside the conservatory at three in the morning.

\speak{The Coroner} Ah, yes. Now can you tell us how you came to be out of doors at that time?

\speak{Duke of D.} \direct{hesitating} I didn't sleep well. I went out for a stroll.

\speak{The Coroner} At three o'clock in the morning?

\speak{Duke of D.} Yes. \direct{With sudden inspiration} You see, my wife's away. \direct{Laughter and some remarks from the back of the room.}

\speak{The Coroner} Silence, please\textellipsis . You mean to say that you got up at that hour of an October night to take a walk in the garden in the pouring rain?

\speak{Duke of D.} Yes, just a stroll. \direct{Laughter.}

\speak{The Coroner} At what time did you leave your bedroom?

\speak{Duke of D.} Oh—oh, about half-past two, I should think.

\speak{The Coroner} Which way did you go out?

\speak{Duke of D.} By the conservatory door.

\speak{The Coroner} The body was not there when you went out?

\speak{Duke of D.} Oh, no!

\speak{The Coroner} Or you would have seen it?

\speak{Duke of D.} Lord, yes! I'd have had to walk over it.

\speak{The Coroner} Exactly where did you go?

\speak{Duke of D.} \direct{vaguely} Oh, just round about.

\speak{The Coroner} You heard no shot?

\speak{Duke of D.} No.

\speak{The Coroner} Did you go far away from the conservatory door and the shrubbery?

\speak{Duke of D.} Well—I was some way away. Perhaps that's why I didn't hear anything. It must have been.

\speak{The Coroner} Were you as much as a quarter of a mile away?

\speak{Duke of D.} I should think I was—oh, yes, quite!

\speak{The Coroner} More than a quarter of a mile away?

\speak{Duke of D.} Possibly. I walked about briskly because it was cold.

\speak{The Coroner} In which direction?

\speak{Duke of D.} \direct{with visible hesitation} Round at the back of the house.  Towards the bowling-green.

\speak{The Coroner} The bowling-green?

\speak{Duke of D.} \direct{more confidently} Yes.

\speak{The Coroner} But if you were more than a quarter of a mile away, you must have left the grounds?

\speak{Duke of D.} I—oh, yes—I think I did. Yes, I walked about on the moor a bit, you know.

\speak{The Coroner} Can you show us the letter you had from Mr~Freeborn?

\speak{Duke of D.} Oh, certainly—if I can find it. I thought I put it in my pocket, but I couldn't find it for that Scotland Yard fellow.

\speak{The Coroner} Can you have accidentally destroyed it?

\speak{Duke of D.} No—I'm sure I remember putting it—Oh—\direct{here the witness paused in very patent confusion, and grew red}—I remember now. I destroyed it.

\speak{The Coroner} That is unfortunate. How was that?

\speak{Duke of D.} I had forgotten; it has come back to me now. I'm afraid it has gone for good.

\speak{The Coroner} Perhaps you kept the envelope?

\speak{Witness} \direct{shook his head.}

\speak{The Coroner} Then you can show the jury no proof of having received it?

\speak{Duke of D.} Not unless Fleming remembers it.

\speak{The Coroner} Ah, yes! No doubt we can check it that way. Thank you, your grace. Call Lady~Mary Wimsey.
\end{dialogue}

The noble lady, who was, until the tragic morning of October 14th, the fiancée of the deceased, aroused a murmur of sympathy on her appearance. Fair and slender, her naturally rose-pink cheeks ashy pale, she seemed overwhelmed with grief. She was dressed entirely in black, and gave her evidence in a very low tone which was at times almost inaudible.\footnote{From the newspaper report—\textit{not} Mr~Parker.}

After expressing his sympathy, the coroner asked, »How long had you been engaged to the deceased?«

\begin{dialogue}
\speak{Witness} About eight months.

\speak{The Coroner} Where did you first meet him?

\speak{Witness} At my sister-in-law's house in London.

\speak{The Coroner} When was that?

\speak{Witness} I think it was June last year.

\speak{The Coroner} You were quite happy in your engagement?

\speak{Witness} Quite.

\speak{The Coroner} You naturally saw a good deal of Captain Cathcart. Did he tell you much about his previous life?

\speak{Witness} Not very much. We were not given to mutual confidences. We usually discussed subjects of common interest.

\speak{The Coroner} You had many such subjects?

\speak{Witness} Oh, yes.

\speak{The Coroner} You never gathered at any time that Captain Cathcart had anything on his mind?

\speak{Witness} Not particularly. He had seemed a little anxious the last few days.

\speak{The Coroner} Did he speak of his life in Paris?

\speak{Witness} He spoke of theatres and amusements there. He knew Paris very well. I was staying in Paris with some friends last February, when he was there, and he took us about. That was shortly after our engagement.

\speak{The Coroner} Did he ever speak of playing cards in Paris?

\speak{Witness} I don't remember.

\speak{The Coroner} With regard to your marriage—had any money settlements been gone into?

\speak{Witness} I don't think so. The date of the marriage was not in any way fixed.

\speak{The Coroner} He always appeared to have plenty of money?

\speak{Witness} I suppose so; I didn't think about it.

\speak{The Coroner} You never heard him complain of being hard up?

\speak{Witness} Everybody complains of that, don't they?

\speak{The Coroner} Was he a man of cheerful disposition?

\speak{Witness} He was very moody, never the same two days together.

\speak{The Coroner} You have heard what your brother says about the deceased wishing to break off the engagement. Had you any idea of this?

\speak{Witness} Not the slightest.

\speak{The Coroner} Can you think of any explanation now?

\speak{Witness} Absolutely none.

\speak{The Coroner} There had been no quarrel?

\speak{Witness} No.

\speak{The Coroner} So far as you knew, on the Wednesday evening, you were still engaged to deceased with every prospect of being married to him shortly?

\speak{Witness} Ye-es. Yes, certainly, of course.

\speak{The Coroner} He was not—forgive me this very painful question—the sort of man who would have been likely to lay violent hands on himself?

\speak{Witness} Oh, I never thought—well, I don't know—I suppose he might have done. That would explain it, wouldn't it?

\speak{The Coroner} Now, Lady~Mary—please don't distress yourself, take your own time—will you tell us exactly what you heard and saw on Wednesday night and Thursday morning.

\speak{Witness} I went up to bed with Mrs~Marchbanks and Mrs~Pettigrew-Robinson at about half-past nine, leaving all the men downstairs. I said good night to Denis, who seemed quite as usual. I was not downstairs when the post came. I went to my room at once. My room is at the back of the house. I heard Mr~Pettigrew-Robinson come up at about ten. The Pettigrew-Robinsons sleep next door to me. Some of the other men came up with him. I did not hear my brother come upstairs. At about a quarter past ten I heard two men talking loudly in the passage, and then I heard someone run downstairs and bang the front door. Afterwards I heard rapid steps in the passage, and finally I heard my brother shut his door. Then I went to bed.

\speak{The Coroner} You did not inquire the cause of the disturbance?

\speak{Witness} \direct{indifferently} I thought it was probably something about the dogs.

\speak{The Coroner} What happened next?

\speak{Witness} I woke up at three o'clock.

\speak{The Coroner} What wakened you?

\speak{Witness} I heard a shot.

\speak{The Coroner} You were not awake before you heard it?

\speak{Witness} I may have been partly awake. I heard it very distinctly. I was sure it was a shot. I listened for a few minutes, and then went down to see if anything was wrong.

\speak{The Coroner} Why did you not call your brother or some other gentleman?

\speak{Witness} \direct{scornfully} Why should I? I thought it was probably only poachers, and I didn't want to make an unnecessary fuss at that unearthly hour.

\speak{The Coroner} Did the shot sound close to the house?

\speak{Witness} Fairly, I think—it is hard to tell when one is wakened by a noise—it always sounds so extra loud.

\speak{The Coroner} It did not seem to be in the house or in the conservatory?

\speak{Witness} No. It was outside.

\speak{The Coroner} So you went downstairs by yourself. That was very plucky of you, Lady~Mary. Did you go immediately?

\speak{Witness} Not quite immediately. I thought it over for a few minutes; then I put on walking-shoes over bare feet, a heavy covert-coat, and a woolly cap. It may have been five minutes after hearing the shot that I left my bedroom. I went downstairs and through the billiard-room to the conservatory.

\speak{The Coroner} Why did you go out that way?

\speak{Witness} Because it was quicker than unbolting either the front door or the back door.
\end{dialogue}

At this point a plan of Riddlesdale Lodge was handed to the jury. It is a roomy, two-storied house, built in a plain style, and leased by the present owner, Mr~Walter Montague, to Lord~Denver for the season, Mr~Montague being in the States.

\begin{dialogue}
\speak{Witness} \direct{resuming} When I got to the conservatory door I saw a man outside, bending over something on the ground. When he looked up I was astonished to see my brother.

\speak{The Coroner} Before you saw who it was, what did you expect?

\speak{Witness} I hardly know—it all happened so quickly. I thought it was burglars, I think.

\speak{The Coroner} His grace has told us that when you saw him you cried out, »O God! you've killed him!« Can you tell us why you did that?

\speak{Witness} \direct{very pale} I thought my brother must have come upon the burglar and fired at him in self-defence—that is, if I thought at all.

\speak{The Coroner} Quite so. You knew that the Duke possessed a revolver?

\speak{Witness} Oh, yes—I think so.

\speak{The Coroner} What did you do next?

\speak{Witness} My brother sent me up to get help. I knocked up Mr~Arbuthnot and Mr~and Mrs~Pettigrew-Robinson. Then I suddenly felt very faint, and went back to my bedroom and took some sal volatile.

\speak{The Coroner} Alone?

\speak{Witness} Yes. Everybody was running about and calling out. I couldn't bear it—I—
\end{dialogue}

Here the witness, who up till this moment had given her evidence very collectedly, though in a low voice, collapsed suddenly, and had to be assisted from the room.


The next witness called was James Fleming, the manservant. He remembered having brought the letters from Riddlesdale at 9:45 on Wednesday evening. He had taken three or four letters to the Duke in the gun-room. He could not remember at all whether one of them had had an Egyptian stamp. He did not collect stamps; his hobby was autographs.

The Hon. Frederick Arbuthnot then gave evidence. He had gone up to bed with the rest at a little before ten. He had heard Denver come up by himself some time later—couldn't say how much later—he was brushing his teeth at the time. \direct{Laughter.} Had certainly heard loud voices and a row going on next door and in the passage. Had heard somebody go for the stairs hell-for-leather. Had stuck his head out and seen Denver in the passage. Had said, »Hello, Denver, what's the row?« The Duke's reply had been inaudible. Denver had bolted into his bedroom and shouted out of the window, »Don't be an ass, man!« He had seemed very angry indeed, but the Hon. Freddy attached no importance to that.  One was always getting across Denver, but it never came to anything.  More dust than kick in his opinion. Hadn't known Cathcart long—always found him all right—no, he didn't \textit{like} Cathcart, but he was all right, you know, nothing wrong about him that he knew of. Good lord, no, he'd never heard it suggested he cheated at cards! Well, no, of course, he didn't go about looking out for people cheating at cards—it wasn't a thing one expected. He'd been had that way in a club at Monte once—he'd had no hand in bringing it to light—hadn't noticed anything till the fun began. Had not noticed anything particular in Cathcart's manner to Lady~Mary, or hers to him. Didn't suppose he ever would notice anything; did not consider himself an observing sort of man. Was not interfering by nature; had thought Wednesday evening's dust-up none of his business. Had gone to bed and to sleep.

\begin{dialogue}

\speak{The Coroner} Did you hear anything further that night?

\speak{Hon. Frederick} Not till poor little Mary knocked me up. Then I toddled down and found Denver in the conservatory, bathing Cathcart's head. We thought we ought to clean the gravel and mud off his face, you know.

\speak{The Coroner} You heard no shot?

\speak{Hon. Frederick} Not a sound. But I sleep pretty heavily.
\end{dialogue}

Colonel and Mrs~Marchbanks slept in the room over what was called the study—more a sort of smoking-room really. They both gave the same account of a conversation which they had had at 11:30. Mrs~Marchbanks had sat up to write some letters after the Colonel was in bed. They had heard voices and someone running about, but had paid no attention.  It was not unusual for members of the party to shout and run about.  At last the Colonel had said, »Come to bed, my dear, it's half-past eleven, and we're making an early start tomorrow. You won't be fit for anything.« He said this because Mrs~Marchbanks was a keen sportswoman and always carried her gun with the rest. She replied, »I'm just coming.« The Colonel said, »You're the only sinner burning the midnight oil—everybody's turned in.« Mrs~Marchbanks replied, »No, the Duke's still up; I can hear him moving about in the study.« Colonel Marchbanks listened and heard it too. Neither of them heard the Duke come up again. They had heard no noise of any kind in the night.

Mr~Pettigrew-Robinson appeared to give evidence with extreme reluctance. He and his wife had gone to bed at ten. They had heard the quarrel with Cathcart. Mr~Pettigrew-Robinson, fearing that something might be going to happen, opened his door in time to hear the Duke say, »If you dare to speak to my sister again I'll break every bone in your body«, or words to that effect. Cathcart had rushed downstairs. The Duke was scarlet in the face. He had not seen Mr~Pettigrew-Robinson, but had spoken a few words to Mr~Arbuthnot, and rushed into his own bedroom. Mr~Pettigrew-Robinson had run out, and said to Mr~Arbuthnot, »I say, Arbuthnot«, and Mr~Arbuthnot had very rudely slammed the door in his face. He had then gone to the Duke's door and said, »I say, Denver«. The Duke had come out, pushing past him, without even noticing him, and gone to the head of the stairs. He had heard him tell Fleming to leave the conservatory door open, as Mr~Cathcart had gone out. The Duke had then returned. Mr~Pettigrew-Robinson had tried to catch him as he passed, and had said again, »I say, Denver, what's up?« The Duke had said nothing, and had shut his bedroom door with great decision. Later on, however, at 11:30 to be precise, Mr~Pettigrew-Robinson had heard the Duke's door open, and stealthy feet moving about the passage. He could not hear whether they had gone downstairs. The bathroom and lavatory were at his end of the passage, and, if anybody had entered either of them, he thought he should have heard. He had not heard the footsteps return. He had heard his travelling clock strike twelve before falling asleep. There was no mistaking the Duke's bedroom door, as the hinge creaked in a peculiar manner.

Mrs~Pettigrew-Robinson confirmed her husband's evidence. She had fallen asleep before midnight, and had slept heavily. She was a heavy sleeper at the beginning of the night, but slept lightly in the early morning. She had been annoyed by all the disturbance in the house that evening, as it had prevented her from getting off. In fact, she had dropped off about 10:30, and Mr~Pettigrew-Robinson had had to wake her an hour after to tell her about the footsteps. What with one thing and another she only got a couple of hours' good sleep. She woke up again at two, and remained broad awake till the alarm was given by Lady~Mary. She could swear positively that she heard no shot in the night. Her window was next to Lady~Mary's, on the opposite side from the conservatory. She had always been accustomed from a child to sleep with her window open. In reply to a question from the Coroner, Mrs~Pettigrew-Robinson said she had never felt there was a real, true affection between Lady~Mary Wimsey and deceased. They seemed very off-hand, but that sort of thing was the fashion nowadays. She had never heard of any disagreement.

Miss Lydia Cathcart, who had been hurriedly summoned from town, then gave evidence about the deceased man. She told the Coroner that she was the Captain's aunt and his only surviving relative. She had seen very little of him since he came into possession of his father's money. He had always lived with his own friends in Paris, and they were such as she could not approve of.

»My brother and I never got on very well,« said Miss Cathcart, »and he had my nephew educated abroad till he was eighteen. I fear Denis's notions were always quite French. After my brother's death Denis went to Cambridge, by his father's desire. I was left executrix of the will, and guardian till Denis came of age. I do not know why, after neglecting me all his life, my brother should have chosen to put such a responsibility upon me at his death, but I did not care to refuse.  My house was open to Denis during his holidays from college, but he preferred, as a rule, to go and stay with his rich friends. I cannot now recall any of their names. When Denis was twenty-one he came into \textsterling 10,000 a year. I believe it was in some kind of foreign property. I inherited a certain amount under the will as executrix, but I converted it all, at once, into good, sound, British securities. I cannot say what Denis did with his. It would not surprise me at all to hear that he had been cheating at cards. I have heard that the persons he consorted with in Paris were most undesirable. I never met any of them.  I have never been in France.«

John Hardraw, the gamekeeper, was next called. He and his wife inhabit a small cottage just inside the gate of Riddlesdale Lodge. The grounds, which measure twenty acres or so, are surrounded at this point by a strong paling; the gate is locked at night. Hardraw stated that he had heard a shot fired at about ten minutes to twelve on Wednesday night, close to the cottage, as it seemed to him. Behind the cottage are ten acres of preserved plantation. He supposed that there were poachers about; they occasionally came in after hares. He went out with his gun in that direction, but saw nobody. He returned home at one o'clock by his watch.

\begin{dialogue}

\speak{The Coroner} Did you fire your gun at any time?

\speak{Witness} No.

\speak{The Coroner} You did not go out again?

\speak{Witness} I did not.

\speak{The Coroner} Nor hear any other shots?

\speak{Witness} Only that one; but I fell asleep after I got back, and was wakened up by the chauffeur going out for the doctor. That would be at about a quarter past three.

\speak{The Coroner} Is it not unusual for poachers to shoot so very near the cottage?

\speak{Witness} Yes, rather. If poachers do come, it is usually on the other side of the preserve, towards the moor.
\end{dialogue}

Dr~Thorpe gave evidence of having been called to see deceased. He lived in Stapley, nearly fourteen miles from Riddlesdale. There was no medical man in Riddlesdale. The chauffeur had knocked him up at 3:45 \textsc{a.m.}, and he had dressed quickly and come with him at once. They were at Riddlesdale Lodge at half-past four. Deceased, when he saw him, he judged to have been dead three or four hours. The lungs had been pierced by a bullet, and death had resulted from loss of blood, and suffocation. Death would not have resulted immediately—deceased might have lingered some time. He had made a post-mortem investigation, and found that the bullet had been deflected from a rib. There was nothing to show whether the wound had been self-inflicted or fired from another hand, at close quarters. There were no other marks of violence.

Inspector~Craikes from Stapley had been brought back in the car with Dr~Thorpe. He had seen the body. It was then lying on its back, between the door of the conservatory and the covered well just outside.  As soon as it became light, Inspector~Craikes had examined the house and grounds. He had found bloody marks all along the path leading to the conservatory, and signs as though a body had been dragged along.  This path ran into the main path leading from the gate to the front door. \direct{Plan produced.} Where the two paths joined, a shrubbery began, and ran down on both sides of the path to the gate and the gamekeeper's cottage. The blood-tracks had led to a little clearing in the middle of the shrubbery, about half-way between the house and the gate. Here the inspector found a great pool of blood, a handkerchief soaked in blood, and a revolver. The handkerchief bore the initials D.C., and the revolver was a small weapon of American pattern, and bore no mark.  The conservatory door was open when the Inspector~arrived, and the key was inside.

Deceased, when he saw him, was in dinner-jacket and pumps, without hat or overcoat. He was wet through, and his clothes, besides being much bloodstained, were very muddy and greatly disordered through the dragging of the body. The pocket contained a cigar-case and a small, flat pocket-knife. Deceased's bedroom had been searched for papers, etc., but so far nothing had been found to shed very much light on his circumstances.

The Duke of Denver was then recalled.
\begin{dialogue}

\speak{The Coroner} I should like to ask your grace whether you ever saw deceased in possession of a revolver?

\speak{Duke of D.} Not since the war.

\speak{The Coroner} You do not know if he carried one about with him?

\speak{Duke of D.} I have no idea.

\speak{The Coroner} You can make no guess, I suppose, to whom this revolver belongs?

\speak{Duke of D.} \direct{in great surprise} That's my revolver—out of the study table drawer. How did you get hold of that? \direct{Sensation.}

\speak{The Coroner} You are certain?

\speak{Duke of D.} Positive. I saw it there only the other day, when I was hunting out some photos of Mary for Cathcart, and I remember saying then that it was getting rusty lying about. There's the speck of rust.

\speak{The Coroner} Did you keep it loaded?

\speak{Duke of D.} Lord, no! I really don't know why it was there. I fancy I turned it out one day with some old Army stuff, and found it among my shooting things when I was up at Riddlesdale in August. I think the cartridges were with it.

\speak{The Coroner} Was the drawer locked?

\speak{Duke of D.} Yes; but the key was in the lock. My wife tells me I'm careless.

\speak{The Coroner} Did anybody else know the revolver was there?

\speak{Duke of D.} Fleming did, I think. I don't know of anybody else.
\end{dialogue}

Detective-Inspector~Parker of Scotland Yard, having only arrived on Friday, had been unable as yet to make any very close investigation.  Certain indications led him to think that some person or persons had been on the scene of the tragedy in addition to those who had taken part in the discovery. He preferred to say nothing more at present.

The Coroner then reconstructed the evidence in chronological order.  At, or a little after, ten o'clock there had been a quarrel between deceased and the Duke of Denver, after which deceased had left the house never to be seen alive again. They had the evidence of Mr~Pettigrew-Robinson that the Duke had gone downstairs at 11:30, and that of Colonel Marchbanks that he had been heard immediately afterwards moving about in the study, the room in which the revolver produced in evidence was usually kept. Against this they had the Duke's own sworn statement that he had not left his bedroom till half-past two in the morning. The jury would have to consider what weight was to be attached to those conflicting statements. Then, as to the shots heard in the night; the gamekeeper had said he heard a shot at ten minutes to twelve, but he had supposed it to be fired by poachers. It was, in fact, quite possible that there had been poachers about. On the other hand, Lady~Mary's statement that she had heard the shot at about three \textsc{a.m.} did not fit in very well with the doctor's evidence that when he arrived at Riddlesdale at 4:30 deceased had been already three or four hours dead. They would remember also that, in Dr~Thorpe's opinion, death had not immediately followed the wound. If they believed this evidence, therefore, they would have to put back the moment of death to between eleven \textsc{p.m.} and midnight, and this might very well have been the shot which the gamekeeper heard. In that case they had still to ask themselves about the shot which had awakened Lady~Mary Wimsey.  Of course, if they liked to put that down to poachers, there was no inherent impossibility.

They next came to the body of deceased, which had been discovered by the Duke of Denver at three \textsc{a.m.} lying outside the door of the small conservatory, near the covered well. There seemed little doubt, from the medical evidence, that the shot which killed deceased had been fired in the shrubbery, about seven minutes' distance from the house, and that the body of deceased had been dragged from that place to the house. Deceased had undoubtedly died as the result of being shot in the lungs. The jury would have to decide whether that shot was fired by his own hand or by the hand of another; and, if the latter, whether by accident, in self-defence, or by malice aforethought with intent to murder. As regards suicide, they must consider what they knew of deceased's character and circumstances. Deceased was a young man in the prime of his strength, and apparently of considerable fortune. He had had a meritorious military career, and was liked by his friends. The Duke of Denver had thought sufficiently well of him to consent to his own sister's engagement to deceased. There was evidence to show that the fiancés, though perhaps not demonstrative, were on excellent terms.  The Duke affirmed that on the Wednesday night deceased had announced his intention of breaking off the engagement. Did they believe that deceased, without even communicating with the lady, or writing a word of explanation or farewell, would thereupon rush out and shoot himself?  Again, the jury must consider the accusation which the Duke of Denver said he had brought against deceased. He had accused him of cheating at cards. In the kind of society to which the persons involved in this inquiry belonged, such a misdemeanour as cheating at cards was regarded as far more shameful than such sins as murder and adultery.  Possibly the mere suggestion of such a thing, whether well-founded or not, might well cause a gentleman of sensitive honour to make away with himself. But was deceased honourable? Deceased had been educated in France, and French notions of the honest thing were very different from British ones. The Coroner himself had had business relations with French persons in his capacity as a solicitor, and could assure such of the jury as had never been in France that they ought to allow for these different standards. Unhappily, the alleged letter giving details of the accusation had not been produced to them. Next, they might ask themselves whether it was not more usual for a suicide to shoot himself in the head. They should ask themselves how deceased came by the revolver. And, finally, they must consider, in that case, who had dragged the body towards the house, and why the person had chosen to do so, with great labour to himself and at the risk of extinguishing any lingering remnant of the vital spark,\footnote{Verbatim.} instead of arousing the household and fetching help.

If they excluded suicide, there remained accident, manslaughter, or murder. As to the first, if they thought it likely that deceased or any other person had taken out the Duke of Denver's revolver that night for any purpose, and that, in looking at, cleaning, shooting with, or otherwise handling the weapon, it had gone off and killed deceased accidentally, then they would return a verdict of death by misadventure accordingly. In that case, how did they explain the conduct of the person, whoever it was, who had dragged the body to the door?

The Coroner then passed on to speak of the law concerning manslaughter.  He reminded them that no mere words, however insulting or threatening, can be an efficient excuse for killing anybody, and that the conflict must be sudden and unpremeditated. Did they think, for example, that the Duke had gone out, wishing to induce his guest to return and sleep in the house, and that deceased had retorted upon him with blows or menaces of assault? If so, and the Duke, having a weapon in his hand, had shot deceased in self-defence, that was only manslaughter. But, in that case, they must ask themselves how the Duke came to go out to deceased with a lethal weapon in his hand? And this suggestion was in direct conflict with the Duke's own evidence.

Lastly, they must consider whether there was sufficient evidence of malice to justify a verdict of murder. They must consider whether any person had a motive, means, and opportunity for killing deceased; and whether they could reasonably account for that person's conduct on any other hypothesis. And, if they thought there \textit{was} such a person, and that his conduct was in any way suspicious or secretive, or that he had wilfully suppressed evidence which might have had a bearing on the case, or (here the Coroner spoke with great emphasis, staring over the Duke's head) fabricated other evidence with intent to mislead—then all these circumstances might be sufficient to amount to a violent presumption of guilt against some party, in which case they were in duty bound to bring in a verdict of wilful murder against that party.  And, in considering this aspect of the question, the Coroner added, they would have to decide in their own minds whether the person who had dragged deceased towards the conservatory door had done so with the object of obtaining assistance or of thrusting the body down the garden well, which, as they had heard from Inspector~Craikes, was situate close by the spot where the body had been found. If the jury were satisfied that deceased had been murdered, but were not prepared to accuse any particular person on the evidence, they might bring in a verdict of murder against an unknown person, or persons; but, if they felt justified in laying the killing at any person's door, then they must allow no respect of persons to prevent them from doing their duty.

Guided by these extremely plain hints, the jury, without very long consultation, returned a verdict of wilful murder against Gerald, Duke of Denver.
