%!TeX root=../cloudstop.tex

\chapter{Manon}
\epigraph{»That one word, my dear Watson, should have told me the whole story, had I been the ideal reasoner which you are so fond of depicting.«}{\textit{Memoirs Of Sherlock Holmes}}

\lettrine[lines=4,ante=`]{T}{hank} God,' said Parker. »Well, that settles it.«

\zz
»It does—and yet again, it doesn't,« retorted Lord Peter. He leaned back against the fat silk cushion in the sofa corner meditatively.

»Of course, it's disagreeable having to give this woman away,« said Parker sensibly and pleasantly, »but these things have to be done.«

»I know. It's all simply awfully nice and all that. And Jerry, who's got the poor woman into this mess, has to be considered first. I know. And if we don't restrain Grimethorpe quite successfully, and he cuts her throat for her, it'll be simply rippin' for Jerry to think of all his life\textellipsis . Jerry! I say, you know, what frightful idiots we were not to see the truth right off! I mean—of course, my sister-in-law is an awfully good woman, and all that, but Mrs Grimethorpe—whew! I told you about the time she mistook me for Jerry. One crowded, split second of glorious all-overishness. I ought to have known then. Our voices are alike, of course, and she couldn't see in that dark kitchen. I don't believe there's an ounce of any feeling left in the woman except sheer terror—but, ye gods! what eyes and skin! Well, never mind. Some undeserving fellows have all the luck. Have you got any really good stories? No? Well, I'll tell you some—enlarge your mind and all that. Do you know the rhyme about the young man at the War Office?«

Mr Parker endured five stories with commendable patience, and then suddenly broke down.

»Hurray!« said Wimsey. »Splendid man! I love to see you melt into a refined snigger from time to time. I'll spare you the really outrageous one about the young housewife and the traveller in bicycle-pumps. You know, Charles, I really should like to know who did Cathcart in. Legally, it's enough to prove Jerry innocent, but, Mrs Grimethorpe or no Mrs Grimethorpe, it doesn't do us credit in a professional capacity. »The father weakens, but the governor is firm«; that is, as a brother I am satisfied—I may say light-hearted—but as a sleuth I am cast down, humiliated, thrown back upon myself, a lodge in a garden of cucumbers. Besides, of all defences an alibi is the most awkward to establish, unless a number of independent and disinterested witnesses combine to make it thoroughly air-tight. If Jerry sticks to his denial, the most they can be sure of is that either he or Mrs Grimethorpe is being chivalrous.«

»But you've got the letter.«

»Yes. But how are we going to prove that it came that evening? The envelope is destroyed. Fleming remembers nothing about it. Jerry might have received it days earlier. Or it might be a complete fake. Who is to say that I didn't put it in the window myself and pretend to find it. After all, I'm hardly what you would call disinterested.«

»Bunter saw you find it.«

»He didn't, Charles. At that precise moment he was out of the room fetching shaving-water.«

»Oh, was he?«

»Moreover, only Mrs Grimethorpe can swear to what is really the important point—the moment of Jerry's arrival and departure. Unless he was at Grider's Hole before 12:30 at least, it's immaterial whether he was there or not.«

»Well,« said Parker, »can't we keep Mrs Grimethorpe up our sleeve, so to speak\longdash«

»Sounds a bit abandoned,« said Lord Peter, »but we will keep her with pleasure if you like.«

»—and meanwhile,« pursued Mr Parker, unheeding, »do our best to find the actual criminal?«

»Oh, yes,« said Lord Peter, »and that reminds me. I made a discovery at the Lodge—at least, I think so. Did you notice that somebody had been forcing one of the study windows?«

»No, really?«

»Yes; I found distinct marks. Of course, it was a long time after the murder, but there were scratches on the catch all right—the sort of thing a penknife would leave.«

»What fools we were not to make an examination at the time!«

»Come to think of it, why should you have? Anyhow, I asked Fleming about it, and he said he did remember, now he came to think of it, that on the Thursday morning he'd found the window open, and couldn't account for it. And here's another thing. I've had a letter from my friend Tim Watchett. Here it is:«

\begin{quote}
\textsc{My Lord},—About our conversation. I have found a Man who was with the Party in question at the »Pig and Whistle« on the night of the 13\textsuperscript{th} \textit{ult.} and he tells me that the Party borrowed his bicycle, and same was found afterwards in the ditch where Party was picked up with the Handlebars bent and wheels buckled.

\begin{flushright}
\begin{minipage}{.5\textwidth}
Trusting to the Continuance of your esteemed favour.\\
~\\
\textsc{Timothy Watchett.}
\end{minipage}
\end{flushright}
\end{quote}

»What do you think of that?«

»Good enough to go on,« said Parker. »At least, we are no longer hampered with horrible doubts.«

»No. And, though she's my sister, I must say that of all the blithering she-asses Mary is the blitheringest. Taking up with that awful bounder to start with\longdash«

»She was jolly fine about it,« said Mr Parker, getting rather red in the face. »It's just because she's your sister that you can't appreciate what a fine thing she did. How should a big, chivalrous nature like hers see through a man like that? She's so sincere and thorough herself, she judges everyone by the same standard. She wouldn't believe anybody could be so thin and wobbly-minded as Goyles till it was proved to her. And even then she couldn't bring herself to think ill of him till he'd given himself away out of his own mouth. It was wonderful, the way she fought for him. Think what it must have meant to such a splendid, straight-forward woman to\longdash«

»All right, all right,« cried Peter, who had been staring at his friend, transfixed with astonishment. »Don't get worked up. I believe you. Spare me. I'm only a brother. All brothers are fools. All lovers are lunatics—Shakespeare says so. Do you want Mary, old man? You surprise me, but I believe brothers always are surprised. Bless you, dear children!«

»Damn it all, Wimsey,« said Parker, very angry, »you've no right to talk like that. I only said how greatly I admired your sister—everyone must admire such pluck and staunchness. You needn't be insulting. I know she's Lady Mary Wimsey and damnably rich, and I'm only a common police official with nothing a year and a pension to look forward to, but there's no need to sneer about it.«

»I'm not sneering,« retorted Peter indignantly. »I can't imagine why anybody should want to marry my sister, but you're a friend of mine and a damn good sort, and you've my good word for what it's worth. Besides—dash it all, man!—to put it on the lowest grounds, do look what it might have been! A Socialist Conchy of neither bowels nor breeding, or a card-sharping dark horse with a mysterious past! Mother and Jerry must have got to the point when they'd welcome a decent, God-fearing plumber, let alone a policeman. Only thing I'm afraid of is that Mary, havin' such beastly bad taste in blokes, won't know how to appreciate a really decent fellow like you, old son.«

Mr Parker begged his friend's pardon for his unworthy suspicions, and they sat a little time in silence. Parker sipped his port, and saw unimaginable visions warmly glowing in its rosy depths. Wimsey pulled out his pocket-book, and began idly turning over its contents, throwing old letters into the fire, unfolding and refolding memoranda, and reviewing a miscellaneous series of other people's visiting-cards. He came at length to the slip of blotting-paper from the study at Riddlesdale, to whose fragmentary markings he had since given scarcely a thought.

Presently Mr Parker, finishing his port and recalling his mind with an effort, remembered that he had been meaning to tell Peter something before the name of Lady Mary had driven all other thoughts out of his head. He turned to his host, open-mouthed for speech, but his remark never got beyond a preliminary click like that of a clock about to strike, for, even as he turned, Lord Peter brought his fist down on the little table with a bang that made the decanters ring, and cried out in the loud voice of complete and sudden enlightenment:

»\textit{Manon Lescaut}!«

»Eh?« said Mr Parker.

»Boil my brains!« said Lord Peter. »Boil `em and mash `em and serve `em up with butter as a dish of turnips, for it's damn well all they're fit for! Look at me!« (Mr Parker scarcely needed this exhortation.) »Here we've been worryin' over Jerry, an' worryin' over Mary, an' huntin' for Goyleses an' Grimethorpes and God knows who—and all the time I'd got this little bit of paper tucked away in my pocket. The blot upon the paper's rim a blotted paper was to him, and it was nothing more. But \textit{Manon}, \textit{Manon}! Charles, if I'd had the grey matter of a woodlouse that book ought to have told me the whole story. And think what we'd have been saved!«

»I wish you wouldn't be so excited,« said Parker. »I'm sure it's perfectly splendid for you to see your way so clearly, but I never read \textit{Manon Lescaut}, and you haven't shown me the blotting-paper, and I haven't the foggiest idea what you've discovered.«

Lord Peter passed the relic over without comment.

»I observe,« said Parker, »that the paper is rather crumpled and dirty, and smells powerfully of tobacco and Russian leather, and deduce that you have been keeping it in your pocket-book.«

»No!« said Wimsey incredulously. »And when you actually saw me take it out! Holmes, how do you do it?«

»At one corner,« pursued Parker, »I see two blots, one rather larger than the other. I think someone must have shaken a pen there. Is there anything sinister about the blot?«

»I haven't noticed anything.«

»Some way below the blots the Duke has signed his name two or three times—or, rather, his title. The inference is that his letters were not to intimates.«

»The inference is justifiable, I fancy.«

»Colonel Marchbanks has a neat signature.«

»He can hardly mean mischief,« said Peter. »He signs his name like an honest man! Proceed.«

»There's a sprawly message about five something of fine something. Do you see anything occult there?«

»The number five may have a cabalistic meaning, but I admit I don't know what it is. There are five senses, five fingers, five great Chinese precepts, five books of Moses, to say nothing of the mysterious entities hymned in the Dilly Song—»Five are the flamboys under the pole.« I must admit that I have always panted to know what the five flamboys were. But, not knowing, I get no help from it in this case.«

»Well, that's all, except a fragment consisting of »oe« on one line, and »is fou\longdash« below it.«

»What do you make of that?«

»»Is found,« I suppose.«

»Do you?«

»That seems the simplest interpretation. Or possibly »his foul«—there seems to have been a sudden rush of ink to the pen just there. Do you think it is »his foul«? Was the Duke writing about Cathcart's foul play? Is that what you mean?«

»No, I don't make that of it. Besides, I don't think it's Jerry's writing.«

»Whose is it?«

»I don't know, but I can guess.«

»And it leads somewhere?«

»It tells the whole story.«

»Oh, cough it up, Wimsey. Even Dr Watson would lose patience.«

»Tut, tut! Try the line above.«

»Well, there's only »oe.««

»Yes, well?«

»Well, I don't know. Poet, poem, manœuvre, Loeb edition, Citroën—it might be anything.«

»Dunno about that. There aren't lashings of English words with »oe« in them—and it's written so close it almost looks like a diphthong at that.«

»Perhaps it isn't an English word.«

»Exactly; perhaps it isn't.«

»Oh! Oh, I see. French?«

»Ah, you're gettin' warm.«

»Sœur—œuvre—œuf—bœuf\longdash«

»No, no. You were nearer the first time.«

»Sœur—Cœur!«

»Cœur. Hold on a moment. Look at the scratch in front of that.«

»Wait a bit—er—cer\longdash«

»How about \textit{percer}?«

»I believe you're right. »Percer le cœur.««

»Yes. Or »perceras le cœur.««

»That's better. It seems to need another letter or two.«

»And now your »is found« line.«

»\textit{Fou!}«

»Who?«

»I didn't say »who«; I said »\textit{fou}«.«

»I know you did. I said who?«

»Who?«

»Who's \textit{fou}?«

»Oh, is. By Jove, »suis«! »Je suis fou.««

»\textit{À la bonne heure!} And I suggest that the next words are »\textit{de douleur}«, or something like it.«

»They might be.«

»Cautious beast! I say they are.«

»Well, and suppose they are?«

»It tells us everything.«

»Nothing!«

»Everything, I say. Think. This was written on the day Cathcart died. Now who in the house would be likely to write these words, »\textit{perceras le cœur \textellipsis  je suis fou de douleur}«? Take everybody. I know it isn't Jerry's fist, and he wouldn't use those expressions. Colonel or Mrs Marchbanks? Not Pygmalion likely! Freddy? Couldn't write passionate letters in French to save his life.«

»No, of course not. It would have to be either Cathcart or—Lady Mary.«

»Rot! It couldn't be Mary.«

»Why not?«

»Not unless she changed her sex, you know.«

»Of course not. It would have to be »je suis folle.« Then Cathcart\longdash«

»Of course. He lived in France all his life. Consider his bank-book. Consider\longdash«

»Lord! Wimsey, we've been blind.«

»Yes.«

»And listen! I was going to tell you. The Sûreté write me that they've traced one of Cathcart's bank-notes.«

»Where to?«

»To a Mr François who owns a lot of house property near the Etoile.«

»And lets it out in \textit{appartements}!«

»No doubt.«

»When's the next train? Bunter!«

»My lord!«

Mr Bunter hurried to the door at the call.

»The next boat-train for Paris?«

»Eight-twenty, my lord, from Waterloo.«

»We're going by it. How long?«

»Twenty minutes, my lord.«

»Pack my toothbrush and call a taxi.«

»Certainly, my lord.«

»But, Wimsey, what light does it throw on Cathcart's murder? Did this woman\longdash«

»I've no time,« said Wimsey hurriedly. »But I'll be back in a day or two. Meanwhile\longdash«

He hunted hastily in the bookshelf.

»Read this.«

He flung the book at his friend and plunged into his bedroom.

At eleven o'clock, as a gap of dirty water disfigured with oil and bits of paper widened between the \textit{Normannia} and the quay; while hardened passengers fortified their sea-stomachs with cold ham and pickles, and the more nervous studied the Boddy jackets in their cabins; while the harbour lights winked and swam right and left, and Lord Peter scraped acquaintance with a second-rate cinema actor in the bar, Charles Parker sat, with a puzzled frown, before the fire at 110 Piccadilly, making his first acquaintance with the delicate masterpiece of the Abbé Prévost.

