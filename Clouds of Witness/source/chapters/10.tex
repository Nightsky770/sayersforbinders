%!TeX root=../cloudstop.tex


\chapter{Nothing Abides at the Noon}


\epigraph{»Alas!« said Hiya, »the sentiments which this person expressed with irreproachable honourableness, when the sun was high in the heavens and the probability of secretly leaving an undoubtedly well-appointed home was engagingly remote, seem to have an entirely different significance when recalled by night in a damp orchard, and on the eve of their fulfilment.«}{\textit{The Wallet Of Kai-Lung}}

\epigraph{And his short minute, after noon, is night.}{Donne}


\lettrine[lines=4]{M}{r} Goyles was interviewed the next day at the police-station. Mr  Murbles was present, and Mary insisted on coming. The young man began by blustering a little, but the solicitor's dry manner made its impression.

»Lord Peter Wimsey identifies you,« said Mr Murbles, »as the man who made a murderous attack upon him last night. With remarkable generosity, he has forborne to press the charge. Now we know further that you were present at Riddlesdale Lodge on the night when Captain Cathcart was shot. You will no doubt be called as a witness in the case. But you would greatly assist justice by making a statement to us now. This is a purely friendly and private interview, Mr Goyles.  As you see, no representative of the police is present. We simply ask for your help. I ought, however, to warn you that, whereas it is, of course, fully competent for you to refuse to answer any of our questions, a refusal might lay you open to the gravest imputations.«

»In fact,« said Goyles, »it's a threat. If I don't tell you, you'll have me arrested on suspicion of murder.«

»Dear me, no, Mr Goyles,« returned the solicitor. »We should merely place what information we hold in the hands of the police, who would then act as they thought fit. God bless my soul, no—anything like a threat would be highly irregular. In the matter of the assault upon Lord Peter, his lordship will, of course, use his own discretion.«

»Well,« said Goyles sullenly, »it's a threat, call it what you like.  However, I don't mind speaking—especially as you'll be jolly well disappointed. I suppose you gave me away, Mary.«

Mary flushed indignantly.

»My sister has been extraordinarily loyal to you, Mr Goyles,« said Lord Peter. »I may tell you, indeed, that she put herself into a position of grave personal inconvenience—not to say danger—on your behalf. You were traced to London in consequence of your having left unequivocal traces in your exceedingly hasty retreat. When my sister accidentally opened a telegram addressed to me at Riddlesdale by my family name she hurried immediately to town, to shield you if she could, at any cost to herself. Fortunately I had already received a duplicate wire at my flat. Even then I was not certain of your identity when I accidentally ran across you at the Soviet Club. Your own energetic efforts, however, to avoid an interview gave me complete certainty, together with an excellent excuse for detaining you. In fact, I'm uncommonly obliged to you for your assistance.«

Mr Goyles looked resentful.

»I don't know how you could think, George\longdash« said Mary.

»Never mind what I think,« said the young man, roughly. »I gather you've told 'em all about it now, anyhow. Well, I'll tell you my story as shortly as I can, and you'll see I know damn all about it. If you don't believe me I can't help it. I came along at about a quarter to three, and parked the 'bus in the lane.«

»Where were you at 11:50?«

»On the road from Northallerton. My meeting didn't finish till 10:45. I can bring a hundred witnesses to prove it.«

Wimsey made a note of the address where the meeting had been held, and nodded to Goyles to proceed.

»I climbed over the wall and walked through the shrubbery.«

»You saw no person, and no body?«

»Nobody, alive or dead.«

»Did you notice any blood or footprints on the path?«

»No. I didn't like to use my torch, for fear of being seen from the house. There was just light enough to see the path. I came to the door of the conservatory just before three. As I came up I stumbled over something. I felt it, and it was like a body. I was alarmed. I thought it might be Mary—ill or fainted or something. I ventured to turn on my light. Then I saw it was Cathcart, dead.«

»You are sure he was dead?«

»Stone dead.«

»One moment,« interposed the solicitor. »You say you saw that it was Cathcart. Had you known Cathcart previously?«

»No, never. I meant that I saw it was a dead man, and learnt afterwards that it was Cathcart.«

»In fact, you do not, now, know of your own knowledge, that it was Cathcart?«

»Yes—at least, I recognized the photographs in the papers afterwards.«

»It is very necessary to be accurate in making a statement, Mr Goyles.  A remark such as you made just now might give a most unfortunate impression to the police or to a jury.«

So saying, Mr Murbles blew his nose, and resettled his pince-nez.

»What next?« inquired Peter.

»I fancied I heard somebody coming up the path. I did not think it wise to be found there with the corpse, so I cleared out.«

»Oh,« said Peter, with an indescribable expression, »that was a very simple solution. You left the girl you were going to marry to make for herself the unpleasant discovery that there was a dead man in the garden and that her gallant wooer had made tracks. What did you expect \textit{her} to think?«

»Well, I thought she'd keep quiet for her own sake. As a matter of fact, I didn't think very clearly about anything. I knew I'd broken in where I had no business, and that if I was found with a murdered man it might look jolly queer for me.«

»In fact,« said Mr Murbles, »you lost your head, young man, and ran away in a very foolish and cowardly manner.«

»You needn't put it that way,« retorted Mr Goyles. »I was in a very awkward and stupid situation to start with.«

»Yes,« said Lord Peter ironically, »and 3 \textsc{a.m.} is a nasty, chilly time of day. Next time you arrange an elopement, make it for six o'clock in the evening, or twelve o'clock at night. You seem better at framing conspiracies than carrying them out. A little thing upsets your nerves, Mr Goyles. I don't really think, you know, that a person of your temperament should carry fire-arms. What in the world, you blitherin' young ass, made you loose off that pop-gun at me last night? You \textit{would} have been in a damned awkward situation then, if you'd accidentally hit me in the head or the heart or anywhere that mattered. If you're so frightened of a dead body, why go about shootin' at people? Why, why, why? That's what beats me. If you're tellin' the truth now, you never stood in the slightest danger. Lord! and to think of the time and trouble we've had to waste catchin' you—you ass! And poor old Mary, workin' away and half killin' herself, because she thought at least you wouldn't have run away unless there was somethin' to run from!«

»You must make allowance for a nervous temperament,« said Mary in a hard voice.

»If you knew what it felt like to be shadowed and followed and badgered\longdash« began Mr Goyles.

»But I thought you Soviet Club people enjoyed being suspected of things,« said Lord Peter. »Why, it ought to be the proudest moment of your life when you're really looked on as a dangerous fellow.«

»It's the sneering of men like you,« said Goyles passionately, »that does more to breed hatred between class and class\longdash«

»Never mind about that,« interposed Mr Murbles. »The law's the law for everybody, and you have managed to put yourself in a very awkward position, young man.« He touched a bell on the table, and Parker entered with a constable. »We shall be obliged to you,« said Mr Murbles, »if you will kindly have this young man kept under observation. We make no charge against him so long as he behaves himself, but he must not attempt to abscond before the Riddlesdale case comes up for trial.«

»Certainly not, sir,« said Mr Parker.

»One moment,« said Mary. »Mr Goyles, here is the ring you gave me.  Good-bye. When next you make a public speech calling for decisive action I will come and applaud it. You speak so well about that sort of thing. But otherwise, I think we had better not meet again.«

»Of course,« said the young man bitterly, »your people have forced me into this position, and you turn round and sneer at me too.«

»I didn't mind thinking you were a murderer,« said Lady Mary spitefully, \enquote{but I \textit{do} mind your being such an ass.}

Before Mr Goyles could reply, Mr Parker, bewildered but not wholly displeased, manœuvred his charge out of the room. Mary walked over to the window, and stood biting her lips.

Presently Lord Peter came across to her. »I say, Polly, old Murbles has asked us to lunch. Would you like to come? Sir Impey Biggs will be there.«

»I don't want to meet him today. It's very kind of Mr Murbles\longdash«

»Oh, come along, old thing. Biggs is some celebrity, you know, and perfectly toppin' to look at, in a marbly kind of way. He'll tell you all about his canaries\longdash«

Mary giggled through her obstinate tears.

»It's perfectly sweet of you, Peter, to try and amuse the baby. But I can't. I'd make a fool of myself. I've been made enough of a fool of for one day.«

»Bosh,« said Peter. »Of course, Goyles didn't show up very well this morning, but, then, he was in an awfully difficult position. \textit{Do} come.«

»I hope Lady Mary consents to adorn my bachelor establishment,« said the solicitor, coming up. »I shall esteem it a very great honour. I really do not think I have entertained a lady in my chambers for twenty years—dear me, twenty years indeed it must be.«

»In that case,« said Lady Mary, »I simply \textit{can't} refuse.«

Mr Murbles inhabited a delightful old set of rooms in Staple Inn, with windows looking out upon the formal garden, with its odd little flower-beds and tinkling fountain. The chambers kept up to a miracle the old-fashioned law atmosphere which hung about his own prim person.  His dining-room was furnished in mahogany, with a Turkey carpet and crimson curtains. On his sideboard stood some pieces of handsome Sheffield plate and a number of decanters with engraved silver labels round their necks. There was a bookcase full of large volumes bound in law calf, and an oil-painting of a harsh-featured judge over the mantelpiece. Lady Mary felt a sudden gratitude for this discreet and solid Victorianism.

»I fear we may have to wait a few moments for Sir Impey,« said Mr  Murbles, consulting his watch. »He is engaged in Quangle \& Hamper v.  \textit{Truth}, but they expect to be through this morning—in fact, Sir Impey fancied that midday would see the end of it. Brilliant man, Sir Impey.  He is defending \textit{Truth}.«

»Astonishin' position for a lawyer, what?« said Peter.

»The newspaper,« said Mr Murbles, acknowledging the pleasantry with a slight unbending of the lips, »against these people who profess to cure fifty-nine different diseases with the same pill. Quangle \& Hamper produced some of their patients in court to testify to the benefits they'd enjoyed from the cure. To hear Sir Impey handling them was an intellectual treat. His kindly manner goes a long way with old ladies.  When he suggested that one of them should show her leg to the Bench the sensation in court was really phenomenal.«

»And did she show it?« inquired Lord Peter.

»Panting for the opportunity, my dear Lord Peter, panting for the opportunity.«

»I wonder they had the nerve to call her.«

»Nerve?« said Mr Murbles. »The nerve of men like Quangle \& Hamper has not its fellow in the universe, to adopt the expression of the great Shakespeare. But Sir Impey is not the man to take liberties with. We are really extremely fortunate to have secured his help.—Ah, I think I hear him!«

A hurried footstep on the stair indeed announced learned counsel, who burst in, still in wig and gown, and full of apology.

»Extremely sorry, Murbles,« said Sir Impey. »We became excessively tedious at the end, I regret to say. I really did my best, but dear old Dowson is getting as deaf as a post, you know, and terribly fumbling in his movements.—And how are you, Wimsey? You look as if you'd been in the wars. Can we bring an action for assault against anybody?«

»Much better than that,« put in Mr Murbles; »attempted murder, if you please.«

»Excellent, excellent,« said Sir Impey.

»Ah, but we've decided not to prosecute,« said Mr Murbles, shaking his head.

»Really! Oh, my dear Wimsey, this will never do. Lawyers have to live, you know. Your sister? I hadn't the pleasure of meeting you at Riddlesdale, Lady Mary. I trust you are fully recovered.«

»Entirely, thank you,« said Mary with emphasis.

»Mr Parker—of course your name is very familiar. Wimsey, here, can't do a thing without you, I know. Murbles, are these gentlemen full of valuable information? I am immensely interested in this case.«

»Not just this moment, though,« put in the solicitor.

»Indeed, no. Nothing but that excellent saddle of mutton has the slightest attraction for me just now. Forgive my greed.«

»Well, well,« said Mr Murbles, beaming mildly, »let's make a start.  I fear, my dear young people, I am old-fashioned enough not to have adopted the modern practice of cocktail-drinking.«

»Quite right too,« said Wimsey emphatically. »Ruins the palate and spoils the digestion. Not an English custom—rank sacrilege in this old Inn. Came from America—result, prohibition. That's what happens to people who don't understand how to drink. God bless me, sir, why, you're giving us the famous claret. It's a sin so much as to mention a cocktail in its presence.«

»Yes,« said Mr Murbles, »yes, that's the Lafite '75. It's very seldom, very seldom, I bring it out for anybody under fifty years of age—but you, Lord Peter, have a discrimination which would do honour to one of twice your years.«

»Thanks very much, sir; that's a testimonial I deeply appreciate. May I circulate the bottle, sir?«

»Do, do—we will wait on ourselves, Simpson, thank you. After lunch,« continued Mr Murbles, »I will ask you to try something really curious.  An odd old client of mine died the other day, and left me a dozen of '47 port.«

»Gad!« said Peter. »'47! It'll hardly be drinkable, will it, sir?«

»I very greatly fear,« replied Mr Murbles, »that it will not. A great pity. But I feel that some kind of homage should be paid to so notable an antiquity.«

»It would be something to say that one had tasted it,« said Peter. »Like goin' to see the divine Sarah, you know. Voice gone, bloom gone, savor gone—but still a classic.«

»Ah,« said Mr Murbles. »I remember her in her great days. We old fellows have the compensation of some very wonderful memories.«

»Quite right, sir,« said Peter, »and you'll pile up plenty more yet.  But what was this old gentleman doing to let a vintage like that get past its prime?«

»Mr Featherstone was a very singular man,« said Mr Murbles. »And yet—I don't know. He may have been profoundly wise. He had the reputation for extreme avarice. Never bought a new suit, never took a holiday, never married, lived all his life in the same dark, narrow chambers he occupied as a briefless barrister. Yet he inherited a huge income from his father, all of which he left to accumulate. The port was laid down by the old man, who died in 1860, when my client was thirty-four. He—the son, I mean—was ninety-six when he deceased. He said no pleasure ever came up to the anticipation, and so he lived like a hermit—doing nothing, but planning all the things he might have done. He wrote an elaborate diary, containing, day by day, the record of this visionary existence which he had never dared put to the test of actuality. The diary described minutely a blissful wedded life with the woman of his dreams. Every Christmas and Easter Day a bottle of the '47 was solemnly set upon his table and solemnly removed, unopened, at the close of his frugal meal. An earnest Christian, he anticipated great happiness after death, but, as you see, he put the pleasure off as long as possible. He died with the words, »He is faithful that promised«—feeling to the end the need of assurance. A very singular man, very singular indeed—far removed from the adventurous spirit of the present generation.«

»How curious and pathetic,« said Mary.

»Perhaps he had at some time set his heart on something unattainable,« said Parker.

»Well, I don't know,« said Mr Murbles. »People used to say that the dream-lady had not always been a dream, but that he never could bring himself to propose.«

»Ah,« said Sir Impey briskly, »the more I see and hear in the courts the more I am inclined to feel that Mr Featherstone chose the better part.«

»And are determined to follow his example—in that respect at any rate?  Eh, Sir Impey!« replied Mr Murbles, with a mild chuckle.

Mr Parker glanced towards the window. It was beginning to rain.

Truly enough the '47 port was a dead thing; the merest ghost of its old flame and flavour hung about it. Lord Peter held his glass poised a moment.

»It is like the taste of a passion that has passed its noon and turned to weariness,« he said, with sudden gravity. »The only thing to do is to recognize bravely that it is dead, and put it away.« With a determined movement, he flung the remainder of the wine into the fire.  The mocking smile came back to his face: 
»\begin{verse}
What I like about Clive\\
Is that he is no longer alive\longdash \\
There is a great deal to be said\\
For being dead.\\
\end{verse}

What classic pith and brevity in those four lines! However, in the matter of this case, we've a good deal to tell you, sir.«

With the assistance of Parker, he laid before the two men of law the whole train of the investigation up to date, Lady Mary coming loyally up to the scratch with her version of the night's proceedings.

»In fact, you see,« said Peter, »this Mr Goyles has lost a lot by \textit{not} being a murderer. We feel he would have cut a fine, sinister figure as a midnight assassin. But things bein' as they are, you see, we must make what we can of him as a witness, what?«

»Well, Lord Peter,« said Mr Murbles slowly, »I congratulate you and Mr Parker on a great deal of industry and ingenuity in working the matter out.«

»I think we may say we have made some progress,« said Parker.

»If only negatively,« added Peter.

»Exactly,« said Sir Impey turning on him with staggering abruptness.  »Very negatively indeed. And, having seriously hampered the case for the defence, what are you going to do next?«

»That's a nice thing to say,« cried Peter indignantly, »when we've cleared up such a lot of points for you!«

»I daresay,« said the barrister, »but they're the sorts of points which are much better left muffled up.«

»Damn it all, we want to get at the truth!«

»Do you?« said Sir Impey drily. »I don't. I don't care twopence about the truth. I want a case. It doesn't matter to me who killed Cathcart, provided I can prove it wasn't Denver. It's really enough if I can throw reasonable doubt on its being Denver. Here's a client comes to me with a story of a quarrel, a suspicious revolver, a refusal to produce evidence of his statements, and a totally inadequate and idiotic alibi. I arrange to obfuscate the jury with mysterious footprints, a discrepancy as to time, a young woman with a secret, and a general vague suggestion of something between a burglary and a \textit{crime passionel}. And here you come explaining the footprints, exculpating the unknown man, abolishing the discrepancies, clearing up the motives of the young woman, and most carefully throwing back suspicion to where it rested in the first place. What \textit{do} you expect?«

»I've always said,« growled Peter, »that the professional advocate was the most immoral fellow on the face of the earth, and now I know for certain.«

»Well, well,« said Mr Murbles, »all this just means that we mustn't rest upon our oars. You must go on, my dear boy, and get more evidence of a positive kind. If this Mr Goyles did not kill Cathcart we must be able to find the person who did.«

»Anyhow,« said Biggs, »there's one thing to be thankful for—and that is, that you were still too unwell to go before the Grand Jury last Thursday, Lady Mary«—Lady Mary blushed—»and the prosecution will be building their case on a shot fired at three \textsc{a.m.} Don't answer any questions if you can help it, and we'll spring it on 'em.«

»But will they believe anything she says at the trial after that?« asked Peter dubiously.

»All the better if they don't. She'll be their witness. You'll get a nasty heckling, Lady Mary, but you mustn't mind that. It's all in the game. Just stick to your story and we'll deliver the goods. See!« Sir Impey wagged a menacing finger.

»I see,« said Mary. »And I'll be heckled like anything. Just go on stubbornly saying, »I am telling the truth now.« That's the idea, isn't it?«

»Exactly so,« said Biggs. »By the way, Denver still refuses to explain his movements, I suppose?«

»Cat-e-gori-cally,« replied the solicitor. »The Wimseys are a very determined family,« he added, »and I fear that, for the present, it is useless to pursue that line of investigation. If we could discover the truth in some other way, and confront the Duke with it, he might then be persuaded to add his confirmation.«

»Well, now,« said Parker, »we have, as it seems to me, still three lines to go upon. First, we must try to establish the Duke's alibi from external sources. Secondly, we can examine the evidence afresh with a view to finding the real murderer. And thirdly, the Paris police may give us some light upon Cathcart's past history.«

»And I fancy I know where to go next for information on the second point,« said Wimsey suddenly. »Grider's Hole.«

»Whew-w!« Parker whistled. »I was forgetting that. That's where that bloodthirsty farmer fellow lives, isn't it, who set the dogs on you?«

»With the remarkable wife. Yes. See here, how does this strike you?  This fellow is ferociously jealous of his wife, and inclined to suspect every man who comes near her. When I went up there that day, and mentioned that a friend of mine might have been hanging about there the previous week, he got frightfully excited and threatened to have the fellow's blood. Seemed to know who I was referrin' to. Now, of course, with my mind full of № 10—Goyles, you know—I never thought but what he was the man. But supposin' it was Cathcart? You see, we know now, Goyles hadn't even been in the neighbourhood til the Wednesday, so you wouldn't expect what's-his-name—Grimethorpe—to know about him, but Cathcart might have wandered over to Grider's Hole any day and been seen. And look here! Here's another thing that fits in. When I went up there Mrs Grimethorpe evidently mistook me for somebody she knew, and hurried down to warn me off. Well, of course, I've been thinkin' all the time she must have seen my old cap and Burberry from the window and mistaken me for Goyles, but, now I come to think of it, I told the kid who came to the door that I was from Riddlesdale Lodge. If the child told her mother, she must have thought it was Cathcart.«

»No, no, Wimsey, that won't do,« put in Parker; »she must have known Cathcart was dead by that time.«

»Oh, damn it! Yes, I suppose she must. Unless that surly old devil kept the news from her. By Jove! that's just what he would do if he'd killed Cathcart himself. He'd never say a word to her—and I don't suppose he would let her look at a paper, even if they take one in. It's a primitive sort of place.«

»But didn't you say Grimethorpe had an alibi?«

»Yes, but we didn't really test it.«

»And how d'you suppose he knew Cathcart was going to be in the thicket that night?«

Peter considered.

»Perhaps he sent for him,« suggested Mary.

»That's right, that's right,« cried Peter eagerly. »You remember we thought Cathcart must somehow or other have heard from Goyles, making an appointment—but suppose the message was from Grimethorpe, threatening to split on Cathcart to Jerry.«

»You are suggesting, Lord Peter,« said Mr Murbles, in a tone calculated to chill Peter's blithe impetuosity, »that, at the very time Mr Cathcart was betrothed to your sister, he was carrying on a disgraceful intrigue with a married woman very much his social inferior.«

»I beg your pardon, Polly,« said Wimsey.

»It's all right,« said Mary, »I—as a matter of fact, it wouldn't surprise me frightfully. Denis was always—I mean, he had rather Continental ideas about marriage and that sort of thing. I don't think he'd have thought that mattered very much. He'd probably have said there was a time and place for everything.«

»One of those watertight compartment minds,« said Wimsey thoughtfully.  Mr Parker, despite his long acquaintance with the seamy side of things in London, had his brows set in a gloomy frown of as fierce a provincial disapproval as ever came from Barrow-in-Furness.

»If you can upset this Grimethorpe's alibi,« said Sir Impey, fitting his right-hand fingertips neatly between the fingers of his left hand, »we might make some sort of a case of it. What do you think, Murbles?«

»After all,« said the solicitor, »Grimethorpe and the servant both admit that he, Grimethorpe, was not at Grider's Hole on Wednesday night. If he can't prove he was at Stapley he may have been at Riddlesdale.«

»By Jove!« cried Wimsey; »driven off alone, stopped somewhere, left the gee, sneaked back, met Cathcart, done him in, and toddled home next day with a tale about machinery.«

»Or he may even have been to Stapley,« put in Parker; »left early or gone late, and put in the murder on the way. We shall have to check the precise times very carefully.«

»Hurray!« cried Wimsey. »I think I'll be gettin' back to Riddlesdale.«

»I'd better stay here,« said Parker. »There may be something from Paris.«

»Right you are. Let me know the minute anything comes through. I say, old thing!«

»Yes?«

»Does it occur to you that what's the matter with this case is that there are too many clues? Dozens of people with secrets and elopements bargin' about all over the place\longdash«

»I hate you, Peter,« said Lady Mary.
