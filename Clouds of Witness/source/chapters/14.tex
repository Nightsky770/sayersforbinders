%!TeX root=../cloudstop.tex


\chapter{The Edge of the Axe Towards Him}

\epigraph{
\textit{Scene 1. Westminster Hall. Enter as to the Parliament, Bolingbroke, Aumerle, Northumberland, Percy, Fitzwater, Surrey, the Bishop of Carlisle, the Abbot of Westminster, and another Lord, Herald, Officers, and Bagot.}\\
\textsc{Bolingbroke}: Call forth Bagot.\\
\indent Now, Bagot, freely speak thy mind;\\
\indent What thou dost know of noble Gloucester's death;\\
\indent Who wrought it with the king, and who performed\\
The bloody office of his timeless end.\\
\textsc{Bagot}: Then set before my face the Lord~Aumerle.}{\textit{King Richard Ⅱ}}


\lettrine[lines=4]{T}{he} historic trial of the Duke of Denver for murder opened as soon as Parliament reassembled after the Christmas vacation. The papers had leaderettes on »Trial by his Peers,« by a Woman Barrister, and »The Privilege of Peers: should it be abolished?« by a Student of History.  The \textit{Evening Banner} got into trouble for contempt by publishing an article entitled »The Silken Rope« (by an Antiquarian), which was deemed to be prejudicial, and the \textit{Daily Trumpet}—the Labour organ—inquired sarcastically why, when a peer was tried, the fun of seeing the show should be reserved to the few influential persons who could wangle tickets for the Royal Gallery.

Mr~Murbles and Detective Inspector~Parker, in close consultation, went about with preoccupied faces, while Sir Impey Biggs retired into a complete eclipse for three days, revolved about by Mr~Glibbery, \textsc{k.c.}, Mr~Brownrigg-Fortescue, \textsc{k.c.}, and a number of lesser satellites. The schemes of the Defence were kept dark indeed—the more so that they found themselves on the eve of the struggle deprived of their principal witness, and wholly ignorant whether or not he would be forthcoming with his testimony.

Lord~Peter had returned from Paris at the end of four days, and had burst in like a cyclone at Great Ormond Street. »I've got it,« he said, »but it's touch and go. Listen!«

For an hour Parker had listened, feverishly taking notes.

»You can work on that,« said Wimsey. »Tell Murbles. I'm off.«

His next appearance was at the American Embassy. The Ambassador, however, was not there, having received a royal mandate to dine. Wimsey damned the dinner, abandoned the polite, horn-rimmed secretaries, and leapt back into his taxi with a demand to be driven to Buckingham Palace. Here a great deal of insistence with scandalized officials produced first a higher official, then a very high official, and, finally, the American Ambassador and a Royal Personage while the meat was yet in their mouths.

»Oh, yes,« said the Ambassador, »of course it can be done\longdash«

»Surely, surely,« said the Personage genially, »we mustn't have any delay. Might cause an international misunderstanding, and a lot of paragraphs about Ellis Island. Terrible nuisance to have to adjourn the trial—dreadful fuss, isn't it? Our secretaries are everlastingly bringing things along to our place to sign about extra policemen and seating accommodation. Good luck to you, Wimsey! Come and have something while they get your papers through. When does your boat go?«

»Tomorrow morning, sir. I'm catching the Liverpool train in an hour—if I can.«

»You surely will,« said the Ambassador cordially, signing a note. »And they say the English can't hustle.«

So, with his papers all in order, his lordship set sail from Liverpool the next morning, leaving his legal representatives to draw up alternative schemes of defence. 

\noindent\hfil\rule{0.5\textwidth}{.4pt}\hfil 

»Then the peers, two by two, in their order, beginning with the youngest baron.«

Garter King-of-Arms, very hot and bothered, fussed unhappily around the three hundred or so British peers who were sheepishly struggling into their robes, while the heralds did their best to line up the assembly and keep them from wandering away when once arranged.

»Of all the farces!« grumbled Lord~Attenbury irritably. He was a very short, stout gentleman of a choleric countenance, and was annoyed to find himself next to the Earl of Strathgillan and Begg, an extremely tall, lean nobleman, with pronounced views on Prohibition and the Legitimation question.

»I say, Attenbury,« said a kindly, brick-red peer, with five rows of ermine on his shoulder, »is it true that Wimsey hasn't come back?  My daughter tells me she heard he'd gone to collect evidence in the States. Why the States?«

»Dunno,« said Attenbury; »but Wimsey's a dashed clever fellow. When he found those emeralds of mine, you know, I said\longdash«

»Your grace, your grace,« cried Rouge Dragon desperately, diving in, »your grace is out of line again.«

»Eh, what?« said the brick-faced peer. »Oh, damme! Must obey orders, I suppose, what?« And was towed away from the mere earls and pushed into position next to the Duke of Wiltshire, who was deaf, and a distant connection of Denver's on the distaff side.

The Royal Gallery was packed. In the seats reserved below the Bar for peeresses sat the Dowager Duchess of Denver, beautifully dressed and defiant. She suffered much from the adjacent presence of her daughter-in-law, whose misfortune it was to become disagreeable when she was unhappy—perhaps the heaviest curse that can be laid on man, who is born to sorrow.

Behind the imposing array of Counsel in full-bottomed wigs in the body of the hall were seats reserved for witnesses, and here Mr~Bunter was accommodated—to be called if the defence should find it necessary to establish the alibi—the majority of the witnesses being pent up in the King's Robing-Room, gnawing their fingers and glaring at one another.  On either side, above the Bar, were the benches for the peers—each in his own right a judge both of fact and law—while on the high dais the great chair of state stood ready for the Lord~High Steward.

The reporters at their little table were beginning to fidget and look at their watches. Muffled by the walls and the buzz of talk, Big Ben dropped eleven slow notes into the suspense. A door opened.  The reporters started to their feet; counsel rose; everybody rose; the Dowager Duchess whispered irrepressibly to her neighbour that it reminded her of the Voice that breathed o'er Eden; and the procession streamed slowly in, lit by a shaft of wintry sunshine from the tall windows.

The proceedings were opened by a Proclamation of Silence from the Sergeant-at-Arms, after which the Clerk of the Crown in Chancery, kneeling at the foot of the throne, presented the Commission under the Great Seal to the Lord~High Steward,\footnote{The Lord~Chancellor held the appointment on this occasion as usual.} who, finding no use for it, returned it with great solemnity to the Clerk of the Crown. The latter accordingly proceeded to read it at dismal and wearisome length, affording the assembly an opportunity of judging just how bad the acoustics of the chamber were. The Sergeant-at-Arms retorted with great emphasis, »God Save the King,« whereupon Garter King-of-Arms and the Gentleman Usher of the Black Rod, kneeling again, handed the Lord~High Steward his staff of office. (»So picturesque, isn't it?« said the Dowager—»quite High Church, you know.«)

The Certiorari and Return followed in a long, sonorous rigmarole, which, starting with George the Fifth by the Grace of God, called upon all the Justices and Judges of the Old Bailey, enumerated the Lord~Mayor of London, the Recorder, and a quantity of assorted aldermen and justices, skipped back to our Lord~the King, roamed about the City of London, Counties of London and Middlesex, Essex, Kent, and Surrey, mentioned our late Sovereign Lord~King William the Fourth, branched off to the Local Government Act one thousand eight hundred and eighty-eight, lost its way in a list of all treasons, murders, felonies, and misdemeanours by whomsoever and in what manner soever done, committed or perpetrated and by whom or to whom, when, how and after what manner and of all other articles and circumstances concerning the premises and every one of them and any of them in any manner whatsoever, and at last, triumphantly, after reciting the names of the whole Grand Jury, came to the presentation of the indictment with a sudden, brutal brevity.

»The Jurors for our Lord~the King upon their oaths present that the most noble and puissant prince Gerald Christian Wimsey, Viscount St~George, Duke of Denver, a Peer of the United Kingdom of Great Britain and Ireland, on the thirteenth day of October in the year of Our Lord~one thousand nine hundred and twenty— in the Parish of Riddlesdale in the County of Yorkshire did kill and murder Denis Cathcart.«

After which, Proclamation\footnote{For Report of the procedure see House of Lords Journal for the dates in question.} was made by the Sergeant-at-Arms for the Gentleman Usher of the Black Rod to call in Gerald Christian Wimsey, Viscount St~George, Duke of Denver, to appear at the Bar to answer his indictment, who, being come to the Bar, kneeled until the Lord~High Steward acquainted him that he might rise.

The Duke of Denver looked very small and pink and lonely in his blue serge suit, the only head uncovered among all his peers, but he was not without a certain dignity as he was conducted to the »Stool placed within the Bar,« which is deemed appropriate to noble prisoners, and he listened to the Lord~High Steward's rehearsal of the charge with a simple gravity which became him very well.

»Then the said Duke of Denver was arraigned by the Clerk of the Parliaments in the usual manner and asked whether he was Guilty or Not Guilty, to which he pleaded Not Guilty.«

Whereupon Sir Wigmore Wrinching, the Attorney-General, rose to open the case for the Crown.

After the usual preliminaries to the effect that the case was a very painful one and the occasion a very solemn one, Sir Wigmore proceeded to unfold the story from the beginning: the quarrel, the shot at 3 \textsc{a.m.}, the pistol, the finding of the body, the disappearance of the letter, and the rest of the familiar tale. He hinted, moreover, that evidence would be called to show that the quarrel between Denver and Cathcart had motives other than those alleged by the prisoner, and that the latter would turn out to have had »good reason to fear exposure at Cathcart's hands.« At which point the accused was observed to glance uneasily at his solicitor. The exposition took only a short time, and Sir Wigmore proceeded to call witnesses.

The prosecution being unable to call the Duke of Denver, the first important witness was Lady~Mary Wimsey. After telling about her relations with the murdered man, and describing the quarrel, »At three o'clock,« she proceeded, »I got up and went downstairs.«

»In consequence of what did you do so?« inquired Sir Wigmore, looking round the Court with the air of a man about to produce his great effect.

»In consequence of an appointment I had made to meet a friend.«

All the reporters looked up suddenly, like dogs expecting a piece of biscuit, and Sir Wigmore started so violently that he knocked his brief over upon the head of the Clerk to the House of Lords sitting below him.

»Indeed! Now, witness, remember you are on your oath, and be very careful. What was it caused you to wake at three o'clock?«

»I was not asleep. I was waiting for my appointment.«

»And while you were waiting did you hear anything?«

»Nothing at all.«

»Now, Lady~Mary, I have here your deposition sworn before the Coroner.  I will read it to you. Please listen very carefully. You say, »At three o'clock I was wakened by a shot. I thought it might be poachers. It sounded very loud, close to the house. I went down to find out what it was.« Do you remember making that statement?«

»Yes, but it was not true.«

»Not true?«

»No.«

»In the face of that statement, you still say that you heard nothing at three o'clock?«

»I heard nothing at all. I went down because I had an appointment.«

»My lords,« said Sir Wigmore, with a very red face, »I must ask leave to treat this witness as a hostile witness.«

Sir Wigmore's fiercest onslaught, however, produced no effect, except a reiteration of the statement that no shot had been heard at any time. With regard to the finding of the body, Lady~Mary explained that when she said, »Oh, God! Gerald, you've killed him,« she was under the impression that the body was that of the friend who had made the appointment. Here a fierce wrangle ensued as to whether the story of the appointment was relevant. The Lords decided that on the whole it was relevant; and the entire Goyles story came out, together with the intimation that Mr~Goyles was in court and could be produced.  Eventually, with a loud snort, Sir Wigmore Wrinching gave up the witness to Sir Impey Biggs, who, rising suavely and looking extremely handsome, brought back the discussion to a point long previous.

»Forgive the nature of the question,« said Sir Impey, bowing blandly, »but will you tell us whether, in your opinion, the late Captain Cathcart was deeply in love with you?«

»No, I am sure he was not; it was an arrangement for our mutual convenience.«

»From your knowledge of his character, do you suppose he was capable of a very deep affection?«

»I think he might have been, for the right woman. I should say he had a very passionate nature.«

»Thank you. You have told us that you met Captain Cathcart several times when you were staying in Paris last February. Do you remember going with him to a jeweller's—Monsieur Briquet's in the Rue de la Paix?«

»I may have done; I cannot exactly remember.«

»The date to which I should like to draw your attention is the sixth.«

»I could not say.«

»Do you recognize this trinket?«

Here the green-eyed cat was handed to witness.

»No; I have never seen it before.«

»Did Captain Cathcart ever give you one like it?«

»Never.«

»Did you ever possess such a jewel?«

»I am quite positive I never did.«

»My lords, I put in this diamond-and-platinum cat. Thank you, Lady~Mary.«

James Fleming, being questioned closely as to the delivery of the post, continued to be vague and forgetful, leaving the Court, on the whole, with the impression that no letter had ever been delivered to the Duke.  Sir Wigmore, whose opening speech had contained sinister allusions to an attempt to blacken the character of the victim, smiled disagreeably, and handed the witness over to Sir Impey. The latter contented himself with extracting an admission that witness could not swear positively one way or the other, and passed on immediately to another point.

»Do you recollect whether any letters came by the same post for any of the other members of the party?«

»Yes; I took three or four into the billiard-room.«

»Can you say to whom they were addressed?«

»There were several for Colonel Marchbanks and one for Captain Cathcart.«

»Did Captain Cathcart open his letter there and then?«

»I couldn't say, sir. I left the room immediately to take his grace's letters to the study.«

»Now will you tell us how the letters are collected for the post in the morning at the Lodge?«

»They are put into the post-bag, which is locked. His grace keeps one key and the post-office has the other. The letters are put in through a slit in the top.«

»On the morning after Captain Cathcart's death were the letters taken to the post as usual?«

»Yes, sir.«

»By whom?«

»I took the bag down myself, sir.«

»Had you an opportunity of seeing what letters were in it?«

»I saw there was two or three when the postmistress took 'em out of the bag, but I couldn't say who they was addressed to or anythink of that.«

»Thank you.«

Sir Wigmore Wrinching here bounced up like a very irritable jack-in-the-box.

»Is this the first time you have mentioned this letter which you say you delivered to Captain Cathcart on the night of his murder?«

»My lords,« cried Sir Impey. »I protest against this language. We have as yet had no proof that any murder was committed.«

This was the first indication of the line of defence which Sir Impey proposed to take, and caused a little rustle of excitement.

»My lords,« went on Counsel, replying to a question of the Lord~High Steward, »I submit that so far there has been no attempt to prove murder, and that, until the prosecution have established the murder, such a word cannot properly be put into the mouth of a witness.«

»Perhaps, Sir Wigmore, it would be better to use some other word.«

»It makes no difference to our case, my lord; I bow to your lordships' decision. Heaven knows that I would not seek, even by the lightest or most trivial word, to hamper the defence on so serious a charge.«

»My lords,« interjected Sir Impey, »if the learned Attorney-General considers the word murder to be a triviality, it would be interesting to know to what words he does attach importance.«

»The learned Attorney-General has agreed to substitute another word,« said the Lord~High Steward soothingly, and nodding to Sir Wigmore to proceed.

Sir Impey, having achieved his purpose of robbing the Attorney-General's onslaught on the witness of some of its original impetus, sat down, and Sir Wigmore repeated his question.

»I mentioned it first to Mr~Murbles about three weeks ago.«

»Mr~Murbles is the solicitor for the accused, I believe.«

»Yes, sir.«

»And how was it,« inquired Sir Wigmore ferociously, settling his pince-nez on his rather prominent nose, and glowering at the witness, »that you did not mention this letter at the inquest or at the earlier proceedings in the case?«

»I wasn't asked about it, sir.«

»What made you suddenly decide to go and tell Mr~Murbles about it?«

»He asked me, sir.«

»Oh, he asked you; and you conveniently remembered it when it was suggested to you?«

»No, sir. I remembered it all the time. That is to say, I hadn't given any special thought to it, sir.«

»Oh, you remembered it all the time, though you hadn't given any thought to it. Now I put it to you that you had not remembered about it at all till it was suggested to you by Mr~Murbles.«

»Mr~Murbles didn't suggest nothing, sir. He asked me whether any other letters came by the post, and then I remembered it.«

»Exactly. When it was suggested to you, you remembered it, and not before.«

»No, sir. That is, if I'd been asked before I should have remembered it and mentioned it, but, not being asked, I didn't think it would be of any importance, sir.«

»You didn't think it of any importance that this man received a letter a few hours before his—decease?«

»No, sir. I reckoned if it had been of any importance the police would have asked about it, sir.«

»Now, James Fleming, I put it to you again that it never occurred to you that Captain Cathcart might have received a letter the night he died till the idea was put into your head by the defence.«

The witness, baffled by this interrogative negative, made a confused reply, and Sir Wigmore, glancing round the house as much as to say, »You see this shifty fellow,« proceeded:

»I suppose it didn't occur to you either to mention to the police about the letters in the post-bag?«

»No, sir.«

»Why not?«

»I didn't think it was my place, sir.«

»Did you think about it at all?«

»No, sir.«

»Do you ever think?«

»No, sir—I mean, yes, sir.«

»Then will you please think what you are saying now.«

»Yes, sir.«

»You say that you took all these important letters out of the house without authority and without acquainting the police?«

»I had my orders, sir.«

»From whom?«

»They was his grace's orders, sir.«

»Ah! His grace's orders. When did you get that order?«

»It was part of my regular duty, sir, to take the bag to the post each morning.«

»And did it not occur to you that in a case like this the proper information of the police might be more important than your orders?«

»No, sir.«

Sir Wigmore sat down with a disgusted look; and Sir Impey took the witness in hand again.

»Did the thought of this letter delivered to Captain Cathcart never pass through your mind between the day of the death and the day when Mr~Murbles spoke to you about it?«

»Well, it did pass through my mind, in a manner of speaking, sir.«

»When was that?«

»Before the Grand Jury, sir.«

»And how was it you didn't speak about it then?«

»The gentleman said I was to confine myself to the questions, and not say nothing on my own, sir.«

»Who was this very peremptory gentleman?«

»The lawyer that came down to ask questions for the Crown, sir.«

»Thank you,« said Sir Impey smoothly, sitting down, and leaning over to say something, apparently of an amusing nature, to Mr~Glibbery.

The question of the letter was further pursued in the examination of the Hon. Freddy. Sir Wigmore Wrinching laid great stress upon this witness's assertion that deceased had been in excellent health and spirits when retiring to bed on the Wednesday evening, and had spoken of his approaching marriage. »He seemed particularly cheerio, you know,« said the Hon. Freddy.

»Particularly what?« inquired the Lord~High Steward.

»Cheerio, my lord,« said Sir Wigmore, with a deprecatory bow.

»I do not know whether that is a dictionary word,« said his lordship, entering it upon his notes with meticulous exactness, »but I take it to be synonymous with cheerful.«

The Hon. Freddy, appealed to, said he thought he meant more than just cheerful, more merry and bright, you know.

»May we take it that he was in exceptionally lively spirits?« suggested Counsel.

»Take it in any spirit you like,« muttered the witness, adding, more happily, »Take a peg of John Begg.«

»The deceased was particularly lively and merry when he went to bed,« said Sir Wigmore, frowning horribly, »and looking forward to his marriage in the near future. Would that be a fair statement of his condition?«

The Hon. Freddy agreed to this.

Sir Impey did not cross-examine as to witness's account of the quarrel, but went straight to his point.

»Do you recollect anything about the letters that were brought in the night of the death?«

»Yes; I had one from my aunt. The Colonel had some, I fancy, and there was one for Cathcart.«

»Did Captain Cathcart read his letter there and then?«

»No, I'm sure he didn't. You see, I opened mine, and then I saw he was shoving his away in his pocket, and I thought\longdash«

»Never mind what you thought,« said Sir Impey. »What did you do?«

»I said, »Excuse me, you don't mind, do you?« And he said, »Not at all«; but he didn't read his; and I remember thinking\longdash«

»We can't have that, you know,« said the Lord~High Steward.

»But that's why I'm so sure he didn't open it,« said the Hon. Freddy, hurt. »You see, I said to myself at the time what a secretive fellow he was, and that's how I know.«

Sir Wigmore, who had bounced up with his mouth open, sat down again.

»Thank you, Mr~Arbuthnot,« said Sir Impey, smiling.

Colonel and Mrs~Marchbanks testified to having heard movements in the Duke's study at 11:30. They had heard no shot or other noise. There was no cross-examination.

Mr~Pettigrew-Robinson gave a vivid account of the quarrel, and asserted very positively that there could be no mistaking the sound of the Duke's bedroom door.

»We were then called up by Mr~Arbuthnot at a little after 3 \textsc{a.m.},« proceeded witness, »and went down to the conservatory, where I saw the accused and Mr~Arbuthnot washing the face of the deceased. I pointed out to them what an unwise thing it was to do this, as they might be destroying valuable evidence for the police. They paid no attention to me. There were a number of footmarks round about the door which I wanted to examine, because it was my theory that\longdash«

»My lords,« cried Sir Impey, »we really cannot have this witness's theory.«

»Certainly not!« said the Lord~High Steward. »Answer the questions, please, and don't add anything on your own account.«

»Of course,« said Mr~Pettigrew-Robinson. »I don't mean to imply that there was anything wrong about it, but I considered\longdash«

»Never mind what you considered. Attend to me, please. When you first saw the body, how was it lying?«

»On its back, with Denver and Arbuthnot washing its face. It had evidently been turned over, because\longdash«

»Sir Wigmore,« interposed the Lord~High Steward, »you really must control your witness.«

»Kindly confine yourself to the evidence,« said Sir Wigmore, rather heated. »We do not want your deductions from it. You say that when you saw the body it was lying on its back. Is that correct?«

»And Denver and Arbuthnot were washing it.«

»Yes. Now I want to pass to another point. Do you remember an occasion when you lunched at the Royal Automobile Club?«

»I do. I lunched there one day in the middle of last August—I think it was about the sixteenth or seventeenth.«

»Will you tell us what happened on that occasion?«

»I had gone into the smoke-room after lunch, and was reading in a high-backed armchair, when I saw the prisoner at the Bar come in with the late Captain Cathcart. That is to say, I saw them in the big mirror over the mantelpiece. They did not notice there was anyone there, or they would have been a little more careful what they said, I fancy.  They sat down near me and started talking, and presently Cathcart leaned over and said something in a low tone which I couldn't catch.  The prisoner leapt up with a horrified face, exclaiming, »For God's sake, don't give me away, Cathcart—there'd be the devil to pay.« Cathcart said something reassuring—I didn't hear what, he had a furtive sort of voice—and the prisoner replied, »Well, don't, that's all. I couldn't afford to let anybody get hold of it.« The prisoner seemed greatly alarmed. Captain Cathcart was laughing. They dropped their voices again, and that was all I heard.«

»Thank you.«

Sir Impey took over the witness with a Belial-like politeness.

»You are gifted with very excellent powers of observation and deduction, Mr~Pettigrew-Robinson,« he began, »and no doubt you like to exercise your sympathetic imagination in a scrutiny of people's motives and characters?«

»I think I may call myself a student of human nature,« replied Mr~Pettigrew-Robinson, much mollified.

»Doubtless, people are inclined to confide in you?«

»Certainly. I may say I am a great repository of human documents.«

»On the night of Captain Cathcart's death your wide knowledge of the world was doubtless of great comfort and assistance to the family?«

»They did not avail themselves of my experience, sir,« said Mr~Pettigrew-Robinson, exploding suddenly. »I was ignored completely. If only my advice had been taken at the time\longdash«

»Thank you, thank you,« said Sir Impey, cutting short an impatient exclamation from the Attorney-General, who thereupon rose and demanded:

»If Captain Cathcart had had any secret or trouble of any kind in his life, you would have expected him to tell you about it?«

»From any right-minded young man I might certainly have expected it,« said Mr~Pettigrew-Robinson blusteringly; »but Captain Cathcart was disagreeably secretive. On the only occasion when I showed a friendly interest in his affairs he was very rude indeed. He called me\longdash«

»That'll do,« interposed Sir Impey hastily, the answer to the question not having turned out as he expected. »What the deceased called you is immaterial.«

Mr~Pettigrew-Robinson retired, leaving behind him the impression of a man with a grudge—an impression which seemed to please Mr~Glibbery and Mr~Brownrigg-Fortescue extremely, for they chuckled continuously through the evidence of the next two witnesses.

Mrs~Pettigrew-Robinson had little to add to her previous evidence at the inquest. Miss Cathcart was asked by Sir Impey about Cathcart's parentage, and explained, with deep disapproval in her voice, that her brother, when an all-too-experienced and middle-aged man of the world, had nevertheless »been entangled by« an Italian singer of nineteen, who had »contrived« to make him marry her. Eighteen years later both parents had died. »No wonder,« said Miss Cathcart, »with the rackety life they led,« and the boy had been left to her care. She explained how Denis had always chafed at her influence, gone about with men she disapproved of, and eventually gone to Paris to make a diplomatic career for himself, since which time she had hardly seen him.

An interesting point was raised in the cross-examination of Inspector~Craikes. A penknife being shown him, he identified it as the one found on Cathcart's body.

\textsc{By Mr~Glibbery}: »Do you observe any marks on the blade?«

»Yes, there is a slight notch near the handle.«

»Might the mark have been caused by forcing back the catch of a window?«

Inspector~Craikes agreed that it might, but doubted whether so small a knife would have been adequate for such a purpose. The revolver was produced, and the question of ownership raised.

»My lords,« put in Sir Impey, »we do not dispute the Duke's ownership of the revolver.«

The Court looked surprised, and, after Hardraw the gamekeeper had given evidence of the shot heard at 11:30, the medical evidence was taken.

\textsc{Sir Impey Biggs}: »Could the wound have been self-inflicted?«

»It could, certainly.«

»Would it have been instantly fatal?«

»No. From the amount of blood found upon the path it was obviously not immediately fatal.«

»Are the marks found, in your opinion, consistent with deceased having crawled towards the house?«

»Yes, quite. He might have had sufficient strength to do so.«

»Would such a wound cause fever?«

»It is quite possible. He might have lost consciousness for some time, and contracted a chill and fever by lying in the wet.«

»Are the appearances consistent with his having lived for some hours after being wounded?«

»They strongly suggest it.«

Re-examining, Sir Wigmore Wrinching established that the wound and general appearance of the ground were equally consistent with the theory that deceased had been shot by another hand at very close quarters, and dragged to the house before life was extinct.

»In your experience is it more usual for a person committing suicide to shoot himself in the chest or in the head?«

»In the head is perhaps more usual.«

»So much as almost to create a presumption of murder when the wound is in the chest?«

»I would not go so far as that.«

»But, other things being equal, you would say that a wound in the head is more suggestive of suicide than a body-wound?«

»That is so.«

\makeatletter
\@ifclasswith{scrbook}{a5paper}
{%
  \enlargethispage{\baselineskip}
}{%
}
\makeatother

\textsc{Sir Impey Biggs}: »But suicide by shooting in the heart is not by any means impossible?«

»Oh, dear, no.«

»There have been such cases?«

»Oh, certainly; many such.«

»There is nothing in the medical evidence before you to exclude the idea of suicide?«

»Nothing whatever.«

This closed the case for the Crown. 
