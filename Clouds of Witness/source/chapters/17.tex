%!TeX root=../cloudstop.tex


\chapter{The Eloquent Dead}

\epigraph{\begin{french}Je connaissais Manon: pourquoi m'affliger tant d'un malheur que j'avais dû prévoir.\end{french}}{\textit{Manon Lescaut}}




\lettrine[lines=4]{T}{he} gale had blown itself out into a wonderful fresh day, with clear
spaces of sky, and a high wind rolling boulders of cumulus down the
blue slopes of air.

\zz
The prisoner had been wrangling for an hour with his advisers when
finally they came into court, and even Sir Impey's classical face
showed flushed between the wings of his wig.

»I'm not going to say anything,« said the Duke obstinately. »Rotten
thing to do. I suppose I can't prevent you callin' her if she insists
on comin'—damn' good of her—makes me feel no end of a beast.«

»Better leave it at that,« said Mr~Murbles. »Makes a good impression,
you know. Let him go into the box and behave like a perfect gentleman.
They'll like it.«

Sir Impey, who had sat through the small hours altering his speech,
nodded.

The first witness that day came as something of a surprise. She gave
her name and address as Eliza Briggs, known as Madame Brigette of New
Bond Street, and her occupation as beauty specialist and perfumer. She
had a large and aristocratic clientele of both sexes, and a branch in
Paris.

Deceased had been a client of hers in both cities for several years.
He had massage and manicure. After the war he had come to her about
some slight scars caused by grazing with shrapnel. He was extremely
particular about his personal appearance, and, if you called that
vanity in a man, you might certainly say he was vain. Thank you. Sir
Wigmore Wrinching made no attempt to cross-examine the witness, and the
noble lords wondered to one another what it was all about.

At this point Sir Impey Biggs leaned forward, and, tapping his brief
impressively with his forefinger, began:

»My lords, so strong is our case that we had not thought it necessary
to present an alibi\longdash« when an officer of the court rushed up from a
little whirlpool of commotion by the door and excitedly thrust a note
into his hand. Sir Impey read, coloured, glanced down the hall, put down
his brief, folded his hands over it, and said in a sudden, loud voice
which penetrated even to the deaf ear of the Duke of Wiltshire:

»My lords, I am happy to say that our missing witness is here. I call
Lord~Peter Wimsey.«

Every neck was at once craned, and every eye focused on the very grubby
and oily figure that came amiably trotting up the long room. Sir Impey
Biggs passed the note down to Mr~Murbles, and, turning to the witness,
who was yawning frightfully in the intervals of grinning at all his
acquaintances, demanded that he should be sworn.

The witness's story was as follows:

»I am Lord~Peter Wimsey, brother of the accused. I live at 110
Piccadilly. In consequence of what I read on that bit of blotting-paper
which I now identify, I went to Paris to look for a certain lady. The
name of the lady is Mademoiselle Simone Vonderaa. I found she had
left Paris in company with a man named Van Humperdinck. I followed
her, and at length came up with her in New York. I asked her to give
me the letter Cathcart wrote on the night of his death.« (Sensation).
»I produce that letter, with Mademoiselle Vonderaa's signature on
the corner, so that it can be identified if Wiggy there tries to put
it over you.« (Joyous sensation, in which the indignant protests of
prosecuting counsel were drowned.) »And I'm sorry I've given you
such short notice of this, old man, but I only got it the day before
yesterday. We came as quick as we could, but we had to come down near
Whitehaven with engine trouble, and if we had come down half a mile
sooner I shouldn't be here now.« (Applause, hurriedly checked by the
Lord~High Steward.)

»My lords,« said Sir Impey, »your lordships are witnesses that I
have never seen this letter in my life before. I have no idea of its
contents; yet so positive am I that it cannot but assist my noble
client's case, that I am willing—nay, eager—to put in this document
immediately, as it stands, without perusal, to stand or fall by the
contents.«

»The handwriting must be identified as that of the deceased,«
interposed the Lord~High Steward.

The ravening pencils of the reporters tore along the paper. The lean
young man who worked for the \textit{Daily Trumpet} scented a scandal in high
life and licked his lips, never knowing what a much bigger one had
escaped him by a bare minute or so.

Miss Lydia Cathcart was recalled to identify the handwriting, and the
letter was handed to the Lord~High Steward, who announced:

»The letter is in French. We shall have to swear an interpreter.«

»You will find,« said the witness suddenly, »that those bits of
words on the blotting-paper come out of the letter. You'll 'scuse my
mentioning it.«

»Is this person put forward as an expert witness?« inquired Sir Wigmore
witheringly.

»Right ho!« said Lord~Peter. »Only, you see, it has been rather sprung
on Biggy as you might say.

\begin{verse}
Biggy and Wiggy\\
Were two pretty men,\\
They went into court\\
When the clock—\end{verse}«

»Sir Impey, I must really ask you to keep your witness in order.«

Lord~Peter grinned, and a pause ensued while an interpreter was fetched
and sworn. Then, at last, the letter was read, amid a breathless
silence:

\begin{quotation}
\begin{french}

\begin{flushright}
Riddlesdale Lodge,\\
Stapley,\\
N.E. Yorks.\\
le 13 Octobre, 192—
\end{flushright}

\textsc{Simone},—Je viens de recevoir ta lettre. Que dire? Inutiles, les prières ou les reproches. Tu ne comprendras—tu ne liras même pas.

N'ai-je pas toujours su, d'ailleurs, que tu devais infailliblement me
trahir? Depuis huit ans déjà je souffre tous les torments que puisse
infliger la jalousie. Je comprends bien que tu n'as jamais voulu me
faire de la peine. C'est tout justement cette insouciance, cette
légèreté, cette façon séduisante d'être malhonnête, que j'adorais en
toi. J'ai tout su, et je t'ai aimée.

Ma foi, non, ma chère, jamais je n'ai eu la moindre illusion. Te
rappelles-tu cette première rencontre, un soir au Casino? Tu avais
dix-sept ans, et tu étais jolie à ravir. Le lendemain tu fus à moi. Tu
m'as dit, si gentiment, que tu m'aimais bien, et que j'étais, moi, le
premier. Ma pauvre enfant, tu en as menti. Tu riais, toute seule, de
ma naïveté—il y avait bien de quoi rire! Dès notre premier baiser,
j'ai prévu ce moment.

Mais écoute, Simone. J'ai la faiblesse de vouloir te montrer
exactement ce que tu as fait de moi. Tu regretteras peut-être en peu.
Mais, non—si tu pouvais regretter quoi que ce fût, tu ne serais plus
Simone.

Il y a huit ans, la veille de la guerre, j'étais riche—moins riche
que ton Américain, mais assez riche pour te donner l'éstablissement
qu'il te fallait. Tu étais moins exigeante avant la guerre,
Simone—qui est-ce qui, pendant mon absence, t'a enseigné le goût du
luxe? Charmante discrétion de ma part de ne jamais te le demander! Eh
bien, une grande partie de ma fortune se trouvant placée en Russie et
en Allemagne, j'en ai perdu plus des trois-quarts. Ce que m'en restait
en France a beaucoup diminué en valeur. Il est vrai que j'avais mon
traitement de capitaine dans l'armée britannique, mais c'est peu de
chose, tu sais. Avant même la fin de la guerre, tu m'avais mangé
toutes mes économies. C'était idiot, quoi? Un jeune homme qui a perdu
les trois-quarts de ses rentes ne se permet plus une maîtresse et
un appartement Avenue Kléber. Ou il congédie madame, ou bien il lui
demande quelques sacrifices. Je n'ai rien osé demander. Si j'étais
venu un jour te dire, »Simone, je suis pauvre«—que m'aurais-tu
répondu?

Sais-tu ce que j'ai fait? Non—tu n'as jamais pensé à demander d'où
venait cet argent. Qu'est-ce que cela pouvait te faire que j'ai
tout jeté—fortune, honneur, bonheur—pour te posséder? J'ai joué,
désespérément, éperdument—j'ai fait pis: j'ai triché au jeu. Je te
vois hausser les épaules—tu ris—tu dis, »Tiens, c'est malin, ça!«
Oui, mais cela ne se fait pas. On m'aurait chassé du régiment. Je
devenais le dernier des hommes.

D'ailleurs, cela ne pouvait durer. Déjà un soir à Paris on m'a fait
une scène désagréable, bien qu'on n'ait rien pu prouver. C'est alors
que je me suis fiancé avec cette demoiselle dont je t'ai parlé, la
fille du duc anglais. Le beau projet, quoi! Entretenir ma maîtresse
avec l'argent de ma femme! Et je l'aurais fait—et je le ferais encore
demain, si c'était pour te reposséder.

Mais tu me quittes. Cet Américain est riche—archi-riche. Depuis
longtemps tu me répètes que ton appartement est trop petit et que tu
t'ennuies à mourir. Cet »ami bienveillant« t'offre les autos, les
diamants, les mille-et-une nuits, la lune! Auprés de ces merveilles,
évidemment, que valent l'amour et l'honneur?

Enfin, le bon duc est d'une stupidité très commode. Il laisse traîner son révolver dans le tiroir de son bureau. D'ailleurs, il vient de me
demander une explication à propos de cette histoire de cartes. Tu vois
qu'en tout cas la partie était finie. Pourquoi t'en vouloir? On mettra
sans doute mon suicide au compte de cet exposé. Tant mieux, je ne veux
pas qu'on affiche mon histoire amoureuse dans les journaux.

Adieu, ma bien-aimée—mon adorée, mon adorée, ma Simone. Sois
heureuse avec ton nouvel amant. Ne pense plus à moi. Qu'est-ce tout
cela peut bien te faire? Mon Dieu, comme je t'ai aimée—comme je
t'aime toujours, malgré moi. Mais c'en est fini. Jamais plus tu ne me
perceras le coeur. Oh! J'enrage—je suis fou de douleur! Adieu.

\begin{flushright}
\textsc{Denis Cathcart.}
\end{flushright}
\end{french}
\end{quotation}

\makeatletter
\@ifclasswith{scrbook}{a5paper}
{%
  \clearpage
}{%
    
}
\makeatother


\begin{center}
\textsc{Translation}
\end{center}

\begin{quotation}
\textsc{Simone},—I have just got your letter. What am I to say? It is
useless to entreat or reproach you. You would not understand, or even
read the letter.

Besides, I always knew you must betray me some day. I have suffered
a hell of jealousy for the last eight years. I know perfectly well
you never meant to hurt me. It was just your utter lightness and
carelessness and your attractive way of being dishonest which was so
adorable. I knew everything, and loved you all the same.

Oh, no, my dear, I never had any illusions. You remember our
first meeting that night at the Casino. You were seventeen, and
heartbreakingly lovely. You came to me the very next day. You told me,
very prettily, that you loved me and that I was the first. My poor
little girl, that wasn't true. I expect, when you were alone, you
laughed to think I was so easily taken in. But there was nothing to
laugh at. From our very first kiss I foresaw this moment.

I'm afraid I'm weak enough, though, to want to tell you just what
you have done for me. You may be sorry. But no—if you could regret
anything, you wouldn't be Simone any longer.

Eight years ago, before the war, I was rich—not so rich as your new
American, but rich enough to give you what you wanted. You didn't
want quite so much before the war, Simone. Who taught you to be so
extravagant while I was away? I think it was very nice of me never to
ask you. Well, most of my money was in Russian and German securities,
and more than three-quarters of it went west. The remainder in France
went down considerably in value. I had my captain's pay, of course,
but that didn't amount to much. Even before the end of the war you
had managed to get through all my savings. Of course, I was a fool. A
young man whose income has been reduced by three-quarters can't afford
an expensive mistress and a flat in the Avenue Kléber. He ought either
to dismiss the lady or to demand a little self-sacrifice. But I didn't
dare demand anything. Suppose I had come to you one day and said,
»Simone, I've lost my money«—what would you have said to me?

What do you think I did? I don't suppose you ever thought about it
at all. You didn't care if I was chucking away my money and my honour
and my happiness to keep you. I gambled desperately. I did worse, I
cheated at cards. I can see you shrug your shoulders and say, »Good
for you!« But it's a rotten thing to do—a rotter's game. If anybody
had found out they'd have cashiered me.

Besides, it couldn't go on for ever. There was one row in Paris,
though they couldn't prove anything. So then I got engaged to the
English girl I told you about—the duke's daughter. Pretty, wasn't it?
I actually brought myself to consider keeping my mistress on my wife's
money! But I'd have done it, and I'd do it again, to get you back.

And now you've chucked me. This American is colossally rich. For a
long time you've been dinning into my ears that the flat is too small
and that you're bored to death. Your »good friend« can offer you cars,
diamonds—Aladdin's palace—the moon! I admit that love and honour look
pretty small by comparison.

Ah, well, the Duke is most obligingly stupid. He leaves his revolver
about in his desk drawer. Besides, he's just been in to ask what about
this card-sharping story. So you see the game's up, anyhow. I don't
blame you. I suppose they'll put my suicide down to fear of exposure.
All the better. I don't want my love-affairs in the Sunday Press.

Good-bye, my dear—oh, Simone, my darling, my darling, good-bye. Be
happy with your new lover. Never mind me—what does it all matter? My
God—how I loved you, and how I still love you in spite of myself.
It's all done with. You'll never break my heart again. I'm mad—mad
with misery! Good-bye.

\end{quotation}