%!TeX root=../cloudstop.tex

\chapter{The Second String}

\epigraph{
O, whan he came to broken briggs\\
He bent his bow and swam,\\
And whan he came to the green grass growin'\\
He slacked his shoone and ran.\\
~\\
O, whan he came to Lord William's gates\\
He baed na to chap na ca',\\
But set his bent bow till his breast,\\
An' lightly lap the wa'.}{\textit{Ballad Of Lady Maisry}}



\lettrine[lines=4]{L}{ord} Peter peered out through the cold scurry of cloud. The thin
struts of steel, incredibly fragile, swung slowly across the gleam
and glint far below, where the wide country dizzied out and spread
like a revolving map. In front the sleek leather back of his companion
humped stubbornly, sheeted with rain. He hoped that Grant was feeling
confident. The roar of the engine drowned the occasional shout he threw
to his passenger as they lurched from gust to gust.

He withdrew his mind from present discomforts and went over that last,
strange, hurried scene. Fragments of conversation spun through his head.

»Mademoiselle, I have scoured two continents in search of you.«

»\textit{Voyons}, then, it is urgent. But be quick for the big bear may come
in and be grumpy, and I do not like \textit{des histoires}.«

There had been a lamp on a low table; he remembered the gleam through
the haze of short gold hair. She was a tall girl, but slender, looking
up at him from the huge black-and-gold cushions.

»Mademoiselle, it is incredible to me that you should ever—dine or
dance—with a person called Van Humperdinck.«

Now what had possessed him to say that—when there was so little time,
and Jerry's affairs were of such importance?

»Monsieur van Humperdinck does not dance. Did you seek me through two
continents to say that?«

»No, I am serious.«

»\textit{Eh bien}, sit down.«

She had been quite frank about it.

»Yes, poor soul. But life was very expensive since the war. I refused
several good things. But always \textit{des histoires}. And so little money.
You see, one must be sensible. There is one's old age. It is necessary
to be provident, \textit{hein}?«

»Assuredly.« She had a little accent—very familiar. At first he could
not place it. Then it came to him—Vienna before the war, that capital
of incredible follies.

»Yes, yes, I wrote. I was very kind, very sensible. I said, \foreignlanguage{french}{\textit{»Je ne suis pas femme à supporter de gros ennuis.« Cela se comprend, n'est-ce pas?}}«

That was readily understood. The `plane dived sickly into a sudden pocket, the propeller whirring helplessly in the void, then steadied and began to nose up the opposite spiral.

»I saw it in the papers—yes. Poor boy! Why should anybody have shot him?«

»Mademoiselle, it is for that I have come to you. My brother, whom I dearly love, is accused of the murder. He may be hanged.«

»Brr!«

»For a murder he did not commit.«

»Mon pauvre enfant\longdash«

»Mademoiselle, I implore you to be serious. My brother is accused, and will be standing his trial\longdash«

Once her attention had been caught she had been all sympathy. Her blue eyes had a curious and attractive trick—a full lower lid that shut them into glimmering slits.

»Mademoiselle, I implore you, try to remember what was in his letter.«

»But, mon pauvre ami, how can I? I did not read it. It was very long, very tedious, full of histoires. The thing was finished—I never bother about what cannot be helped, do you?«

But his real agony at this failure had touched her.

»Listen, then; all is perhaps not lost. It is possible the letter is still somewhere about. Or we will ask Adèle. She is my maid. She collects letters to blackmail people—oh, yes, I know! But she is habile comme tout pour la toilette. Wait—we will look first.«

Tossing out letters, trinkets, endless perfumed rubbish from the little gimcrack secretaire, from drawers full of lingerie (»I am so untidy—I am Adèle's despair«) from bags—hundreds of bags—and at last Adèle, thin-lipped and wary-eyed, denying everything till her mistress suddenly slapped her face in a fury, and called her ugly little names in French and German.

»It is useless, then,« said Lord Peter. »What a pity that Mademoiselle Adèle cannot find a thing so valuable to me.«

The word »valuable« suggested an idea to Adèle. There was Mademoiselle's jewel-case which had not been searched. She would fetch it.

»C'est cela que cherche monsieur?«

After that the sudden arrival of Mr Cornelius van Humperdinck, very rich and stout and suspicious, and the rewarding of Adèle in a tactful, unobtrusive fashion by the elevator shaft.

Grant shouted, but the words flipped feebly away into the blackness and were lost. »What?« bawled Wimsey in his ear. He shouted again, and this time the word »juice« shot into sound and fluttered away. But whether the news was good or bad Lord Peter could not tell.

Mr Murbles was aroused a little after midnight by a thunderous knocking upon his door. Thrusting his head out of the window in some alarm, he saw the porter with his lantern steaming through the rain, and behind him a shapeless figure which for the moment Mr Murbles could not make out.

»What's the matter?« said the solicitor.

»Young lady askin' urgently for you, sir.«

The shapeless figure looked up, and he caught the spangle of gold hair in the lantern-light under the little tight hat.

»Mr Murbles, please come. Bunter rang me up. There's a woman come to give evidence. Bunter doesn't like to leave her—she's frightened—but he says it's frightfully important, and Bunter's always right, you know.«

»Did he mention the name?«

»A Mrs Grimethorpe.«

»God bless me! Just a moment, my dear young lady, and I will let you in.«

And, indeed, more quickly than might have been expected, Mr Murbles made his appearance in a Jaeger dressing-gown at the front door.

»Come in, my dear. I will get dressed in a very few minutes. It was quite right of you to come to me. I'm very, very glad you did. What a terrible night! Perkins, would you kindly wake up Mr Murphy and ask him to oblige me with the use of his telephone?«

Mr Murphy—a noisy Irish barrister with a hearty manner—needed no waking. He was entertaining a party of friends, and was delighted to be of service.

»Is that you Biggs? Murbles speaking. That alibi\longdash«

»Yes!«

»Has come along of its own accord.«

»My God! You don't say so!«

»Can you come round to 110 Piccadilly?«

»Straight away.«

It was a strange little party gathered round Lord Peter's fire—the white-faced woman, who started at every sound; the men of law, with their keen, disciplined faces; Lady Mary; Bunter, the efficient. Mrs Grimethorpe's story was simple enough. She had suffered the torments of knowledge ever since Lord Peter had spoken to her. She had seized an hour when her husband was drunk in the »Lord in Glory«, and had harnessed the horse and driven in to Stapley.

»I couldn't keep silence. It's better my man should kill me, for I'm unhappy enough, and maybe I couldn't be any worse off in the Lord's hand—rather than they should hang him for a thing he never done. He was kind, and I was desperate miserable, that's the truth, and I'm hoping his lady won't be hard on him when she knows it all.«

»No, no,« said Mr Murbles, clearing his throat. »Excuse me a moment, madam. Sir Impey\longdash«

The lawyers whispered together in the window-seat.

»You see,« said Sir Impey, »she has burnt her boats pretty well now by coming at all. The great question for us is, Is it worth the risk? After all, we don't know what Wimsey's evidence amounts to.«

»No, that is why I feel inclined—in spite of the risk—to put this evidence in,« said Mr Murbles.

»I am ready to take the risk,« interposed Mrs Grimethorpe starkly.

»We quite appreciate that,« replied Sir Impey. »It is the risk to our client we have to consider first of all.«

»Risk?« cried Mary. »But surely this clears him!«

»Will you swear absolutely to the time when his grace of Denver arrived at Grider's Hole, Mrs Grimethorpe?« went on the lawyer, as though he had not heard her.

»It was a quarter past twelve by the kitchen clock—'tis a very good clock.«

»And he left you at\longdash«

»About five minutes past two.«

»And how long would it take a man, walking quickly, to get back to Riddlesdale Lodge?«

»Oh, wellnigh an hour. It's rough walking, and a steep bank up and down to the beck.«

»You mustn't let the other counsel upset you on those points, Mrs Grimethorpe, because they will try to prove that he had time to kill Cathcart either before he started or after he returned, and by admitting that the Duke had something in his life that he wanted kept secret we shall be supplying the very thing the prosecution lack—\textit{a motive for murdering anyone who might have found him out.}«

There was a stricken silence.

»If I may ask, madam,« said Sir Impey, »has any person any suspicion?«

»My husband guessed,« she answered hoarsely. »I am sure of it. He has always known. But he couldn't prove it. That very night\longdash«

»What night?«

»The night of the murder—he laid a trap for me. He came back from Stapley in the night, hoping to catch us and do murder. But he drank too much before he started, and spent the night in the ditch, or it might be Gerald's death you'd be inquiring into, and mine, as well as the other.«

It gave Mary an odd shock to hear her brother's name spoken like that, by that speaker and in that company. She asked suddenly, apropos of nothing, »Isn't Mr Parker here?«

»No, my dear,« said Mr Murbles reprovingly, »this is not a police matter.«

»The best thing we can do, I think,« said Sir Impey, »is to put in the evidence, and, if necessary, arrange for some kind of protection for this lady. In the meantime\longdash«

»She is coming round with me to mother,« said Lady Mary determinedly.

»My dear lady,« expostulated Mr Murbles, »that would be very unsuitable in the circumstances. I think you hardly grasp\longdash«

»Mother said so,« retorted her ladyship. »Bunter, call a taxi.«

Mr Murbles waved his hands helplessly, but Sir Impey was rather amused. »It's no good, Murbles,« he said. »Time and trouble will tame an advanced young woman, but an advanced old woman is uncontrollable by any earthly force.«

So it was from the Dowager's town house that Lady Mary rang up Mr Charles Parker to tell him the news.