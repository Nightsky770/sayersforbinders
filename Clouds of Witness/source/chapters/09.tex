%!TeX root=../cloudstop.tex


\chapter{Goyles}

\epigraph{»—and the moral of that is\longdash« said the Duchess.}{\textit{Alice's Adventures In Wonderland}}

\lettrine[lines=4]{A}{} party of four were assembled next morning at a very late breakfast, or very early lunch, in Lord Peter's flat. Its most cheerful member, despite a throbbing shoulder and a splitting headache, was undoubtedly Lord Peter himself, who lay upon the Chesterfield surrounded with cushions and carousing upon tea and toast. Having been brought home in an ambulance, he had instantly fallen into a healing sleep, and had woken at nine o'clock aggressively clear and active in mind. In consequence, Mr Parker had been dispatched in a hurry, half-fed and burdened with the secret memory of last night's disclosures, to Scotland Yard. Here he had set in motion the proper machinery for catching Lord Peter's assassin. »Only don't you say anything about the attack on me,« said his lordship. »Tell 'em he's to be detained in connection with the Riddlesdale case. That's good enough for them.« It was now eleven, and Mr Parker had returned, gloomy and hungry, and was consuming a belated omelette and a glass of claret.

Lady Mary Wimsey was hunched up in the window-seat. Her bobbed golden hair made a little blur of light about her in the pale autumn sunshine.  She had made an attempt to breakfast earlier, and now sat gazing out into Piccadilly. Her first appearance that morning had been made in Lord Peter's dressing-gown, but she now wore a serge skirt and jade-green jumper, which had been brought to town for her by the fourth member of the party, now composedly eating a mixed grill and sharing the decanter with Parker.

This was a rather short, rather plump, very brisk elderly lady, with bright black eyes like a bird's, and very handsome white hair exquisitely dressed. Far from looking as though she had just taken a long night journey, she was easily the most composed and trim of the four. She was, however, annoyed, and said so at considerable length.  This was the Dowager Duchess of Denver.

»It is not so much, Mary, that you went off so abruptly last night—just before dinner, too—inconveniencing and alarming us very much—indeed, poor Helen was totally unable to eat her dinner, which was extremely distressing to her feelings, because, you know, she always makes such a point of never being upset about anything—I really don't know why, for some of the greatest men have not minded showing their feelings, I don't mean Southerners necessarily, but, as Mr  Chesterton very rightly points out—Nelson, too, who was certainly English if he wasn't Irish or Scotch, I forget, but United Kingdom, anyway (if that means anything nowadays with a Free State—such a ridiculous title, especially as it always makes one think of the Orange Free State, and I'm sure they wouldn't care to be mixed up with that, being so very green themselves). And going off without even proper clothes, and taking the car, so that I had to wait till the 1:15 from Northallerton—a ridiculous time to start, and such a bad train, too, not getting up till 10:30. Besides, if you \textit{must} run off to town, why do it in that unfinished manner? If you had only looked up the trains before starting you would have seen you would have half an hour's wait at Northallerton, and you could quite easily have packed a bag.  It's so much better to do things neatly and thoroughly—even stupid things. And it was very stupid of you indeed to dash off like that, to embarrass and bore poor Mr Parker with a lot of twaddle—though I suppose it was Peter you meant to see. You know, Peter, if you will haunt low places full of Russians and sucking Socialists taking themselves seriously, you ought to know better than to encourage them by running after them, however futile, and given to drinking coffee and writing poems with no shape to them, and generally ruining their nerves. And, in any case, it makes not the slightest difference; I could have told Peter all about it myself, if he doesn't know already, as he probably does.«

Lady Mary turned very white at this and glanced at Parker, who replied rather to her than to the Dowager:

»No, Lord Peter and I haven't had time to discuss anything yet.«

»Lest it should ruin my shattered nerves and bring a fever to my aching brow,« added that nobleman amiably. »You're a kind, thoughtful soul, Charles, and I don't know what I should do without you. I wish that rotten old second-hand dealer had been a bit brisker about takin' in his stock-in-trade for the night, though. Perfectly 'straor'nary number of knobs there are on a brass bedstead. Saw it comin', y'know, an' couldn't stop myself. However, what's a mere brass bedstead? The great detective, though at first stunned and dizzy from his brutal treatment by the fifteen veiled assassins all armed with meat-choppers, soon regained his senses, thanks to his sound constitution and healthy manner of life. Despite the severe gassing he had endured in the underground room—eh? A telegram? Oh, thanks, Bunter.«

Lord Peter appeared to read the message with great inward satisfaction, for his long lips twitched at the corners, and he tucked the slip of paper away in his pocket-book with a little sigh of satisfaction. He called to Bunter to take away the breakfast-tray and to renew the cooling bandage about his brow. This done, Lord Peter leaned back among his cushions, and with an air of malicious enjoyment launched at Mr Parker the inquiry:

»Well, now, how did you and Mary get on last night? Polly, did you tell him you'd done the murder?«

Few things are more irritating than to discover, after you have been at great pains to spare a person some painful intelligence, that he has known it all along and is not nearly so much affected by it as he properly should be. Mr Parker quite simply and suddenly lost his temper. He bounded to his feet, and exclaimed, without the least reason: »Oh, it's perfectly hopeless trying to do anything!«

Lady Mary sprang from the window-seat.

»Yes, I did,« she said. »It's quite true. Your precious case is finished, Peter.«

The Dowager said, without the least discomposure: »You must allow your brother to be the best judge of his own affairs, my dear.«

»As a matter of fact,« replied his lordship, »I rather fancy Polly's right. Hope so, I'm sure. Anyway, we've got the fellow, so now we shall know.«

Lady Mary gave a sort of gasp, and stepped forward with her chin up and her hands tightly clenched. It caught at Parker's heart to see overwhelming catastrophe so bravely faced. The official side of him was thoroughly bewildered, but the human part ranged itself instantly in support of that gallant defiance.

»Whom have they got?« he demanded, in a voice quite unlike his own.

»The Goyles person,« said Lord Peter carelessly. »Uncommon quick work, what? But since he'd no more original idea than to take the boat-train to Folkestone they didn't have much difficulty.«

»It isn't true,« said Lady Mary. She stamped. »It's a lie. He wasn't there. He's innocent. I killed Denis.«

»Fine,« thought Parker, »fine! Damn Goyles, anyway, what's he done to deserve it?«

Lord Peter said: »Mary, don't be an ass.«

»Yes,« said the Dowager placidly. »I was going to suggest to you, Peter, that this Mr Goyles—such a terrible name, Mary dear, I can't say I ever cared for it, even if there had been nothing else against him—especially as he would sign himself Geo. Goyles—G. e. o. you know, Mr Parker, for George, and I never \textit{could} help reading it as Gargoyles—I very nearly wrote to you, my dear, mentioning Mr Goyles, and asking if you could see him in town, because there was something, when I came to think of it, about that ipecacuanha business that made me feel he might have something to do with it.«

»Yes,« said Peter, with a grin, »you always did find him a bit sickenin', didn't you?«

»How can you, Wimsey?« growled Parker reproachfully, with his eyes on Mary's face.

»Never mind him,« said the girl. »If you can't be a gentleman, Peter\longdash«

»Damn it all!« cried the invalid explosively. »Here's a fellow who, without the slightest provocation, plugs a bullet into my shoulder, breaks my collar-bone, brings me up head foremost on a knobbly second-hand brass bedstead and vamooses, and when, in what seems to me jolly mild, parliamentary language, I call him a sickenin' feller my own sister says I'm no gentleman. Look at me! In my own house, forced to sit here with a perfectly beastly headache, and lap up toast and tea, while you people distend and bloat yourselves on mixed grills and omelettes and a damn good vintage claret\longdash«

»Silly boy,« said the Duchess, »don't get so excited. And it's time for your medicine. Mr Parker, kindly touch the bell.«

Mr Parker obeyed in silence. Lady Mary came slowly across, and stood looking at her brother.

»Peter,« she said, »what makes you say that \textit{he} did it?«

»Did what?«

»Shot—you?« The words were only a whisper.

The entrance of Mr Bunter at this moment with a cooling draught dissipated the tense atmosphere. Lord Peter quaffed his potion, had his pillows rearranged, submitted to have his temperature taken and his pulse counted, asked if he might not have an egg for his lunch, and lit a cigarette. Mr Bunter retired, people distributed themselves into more comfortable chairs, and felt happier.

»Now, Polly, old girl,« said Peter, »cut out the sob-stuff. I accidentally ran into this Goyles chap last night at your Soviet Club.  I asked that Miss Tarrant to introduce me, but the minute Goyles heard my name, he made tracks. I rushed out after him, only meanin' to have a word with him, when the idiot stopped at the corner of Newport Court, potted me, and bunked. Silly-ass thing to do. I knew who he was. He couldn't help gettin' caught.«

»Peter\longdash« said Mary in a ghastly voice.

»Look here, Polly,« said Wimsey. »I did think of you. Honest injun, I did. I haven't had the man arrested. I've made no charge at all—have I, Parker? What did you tell 'em to do when you were down at the Yard this morning?«

»To detain Goyles pending inquiries, because he was wanted as a witness in the Riddlesdale case,« said Parker slowly.

»He knows nothing about it,« said Mary, doggedly now. »He wasn't anywhere near. He is innocent of \textit{that}!«

»Do you think so?« said Lord Peter gravely. »If you know he is innocent, why tell all these lies to screen him? It won't do, Mary. You know he was there—and you think he is guilty.«

»No!«

»Yes,« said Wimsey, grasping her with his sound hand as she shrank away. »Mary, have you thought what you are doing? You are perjuring yourself and putting Gerald in peril of his life, in order to shield from justice a man whom you suspect of murdering your lover and who has most certainly tried to murder me.«

»Oh,« cried Parker, in an agony, »all this interrogation is horribly irregular.«

»Never mind him,« said Peter. »Do you really think you're doing the right thing, Mary?«

The girl looked helplessly at her brother for a minute or two. Peter cocked up a whimsical, appealing eye from under his bandages. The defiance melted out of her face.

»I'll tell the truth,« said Lady Mary.

»Good egg,« said Peter, extending a hand. »I'm sorry. I know you like the fellow, and we appreciate your decision enormously. Truly, we do.  Now, sail ahead, old thing, and you take it down, Parker.«

»Well, it really all started years ago with George. You were at the Front then, Peter, but I suppose they told you about it—and put everything in the worst possible light.«

»I wouldn't say that, dear,« put in the Duchess. »I think I told Peter that your brother and I were not altogether pleased with what he had seen of the young man—which was not very much, if you remember. He invited himself down one weekend when the house was very full, and he seemed to make a point of consulting nobody's convenience but his own. And you know, dear, you even said yourself you thought he was unnecessarily rude to poor old Lord Mountweazle.«

»He said what he thought,« said Mary. »Of course, Lord Mountweazle, poor dear, doesn't understand that the present generation is accustomed to discuss things with its elders, not just kow-tow to them. When George gave his opinion, he thought he was just contradicting.«

»To be sure,« said the Dowager, »when you flatly deny everything a person says it does sound like contradiction to the uninitiated. But all I remember saying to Peter was that Mr Goyles's manners seemed to me to lack polish, and that he showed a lack of independence in his opinions.«

»A lack of independence?« said Mary, wide-eyed.

»Well, dear, I thought so. What oft was thought and frequently much better expressed, as Pope says—or was it somebody else? But the worse you express yourself these days the more profound people think you—though that's nothing new. Like Browning and those quaint metaphysical people, when you never know whether they really mean their mistress or the Established Church, so bridegroomy and biblical—to say nothing of dear S. Augustine—the Hippo man, I mean, not the one who missionised over here, though I daresay he was delightful too, and in those days I suppose they didn't have annual sales of work and tea in the parish room, so it doesn't seem quite like what we mean nowadays by missionaries—he knew all about it—you remember about that mandrake—or is that the thing you had to get a big black dog for?  Manichee, that's the word. What was his name? Was it Faustus? Or am I mixing him up with the old man in the opera?«

»Well, anyway,« said Mary, without stopping to disentangle the Duchess's sequence of ideas, »George was the only person I really cared about—he still is. Only it did seem so hopeless. Perhaps you didn't say much about him, mother, but Gerald said \textit{lots}—dreadful things!«

»Yes,« said the Duchess, »he said what he thought. The present generation does, you know. To the uninitiated, I admit, dear, it does sound a little rude.«

Peter grinned, but Mary went on unheeding.

»George had simply \textit{no} money. He'd really given everything he had to the Labour Party one way and another, and he'd lost his job in the Ministry of Information: they found he had too much sympathy with the Socialists abroad. It was awfully unfair. Anyhow, one couldn't be a burden on him; and Gerald was a beast, and said he'd absolutely stop my allowance if I didn't send George away. So I did, but of course it didn't make a bit of difference to the way we both felt. I will say for mother she was a bit more decent. She said she'd help us if George got a job; but, as I pointed out, if George got a job we shouldn't \textit{need} helping!«

»But, my dear, I could hardly insult Mr Goyles by suggesting that he should live on his mother-in-law,« said the Dowager.

»Why not?« said Mary. »George doesn't believe in those old-fashioned ideas about property. Besides, if you'd given it to me, it would be \textit{my} money. We believe in men and women being equal. Why should the one always be the bread-winner more than the other?«

»I can't imagine, dear,« said the Dowager. »Still, I could hardly expect poor Mr Goyles to live on unearned increment when he didn't believe in inherited property.«

»That's a fallacy,« said Mary, rather vaguely. »Anyhow,« she added hastily, »that's what happened. Then, after the war, George went to Germany to study Socialism and Labour questions there, and nothing seemed any good. So when Denis Cathcart turned up, I said I'd marry him.«

»Why?« asked Peter. »He never sounded to me a bit the kind of bloke for you. I mean, as far as I could make out, he was Tory and diplomatic and—well, quite crusted old tawny, so to speak, I shouldn't have thought you had an idea in common.«

»No; but then he didn't care twopence whether I had any ideas or not. I made him promise he wouldn't bother me with diplomats and people, and he said no, I could do as I liked, provided I didn't compromise him.  And we were to live in Paris and go our own ways and not bother. And anything was better than staying here, and marrying somebody in one's own set, and opening bazaars and watching polo and meeting the Prince of Wales. So I said I'd marry Denis, because I didn't care about him, and I'm pretty sure he didn't care a half-penny about me, and we should have left each other alone. I did so want to be left alone!«

»Was Jerry all right about your money?« inquired Peter.

»Oh, yes. He said Denis was no great catch—I do wish Gerald wasn't so vulgar, in that flat, early-Victorian way—but he said that, after George, he could only thank his stars it wasn't worse.«

»Make a note of that, Charles,« said Wimsey.

»Well, it seemed all right at first, but, as things went on, I got more and more depressed. Do you know, there was something a little alarming about Denis. He was so extraordinarily reserved. I know I wanted to be left alone, but—well, it was uncanny! He was correct. Even when he went off the deep end and was passionate—which didn't often happen—he was correct about it. Extraordinary. Like one of those odd French novels, you know, Peter: frightfully hot stuff, but absolutely impersonal.«

»Charles, old man!« said Lord Peter.

»M'm?«

»That's important. You realize the bearing of that?«

»No.«

»Never mind. Drive on, Polly.«

»Aren't I making your head ache?«

»Damnably; but I like it. Do go on. I'm not sprouting a lily with anguish moist and fever-dew, or anything like that. I'm getting really thrilled. What you've just said is more illuminating than anything I've struck for a week.«

»Really!« Mary stared at Peter with every trace of hostility vanished.  »I thought you'd never understand that part.«

»Lord!« said Peter. »Why not?«

Mary shook her head. »Well, I'd been corresponding all the time with George, and suddenly he wrote to me at the beginning of this month to say he'd come back from Germany, and had got a job on the \textit{Thunderclap}—the Socialist weekly, you know—at a beginning screw of \textsterling 4 a week, and wouldn't I chuck these capitalists and so on, and come and be an honest working woman with him. He could get me a secretarial job on the paper. I was to type and so on for him, and help him get his articles together. And he thought between us we should make \textsterling 6 or \textsterling 7 a week, which would be heaps to live on. And I was getting more frightened of Denis every day. So I said I would. But I knew there'd be an awful row with Gerald. And really I was rather ashamed—the engagement had been announced and there'd be a ghastly lot of talk and people trying to persuade me. And Denis might have made things horribly uncomfortable for Gerald—he was rather that sort. So we decided the best thing to do would be just to run away and get married first, and escape the wrangling.«

»Quite so,« said Peter. »Besides, it would look rather well in the paper, wouldn't it? »Peer's Daughter Weds Socialist—Romantic Side-car Elopement—»\textsterling 6 a Week Plenty,« says Her Ladyship.««

»Pig!« said Lady Mary.

»Very good,« said Peter, »I get you! So it was arranged that the romantic Goyles should fetch you away from Riddlesdale—why Riddlesdale? It would be twice as easy from London or Denver.«

»No. For one thing he had to be up North. And everybody knows one in town, and—anyhow, we didn't want to wait.«

\enquote{Besides, one would miss the Young Lochinvar touch. Well, then, why at the unearthly hour of 3 \textsc{a.m.}?}

»He had a meeting on Wednesday night at Northallerton. He was going to come straight on and pick me up, and run me down to town to be married by special license. We allowed ample time. George had to be at the office next day.«

»I see. Well, I'll go on now, and you stop me if I'm wrong. You went up at 9:30 on Wednesday night. You packed a suit-case. You—did you think of writing any sort of letter to comfort your sorrowing friends and relations?«

»Yes, I wrote one. But I\longdash«

»Of course. Then you went to bed, I fancy, or, at any rate, turned the clothes back and lay down.«

»Yes. I lay down. It was a good thing I did, as it happened\longdash«

»True, you wouldn't have had much time to make the bed look probable in the morning, and we should have heard about it. By the way, Parker, when Mary confessed her sins to you last night, did you make any notes?«

»Yes,« said Parker, »if you can read my shorthand.«

»Quite so,« said Peter. »Well, the rumpled bed disposes of your story about never having gone to bed at all, doesn't it?«

»And I thought it was such a good story!«

»Want of practice,« replied her brother kindly.

»You'll do better next time. It's just as well, really, that it's so hard to tell a long, consistent lie. \textit{Did} you, as a matter of fact, hear Gerald go out at 11:30, as Pettigrew-Robinson (damn his ears!) said?«

»I fancy I did hear somebody moving about,« said Mary, »but I didn't think much about it.«

»Quite right,« said Peter. »When I hear people movin' about the house at night, I'm much too delicate-minded to think anything at all.«

»Of course,« interposed the Duchess, »particularly in England, where it is so oddly improper to think. I will say for Peter that, if he can put a continental interpretation on anything, he will—so considerate of you, dear, as soon as you took to doing it in silence and not mentioning it, as you so intelligently did as a child. You were really a very observant little boy, dear.«

»And still is,« said Mary, smiling at Peter with surprising friendliness.

»Old bad habits die hard,« said Wimsey. »To proceed. At three o'clock you went down to meet Goyles. Why did he come all the way up to the house? It would have been safer to meet him in the lane.«

»I knew I couldn't get out of the lodge-gate without waking Hardraw, and so I'd have to get over the palings somewhere. I might have managed alone, but not with a heavy suit-case. So, as George would have to climb over, anyhow, we thought he'd better come and help carry the suit-case. And then we couldn't miss each other by the conservatory door. I sent him a little plan of the path.«

»Was Goyles there when you got downstairs?«

»No—at least—no, I didn't see him. But there was poor Denis's body, and Gerald bending over it. My first idea was that Gerald had killed George. That's why I said, »O God! you've killed him!«« (Peter glanced across at Parker and nodded.) »Then Gerald turned him over, and I saw it was Denis—and then I'm sure I heard something moving a long way off in the shrubbery—a noise like twigs snapping—and it suddenly came over me, where was George? Oh, Peter, I saw everything then, so clearly. I saw that Denis must have come on George waiting there, and attacked him—I'm sure Denis must have attacked him. Probably he thought it was a burglar. Or he found out who he was and tried to drive him away. And in the struggle George must have shot him. It was awful!«

Peter patted his sister on the shoulder. »Poor kid,« he said.

»I didn't know what to do,« went on the girl. »I'd so awfully little time, you see. My one idea was that nobody must suspect anybody had been there. So I had quickly to invent an excuse for being there myself. I shoved my suit-case behind the cactus-plants to start with.  Jerry was taken up with the body and didn't notice—you know, Jerry never \textit{does} notice things till you shove them under his nose. But I knew if there'd been a shot Freddy and the Marchbankses must have heard it. So I pretended I'd heard it too, and rushed down to look for burglars. It was a bit lame, but the best thing I could think of.  Gerald sent me up to alarm the house, and I had the story all ready by the time I reached the landing. Oh, and I was quite proud of myself for not forgetting the suit-case!«

»You dumped it into the chest,« said Peter.

»Yes. I had a horrible shock the other morning when I found you looking in.«

»Nothing like the shock I had when I found the silver sand there.«

»Silver sand?«

»Out of the conservatory.«

»Good gracious!« said Mary.

»Well, go on. You knocked up Freddy and the Pettigrew-Robinsons. Then you had to bolt into your room to destroy your farewell letter and take your clothes off.«

»Yes. I'm afraid I didn't do that very naturally. But I couldn't expect anybody to believe that I went burglar-hunting in a complete set of silk undies and a carefully knotted tie with a gold safety-pin.«

»No. I see your difficulty.«

»It turned out quite well, too, because they were all quite ready to believe that I wanted to escape from Mrs Pettigrew-Robinson—except Mrs P. herself, of course.«

»Yes; even Parker swallowed that, didn't you, old man?«

»Oh, quite, quite so,« said Parker gloomily.

»I made a dreadful mistake about that shot,« resumed Lady Mary. »You see, I explained it all so elaborately—and then I found that nobody had heard a shot at all. And afterwards they discovered that it had all happened in the shrubbery—and the time wasn't right, either. Then at the inquest I \textit{had} to stick to my story—and it got to look worse and worse—and then they put the blame on Gerald. In my wildest moments I'd never thought of that. Of course, I see now how my wretched evidence helped.«

»Hence the ipecacuanha,« said Peter.

»I'd got into such a frightful tangle,« said poor Lady Mary, »I thought I had better shut up altogether for fear of making things still worse.«

»And did you still think Goyles had done it?«

»I—I didn't know what to think,« said the girl. \enquote{I don't know. Peter, who else \textit{could} have done it?}

»Honestly, old thing,« said his lordship, »if he didn't do it, I don't know who did.«

»He ran away, you see,« said Lady Mary.

»He seems rather good at shootin' and runnin' away,« said Peter grimly.

»If he hadn't done that to you,« said Mary slowly, »I'd never have told you. I'd have died first. But of course, with his revolutionary doctrines—and when you think of red Russia and all the blood spilt in riots and insurrections and things—I suppose it does teach a contempt for human life.«

»My dear,« said the Duchess, »it seems to me that Mr Goyles shows no especial contempt for his own life. You must try to look at the thing fairly. Shooting people and running away is not very heroic—according to \textit{our} standards.«

»The thing I don't understand,« struck in Wimsey hurriedly, »is how Gerald's revolver got into the shrubbery.«

»The thing \textit{I} should like to know about,« said the Duchess, »is, was Denis really a card-sharper?«

»The thing \textit{I} should like to know about,« said Parker, »is the green-eyed cat.«

»Denis \textit{never} gave me a cat,« said Mary. »That was a tarradiddle.«

»Were you ever in a jeweller's with him in the Rue de la Paix?«

»Oh, yes; heaps of times. And he gave me a diamond and tortoiseshell comb. But never a cat.«

»Then we may disregard the whole of last night's elaborate confession,« said Lord Peter, looking through Parker's notes, with a smile. »It's really not bad, Polly, not bad at all. You've quite a talent for romantic fiction—no, I mean it! Just here and there you need more attention to detail. For instance, you \textit{couldn't} have dragged that badly wounded man all up the path to the house without getting blood all over your coat, you know. By the way, did Goyles know Cathcart at all?«

»Not to my knowledge.«

»Because Parker and I had an alternative theory, which would clear Goyles from the worst part of the charge, anyhow. Tell her, old man; it was your idea.«

Thus urged, Parker outlined the blackmail and suicide theory.

»That sounds plausible,« said Mary—»academically speaking, I mean; but it isn't a bit like George—I mean, blackmail is so \textit{beastly}, isn't it?«

»Well,« said Peter, »I think the best thing is to go and see Goyles.  Whatever the key to Wednesday night's riddle is, he holds it. Parker, old man, we're nearing the end of the chase.«
