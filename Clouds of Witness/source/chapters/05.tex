%!TeX root=../cloudstop.tex
\chapter{The Rue St Honoré and the Rue de la Paix}

\epigraph{I think it was the cat.}{\textit{H.M.S. Pinafore}}



\lettrine[lines=4]{M}{r} Parker sat disconsolate in a small \foreignlanguage{french}{\textit{appartement}} in the Rue St  Honoré. It was three o'clock in the afternoon. Paris was full of a subdued but cheerful autumn sunlight, but the room faced north, and was depressing, with its plain, dark furniture and its deserted air.  It was a man's room, well appointed after the manner of a discreet club; a room that kept its dead owner's counsel imperturbably. Two large saddlebag chairs in crimson leather stood by the cold hearth.  On the mantelpiece was a bronze clock, flanked by two polished German shells, a stone tobacco-jar, and an Oriental brass bowl containing a long-cold pipe. There were several excellent engravings in narrow pearwood frames, and the portrait in oils of a rather florid lady of the period of Charles Ⅱ. The window-curtains were crimson, and the floor covered with a solid Turkey carpet. Opposite the fireplace stood a tall mahogany bookcase with glass doors, containing a number of English and French classics, a large collection of books on history and international politics, various French novels, a number of works on military and sporting subjects, and a famous French edition of the \textit{Decameron} with the additional plates. Under the window stood a large bureau.

Parker shook his head, took out a sheet of paper, and began to write a report. He had breakfasted on coffee and rolls at seven; he had made an exhaustive search of the flat; he had interviewed the concierge, the manager of the Crédit Lyonnais, and the Prefect of Police for the Quartier, and the result was very poor indeed.

Information obtained from Captain Cathcart's papers:

\begin{quotation}
Before the war Denis Cathcart had undoubtedly been a rich man. He had considerable investments in Russia and Germany and a large share in a prosperous vineyard in Champagne. After coming into his property at the age of twenty-one he had concluded his three years' residence at Cambridge, and had then travelled a good deal, visiting persons of importance in various countries, and apparently studying with a view to a diplomatic career. During the period from 1913 to 1918 the story told by the books became intensely interesting, baffling, and depressing. At the outbreak of war he had taken a commission in the 15\textsuperscript{th} ------shires.  With the help of the check-book, Parker reconstructed the whole economic life of a young British officer\allowbreak---\allowbreak clothes, horses, equipment, travelling, wine and dinners when on leave, bridge debts, rent of the flat in the Rue St Honoré, club subscriptions, and what not. This outlay was strictly moderate and proportioned to his income. Receipted bills, neatly docketed, occupied one drawer of the bureau, and a careful comparison of these with the check-book and the returned checks revealed no discrepancy. But, beyond these, there appeared to have been another heavy drain upon Cathcart's resources. Beginning in 1913, certain large checks, payable to self, appeared regularly at every quarter, and sometimes at shorter intervals. As to the destination of these sums, the bureau preserved the closest discretion; there were no receipts, no memoranda of their expenditure.

The great crash which in 1914 shook the credits of the world was mirrored in little in the pass-book. The credits from Russian and German sources stopped dead; those from the French shares slumped to a quarter of the original amount, as the tide of war washed over the vineyards and carried the workers away. For the first year or so there were substantial dividends from capital invested in French \textit{rentes}; then came an ominous entry of 20,000 francs on the credit side of the account, and, six months after, another of 30,000 francs. After that the landslide followed fast. Parker could picture those curt notes from the Front, directing the sale of Government securities, as the savings of the past six years whirled away in the maelstrom of rising prices and collapsing currencies. The dividends grew less and less and ceased; then, more ominous still, came a series of debits representing the charges on renewal of promissory notes.

About 1918 the situation had become acute, and several entries showed a desperate attempt to put matters straight by gambling in foreign exchanges. There were purchases, through the bank, of German marks, Russian roubles, and Roumanian lei. Mr Parker sighed sympathetically, when he saw this, thinking of \textsterling 12 worth of these delusive specimens of the engraver's art laid up in his own desk at home. He knew them to be waste-paper, yet his tidy mind could not bear the thought of destroying them. Evidently Cathcart had found marks and roubles very broken reeds.

It was about this time that Cathcart's pass-book began to reveal the paying in of various sums in cash, some large, some small, at irregular dates and with no particular consistency. In December, 1919, there had been one of these amounting to as much as 35,000 francs.  Parker at first supposed that these sums might represent dividends from some separate securities which Cathcart was handling for himself without passing them through the bank. He made a careful search of the room in the hope of finding either the bonds themselves or at least some memorandum concerning them, but the search was in vain, and he was forced to conclude either that Cathcart had deposited them in some secret place or that the credits in question represented some different source of income.

Cathcart had apparently contrived to be demobilized almost at once (owing, no doubt, to his previous frequentation of distinguished governmental personages), and to have taken a prolonged holiday upon the Riviera. Subsequently a visit to London coincided with the acquisition of \textsterling 700, which, converted into francs at the then rate of exchange, made a very respectable item in the account. From that time on, the outgoings and receipts presented a similar aspect and were more or less evenly balanced, the checks to self becoming rather larger and more frequent as time went on, while during 1921 the income from the vineyard began to show signs of recovery.

\end{quotation}

Mr Parker noted down all this information in detail, and, leaning back in his chair, looked round the flat. He felt, not for the first time, a distaste for his profession, which cut him off from the great masculine community whose members take each other for granted and respect their privacy. He relighted his pipe, which had gone out, and proceeded with his report.

Information obtained from Monsieur Turgeot, the manager of the Crédit Lyonnais, confirmed the evidence of the pass-book in every particular.  Monsieur Cathcart had recently made all his payments in notes, usually in notes of small denominations. Once or twice he had had an overdraft\allowbreak---\allowbreak never very large, and always made up within a few months. He had, of course, suffered a diminution of income, like everybody else, but the account had never given the bank any uneasiness. At the moment it was some 14,000 francs on the right side. Monsieur Cathcart was always very agreeable, but not communicative---\foreignlanguage{french}{\textit{très correct}}.


Information obtained from the concierge:
\begin{quote}
One did not see much of Monsieur Cathcart, but he was \foreignlanguage{french}{\textit{très gentil}}.  He never failed to say, \foreignquote{french}{\textit{Bonjour, Bourgois},} when he came in or out. He received visitors sometimes\allowbreak---\allowbreak gentlemen in evening dress. One made card-parties. Monsieur Bourgois had never directed any ladies to his rooms; except once, last February, when he had given a lunch-party to some ladies \textit{très comme il faut} who brought with them his fiancée, \foreignlanguage{french}{\textit{une jolie blonde}}. Monsieur Cathcart used the flat as a \foreignlanguage{french}{\textit{pied-à-terre}}, and often he would shut it up and go away for several weeks or months. He was \foreignlanguage{french}{\textit{un jeune homme très rangé}}. He had never kept a valet. Madame Leblanc, the cousin of one's late wife, kept his \textit{appartement} clean. Madame Leblanc was very respectable. But certainly monsieur might have Madame Leblanc's address.
\end{quote}

Information obtained from Madame Leblanc:

\begin{quote}
Monsieur Cathcart was a charming young man, and very pleasant to work for. Very generous and took a great interest in the family. Madame Leblanc was desolated to hear that he was dead, and on the eve of his marriage to the daughter of the English milady. Madame Leblanc had seen Mademoiselle last year when she visited Monsieur Cathcart in Paris; she considered the young lady very fortunate. Very few young men were as serious as Monsieur Cathcart, especially when they were so good-looking. Madame Leblanc had had experience of young men, and she could relate many histories if she were disposed, but none of Monsieur Cathcart. He would not always be using his rooms; he had the habit of letting her know when he would be at home, and she then went round to put the flat in order. He kept his things very tidy; he was not like English gentlemen in that respect. Madame Leblanc had known many of them, who kept their affairs \foreignlanguage{french}{\textit{sans dessus dessous}}. Monsieur Cathcart was always very well dressed; he was particular about his bath; he was like a woman for his toilet, the poor gentleman. And so he was dead. \foreignlanguage{french}{\textit{Le pauvre garçon!}} Really it had taken away Madame Leblanc's appetite.
\end{quote}

Information obtained from Monsieur the Prefect of Police:

\begin{quote}
Absolutely nothing. Monsieur Cathcart had never caught the eye of the police in any way. With regard to the sums of money mentioned by Monsieur Parker, if monsieur would give him the numbers of some of the notes, efforts would be made to trace them.
\end{quote}

Where had the money gone? Parker could think only of two destinations\allowbreak---\allowbreak an irregular establishment or a blackmailer. Certainly a handsome man like Cathcart might very well have a woman or two in his life, even without the knowledge of the concierge. Certainly a man who habitually cheated at cards\allowbreak---\allowbreak if he did cheat at cards\allowbreak---\allowbreak might very well have got himself into the power of somebody who knew too much.  It was noteworthy that his mysterious receipts in cash began just as his economies were exhausted; it seemed likely that they represented irregular gains from gambling\allowbreak---\allowbreak in the casinos, on the exchange, or, if Denver's story had any truth in it, from crooked play. On the whole, Parker rather inclined to the blackmailing theory. It fitted in with the rest of the business, as he and Lord Peter had reconstructed it at Riddlesdale.

Two or three things, however, still puzzled Parker. Why should the blackmailer have been trailing about the Yorkshire moors with a cycle and side-car? Whose was the green-eyed cat? It was a valuable trinket.  Had Cathcart offered it as part of his payment? That seemed somehow foolish. One could only suppose that the blackmailer had tossed it away with contempt. The cat was in Parker's possession, and it occurred to him that it might be worthwhile to get a jeweller to estimate its value.  But the side-car was a difficulty, the cat was a difficulty, and, more than all, Lady Mary was a difficulty.

Why had Lady Mary lied at the inquest? For that she had lied, Parker had no manner of doubt. He disbelieved the whole story of the second shot which had awakened her. What had brought her to the conservatory door at three o'clock in the morning? Whose was the suit-case\allowbreak---\allowbreak if it was a suit-case\allowbreak---\allowbreak that had lain concealed among the cactus plants? Why this prolonged nervous breakdown, with no particular symptoms, which prevented Lady Mary from giving evidence before the magistrate or answering her brother's inquiries? Could Lady Mary have been present at the interview in the shrubbery? If so, surely Wimsey and he would have found her footprints. Was she in league with the blackmailer? That was an unpleasant thought. Was she endeavouring to help her fiancé? She had an allowance of her own\allowbreak---\allowbreak a generous one, as Parker knew from the Duchess. Could she have tried to assist Cathcart with money? But in that case, why not tell all she knew? The worst about Cathcart\allowbreak---\allowbreak always supposing that card-sharping were the worst\allowbreak---\allowbreak was now matter of public knowledge, and the man himself was dead. If she knew the truth, why did she not come forward and save her brother?

And at this point he was visited by a thought even more unpleasant. If, after all, it had not been Denver whom Mrs Marchbanks had heard in the library, but someone else\allowbreak---\allowbreak someone who had likewise an appointment with the blackmailer\allowbreak---\allowbreak someone who was on his side as against Cathcart\allowbreak---\allowbreak who knew that there might be danger in the interview. Had he himself paid proper attention to the grass lawn between the house and the thicket?  Might Thursday morning perhaps have revealed here and there a trodden blade that rain and sap had since restored to uprightness? Had Peter and he found \textit{all} the footsteps in the wood? Had some more trusted hand fired that shot at close quarters? Once again---\textit{whose was the green-eyed cat}?

Surmises and surmises, each uglier than the last, thronged into Parker's mind. He took up a photograph of Cathcart with which Wimsey had supplied him, and looked at it long and curiously. It was a dark, handsome face; the hair was black, with a slight wave, the nose large and well shaped, the big, dark eyes at once pleasing and arrogant. The mouth was good, though a little thick, with a hint of sensuality in its close curves; the chin showed a cleft. Frankly, Parker confessed to himself, it did not attract him; he would have been inclined to dismiss the man as a »Byronic blighter«, but experience told him that this kind of face might be powerful with a woman, either for love or hatred.

Coincidences usually have the air of being practical jokes on the part of Providence. Mr Parker was shortly to be favoured\allowbreak---\allowbreak if the term is a suitable one\allowbreak---\allowbreak with a special display of this Olympian humour. As a rule, that kind of thing did not happen to him; it was more in Wimsey's line. Parker had made his way from modest beginnings to a respectable appointment in the C.I.D. rather by a combination of hard work, shrewdness, and caution than by spectacular displays of happy guesswork or any knack for taking fortune's tide at the flood. This time, however, he was given a »leading« from above, and it was only part of the nature of things and men that he should have felt distinctly ungrateful for it.

He finished his report, replaced everything tidily in the desk and went round to the police-station to arrange with the Prefect about the keys and the fixing of the seals. It was still early evening and not too cold; he determined, therefore, to banish gloomy thoughts by a \foreignlanguage{french}{\textit{café-cognac}} in the Boul' Mich', followed by a stroll through the Paris of the shops. Being of a kindly, domestic nature, indeed, he turned over in his mind the idea of buying something Parisian for his elder sister, who was unmarried and lived a rather depressing life in Barrow-in-Furness. Parker knew that she would take pathetic delight in some filmy scrap of lace underwear which no one but herself would ever see. Mr Parker was not the kind of man to be deterred by the difficulty of buying ladies' underwear in a foreign language; he was not very imaginative. He remembered that a learned judge had one day asked in court what a camisole was, and recollected that there had seemed to be nothing particularly embarrassing about the garment when explained. He determined that he would find a really Parisian shop, and ask for a camisole. That would give him a start, and then mademoiselle would show him other things without being asked further.

Accordingly, towards six o'clock, he was strolling along the Rue de la Paix with a little carton under his arm. He had spent rather more money than he intended, but he had acquired knowledge. He knew for certain what a camisole was, and he had grasped for the first time in his life that crêpe-de-Chine had no recognizable relation to crape, and was astonishingly expensive for its bulk. The young lady had been charmingly sympathetic, and, without actually insinuating anything, had contrived to make her customer feel just a little bit of a dog. He felt that his French accent was improving. The street was crowded with people, slowly sauntering past the brilliant shop-windows. Mr Parker stopped and gazed nonchalantly over a gorgeous display of jewellery, as though hesitating between a pearl necklace valued at 80,000 francs and a pendant of diamonds and aquamarines set in platinum.

And there, balefully winking at him from under a label inscribed \foreignquote{french}{\textit{Bonne fortune}} hung a green-eyed cat.

The cat stared at Mr Parker, and Mr Parker stared at the cat. It was no ordinary cat. It was a cat with a personality. Its tiny arched body sparkled with diamonds, and its platinum paws, set close together, and its erect and glittering tail were instinct in every line with the sensuous delight of friction against some beloved object. Its head, cocked slightly to one side, seemed to demand a titillating finger under the jaw. It was a minute work of art, by no journeyman hand. Mr  Parker fished in his pocket-book. He looked from the cat in his hand to the cat in the window. They were alike. They were astonishingly alike.  They were identical. Mr Parker marched into the shop.

»I have here,« said Mr Parker to the young man at the counter, »a diamond cat which greatly resembles one which I perceive in your window. Could you have the obligingness to inform me what would be the value of such a cat?«

The young man replied instantly:

»But certainly, monsieur. The price of the cat is 5,000 francs. It is, as you perceive, made of the finest materials. Moreover, it is the work of an artist; it is worth more than the market value of the stones.«

»It is, I suppose, a mascot?«

»Yes, monsieur; it brings great good luck, especially at cards. Many ladies buy these little objects. We have here other mascots, but all of this special design are of similar quality and price. Monsieur may rest assured that his cat is a cat of pedigree.«

»I suppose that such cats are everywhere obtainable in Paris,« said Mr  Parker nonchalantly.

»But no, monsieur. If you desire to match your cat I recommend you to do it quickly. Monsieur Briquet had only a score of these cats to begin with, and there are now only three left, including the one in the window. I believe that he will not make any more. To repeat a thing often is to vulgarize it. There will, of course, be other cats\longdash«

»I don't want another cat,« said Mr Parker, suddenly interested. »Do I understand you to say that cats such as this are only sold by Monsieur Briquet? That my cat originally came from this shop?«

»Undoubtedly, monsieur, it is one of our cats. These little animals are made by a workman of ours\allowbreak---\allowbreak a genius who is responsible for many of our finest articles.«

»It would, I imagine, be impossible to find out to whom this cat was originally sold?«

»If it was sold over the counter for cash it would be difficult, but if it was entered in our books it might not be impossible to discover, if monsieur desired it.«

»I do desire it very much,« said Parker, producing his card. »I am an agent of the British police, and it is of great importance that I should know to whom this cat originally belonged.«

»In that case,« said the young man, »I shall do better to inform monsieur the proprietor.«

He carried away the card into the back premises, and presently emerged with a stout gentleman, whom he introduced as Monsieur Briquet.

In Monsieur Briquet's private office the books of the establishment were brought out and laid on the desk.

»You will understand, monsieur,« said Monsieur Briquet, »that I can only inform you of the names and addresses of such purchasers of these cats as have had an account sent them. It is, however, unlikely that an object of such value was paid for in cash. Still, with rich Anglo-Saxons, such an incident may occur. We need not go back further than the beginning of the year, when these cats were made.« He ran a podgy finger down the pages of the ledger. »The first purchase was on January 19\textsuperscript{th}.«

Mr Parker noted various names and addresses, and at the end of half an hour Monsieur Briquet said in a final manner:

»That is all, monsieur. How many names have you there?«

»Thirteen,« said Parker.

»And there are still three cats in stock\allowbreak---\allowbreak the original number was twenty\allowbreak---\allowbreak so that four must have been sold for cash. If monsieur wishes to verify the matter we can consult the day-book.«

The search in the day-book was longer and more tiresome, but eventually four cats were duly found to have been sold; one on January 31\textsuperscript{st}, another on February 6\textsuperscript{th}, the third on May 17\textsuperscript{th}, and the last on August 9\textsuperscript{th}.

Mr Parker had risen, and embarked upon a long string of compliments and thanks, when a sudden association of ideas and dates prompted him to hand Cathcart's photograph to Monsieur Briquet and ask whether he recognized it.

Monsieur Briquet shook his head.

»I am sure he is not one of our regular customers,« he said, »and I have a very good memory for faces. I make a point of knowing anyone who has any considerable account with me. And this gentleman has not everybody's face. But we will ask my assistants.«

The majority of the staff failed to recognize the photograph, and Parker was on the point of putting it back in his pocket-book when a young lady, who had just finished selling an engagement ring to an obese and elderly Jew, arrived, and said, without any hesitation:

»\foreignlanguage{french}{\textit{Mais oui, je l'ai vu, ce monsieur-là.}} It is the Englishman who bought a diamond cat for the \foreignlanguage{french}{\textit{jolie blonde}}.«

»Mademoiselle,« said Parker eagerly, »I beseech you to do me the favour to remember all about it.«

»\foreignlanguage{french}{\textit{Parfaitement},}« said she. »It is not the face one would forget, especially when one is a woman. The gentleman bought a diamond cat and paid for it\allowbreak---\allowbreak no, I am wrong. It was the lady who bought it, and I remember now to have been surprised that she should pay like that at once in money, because ladies do not usually carry such large sums. The gentleman bought too. He bought a diamond and tortoiseshell comb for the lady to wear, and then she said she must give him something \foreignlanguage{french}{\textit{pour porter bonheur}}, and asked me for a mascot that was good for cards. I showed her some jewels more suitable for a gentleman, but she saw these cats and fell in love with them, and said he should have a cat and nothing else; she was sure it would bring him good hands. She asked me if it was not so, and I said, »Undoubtedly, and monsieur must be sure never to play without it,« and he laughed very much, and promised always to have it upon him when he was playing.«

»And how was she, this lady?«

»Blond, monsieur, and very pretty; rather tall and svelte, and very well dressed. A big hat and dark blue costume. \foreignlanguage{french}{\textit{Quoi encore? Voyons}}---yes, she was a foreigner.«

»English?«

»I do not know. She spoke French very, very well, almost like a French person, but she had just the little suspicion of accent.«

»What language did she speak with the gentleman?«

»French, monsieur. You see, we were speaking together, and they both appealed to me continually, and so all the talk was in French. The gentleman spoke French \foreignlanguage{french}{\textit{à merveille}}, it was only by his clothes and a \foreignlanguage{french}{\textit{je ne sais quoi}} in his appearance that I guessed he was English. The lady spoke equally fluently, but one remarked just the accent from time to time. Of course, I went away from them once or twice to get goods from the window, and they talked then; I do not know in what language.«

»Now, mademoiselle, can you tell me how long ago this was?«

\foreignquote{french}{\textit{Ah, mon Dieu, ça c'est plus difficile. Monsieur sait que les jours se suivent et se ressemblent. Voyons.}}

»We can see by the day-book,« put in Monsieur Briquet, »on what occasion a diamond comb was sold with a diamond cat.«

»Of course,« said Parker hastily. »Let us go back.«

They went back and turned to the January volume, where they found no help. But on February 6\textsuperscript{th} they read:

~\\
\begin{tabular} { l l } 
\foreignlanguage{french}{\textit{Peigne en écaille et diamants}}&\textit{f.7,500}\\
\foreignlanguage{french}{\textit{Chat en diamants (Dessin C-5)}}&\textit{f.5,000}\\
\end{tabular}
~\\

»That settles it,« said Parker gloomily.

»Monsieur does not appear content,« suggested the jeweller.

»Monsieur,« said Parker, »I am more grateful than I can say for your very great kindness, but I will frankly confess that, of all the twelve months in the year, I had rather it had been any other.«

Parker found this whole episode so annoying to his feelings that he bought two comic papers and, carrying them away to Boudet's at the corner of the Rue Auguste Léopold, read them solemnly through over his dinner, by way of settling his mind. Then, returning to his modest hotel, he ordered a drink, and sat down to compose a letter to Lord Peter. It was a slow job, and he did not appear to relish it very much.  His concluding paragraph was as follows:

\begin{quote}

I have put all these things down for you without any comment. You will be able to draw your own inferences as well as I can\allowbreak---\allowbreak better, I hope, for my own are perplexing and worrying me no end. They may be all rubbish\allowbreak---\allowbreak I hope they are; I daresay something will turn up at your end to put quite a different interpretation upon the facts. But I do feel that they must be cleared up. I would offer to hand over the job, but another man might jump at conclusions even faster than I do, and make a mess of it. But of course, if you say so, I will be taken suddenly ill at any moment. Let me know. If you think I'd better go on grubbing about over here, can you get hold of a photograph of Lady Mary Wimsey, and find out if possible about the diamond comb and the green-eyed cat\allowbreak---\allowbreak also at exactly what date Lady Mary was in Paris in February. Does she speak French as well as you do? Let me know how you are getting on.

\begin{flushright}
Yours ever,\\
\textsc{Charles Parker.}
\end{flushright}
\end{quote}


He re-read the letter and report carefully and sealed them up. Then he wrote to his sister, did up his parcel neatly, and rang for the valet de chambre.

»I want this letter sent off at once, registered,« he said, \enquote{and the parcel is to go tomorrow as a \textit{colis postal}.}

After which he went to bed, and read himself to sleep with a commentary on the Epistle to the Hebrews.

\noindent\hfil\rule{0.5\textwidth}{.4pt}\hfil

Lord Peter's reply arrived by return:

\begin{quotation}

\textsc{Dear Charles},---Don't worry. I don't like the look of things myself frightfully, but I'd rather you tackled the business than anyone else. As you say, the ordinary police bloke doesn't mind whom he arrests, provided he arrests someone, and is altogether a most damnable fellow to have poking into one's affairs. I'm putting my mind to getting my brother cleared\allowbreak---\allowbreak that \textit{is} the first consideration, after all, and really anything else would be better than having Jerry hanged for a crime he didn't commit. Whoever did it, it's better the right person should suffer than the wrong. So go ahead.

I enclose two photographs\allowbreak---\allowbreak all I can lay hands on for the moment. The one in nursing-kit is rather rotten, and the other's all smothered up in a big hat.

I had a damn' queer little adventure here on Wednesday, which I'll tell you about when we meet. I've found a woman who obviously knows more than she ought, and a most promising ruffian\allowbreak---\allowbreak only I'm afraid he's got an alibi. Also I've got a faint suggestion of a clue about № 10. Nothing much happened at Northallerton, except that Jerry was of course committed for trial. My mother is here, thank God! and I'm hoping she'll get some sense out of Mary, but she's been worse the last two days\allowbreak---\allowbreak Mary, I mean, not my mother\allowbreak---\allowbreak beastly sick and all that sort of thing. Dr Thingummy\allowbreak---\allowbreak who is an ass\allowbreak---\allowbreak can't make it out. Mother says it's as clear as noon-day, and she'll stop it if I have patience a day or two. I made her ask about the comb and the cat. M. denies the cat altogether, but admits to a diamond comb bought in Paris\allowbreak---\allowbreak says she bought it herself. It's in town\allowbreak---\allowbreak I'll get it and send it on. She says she can't remember where she bought it, has lost the bill, but it didn't cost anything like 7,500 francs. She was in Paris from February 2\textsuperscript{nd} to February 20\textsuperscript{th}. My chief business now is to see Lubbock and clear up a little matter concerning silver sand.

The Assizes will be the first week in November\allowbreak---\allowbreak in fact, the end of next week. This rushes things a bit, but it doesn't matter, because they can't try him there; nothing will matter but the Grand Jury, who are bound to find a true bill on the face of it. After that we can hang matters up as long as we like. It's going to be a deuce of a business, Parliament sitting and all. Old Biggs is fearfully perturbed under that marble outside of his. I hadn't really grasped what a fuss it was to try peers. It's only happened about once in every sixty years, and the procedure's about as old as Queen Elizabeth. They have to appoint a Lord High Steward for the occasion, and God knows what. They have to make it frightfully clear in the Commission that it \textit{is} only for the occasion, because, somewhere about Richard Ⅲ's time, the L.H.S. was such a terrifically big pot that he got to ruling the roost. So when Henry Ⅳ came to the throne, and the office came into the hands of the Crown, he jolly well kept it there, and now they only appoint a man \textit{pro tem.} for the Coronation and shows like Jerry's. The King always pretends not to know there isn't a L.H.S. till the time comes, and is no end surprised at having to think of somebody to take on the job. Did you know all this? I didn't. I got it out of Biggy.

Cheer up. Pretend you don't know that any of these people are relations of mine. My mother sends you her kindest regards and what not, and hopes she'll see you again soon. Bunter sends something correct and respectful; I forget what.

\begin{flushright}
Yours in the brotherhood of detection.\\
\textsc{P.W.}
\end{flushright}
\end{quotation}
It may as well be said at once that the evidence from the photographs was wholly inconclusive.
