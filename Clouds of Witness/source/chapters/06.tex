%!TeX root=../cloudstop.tex 
\chapter{Mary Quite Contrary}

\epigraph{I am striving to take into public life what any man gets from his mother.}{Lady Astor}


\lettrine[lines=4]{O}{n} the opening day of the York Assizes, the Grand Jury brought in a true bill, against Gerald, Duke of Denver, for murder. Gerald, Duke of Denver, being accordingly produced in the court, the Judge affected to discover—what, indeed, every newspaper in the country had been announcing to the world for the last fortnight—that he, being but a common or garden judge with a plebeian jury, was incompetent to try a peer of the realm. He added, however, that he would make it his business to inform the Lord Chancellor (who also, for the last fortnight, had been secretly calculating the accommodation in the Royal Gallery and choosing lords to form the Select Committee). Order being taken accordingly, the noble prisoner was led away.

\noindent\hfil\rule{0.5\textwidth}{.4pt}\hfil

A day or two later, in the gloom of a London afternoon, Mr Charles Parker rang the bell of a second-floor flat at No. 110 Piccadilly. The door was opened by Bunter, who informed him with a gracious smile that Lord Peter had stepped out for a few minutes but was expecting him, and would he kindly come in and wait.

»We only came up this morning,« added the valet, »and are not quite straight yet, sir, if you will excuse us. Would you feel inclined for a cup of tea?«

Parker accepted the offer, and sank luxuriously into a corner of the Chesterfield. After the extraordinary discomfort of French furniture there was solace in the enervating springiness beneath him, the cushions behind his head, and Wimsey's excellent cigarettes. What Bunter had meant by saying that things were »not quite straight yet« he could not divine. A leaping wood fire was merrily reflected in the spotless surface of the black baby grand; the mellow calf bindings of Lord Peter's rare editions glowed softly against the black and primrose walls; the vases were filled with tawny chrysanthemums; the latest editions of all the papers were on the table—as though the owner had never been absent.

Over his tea Mr Parker drew out the photographs of Lady Mary and Denis Cathcart from his breast pocket. He stood them up against the teapot and stared at them, looking from one to the other as if trying to force a meaning from their faintly smirking, self-conscious gaze. He referred again to his Paris notes, ticking off various points with a pencil.  »Damn!« said Mr Parker, gazing at Lady Mary. »Damn—damn—damn\longdash«

The train of thought he was pursuing was an extraordinarily interesting one. Image after image, each rich in suggestion, crowded into his mind. Of course, one couldn't think properly in Paris—it was so uncomfortable and the houses were central heated. Here, where so many problems had been unravelled, there was a good fire. Cathcart had been sitting before the fire. Of course, he wanted to think out a problem.  When cats sat staring into the fire they were thinking out problems. It was odd he should not have thought of that before. When the green-eyed cat sat before the fire one sank right down into a sort of rich, black, velvety suggestiveness which was most important. It was luxurious to be able to think so lucidly as this, because otherwise it would be a pity to exceed the speed limit—and the black moors were reeling by so fast.  But now he had really got the formula he wouldn't forget it again. The connection was just there—close, thick, richly coherent.

»The glass-blower's cat is bompstable,« said Mr Parker aloud and distinctly.

»I'm charmed to hear it,« replied Lord Peter, with a friendly grin.  »Had a good nap, old man?«

»I—what?« said Mr Parker. »Hullo! Watcher mean, nap? I had got hold of a most important train of thought, and you've put it out of my head.  What was it? Cat—cat—cat\longdash« He groped wildly.

\enquote{You \textit{said} »The glass-blower's cat is bompstable,«} retorted Lord Peter. »It's a perfectly rippin' word, but I don't know what you mean by it.«

»Bompstable?« said Mr Parker, blushing slightly. »Bomp—oh, well, perhaps you're right—I may have dozed off. But, you know, I thought I'd just got the clue to the whole thing. I attached the greatest importance to that phrase. Even now—No, now I come to think of it, my train of thought doesn't seem quite to hold together. What a pity. I thought it was so lucid.«

»Never mind,« said Lord Peter. »Just back?«

»Crossed last night. Any news?«

»Lots.«

»Good?«

»No.«

Parker's eyes wandered to the photographs.

»I don't believe it,« he said obstinately. »I'm damned if I'm going to believe a word of it.«

»A word of what?«

»Of whatever it is.«

»You'll have to believe it, Charles, as far as it goes,« said his friend softly, filling his pipe with decided little digs of the fingers. »I don't say«—dig—»that Mary«—dig—»shot Cathcart«—dig, dig—»but she has lied«—dig—»again and again.«—Dig, dig—»She knows who did it«—dig—»she was prepared for it«—dig—»she's malingering and lying to keep the fellow shielded«—dig—»and we shall have to make her speak.« Here he struck a match and lit the pipe in a series of angry little puffs.

»If you can think,« said Mr Parker, with some heat, »that that woman«—he indicated the photographs—»had any hand in murdering Cathcart, I don't care what your evidence is, you—hang it all, Wimsey, she's your own sister.«

»Gerald is my brother,« said Wimsey quietly. »You don't suppose I'm exactly enjoying this business, do you? But I think we shall get along very much better if we try to keep our tempers.«

»I'm awfully sorry,« said Parker. »Can't think why I said that—rotten bad form—beg pardon, old man.«

»The best thing we can do,« said Wimsey, »is to look the evidence in the face, however ugly. And I don't mind admittin' that some of it's a positive gargoyle.«

\enquote{My mother turned up at Riddlesdale on Friday. She marched upstairs at once and took possession of Mary, while I drooped about in the hall and teased the cat, and generally made a nuisance of myself. \textit{You} know. Presently old Dr Thorpe called. I went and sat on the chest on the landing. Presently the bell rings and Ellen comes upstairs. Mother and Thorpe popped out and caught her just outside Mary's room, and they jibber-jabbered a lot, and presently mother came barging down the passage to the bathroom with her heels tapping and her earrings simply dancing with irritation. I sneaked after 'em to the bathroom door, but I couldn't see anything, because they were blocking the doorway, but I heard mother say, »There, now, what did I tell you«; and Ellen said, »Lawks! your grace, who'd 'a' thought it?«; and my mother said, \enquote{All I can say is, if I had to depend on you people to save me from being murdered with arsenic or that other stuff with the name like anemones\footnote{Antinomy? The Duchess appears to have had Dr Pritchard's case in mind.}}—you know what I mean—that that very attractive-looking man with the preposterous beard used to make away with his wife and mother-in-law (who was vastly the more attractive of the two, poor thing), I might be being cut up and analysed by Dr Spilsbury now—such a horrid, distasteful job he must have of it, poor man, and the poor little rabbits, too.} Wimsey paused for breath, and Parker laughed in spite of his anxiety.

»I won't vouch for the exact words,« said Wimsey, \enquote{but it was to that effect—you know my mother's style. Old Thorpe tried to look dignified, but mother ruffled up like a little hen and said, looking beadily at him: \enquote{In \textit{my} day we called that kind of thing hysterics and naughtiness. \textit{We} didn't let girls pull the wool over our eyes like that. I suppose \textit{you} call it a neurosis, or a suppressed desire, or a reflex, and coddle it. You might have let that silly child make herself really ill. You are all perfectly ridiculous, and no more fit to take care of yourselves than a lot of babies—not but what there are plenty of poor little things in the slums that look after whole families and show more sense than the lot of you put together. I am very angry with Mary, advertising herself in this way, and she's not to be pitied.} You know,} said Wimsey, »I think there's often a great deal in what one's mother says.«

»I believe you,« said Parker.

»Well, I got hold of mother afterwards and asked her what it was all about. She said Mary wouldn't tell her anything about herself or her illness; just asked to be let alone. Then Thorpe came along and talked about nervous shock—said he couldn't understand these fits of sickness, or the way Mary's temperature hopped about. Mother listened, and told him to go and see what the temperature was now. Which he did, and in the middle mother called him away to the dressing-table. But, bein' a wily old bird, you see, she kept her eyes on the looking-glass, and nipped round just in time to catch Mary stimulatin' the thermometer to terrific leaps on the hotwater bottle.«

»Well, I'm damned!« said Parker.

»So was Thorpe. All mother said was, that if he wasn't too old a bird yet to be taken in by that hoary trick he'd no business to be gettin' himself up as a grey-haired family practitioner. So then she asked the girl about the sick fits—when they happened, and how often, and was it after meals or before, and so on, and at last she got out of them that it generally happened a bit after breakfast and occasionally at other times. Mother said she couldn't make it out at first, because she'd hunted all over the room for bottles and things, till at last she asked who made the bed, thinkin', you see, Mary might have hidden something under the mattress. So Ellen said she usually made it while Mary had her bath. »When's that?« says mother. »Just before her breakfast,« bleats the girl. »God forgive you all for a set of nincompoops,« says my mother. »Why didn't you say so before?« So away they all trailed to the bathroom, and there, sittin' up quietly on the bathroom shelf among the bath salts and the Elliman's embrocation and the Kruschen feelings and the toothbrushes and things, was the family bottle of ipecacuanha—three-quarters empty! Mother said—well, I told you what she said. By the way, how do you spell ipecacuanha?«

Mr Parker spelt it.

»Damn you!« said Lord Peter. \enquote{I \textit{did} think I'd stumped you that time.  I believe you went and looked it up beforehand. \textit{No} decent-minded person would know how to spell ipecacuanha out of his own head. Anyway, as you were saying, it's easy to see which side of the family has the detective instinct.}

»I didn't say so\longdash«

»I know. Why didn't you? I think my mother's talents deserve a little acknowledgment. I said so to her, as a matter of fact, and she replied in these memorable words: »My dear child, you can give it a long name if you like, but I'm an old-fashioned woman and I call it mother-wit, and it's so rare for a man to have it that if he does you write a book about him and call him Sherlock Holmes.« However, apart from all that, I said to mother (in private, of course), »It's all very well, but I can't believe that Mary has been going to all this trouble to make herself horribly sick and frighten us all just to show off. Surely she isn't that sort.« Mother looked at me as steady as an owl, and quoted a whole lot of examples of hysteria, ending up with the servant-girl who threw paraffin about all over somebody's house to make them think it was haunted, and finished up—that if all these new-fangled doctors went out of their way to invent subconsciousness and kleptomania, and complexes and other fancy descriptions to explain away when people had done naughty things, she thought one might just as well take advantage of the fact.«

»Wimsey,« said Parker, much excited, »did she mean she suspected something?«

»My dear old chap,« replied Lord Peter, \enquote{whatever can be known about Mary by putting two and two together my mother knows. I told her all \textit{we} knew up to that point, and she took it all in, in her funny way, you know, never answering anything directly, and then she put her head on one side and said: »If Mary had listened to me, and done something useful instead of that V.A.D. work, which never came to much, if you ask me—not that I have anything against V.A.D.'s in a general way, but that silly woman Mary worked under was the most terrible snob on God's earth—and there were very much more sensible things which Mary might really have done well, only that she was so crazy to get to London—I shall always say it was the fault of that ridiculous club—what could you expect of a place where you ate such horrible food, all packed into an underground cellar painted pink and talking away at the tops of their voices, and never any evening dress—only Soviet jumpers and side-whiskers. Anyhow, I've told that silly old man what to say about it, and they'll never be able to think of a better explanation for themselves.« Indeed, you know,} said Peter, »I think if any of them start getting inquisitive, they'll have mother down on them like a ton of bricks.«

»What do you really think yourself?« asked Parker.

»I haven't come yet to the unpleasantest bit of the lot,« said Peter. »I've only just heard it, and it did give me a nasty jar, I'll admit.  Yesterday I got a letter from Lubbock saying he would like to see me, so I trotted up here and dropped in on him this morning. You remember I sent him a stain off one of Mary's skirts which Bunter had cut out for me? I had taken a squint at it myself, and didn't like the look of it, so I sent it up to Lubbock, \textit{ex abundantia cautelœ}; and I'm sorry to say he confirms me. It's human blood, Charles, and I'm afraid it's Cathcart's.«

»But—I've lost the thread of this a bit.«

»Well, the skirt must have got stained the day Cathcart—died, as that was the last day on which the party was out on the moors, and if it had been there earlier Ellen would have cleaned it off. Afterwards Mary strenuously resisted Ellen's efforts to take the skirt away, and made an amateurish effort to tidy it up herself with soap. So I think we may conclude that Mary knew the stains were there, and wanted to avoid discovery. She told Ellen that the blood was from a grouse—which must have been a deliberate untruth.«

»Perhaps,« said Parker, struggling against hope to make out a case for Lady Mary, »she only said, »Oh! one of the birds must have bled,« or something like that.«

»I don't believe,« said Peter, »that one could get a great patch of human blood on one's clothes like that and not know what it was. She must have knelt right in it. It was three or four inches across.«

Parker shook his head dismally, and consoled himself by making a note.

»Well, now,« went on Peter, \enquote{on Wednesday night everybody comes in and dines and goes to bed except Cathcart, who rushes out and stays out. At 11:50 the gamekeeper, Hardraw, hears a shot which may very well have been fired in the clearing where the—well, let's say the accident—took place. The time also agrees with the medical evidence about Cathcart having already been dead three or four hours when he was examined at 4:30. Very well. At 3 \textsc{a.m.} Jerry comes home from somewhere or other and finds the body. As he is bending over it, Mary arrives in the most apropos manner from the house in her coat and cap and walking shoes. Now what is her story? She says that at three o'clock she was awakened by a shot. Now nobody else heard that shot, and we have the evidence of Mrs Pettigrew-Robinson, who slept in the next room to Mary, with her window open according to her immemorial custom, that she lay broad awake from 2 \textsc{a.m.} till a little after 3 \textsc{a.m.}, when the alarm was given, and heard no shot. According to Mary, the shot was loud enough to waken her on the other side of the building. It's odd, isn't it, that the person already awake should swear so positively that she heard nothing of a noise loud enough to waken a healthy young sleeper next door? And, in any case, \textit{if} that was the shot that killed Cathcart, he can barely have been dead when my brother found him—and again, in that case, how was there time for him to be carried up from the shrubbery to the conservatory?}

»We've been over all this ground,« said Parker, with an expression of distaste. »We agreed that we couldn't attach any importance to the story of the shot.«

»I'm afraid we've got to attach a great deal of importance to it,« said Lord Peter gravely. »Now, what does Mary do? Either she thought the shot\longdash«

»There was no shot.«

»I know that. But I'm examining the discrepancies of her story. She said she did not give the alarm because she thought it was probably only poachers. But, if it was poachers, it would be absurd to go down and investigate. So she explains that she thought it might be burglars.  Now how does she dress to go and look for burglars? What would you or I have done? I think we would have taken a dressing-gown, a stealthy kind of pair of slippers, and perhaps a poker or a stout stick—not a pair of walking shoes, a coat, and a cap, of all things!«

»It was a wet night,« mumbled Parker.

»My dear chap, if it's burglars you're looking for you don't expect to go and hunt them round the garden. Your first thought is that they're getting into the house, and your idea is to slip down quietly and survey them from the staircase or behind the dining-room door. Anyhow, fancy a present-day girl, who rushes about bare-headed in all weathers, stopping to embellish herself in a cap for a burglar-hunt—damn it all, Charles, it won't wash, you know! And she walks straight off to the conservatory and comes upon the corpse, exactly as if she knew where to look for it beforehand.«

Parker shook his head again.

\enquote{Well, now. She sees Gerald stooping over Cathcart's body. What does she say? Does she ask what's the matter? Does she ask who it is? She exclaims: »O God! Gerald, you've killed him,« and \textit{then} she says, as if on second thoughts, »Oh, it's Denis! What has happened? Has there been an accident?« Now, does that strike you as natural?}

»No. But it rather suggests to me that it wasn't Cathcart she expected to see there, but somebody else.«

»Does it? It rather sounds to me as if she was pretending not to know who it was. First she says, »You've killed him!« and then, recollecting that she isn't supposed to know who »he« is, she says, »Why, it's Denis!««

»In any case, then, if her first exclamation was genuine, she didn't expect to find the man dead.«

\enquote{No—no—we must remember that. The death \textit{was} a surprise. Very well.  Then Gerald sends Mary up for help. And here's where a little bit of evidence comes in that you picked up and sent along. Do you remember what Mrs Pettigrew-Robinson said to you in the train?}

»About the door slamming on the landing, do you mean?«

\enquote{Yes. Now I'll tell you something that happened to me the other morning. I was burstin' out of the bathroom in my usual breezy way when I caught myself a hell of a whack on that old chest on the landin', and the lid lifted up and shut down, \textit{plonk!} That gave me an idea, and I thought I'd have a squint inside. I'd got the lid up and was lookin' at some sheets and stuff that were folded up at the bottom, when I heard a sort of gasp, and there was Mary, starin' at me, as white as a ghost.  She gave me a turn, by Jove, but nothin' like the turn I'd given her.  Well, she wouldn't say anything to me, and got hysterical, and I hauled her back to her room. But I'd seen something on those sheets.}

»What?«

»Silver sand.«

»Silver\longdash«

»D'you remember those cacti in the greenhouse, and the place where somebody'd put a suit-case or something down?«

»Yes.«

»Well, there was a lot of silver sand scattered about—the sort people stick round some kinds of bulbs and things.«

»And that was inside the chest too?«

»Yes. Wait a moment. After the noise Mrs Pettigrew-Robinson heard, Mary woke up Freddy and then the Pettigrew-Robinsons—and then what?«

»She locked herself into her room.«

»Yes. And shortly afterwards she came down and joined the others in the conservatory, and it was at this point everybody remembered noticing that she was wearing a cap and coat and walking shoes over pyjamas and bare feet.«

»You are suggesting,« said Parker, »that Lady Mary was already awake and dressed at three o'clock, that she went out by the conservatory door with her suit-case, expecting to meet the—the murderer of her—damn it, Wimsey!«

»We needn't go so far as that,« said Peter; \enquote{we decided that she \textit{didn't} expect to find Cathcart dead.}

»No. Well, she went, presumably to meet somebody.«

\enquote{Shall we say, \textit{pro tem.}, she went to meet № 10?} suggested Wimsey softly.

»I suppose we may as well say so. When she turned on the torch and saw the Duke stooping over Cathcart she thought—by Jove, Wimsey, I was right after all! When she said, »You've killed him!« she meant No.  10—she thought it was № 10's body.«

»Of course!« cried Wimsey. »I'm a fool! Yes. Then she said, »It's Denis—what has happened?« That's quite clear. And, meanwhile, what did she do with the suit-case?«

»I see it all now,« cried Parker. \enquote{When she saw that the body wasn't the body of № 10 she realized that № 10 must be the murderer. So her game was to prevent anybody knowing that № 10 had been there.  So she shoved the suit-case behind the cacti. Then, when she went upstairs, she pulled it out again, and hid it in the oak chest on the landing. She couldn't take it to her room, of course, because if anybody'd heard her come upstairs it would seem odd that she should run to her room before calling the others. Then she knocked up Arbuthnot and the Pettigrew-Robinsons—she'd be in the dark, and they'd be flustered and wouldn't see exactly what she had on. Then she escaped from Mrs P., ran into her room, took off the skirt in which she had knelt by Cathcart's side, and the rest of her clothes, and put on her pyjamas and the cap, which someone might have noticed, and the coat, which they \textit{must} have noticed, and the shoes, which had probably left footmarks already. Then she could go down and show herself. Meantime she'd concocted the burglar story for the Coroner's benefit.}

»That's about it,« said Peter. »I suppose she was so desperately anxious to throw us off the scent of № 10 that it never occurred to her that her story was going to help implicate her brother.«

»She realized it at the inquest,« said Parker eagerly. »Don't you remember how hastily she grasped at the suicide theory?«

»And when she found that she was simply saving her—well, № 10—in order to hang her brother, she lost her head, took to her bed, and refused to give any evidence at all. Seems to me there's an extra allowance of fools in my family,« said Peter gloomily.

»Well, what could she have done, poor girl?« asked Parker. He had been growing almost cheerful again. »Anyway, she's cleared\longdash«

»After a fashion,« said Peter, »but we're not out of the wood yet by a long way. Why is she hand-in-glove with № 10 who is at least a blackmailer if not a murderer? How did Gerald's revolver come on the scene? And the green-eyed cat? How much did Mary know of that meeting between № 10 and Denis Cathcart? And if she was seeing and meeting the man she might have put the revolver into his hands any time.«

»No, no,« said Parker. »Wimsey, don't think such ugly things as that.«

»Hell!« cried Peter, exploding. »I'll have the truth of this beastly business if we all go to the gallows together!«

At this moment Bunter entered with a telegram addressed to Wimsey. Lord Peter read as follows:

\begin{quote}

\textsc{Party traced London; seen Marylebone Friday. Further information from Scotland Yard.—Police-Superintendent Gosling, Ripley.}

\end{quote}

»Good egg!« cried Wimsey. »Now we're gettin' down to it. Stay here, there's a good man, in case anything turns up. I'll run round to the Yard now. They'll send you up dinner, and tell Bunter to give you a bottle of the Château Yquem—it's rather decent. So long.«

He leapt out of the flat, and a moment later his taxi buzzed away up Piccadilly.
