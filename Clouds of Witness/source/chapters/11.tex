%!TeX root=../cloudstop.tex

\chapter{Meribah}

\epigraph{Oh-ho, my friend! You are gotten into Lob's pond.}{\textit{Jack the Giant-Killer}}


\lettrine[lines=4]{L}{ord} Peter broke his journey north at York, whither the Duke of Denver had been transferred after the Assizes, owing to the imminent closing-down of Northallerton Gaol. By dint of judicious persuasion, Peter contrived to obtain an interview with his brother. He found him looking ill at ease, and pulled down by the prison atmosphere, but still unquenchably defiant.

»Bad luck, old man,« said Peter, »but you're keepin' your tail up fine. Beastly slow business, all this legal stuff, what? But it gives us time, an' that's all to the good.«

»It's a confounded nuisance,« said his grace. »And I'd like to know what Murbles means. Comes down and tries to bully me—damned impudence! Anybody'd think he suspected me.«

»Look here, Jerry,« said his brother earnestly, »why can't you let up on that alibi of yours? It'd help no end, you know. After all, if a fellow won't say what he's been doin'\longdash«

»It ain't my business to prove anything,« retorted his grace, with dignity. »They've got to show I was there, murderin' the fellow. I'm not bound to say where I was. I'm presumed innocent, aren't I, till they prove me guilty? I call it a disgrace. Here's a murder committed, and they aren't taking the slightest trouble to find the real criminal. I give `em my word of honour, to say nothin' of an oath, that I didn't kill Cathcart—though, mind you, the swine deserved it—but they pay no attention. Meanwhile, the real man's escapin' at his confounded leisure. If I were only free, I'd make a fuss about it.«

»Well, why the devil don't you cut it short, then?« urged Peter. »I don't mean here and now to me«—with a glance at the warder, within earshot—»but to Murbles. Then we could get to work.«

»I wish you'd jolly well keep out of it,« grunted the Duke. »Isn't it all damnable enough for Helen, poor girl, and mother, and everyone, without you makin' it an opportunity to play Sherlock Holmes? I'd have thought you'd have had the decency to keep quiet, for the family's sake. I may be in a damned rotten position, but I ain't makin' a public spectacle of myself, by Jove!«

»Hell!« said Lord Peter, with such vehemence that the wooden-faced warder actually jumped. »It's you that's makin' the spectacle! It need never have started, but for you. Do you think I \textit{like} havin' my brother and sister dragged through the Courts, and reporters swarmin' over the place, and paragraphs and news-bills with your name starin' at me from every corner, and all this ghastly business, endin' up in a great show in the House of Lords, with a lot of people togged up in scarlet and ermine, and all the rest of the damn-fool jiggery-pokery? People are beginnin' to look oddly at me in the Club, and I can jolly well hear `em whisperin' that »Denver's attitude looks jolly fishy, b'gad!« Cut it out, Jerry.«

»Well, we're in for it now,« said his brother, »and thank heaven there are still a few decent fellows left in the peerage who'll know how to take a gentleman's word, even if my own brother can't see beyond his rotten legal evidence.«

As they stared angrily at one another, that mysterious sympathy of the flesh which we call family likeness sprang out from its hiding-place, stamping their totally dissimilar features with an elfish effect of mutual caricature. It was as though each saw himself in a distorting mirror, while the voices might have been one voice with its echo.

»Look here, old chap,« said Peter, recovering himself, »I'm frightfully sorry. I didn't mean to let myself go like that. If you won't say anything, you won't. Anyhow, we're all working like blazes, and we're sure to find the right man before very long.«

»You'd better leave it to the police,« said Denver. »I know you like playin' at detectives, but I do think you might draw the line somewhere.«

»That's a nasty one,« said Wimsey. »But I don't look on this as a game, and I can't say I'll keep out of it, because I know I'm doin' valuable work. Still, I can—honestly, I can—see your point of view. I'm jolly sorry you find me such an irritatin' sort of person. I suppose it's hard for you to believe I feel anything. But I do, and I'm goin' to get you out of this, if Bunter and I both perish in the attempt. Well, so long—that warder's just wakin' up to say, »Time, gentlemen.« Cheer-oh, old thing! Good luck!«

He rejoined Bunter outside.

»Bunter,« he said, as they walked through the streets of the old city, »is my manner \textit{really} offensive, when I don't mean it to be?«

»It is possible, my lord, if your lordship will excuse my saying so, that the liveliness of your lordship's manner may be misleading to persons of limited\longdash«

»Be careful, Bunter!«

»Limited imagination, my lord.«

»Well-bred English people never have imagination, Bunter.«

»Certainly not, my lord. I meant nothing disparaging.«

»Well, Bunter—oh, lord! there's a reporter! Hide me, quick!«

»In here, my lord.«

Mr Bunter whisked his master into the cool emptiness of the Cathedral.

»I venture to suggest, my lord,« he urged in a hurried whisper, »that we adopt the attitude and external appearance of prayer, if your lordship will excuse me.«

Peeping through his fingers, Lord Peter saw a verger hastening towards them, rebuke depicted on his face. At that moment, however, the reporter entered in headlong pursuit, tugging a note-book from his pocket. The verger leapt swiftly on this new prey.

»The winder h'under which we stand,« he began in a reverential monotone, »is called the Seven Sisters of York. They say\longdash«

Master and man stole quietly out.

For his visit to the market town of Stapley Lord Peter attired himself in an aged Norfolk suit, stockings with sober tops, an ancient hat turned down all round, stout shoes, and carried a heavy ashplant. It was with regret that he abandoned his favourite stick—a handsome malacca, marked off in inches for detective convenience, and concealing a sword in its belly and a compass in its head. He decided, however, that it would prejudice the natives against him, as having a town-bred, not to say supercilious, air about it. The sequel to this commendable devotion to his art forcibly illustrated the truth of Gertrude Rhead's observation, »All this self-sacrifice is a sad mistake.«

The little town was sleepy enough as he drove into it in one of the Riddlesdale dog-carts, Bunter beside him, and the under-gardener on the back seat. For choice, he would have come on a market-day, in the hope of meeting Grimethorpe himself, but things were moving fast now, and he dared not lose a day. It was a raw, cold morning, inclined to rain.

»Which is the best inn to put up at, Wilkes?«

»There's t' »Bricklayers' Arms«, my lord—a fine, well-thought of place, or t' »Bridge and Bottle«, i' t' square, or t' »Rose and Crown«, t'other side o' square.«

»Where do the folks usually put up on market-days?«

»Mebbe »Rose and Crown« is most popular, so to say—Tim Watchett, t' landlord, is a rare gossip. Now Greg Smith ower t'way at »Bridge and Bottle«, he's nobbut a grimly, surly man, but he keeps good drink.«

»H'm—I fancy, Bunter, our man will be more attracted by surliness and good drink than by a genial host. The »Bridge and Bottle« for us, I fancy, and, if we draw blank there, we'll toddle over to the »Rose and Crown«, and pump the garrulous Watchett.«

Accordingly they turned into the yard of a large, stony-faced house, whose long-unpainted sign bore the dim outline of a »Bridge Embattled«, which local etymology had (by a natural association of ideas) transmogrified into the »Bridge and Bottle«. To the grumpy ostler who took the horse Peter, with his most companionable manner, addressed himself:

»Nasty raw morning, isn't it?«

»Eea.«

»Give him a good feed. I may be here some time.«

»Ugh!«

»Not many people about today, what?«

»Ugh!«

»But I expect you're busy enough market-days.«

»Eea.«

»People come in from a long way round, I suppose.«

»Co-oop!« said the ostler. The horse walked three steps forward.

»Wo!« said the ostler. The horse stopped, with the shafts free of the tugs; the man lowered the shafts, to grate viciously on the gravel.

»Coom on oop!« said the ostler, and walked calmly off into the stable, leaving the affable Lord Peter as thoroughly snubbed as that young sprig of the nobility had ever found himself.

»I am more and more convinced,« said his lordship, »that this is Farmer Grimethorpe's usual house of call. Let's try the bar. Wilkes, I shan't want you for a bit. Get yourself lunch if necessary. I don't know how long we shall be.«

»Very good, my lord.«

In the bar of the »Bridge and Bottle« they found Mr Greg Smith gloomily checking a long invoice. Lord Peter ordered drinks for Bunter and himself. The landlord appeared to resent this as a liberty, and jerked his head towards the barmaid. It was only right and proper that Bunter, after respectfully returning thanks to his master for his half-pint, should fall into conversation with the girl, while Lord Peter paid his respects to Mr Smith.

»Ah!« said his lordship, »good stuff, that, Mr Smith. I was told to come here for real good beer, and, by Jove! I've been sent to the right place.«

»Ugh!« said Mr Smith, »'tisn't what it was. Nowt's good these times.«

»Well, I don't want better. By the way, is Mr Grimethorpe here today?«

»Eh?«

»Is Mr Grimethorpe in Stapley this morning, d'you know?«

»How'd I know?«

»I thought he always put up here.«

»Ah!«

»Perhaps I mistook the name. But I fancied he'd be the man to go where the best beer is.«

»Ay?«

»Oh, well, if you haven't seen him, I don't suppose he's come over today.«

»Coom where?«

»Into Stapley.«

»Doosn't `e live here? He can go and coom without my knowing.«

»Oh, of course!« Wimsey staggered under the shock, and then grasped the misunderstanding. »I don't mean Mr Grimethorpe of Stapley, but Mr Grimethorpe of Grider's Hole.«

»Why didn't tha say so? Oh, him? Ay.«

»He's here today?«

»Nay, I knaw nowt about `un.«

»He comes in on market-days, I expect.«

»Sometimes.«

»It's a longish way. One can put up for the night, I suppose?«

»Doosta want t'stay t'night?«

»Well, no, I don't think so. I was thinking about my friend Mr Grimethorpe. I daresay he often has to stay the night.«

»Happen a does.«

»Doesn't he stay here, then?«

»Naay.«

»Oh!« said Wimsey, and thought impatiently: »If all these natives are as oyster-like I \textit{shall} have to stay the night\dots« »Well, well,« he added aloud, »next time he drops in say I asked after him.«

»And who mought tha be?« inquired Mr Smith in a hostile manner.

»Oh, only Brooks of Sheffield,« said Lord Peter, with a happy grin. »Good morning. I won't forget to recommend your beer.«

Mr Smith grunted. Lord Peter strolled slowly out, and before long Mr Bunter joined him, coming out with a brisk step and the lingering remains of what, in anyone else, might have been taken for a smirk.

»Well?« inquired his lordship. »I hope the young lady was more communicative than that fellow.«

»I found the young person« (»Snubbed again,« muttered Lord Peter) »perfectly amiable, my lord, but unhappily ill-informed. Mr Grimethorpe is not unknown to her, but he does not stay here. She has sometimes seen him in company with a man called Zedekiah Bone.«

»Well,« said his lordship, »suppose you look for Bone, and come and report progress to me in a couple of hours' time. I'll try the »Rose and Crown«. We'll meet at noon under that thing.«

»That thing,« was a tall erection in pink granite, neatly tooled to represent a craggy rock, and guarded by two petrified infantry-men in trench helmets. A thin stream of water gushed from a bronze knob half-way up, a roll of honour was engraved on the octagonal base, and four gas-lamps on cast-iron standards put the finishing touch to a very monument of incongruity. Mr Bunter looked carefully at it, to be sure of recognizing it again, and moved respectfully away. Lord Peter walked ten brisk steps in the direction of the »Rose and Crown«, then a thought struck him.

»Bunter!«

Mr Bunter hurried back to his side.

»Oh, nothing!« said his lordship. »Only I've just thought of a name for it.«

»For\longdash«

»That memorial,« said Lord Peter. »I choose to call it »Meribah.««

»Yes, my lord. The waters of strife. Exceedingly apt, my lord. Nothing harmonious about it, if I may say so. Will there be anything further, my lord?«

»No, that's all.«

Mr Timothy Watchett of the »Rose and Crown« was certainly a contrast to Mr Greg Smith. He was a small, spare, sharp-eyed man of about fifty-five, with so twinkling and humorous an eye and so alert a cock of the head that Lord Peter summed up his origin the moment he set eyes on him.

»Morning, landlord,« said he genially, »and when did \textit{you} last see Piccadilly Circus?«

»'Ard to say, sir. Gettin' on for thirty-five year, I reckon. Many's the time I said to my wife, »Liz, I'll tike you ter see the `Olborn Empire afore I die.« But, with one thing and another, time slips aw'y. One day's so like another—blowed if I ever remember `ow old I'm gettin', sir.«

»Oh, well, you've lots of time yet,« said Lord Peter.

»I `ope so, sir. I ain't never wot you may call got used ter these Northerners. That slow, they are, sir—it fair giv' me the `ump when I first come. And the w'y they speak—that took some gettin' used to. Call that English, I useter say, give me the Frenchies in the Chantycleer Restaurong, I ses. But there, sir, custom's everything. Blowed if I didn't ketch myself a-syin' `yon side the square' the other day. Me!«

»I don't think there's much fear of your turning into a Yorkshire man,« said Lord Peter, »didn't I know you the minute I set eyes on you? In Mr Watchett's bar I said to myself, `My foot is on my native paving-stones.'«

»That's raight, sir. And, bein' there, sir, what can I `ave the pleasure of offerin' you?\dots Excuse me, sir, but `aven't I seen your fice somewhere?«

»I don't think so,« said Peter; »but that reminds me. Do you know one Mr Grimethorpe?«

»I know five Mr Grimethorpes. W'ich of `em was you meanin', sir?«

»Mr Grimethorpe of Grider's Hole.«

The landlord's cheerful face darkened.

»Friend of yours, sir?«

»Not exactly. An acquaintance.«

»There naow!« cried Mr Watchett, smacking his hand down upon the counter. »I knowed as I knowed your fice! Don't you live over at Riddlesdale, sir?«

»I'm stayin' there.«

»I knowed it,« retorted Mr Watchett triumphantly. He dived behind the counter and brought up a bundle of newspapers, turning over the sheets excitedly with a well-licked thumb. »There! Riddlesdale! That's it, of course.«

He smacked open a \textit{Daily Mirror} of a fortnight or so ago. The front page bore a heavy block headline: \textsc{The Riddlesdale Mystery}. And beneath was a lifelike snapshot entitled, »\textit{Lord Peter Wimsey, the Sherlock Holmes of the West End, who is devoting all his time and energies to proving the innocence of his brother, the Duke of Denver.}« Mr Watchett gloated.

»You won't mind my syin' `ow proud I am to `ave you in my bar, my lord.—'Ere, Jem, you attend ter them gentlemen; don't you see they're wytin'?—Follered all yer caises I `ave, my lord, in the pipers—jest like a book they are. An' ter think\longdash«

»Look here, old thing,« said Lord Peter, »d'you mind not talkin' quite so loud. Seein' dear old Felix is out of the bag, so to speak, do you think you could give me some information and keep your mouth shut, what?«

»Come be'ind into the bar-parlour, my lord. Nobody'll `ear us there,« said Mr Watchett eagerly, lifting up the flap. »Jem, `ere! Bring a bottle of—what'll you `ave, my lord?«

»Well, I don't know how many places I may have to visit,« said his lordship dubiously.

»Jem, bring a quart of the old ale.—It's special, that's wot it is, my lord. I ain't never found none like it, except it might be once at Oxford. Thanks, Jem. Naow you get along sharp and attend to the customers. Now, my lord.«

Mr Watchett's information amounted to this. That Mr Grimethorpe used to come to the »Rose and Crown« pretty often, especially on market-days. About ten days previously he had come in lateish, very drunk and quarrelsome, with his wife, who seemed, as usual, terrified of him. Grimethorpe had demanded spirits, but Mr Watchett had refused to serve him. There had been a row, and Mrs Grimethorpe had endeavoured to get her husband away. Grimethorpe had promptly knocked her down, with epithets reflecting upon her virtue, and Mr Watchett had at once called upon the potmen to turn Grimethorpe out, refusing to have him in the house again. He had heard it said on all sides that Grimethorpe's temper, always notoriously bad, had become positively diabolical of late.

»Could you hazard, so to speak, a calculation as to how long, or since when?«

»Well, my lord, come to think of it, especially since the middle of last month—p'r'aps a bit earlier.«

»M'm!«

»Not that I'd go for to insinuate anythink, nor your lordship, neither, of course,« said Mr Watchett quickly.

»Certainly not,« said Lord Peter. »What about?«

»Ah!« said Mr Watchett, »there it is, wot abaht?«

»Tell me,« said Lord Peter, »do you recollect Grimethorpe comin' into Stapley on October 13\textsuperscript{th}—a Wednesday, it was.«

»That would be the day of the—ah! to be sure! Yes, I do recollect it, for I remember thinking it was odd him comin' here except on a market-day. Said he `ad ter look at some machinery—drills and such, that's raight. `E was `ere raight enough.«

»Do you remember what time he came in?«

»Well, naow, I've a fancy `e was `ere ter lunch. The waitress'd know. `Ere, Bet!« he called through the side door, »d'yer `appen to recollect whether Mr Grimethorpe lunched `ere October the 13\textsuperscript{th}—Wednesday it were, the d'y the pore gent was murdered over at Riddlesdale?«

»Grimethorpe o' Grider's Hole?« said the girl, a well-grown young Yorkshire woman. »Yes! `E took loonch, and coom back to sleep. Ah'm not mistook, for ah waited on `un, an' took up `is watter i' t'morning, and `e only gied me tuppence.«

»Monstrous!« said Lord Peter. »Look here, Miss Elizabeth, you're sure it was the thirteenth? Because I've got a bet on it with a friend, and I don't want to lose the money if I can help it. You're positive it was Wednesday night he slept here? I could have sworn it was Thursday.«

»Naay, sir, t'wor Wednesday for I remember hearing the men talking o' t'murder i' t'bar, an' telling Mester Grimethorpe next daay.«

»Sounds conclusive. What did Mr Grimethorpe say about it?«

»There now,« cried the young woman, »'tis queer you should ask that; everyone noticed how strange he acted. He turned all white like a sheet, and looked at both his hands, one after the other, and then he pushes `es hair off `s forehead—dazed-like. We reckoned he hadn't got over the drink. He's more often drunk than not. Ah wouldn't be his wife for five hundred pounds.«

»I should think not,« said Peter; »you can do a lot better than that. Well, I suppose I've lost my money, then. By the way, what time did Mr Grimethorpe come in to bed?«

»Close on two i' t'morning,« said the girl, tossing her head. »He were locked oot, an' Jem had to go down and let `un in.«

»That so?« said Peter. »Well, I might try to get out on a technicality, eh, Mr Watchett? Two o'clock is Thursday, isn't it? I'll work that for all it's worth. Thanks frightfully. That's all I want to know.«

Bet grinned and giggled herself away, comparing the generosity of the strange gentleman with the stinginess of Mr Grimethorpe. Peter rose.

»I'm no end obliged, Mr Watchett,« he said. »I'll just have a word with Jem. Don't say anything, by the way.«

»Not me,« said Mr Watchett; »I knows wot's wot. Good luck, my lord.«

Jem corroborated Bet. Grimethorpe had returned at about 1:50 \textsc{a.m.} on October 14\textsuperscript{th}, drunk, and plastered with mud. He had muttered something about having run up against a man called Watson.

The ostler was next interrogated. He did not think that anybody could get a horse and trap out of the stable at night without his knowing it. He knew Watson. He was a carrier by trade, and lived in Windon Street. Lord Peter rewarded his informant suitably, and set out for Windon Street.

But the recital of his quest would be tedious. At a quarter past noon he joined Bunter at the Meribah memorial.

»Any luck?«

»I have secured certain information, my lord, which I have duly noted. Total expenditure on beer for self and witnesses 7\textit{s}. 2\textit{d}., my lord.«

Lord Peter paid the 7\textit{s}. 2\textit{d}. without a word, and they adjourned to the »Rose and Crown«. Being accommodated in a private parlour, and having ordered lunch, they proceeded to draw up the following schedule:

\makeatletter
\@ifclasswith{scrbook}{a5paper}
{%
\clearpage
}{%

}
\makeatother

\begin{center}
\textsc{Grimethorpe's Movements.}\\Wednesday, October 13\textsuperscript{th} \textit{to} Thursday, October 14\textsuperscript{th}.
\end{center}

\begin{samepage}
\textbf{October 13\textsuperscript{th}:}	\\
\begin{tabular} {r l p{0.75\linewidth} } 
12:30&\textsc{p.m.}&Arrives »Rose and Crown«.\\
1:00&\textsc{p.m.}&Lunches.\\
3:00&\textsc{p.m.}&Orders two drills from man called Gooch in Trimmer's Lane.\\
4:30&\textsc{p.m.}&Drink with Gooch to clinch bargain.\\
5:00&\textsc{p.m.}&Calls at house of John Watson, carrier, about delivering some dog-food. Watson absent. Mrs Watson says W. expected back that night. G. says will call again.\\
5:30&\textsc{p.m.}&Calls on Mark Dolby, grocer, to complain about some tinned salmon.\\
5:45&\textsc{p.m.}&Calls on Mr Hewitt, optician, to pay bill for spectacles and dispute the amount.\\
6:00&\textsc{p.m.}&Drinks with Zedekiah Bone at »Bridge and Bottle«.\\
6:45&\textsc{p.m.}&Calls again on Mrs Watson. Watson not yet home.\\
7:00&\textsc{p.m.}&Seen by Constable Z15 drinking with several men at »Pig and Whistle.« Heard to use threatening language with regard to some person unknown.\\
7:20&\textsc{p.m.}&Seen to leave »Pig and Whistle« with two men (not yet identified).\\
\end{tabular}
\end{samepage}

\begin{samepage}

\textbf{October 14\textsuperscript{th}:}	\\
\begin{tabular} {r l p{0.75\linewidth} } 
1:15&\textsc{a.m.}&Picked up by Watson, carrier, about a mile out on road to Riddlesdale, very dirty and ill-tempered, and not quite sober.\\
1:45&\textsc{a.m.}&Let into »Rose and Crown« by James Johnson, potman.\\
9:00&\textsc{a.m.}&Called by Elizabeth Dobbin.\\
9:30&\textsc{a.m.}&In Bar of »Rose and Crown«. Hears of man murdered at Riddlesdale. Behaves suspiciously.\\
10:15&\textsc{a.m.}&Cashes check \textsterling 129 17\textit{s}. 8\textit{d}. at Lloyds Bank.\\
10:30&\textsc{a.m.}&Pays Gooch for drills.\\
11:50&\textsc{a.m.}&Leaves »Rose and Crown« for Grider's Hole.\\
\end{tabular}
\end{samepage}

Lord Peter looked at this for a few minutes, and put his finger on the great gap of six hours after 7:20.

»How far to Riddlesdale, Bunter?«

»About thirteen and three-quarter miles, my lord.«

»And the shot was heard at 10:55. It couldn't be done on foot. Did Watson explain why he didn't get back from his round till two in the morning?«

»Yes, my lord. He says he reckons to be back about eleven, but his horse cast a shoe between King's Fenton and Riddlesdale. He had to walk him quietly into Riddlesdale—about 3½ miles—getting there about ten, and knock up the blacksmith. He turned in to the »Lord in Glory« till closing time, and then went home with a friend and had a few more. At 12:40 he started off home, and picked Grimethorpe up a mile or so out, near the cross roads.«

»Sounds circumstantial. The blacksmith and the friend ought to be able to substantiate it. But we simply must find those men at the »Pig and Whistle«.«

»Yes, my lord. I will try again after lunch.«

It was a good lunch. But that seemed to exhaust their luck for the day, for by three o'clock the men had not been identified, and the scent seemed cold.

Wilkes, the groom, however, had his own contribution to the inquiry. He had met a man from King's Fenton at lunch, and they had, naturally, got to talking over the mysterious murder at the Lodge, and the man had said that he knew an old man living in a hut on the Fell, who said that on the night of the murder he'd seen a man walking over Whemmeling Fell in the middle of the night. »And it coom to me, all of a sooden, it mought be his grace,« said Wilkes brightly.

Further inquiries elicited that the old man's name was Groot, and that Wilkes could easily drop Lord Peter and Bunter at the beginning of the sheep-path which led up to his hut.

Now, had Lord Peter taken his brother's advice, and paid more attention to English country sports than to incunabula and criminals in London—or had Bunter been brought up on the moors, rather than in a Kentish village—or had Wilkes (who was a Yorkshire man bred and born, and ought to have known better) not been so outrageously puffed up with the sense of his own importance in suggesting a clue, and with impatience to have that clue followed up without delay—or had any one of the three exercised common sense—this preposterous suggestion would never have been made, much less carried out, on a November day in the North Riding. As it was, however, Lord Peter and Bunter left the trap at the foot of the moor-path at ten minutes to four, and, dismissing Wilkes, climbed steadily up to the wee hut on the edge of the Fell.

The old man was extremely deaf, and, after half an hour of interrogation, his story did not amount to much. On a night in October, which he thought might be the night of the murder, he had been sitting by his peat fire when—about midnight, as he guessed—a tall man had loomed up out of the darkness. He spoke like a Southerner, and said he had got lost on the moor. Old Groot had come to his door and pointed out the track down towards Riddlesdale. The stranger had then vanished, leaving a shilling in his hand. He could not describe the stranger's dress more particularly than that he wore a soft hat and an overcoat, and, he thought, leggings. He was pretty near sure it was the night of the murder, because afterwards he had turned it over in his mind and made out that it might have been one of yon folk at the Lodge—possibly the Duke. He had only arrived at this result by a slow process of thought, and had not »come forward,« not knowing whom or where to come to.

With this the inquirers had to be content, and, presenting Groot with half a crown, they emerged upon the moor at something after five o'clock.

»Bunter,« said Lord Peter through the dusk, »I am abso-bally-lutely positive that the answer to all this business is at Grider's Hole.«

»Very possibly, my lord.«

Lord Peter extended his finger in a south-easterly direction. »That is Grider's Hole,« he said. »Let's go.«

»Very good, my lord.«

So, like two Cockney innocents, Lord Peter and Bunter set forth at a brisk pace down the narrow moor-track towards Grider's Hole, with never a glance behind them for the great white menace rolling silently down through the November dusk from the wide loneliness of Whemmeling Fell.

»Bunter!«

»Here, my lord!«

The voice was close at his ear.

»Thank God! I thought you'd disappeared for good. I say, we ought to have known.«

»Yes, my lord.«

It had come on them from behind, in a single stride, thick, cold, choking—blotting each from the other, though they were only a yard or two apart.

»I'm a fool, Bunter,« said Lord Peter.

»Not at all, my lord.«

»Don't move; go on speaking.«

»Yes, my lord.«

Peter groped to the right and clutched the other's sleeve.

»Ah! Now what are we to do?«

»I couldn't say, my lord, having no experience. Has the—er—\newline phenomenon any habits, my lord?«

»No regular habits, I believe. Sometimes it moves. Other times it stays in one place for days. We can wait all night, and see if it lifts at daybreak.«

»Yes, my lord. It is unhappily somewhat damp.«

»Somewhat—as you say,« agreed his lordship, with a short laugh.

Bunter sneezed, and begged pardon politely.

»If we go on going south-east,« said his lordship, »we shall get to Grider's Hole all right, and they'll jolly well \textit{have} to put us up for the night—or give us an escort. I've got my torch in my pocket, and we can go by compass—oh, hell!«

»My lord?«

»I've got the wrong stick. This beastly ash! No compass, Bunter—we're done in.«

»Couldn't we keep on going downhill, my lord?«

Lord Peter hesitated. Recollections of what he had heard and read surged up in his mind to tell him that uphill or downhill seems much the same thing in a fog. But man walks in a vain shadow. It is hard to believe that one is really helpless. The cold was icy. »We might try,« he said weakly.

»I have heard it said, my lord, that in a fog one always walked round in a circle,« said Mr Bunter, seized with a tardy diffidence.

»Not on a slope, surely,« said Lord Peter, beginning to feel bold out of sheer contrariness.

Bunter, being out of his element, had, for once, no good counsel to offer.

»Well, we can't be much worse off than we are,« said Lord Peter. »We'll try it, and keep on shouting.«

He grasped Bunter's hand, and they strode gingerly forward into the thick coldness of the fog.

How long that nightmare lasted neither could have said. The world might have died about them. Their own shouts terrified them; when they stopped shouting the dead silence was more terrifying still. They stumbled over tufts of thick heather. It was amazing how, deprived of sight, they exaggerated the inequalities of the ground. It was with very little confidence that they could distinguish uphill from downhill. They were shrammed through with cold, yet the sweat was running from their faces with strain and terror.

Suddenly—from directly before them as it seemed, and only a few yards away—there rose a long, horrible shriek—and another—and another.

»My God! What's that?«

»It's a horse, my lord.«

»Of course.« They remembered having heard horses scream like that. There had been a burning stable near Poperinghe—

»Poor devil,« said Peter. He started off impulsively in the direction of the sound, dropping Bunter's hand.

»Come back, my lord,« cried the man in a sudden agony. And then, with a frightened burst of enlightenment:

»For God's sake stop, my lord—the bog!«

A sharp shout in the utter blackness.

»Keep away there—don't move—it's got me!«

And a dreadful sucking noise.

