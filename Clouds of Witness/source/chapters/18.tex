%!TeX root=../cloudstop.tex


\chapter{The Speech for the Defence}

\epigraph{Nobody; I myself; farewell.}{\textit{Othello}}



\lettrine[lines=4]{A}{fter} the reading of Cathcart's letter even the appearance of the prisoner in the witness-box came as an anti-climax. In the face of the Attorney-General's cross-examination he maintained stoutly that he had wandered on the moor for several hours without meeting anybody, though he was forced to admit that he had gone downstairs at 11:30, and not at 2:30, as he had stated at the inquest. Sir Wigmore Wrinching made a great point of this, and, in a spirited endeavour to suggest that Cathcart was blackmailing Denver, pressed his questions so hard that Sir Impey Biggs, Mr Murbles, Lady Mary, and Bunter had a nervous feeling that learned counsel's eyes were boring through the walls to the side-room where, apart from the other witnesses, Mrs Grimethorpe sat waiting. After lunch Sir Impey Biggs rose to make his plea for the defence.

\begin{dialogue}
\speak{Counsel} My lords,—Your lordships have now heard—and I, who have watched and pleaded here for these three anxious days, know with what eager interest and with what ready sympathy you have heard—the evidence brought by my noble client to defend him against this dreadful charge of murder. You have listened while as it were from his narrow grave, the dead man has lifted his voice to tell you the story of that fatal night of the thirteenth of October, and I feel sure you can have no doubt in your hearts that that story is the true one. As your lordships know, I was myself totally ignorant of the contents of that letter until I heard it read in Court just now, and, by the profound impression it made upon my own mind, I can judge how tremendously and how painfully it must have affected your lordships. In my long experience at the criminal bar, I think I have never met with a history more melancholy than that of the unhappy young man whom a fatal passion—for here indeed we may use that well-worn expression in all the fullness of its significance—whom a truly fatal passion thus urged into deep after deep of degradation, and finally to a violent death by his own hand.

\smallskip 

The noble peer at the Bar has been indicted before your lordships of the murder of this young man. That he is wholly innocent of the charge must, in the light of what we have heard, be so plain to your lordships that any words from me might seem altogether superfluous. In the majority of cases of this kind the evidence is confused, contradictory; here, however, the course of events is so clear, so coherent, that had we ourselves been present to see the drama unrolled before us, as before the all-seeing eye of God, we could hardly have a more vivid or a more accurate vision of that night's adventures. Indeed, had the death of Denis Cathcart been the sole event of the night, I will venture to say that the truth could never have been one single moment in doubt. Since, however, by a series of unheard-of coincidences, the threads of Denis Cathcart's story became entangled with so many others, I will venture to tell it once again from the beginning, lest, in the confusion of so great a cloud of witnesses, any point should still remain obscure.

\smallskip 

Let me, then, go back to the beginning. You have heard how Denis Cathcart was born of mixed parentage—from the union of a young and lovely southern girl with an Englishman twenty years older than herself: imperious, passionate, and cynical. Till the age of 18 he lives on the Continent with his parents, travelling from place to place, seeing more of the world even than the average young Frenchman of his age, learning the code of love in a country where the \textit{crime passionel} is understood and forgiven as it never can be over here.

\smallskip 

At the age of 18 a terrible loss befalls him. In a very short space of time he loses both his parents—his beautiful and adored mother and his father, who might, had he lived, have understood how to guide the impetuous nature which he had brought into the world. But the father dies, expressing two last wishes, both of which, natural as they were, turned out in the circumstances to be disastrously ill-advised. He left his son to the care of his sister, whom he had not seen for many years, with the direction that the boy should be sent to his own old University.

\smallskip 

My lords, you have seen Miss Lydia Cathcart, and heard her evidence. You will have realized how uprightly, how conscientiously, with what Christian disregard of self, she performed the duty entrusted to her, and yet how inevitably she failed to establish any real sympathy between herself and her young ward. He, poor lad, missing his parents at every turn, was plunged at Cambridge into the society of young men of totally different upbringing from himself. To a young man of his cosmopolitan experience the youth of Cambridge, with its sports and rags and naïve excursions into philosophy o' nights, must have seemed unbelievably childish. You all, from your own recollections of your Alma Mater, can reconstruct Denis Cathcart's life at Cambridge, its outward gaiety, its inner emptiness.

\smallskip 

Ambitious of embracing a diplomatic career, Cathcart made extensive acquaintances among the sons of rich and influential men. From a worldly point of view he was doing well, and his inheritance of a handsome fortune at the age of twenty-one seemed to open up the path to very great success. Shaking the academic dust of Cambridge from his feet as soon as his Tripos was passed, he went over to France, established himself in Paris, and began, in a quiet, determined kind of way, to carve out a little niche for himself in the world of international politics.

\smallskip 

But now comes into his life that terrible influence which was to rob him of fortune, honour, and life itself. He falls in love with a young woman of that exquisite, irresistible charm and beauty for which the Austrian capital is world-famous. He is enthralled body and soul, as utterly as any Chevalier des Grieux, by Simone Vonderaa.

\smallskip 

Mark that in this matter he follows the strict, continental code: complete devotion, complete discretion. You have heard how quietly he lived, how \textit{rangé} he appeared to be. We have had in evidence his discreet banking-account, with its generous checks drawn to self, and cashed in notes of moderate denominations, and with its regular accumulation of sufficient »economies« quarter by quarter. Life has expanded for Denis Cathcart. Rich, ambitious, possessed of a beautiful and complaisant mistress, the world is open before him.

\smallskip 

Then, my lords, across this promising career there falls the thunderbolt of the Great War—ruthlessly smashing through his safeguards, overthrowing the edifice of his ambition, destroying and devastating here, as everywhere, all that made life beautiful and desirable.

\smallskip 

You have heard the story of Denis Cathcart's distinguished army career. On that I need not dwell. Like thousands of other young men, he went gallantly through those five years of strain and disillusionment, to find himself left, in the end, with his life and health indeed, and, so far, happy beyond many of his comrades, but with his life in ruins about him.

\smallskip 

Of his great fortune—all of which had been invested in Russian and German securities—literally nothing is left to him. What, you say, did that matter to a young man so well equipped, with such excellent connections, with so many favourable openings, ready to his hand? He needed only to wait quietly for a few years, to reconstruct much of what he had lost. Alas! my lords, he could not afford to wait. He stood in peril of losing something dearer to him than fortune or ambition; he needed money in quantity, and at once.

\smallskip 

My lords, in that pathetic letter which we have heard read nothing is more touching and terrible than that confession: »I knew you could not but be unfaithful to me.« All through that time of seeming happiness he knew—none better—that his house was built on sand. »I was never deceived by you,« he says. From their earliest acquaintance she had lied to him, and he knew it, and that knowledge was yet powerless to loosen the bands of his fatal fascination. If any of you, my lords, have known the power of love exercised in this irresistible—I may say, this predestined manner—let your experience interpret the situation to you better than any poor words of mine can do. One great French poet and one great English poet have summed the matter up in a few words. Racine says of such a fascination:

\begin{quote}C'est Vénus tout entière à sa proie attachée.\end{quote}

\smallskip 

And Shakespeare has put the lover's despairing obstinacy into two piteous lines:

\begin{verse}
If my love swears that she is made of truth\\
I will believe her, though I know she lies.\\
\end{verse}

\smallskip 

My lords, Denis Cathcart is dead; it is not our place to condemn him, but only to understand and pity him.

\smallskip 

My lords, I need not put before you in detail the shocking shifts to which this soldier and gentleman unhappily condescended. You have heard the story in all its cold, ugly details upon the lips of Monsieur du Bois-Gobey Houdin, and, accompanied by unavailing expressions of shame and remorse, in the last words of the deceased. You know how he gambled, at first honestly—then dishonestly. You know from whence he derived those large sums of money which came at irregular intervals, mysteriously and in cash, to bolster up a bank-account always perilously on the verge of depletion. We need not, my lords, judge too harshly of the woman. According to her own lights, she did not treat him unfairly. She had her interests to consider. While he could pay for her she could give him beauty and passion and good humour and a moderate faithfulness. When he could pay no longer she would find it only reasonable to take another position. This Cathcart understood. Money he must have, by hook or by crook. And so, by an inevitable descent, he found himself reduced to the final deep of dishonour.

\smallskip 

It is at this point, my lords, that Denis Cathcart and his miserable fortunes come into the life of my noble client and of his sister. From this point begin all those complications which led to the tragedy of October 14\textsuperscript{th}, and which we are met in this solemn and historic assembly to unravel.

\smallskip 

About eighteen months ago Cathcart, desperately searching for a secure source of income, met the Duke of Denver, whose father had been a friend of Cathcart's father many years before. The acquaintance prospered, and Cathcart was introduced to Lady Mary Wimsey, at that time (as she has very frankly told us) »at a loose end,« »fed up,« and distressed by the dismissal of her fiancé, Mr Goyles. Lady Mary felt the need of an establishment of her own, and accepted Denis Cathcart, with the proviso that she should be considered a free agent, living her own life in her own way, with the minimum of interference. As to Cathcart's object in all this, we have his own bitter comment, on which no words of mine could improve: »I actually brought myself to consider keeping my mistress on my wife's money.«

\smallskip 

So matters go on until October of this year. Cathcart is now obliged to pass a good deal of his time in England with his fiancée, leaving Simone Vonderaa unguarded in the Avenue Kléber. He seems to have felt fairly secure so far; the only drawback was that Lady Mary, with a natural reluctance to commit herself to the hands of a man she could not really love, had so far avoided fixing a definite date for the wedding. Money is shorter than it used to be in the Avenue Kléber, and the cost of robes and millinery, amusements and so forth, has not diminished. And, meanwhile, Mr Cornelius van Humperdinck, the American millionaire, has seen Simone in the Bois, at the races, at the opera, in Denis Cathcart's flat.

\smallskip 

But Lady Mary is becoming more and more uneasy about her engagement. And at this critical moment, Mr Goyles suddenly sees the prospect of a position, modest but assured, which will enable him to maintain a wife. Lady Mary makes her choice. She consents to elope with Mr Goyles, and by an extraordinary fatality the day and hour selected are 3 \textsc{a.m.} on the morning of October 14\textsuperscript{th}.

\smallskip 

At about 9:30 on the night of Wednesday, October 13\textsuperscript{th}, the party at Riddlesdale Lodge are just separating to go to bed. The Duke of Denver was in the gun-room, the other men were in the billiard-room, the ladies had already retired, when the manservant, Fleming, came up from the village with the evening post. To the Duke of Denver he brought a letter with news of a startling and very unpleasant kind. To Denis Cathcart he brought another letter—one which we shall never see, but whose contents it is easy enough to guess.

\smallskip 

You have heard the evidence of Mr Arbuthnot that, before reading this letter, Cathcart had gone upstairs gay and hopeful, mentioning that he hoped soon to get a date fixed for the marriage. At a little after ten, when the Duke of Denver went up to see him, there was a great change. Before his grace could broach the matter in hand Cathcart spoke rudely and harshly, appearing to be all on edge, and entreating to be left alone. Is it very difficult, my lords, in the face of what we have heard today—in the face of our knowledge that Mademoiselle Vonderaa crossed to New York on the \textit{Berengaria} on October 15\textsuperscript{th}—to guess what news had reached Denis Cathcart in that interval to change his whole outlook upon life?

\smallskip 

At this unhappy moment, when Cathcart is brought face to face with the stupefying knowledge that his mistress has left him, comes the Duke of Denver with a frightful accusation. He taxes Cathcart with the vile truth—that this man, who has eaten his bread and sheltered under his roof, and who is about to marry his sister, is nothing more nor less than a card-sharper. And when Cathcart refuses to deny the charge—when he, most insolently, as it seems, declares that he is no longer willing to wed the noble lady to whom he is affianced—is it surprising that the Duke should turn upon the impostor and forbid him ever to touch or speak to Lady Mary Wimsey again? I say, my lords, that no man with a spark of honourable feeling would have done otherwise. My client contents himself with directing Cathcart to leave the house next day; and when Cathcart rushes madly out into the storm he calls after him to return, and even takes the trouble to direct the footman to leave open the conservatory door for Cathcart's convenience. It is true that he called Cathcart a dirty scoundrel, and told him he should have been kicked out of his regiment, but he was justified; while the words he shouted from the window—»Come back, you fool,« or even, according to one witness, »you b— fool«—have almost an affectionate ring in them. \direct{Laughter.}

\smallskip 

And now I will direct your lordships' attention to the extreme weakness of the case against my noble client from the point of view of motive. It has been suggested that the cause of the quarrel between them was not that mentioned by the Duke of Denver in his evidence, but something even more closely personal to themselves. Of this contention not a jot or tittle, not the slightest shadow of evidence, has been put forward except, indeed, that of the extraordinary witness, Robinson, who appears to bear a grudge against his whole acquaintance, and to have magnified some trifling allusion into a matter of vast importance. Your lordships have seen this person's demeanour in the box, and will judge for yourselves how much weight is to be attached to his observations. While we on our side have been able to show that the alleged cause of complaint was perfectly well founded in fact.

\smallskip 

So Cathcart rushes out into the garden. In the pelting rain he paces heedlessly about, envisaging a future stricken at once suddenly barren of love, wealth, and honour.

\smallskip 

And, meanwhile, a passage door opens, and a stealthy foot creeps down the stair. We know now whose it is—Mrs Pettigrew-Robinson has not mistaken the creak of the door. It is the Duke of Denver.

\smallskip 

That is admitted. But from this point we join issue with my learned friend for the prosecution. It is suggested that the Duke, on thinking matters over, determines that Cathcart is a danger to society and better dead—or that his insult to the Denver family can only be washed out in blood. And we are invited to believe that the Duke creeps downstairs, fetches his revolver from the study table, and prowls out into the night to find Cathcart and make away with him in cold blood.

\smallskip 

My lords, is it necessary for me to point out the inherent absurdity of this suggestion? What conceivable reason could the Duke of Denver have for killing, in this cold-blooded manner, a man of whom a single word has rid him already and for ever? It has been suggested to you that the injury had grown greater in the Duke's mind by brooding—had assumed gigantic proportions. Of that suggestion, my lords, I can only say that a more flimsy pretext for fixing an impulse to murder upon the shoulders of an innocent man was never devised, even by the ingenuity of an advocate. I will not waste my time or insult you by arguing about it. Again it has been suggested that the cause of quarrel was not what it appeared, and the Duke had reason to fear some disastrous action on Cathcart's part. Of this contention I think we have already disposed; it is an assumption constructed \textit{in vacuo}, to meet a set of circumstances which my learned friend is at a loss to explain in conformity with the known facts. The very number and variety of motives suggested by the prosecution is proof that they are aware of the weakness of their own case. Frantically they cast about for any sort of explanation to give colour to this unreasonable indictment.

\smallskip 

And here I will direct your lordships' attention to the very important evidence of Inspector Parker in the matter of the study window. He has told you that it was forced from outside by the latch being slipped back with a knife. If it was the Duke of Denver, who was in the study at 11:30, what need had he to force the window? He was already inside the house. When, in addition, we find that Cathcart had in his pocket a knife, and that there are scratches upon the blade such as might come from forcing back a metal catch, it surely becomes evident that not the Duke, but Cathcart himself forced the window and crept in for the pistol, not knowing that the conservatory door had been left open for him.

\smallskip 

But there is no need to labour this point—we \textit{know} that Captain Cathcart was in the study at that time, for we have seen in evidence the sheet of blotting-paper on which he blotted his letter to Simone Vonderaa, and Lord Peter Wimsey has told us how he himself removed that sheet from the study blotting-pad a few days after Cathcart's death.

\smallskip 

And let me here draw your attention to the significance of one point in the evidence. The Duke of Denver has told us that he saw the revolver in his drawer a short time before the fatal 13\textsuperscript{th}, when he and Cathcart were together.


\speak{The Lord High Steward} One moment, Sir Impey, that is not quite as I have it in my notes.

\speak{Counsel} I beg your lordship's pardon if I am wrong.

\speak{L.H.S.} I will read what I have. »I was hunting for an old photograph of Mary to give Cathcart, and that was how I came across it.« There is nothing about Cathcart being there.

\speak{Counsel} If your lordship will read the next sentence\longdash

\speak{L.H.S.} Certainly. The next sentence is: »I remember saying at the time how rusty it was getting.«

\speak{Counsel} And the next?

\speak{L.H.S.} »To whom did you make that observation?« Answer: »I really don't know, but I distinctly remember saying it.«

\speak{Counsel} I am much obliged to your lordship. When the noble peer made that remark he was looking out some photographs to give to Captain Cathcart. I think we may reasonably infer that the remark was made to the deceased.

\speak{L.H.S.} \direct{to the House} My lords, your lordships will, of course, use your own judgment as to the value of this suggestion.

\speak{Counsel} If your lordships can accept that Denis Cathcart may have known of the existence of the revolver, it is immaterial at what exact moment he saw it. As you have heard, the table-drawer was always left with the key in it. He might have seen it himself at any time, when searching for an envelope or sealing-wax or what not. In any case, I contend that the movements heard by Colonel and Mrs Marchbanks on Wednesday night were those of Denis Cathcart. While he was writing his farewell letter, perhaps with the pistol before him on the table—yes, at that very moment the Duke of Denver slipped down the stairs and out through the conservatory door. Here is the incredible part of this affair—that again and again we find two series of events, wholly unconnected between themselves, converging upon the same point of time, and causing endless confusion. I have used the word »incredible«—not because any coincidence is incredible, for we see more remarkable examples every day of our lives than any writer of fiction would dare to invent—but merely in order to take it out of the mouth of the learned Attorney-General, who is preparing to make it return, boomerang-fashion, against me. \direct{Laughter.}

\smallskip 

My lords, this is the first of these incredible—I am not afraid of the word—coincidences. At 11:30 the Duke goes downstairs and Cathcart enters the study. The learned Attorney-General, in his cross-examination of my noble client, very justifiably made what capital he could out of the discrepancy between witness's statement at the inquest—which was that he did not leave the house till 2:30—and his present statement—that he left it at half-past eleven. My lords, whatever interpretation you like to place upon the motives of the noble Duke in so doing, I must remind you once more that at the time when that first statement was made everybody supposed that the shot had been fired at three o'clock, and that the misstatement was then useless for the purpose of establishing an alibi.

\smallskip

Great stress, too, has been laid on the noble Duke's inability to establish this alibi for the hours from 11:30 to 3 \textsc{a.m.} But, my lords, if he is telling the truth in saying that he walked all that time upon the moors without meeting anyone, what alibi could he establish? He is not bound to supply a motive for all his minor actions during the twenty-four hours. No rebutting evidence has been brought to discredit his story. And it is perfectly reasonable that, unable to sleep after the scene with Cathcart, he should go for a walk to calm himself down.

\smallskip

Meanwhile, Cathcart has finished his letter and tossed it into the post-bag. There is nothing more ironical in the whole of this case than that letter. While the body of a murdered man lay stark upon the threshold, and detectives and doctors searched everywhere for clues, the normal routine of an ordinary English household went, unquestioned, on. That letter, which contained the whole story, lay undisturbed in the post-bag, till it was taken away and put in the post as a matter of course, to be fetched back again, at enormous cost, delay, and risk of life, two months later in vindication of the great English motto: »Business as usual.«

\smallskip

Upstairs, Lady Mary Wimsey was packing her suit-case and writing a farewell letter to her people. At length Cathcart signs his name; he takes up the revolver and hurries out into the shrubbery. Still he paces up and down, with what thoughts God alone knows—reviewing the past, no doubt, racked with vain remorse, most of all, bitter against the woman who has ruined him. He bethinks him of the little love-token, the platinum-and-diamond cat which his mistress gave him for good luck! At any rate, he will not die with that pressing upon his heart. With a furious gesture he hurls it far from him. He puts the pistol to his head.

\smallskip

But something arrests him. Not that! Not that! He sees in fancy his own hideously disfigured corpse—the shattered jaw—the burst eyeball—blood and brains horribly splashed about. No. Let the bullet go cleanly to the heart. Not even in death can he bear the thought of looking—\textit{so}!

\smallskip

He places the revolver against his breast and draws the trigger. With a little moan, he drops to the sodden ground. The weapon falls from his hand; his fingers scrabble a little at his breast.

\smallskip

The gamekeeper who heard the shot is puzzled that poachers should come so close. Why are they not on the moors? He thinks of the hares in the plantation. He takes his lantern and searches in the thick drizzle. Nothing. Only soggy grass and dripping trees. He is human. He concludes his ears deceived him, and he returns to his warm bed. Midnight passes. One o'clock passes.

\smallskip

The rain is less heavy now. Look! In the shrubbery—what was that? A movement. The shot man is moving—groaning a little—crawling to his feet. Chilled to the bone, weak from loss of blood, shaking with the fever of his wound, he but dimly remembers his purpose. His groping hands go to the wound in his breast. He pulls out a handkerchief and presses it upon the place. He drags himself up, slipping and stumbling. The handkerchief slides to the ground, and lies there beside the revolver among the fallen leaves.

\smallskip

Something in his aching brain tells him to crawl back to the house. He is sick, in pain, hot and cold by turns, and horribly thirsty. There someone will take him in and be kind to him—give him things to drink. Swaying and starting, now falling on hands and knees, now reeling to and fro, he makes that terrible nightmare journey to the house. Now he walks, now he crawls, dragging his heavy limbs after him. At last, the conservatory door! Here there will be help. And water for his fever in the trough by the well. He crawls up to it on hands and knees, and strains to lift himself. It is growing very difficult to breathe—a heavy weight seems to be bursting his chest. He lifts himself—a frightful hiccuping cough catches him—the blood rushes from his mouth. He drops down. It is indeed all over.

\smallskip

Once more the hours pass. Three o'clock, the hour of rendezvous, draws on. Eagerly the young lover leaps the wall and comes hurrying through the shrubbery to greet his bride to be. It is cold and wet, but his happiness gives him no time to think of his surroundings. He passes through the shrubbery without a thought. He reaches the conservatory door, through which in a few moments love and happiness will come to him. And in that moment he stumbles across—the dead body of a man!

\smallskip

Fear possesses him. He hears a distant footstep. With but one idea—escape from this horror of horrors—he dashes into the shrubbery, just as, fatigued perhaps a little, but with a mind soothed by his little expedition, the Duke of Denver comes briskly up the path, to meet the eager bride over the body of her betrothed.

\smallskip

My lords, the rest is clear. Lady Mary Wimsey, forced by a horrible appearance of things into suspecting her lover of murder, undertook—with what courage every man amongst you will realize—to conceal that George Goyles ever was upon the scene. Of this ill-considered action of hers came much mystery and perplexity. Yet, my lords, while chivalry holds its own, not one amongst us will breathe one word of blame against that gallant lady. As the old song says:

\begin{verse}
God send each man at his end\\
Such hawks, such hounds, and such a friend.\\
\end{verse}


I think, my lords, that there is nothing more for me to say. To you I leave the solemn and joyful task of freeing the noble peer, your companion, from this unjust charge. You are but human, my lords, and some among you will have grumbled, some will have mocked on assuming these medieval splendours of scarlet and ermine, so foreign to the taste and habit of a utilitarian age. You know well enough that

\begin{verse}
'Tis not the balm, the sceptre and the ball,\\
The sword, the mace, the crown imperial,\\
The intertissued robe of gold and pearl,\\
The farcèd title, nor the tide of pomp\\
That beats upon the high shores of the world\\
\end{verse}

that can add any dignity to noble blood. And yet, to have beheld, day after day, the head of one of the oldest and noblest houses in England standing here, cut off from your fellowship, stripped of his historic honours, robed only in the justice of his cause—this cannot have failed to move your pity and indignation.

\smallskip

My lords, it is your happy privilege to restore to his grace the Duke of Denver these traditional symbols of his exalted rank. When the clerk of this House shall address to you severally the solemn question: Do you find Gerald, Duke of Denver, Viscount St George, guilty or not guilty of the dreadful crime of murder, every one of you may, with a confidence unmarred by any shadow of doubt, lay his hand upon his heart and say, »Not guilty, upon my honour.«

\end{dialogue}
