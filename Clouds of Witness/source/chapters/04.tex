%!TeX root=../cloudstop.tex

\chapter{—And His Daughter Much-Afraid}

\epigraph{The women also looked pale and wan.}{\textit{The Pilgrim's Progress}}


\lettrine[lines=4]{M}{r} Bunter brought Parker's letter up to Lord Peter in bed on the Wednesday morning. The house was almost deserted, everybody having gone to attend the police-court proceedings at Northallerton. The thing would be purely formal, of course, but it seemed only proper that the family should be fully represented. The Dowager Duchess, indeed, was there—she had promptly hastened to her son's side and was living heroically in furnished lodgings, but the younger Duchess thought her mother-in-law more energetic than dignified. There was no knowing what she might do if left to herself. She might even give an interview to a newspaper reporter. Besides, at these moments of crisis a wife's right place is at her husband's side. Lady Mary was ill, and nothing could be said about that, and if Peter chose to stay smoking cigarettes in his pyjamas while his only brother was undergoing public humiliation, that was only what might be expected. Peter took after his mother. How that eccentric strain had got into the family her grace could not imagine, for the Dowager came of a good Hampshire family; there must have been some foreign blood somewhere. Her own duty was clear, and she would do it.

Lord Peter was awake, and looked rather fagged, as though he had been sleuthing in his sleep. Mr Bunter wrapped him solicitously in a brilliant Oriental robe, and placed the tray on his knees.

»Bunter,« said Lord Peter rather fretfully, »your \textit{café au lait} is the one tolerable incident in this beastly place.«

»Thank you, my lord. Very chilly again this morning, my lord, but not actually raining.«

Lord Peter frowned over his letter.

»Anything in the paper, Bunter?«

»Nothing urgent, my lord. A sale next week at Northbury Hall—Mr  Fleetwhite's library, my lord—a Caxton \textit{Confessio Amantis}\longdash«

»What's the good of tellin' me that when we're stuck up here for God knows how long? I wish to heaven I'd stuck to books and never touched crime. Did you send those specimens up to Lubbock?«

»Yes, my lord,« said Bunter gently. Dr Lubbock was the »analytical gentleman.«

»Must have facts,« said Lord Peter, »facts. When I was a small boy I always hated facts. Thought of `em as nasty, hard things, all knobs.  Uncompromisin'.«

»Yes, my lord. My old mother\longdash«

»Your mother, Bunter? I didn't know you had one. I always imagined you were turned out ready-made, so to speak. `Scuse me. Infernally rude of me. Beg your pardon, I'm sure.«

»Not at all, my lord. My mother lives in Kent, my lord, near Maidstone.  Seventy-five, my lord, and an extremely active woman for her years, if you'll excuse my mentioning it. I was one of seven.«

»That is an invention, Bunter. I know better. You are unique. But I interrupted you. You were goin' to tell me about your mother.«

»She always says, my lord, that facts are like cows. If you look them in the face hard enough they generally run away. She is a very courageous woman, my lord.«

Lord Peter stretched out his hand impulsively, but Mr Bunter was too well trained to see it. He had, indeed, already begun to strop a razor.  Lord Peter suddenly bundled out of bed with a violent jerk and sped
across the landing to the bathroom.

Here he revived sufficiently to lift up his voice in »Come unto these Yellow Sands«. Thence, feeling in a Purcellish mood, he passed to »I Attempt from Love's Fever to Fly«, with such improvement of spirits that, against all custom, he ran several gallons of cold water into the bath and sponged himself vigorously. Wherefore, after a rough toweling, he burst explosively from the bathroom, and caught his shin somewhat violently against the lid of a large oak chest which stood at the head of the staircase—so violently, indeed, that the lid lifted with the shock and shut down with a protesting bang.

Lord Peter stopped to say something expressive and to caress his leg softly with the palm of his hand. Then a thought struck him. He set down his towels, soap, sponge, loofah, bath-brush, and other belongings, and quietly lifted the lid of the chest.

Whether, like the heroine of \textit{Northanger Abbey}, he expected to find anything gruesome inside was not apparent. It is certain that, like her, he beheld nothing more startling than certain sheets and counterpanes neatly folded at the bottom. Unsatisfied, he lifted the top one of these gingerly and inspected it for a few moments in the light of the staircase window. He was just returning it to its place, whistling softly the while, when a little hiss of indrawn breath caused him to look up with a start.

His sister was at his elbow. He had not heard her come, but she stood there in her dressing-gown, her hands clutched together on her breast.  Her blue eyes were dilated till they looked almost black, and her skin seemed nearly the colour of her ash-blond hair. Wimsey stared at her over the sheet he held in his arms, and the terror in her face passed over into his, stamping them suddenly with the mysterious likeness of blood-relationship.

Peter's own impression was that he stared »like a stuck pig« for about a minute. He knew, as a matter of fact, that he had recovered himself in a fraction of a second. He dropped the sheet into the chest and stood up.

»Hullo, Polly, old thing,« he said, »where've you been hidin' all this time? First time I've seen you. `Fraid you've been havin' a pretty thin time of it.«

He put his arm round her, and felt her shrink.

»What's the matter?« he demanded. »What's up, old girl? Look here, Mary, we've never seen enough of each other, but I am your brother. Are you in trouble? Can't I\longdash«

»Trouble?« she said. »Why, you silly old Peter, of course I'm in trouble. Don't you know they've killed my man and put my brother in prison? Isn't that enough to be in trouble about?« She laughed, and Peter suddenly thought, »She's talking like somebody in a blood-and-thunder novel.« She went on more naturally. »It's all right, Peter, truly—only my head's so bad. I really don't know what I'm doing. What are you after? You made such a noise, I came out. I thought it was a door banging.«

»You'd better toddle back to bed,« said Lord Peter. »You're gettin' all cold. Why do girls wear such mimsy little pajimjams in this damn cold climate? There, don't you worry. I'll drop in on you later and we'll have a jolly old pow-wow, what?«

»Not today—not today, Peter. I'm going mad, I think.« (»Sensation fiction again,« thought Peter.) »Are they trying Gerald today?«

»Not exactly trying,« said Peter, urging her gently along to her room.  »It's just formal, y'know. The jolly old magistrate bird hears the charge read, and then old Murbles pops up and says please he wants only formal evidence given as he has to instruct counsel. That's Biggy, y'know. Then they hear the evidence of arrest, and Murbles says old Gerald reserves his defence. That's all till the Assizes—evidence before the Grand Jury—a lot of bosh! That'll be early next month, I suppose. You'll have to buck up and be fit by then.«

Mary shuddered.

»No—no! Couldn't I get out of it? I couldn't go through it all again.  I should be sick. I'm feeling awful. No, don't come in. I don't want you. Ring the bell for Ellen. No, let go; go away! I don't want you, Peter!«

Peter hesitated, a little alarmed.

»Much better not, my lord, if you'll excuse me,« said Bunter's voice at his ear. »Only produce hysterics,« he added, as he drew his master gently from the door. »Very distressing for both parties, and altogether unproductive of results. Better to wait for the return of her grace, the Dowager.«

»Quite right,« said Peter. He turned back to pick up his paraphernalia, but was dexterously forestalled. Once again he lifted the lid of the chest and looked in.

»What did you say you found on that skirt, Bunter?«

»Gravel, my lord, and silver sand.«

»Silver sand.«

\noindent\hfil\rule{0.5\textwidth}{.4pt}\hfil

Behind Riddlesdale Lodge the moor stretched starkly away and upward.  The heather was brown and wet, and the little streams had no colour in them. It was six o'clock, but there was no sunset. Only a paleness had moved behind the thick sky from east to west all day. Lord Peter, tramping back after a long and fruitless search for tidings of the man with the motor-cycle, voiced the dull suffering of his gregarious spirit. »I wish old Parker was here,« he muttered, and squelched down a sheep-track.

He was making, not directly for the Lodge, but for a farmhouse about two and a half miles distant from it, known as Grider's Hole. It lay almost due north of Riddlesdale village, a lonely outpost on the edge of the moor, in a valley of fertile land between two wide swells of heather. The track wound down from the height called Whemmeling Fell, skirted a vile swamp, and crossed the little river Ridd about half a mile before reaching the farm. Peter had small hope of hearing any news at Grider's Hole, but he was filled with a sullen determination to leave no stone unturned. Privately, however, he felt convinced that the motor-cycle had come by the high road, Parker's investigations notwithstanding, and perhaps passed directly through King's Fenton without stopping or attracting attention. Still, he had said he would search the neighbourhood, and Grider's Hole was in the neighbourhood. He paused to relight his pipe, then squelched steadily on. The path was marked with stout white posts at regular intervals, and presently with hurdles. The reason for this was apparent as one came to the bottom of the valley, for only a few yards on the left began the stretch of rough, reedy tussocks, with slobbering black bog between them, in which anything heavier than a water-wagtail would speedily suffer change into a succession of little bubbles. Wimsey stooped for an empty sardine-tin which lay, horridly battered, at his feet, and slung it idly into the quag. It struck the surface with a noise like a wet kiss, and vanished instantly. With that instinct which prompts one, when depressed, to wallow in every circumstance of gloom, Peter leaned sadly upon the hurdles and abandoned himself to a variety of shallow considerations upon ⑴ The vanity of human wishes; ⑵ Mutability; ⑶ First love; ⑷ The decay of idealism; ⑸ The aftermath of the Great War; ⑹ Birth-control; and ⑺ The fallacy of free-will. This was his nadir, however. Realizing that his feet were cold and his stomach empty, and that he had still some miles to go, he crossed the stream on a row of slippery stepping-stones and approached the gate of the farm, which was not an ordinary five-barred one, but solid and uncompromising. A man was leaning over it, sucking a straw. He made no attempt to move at Wimsey's approach. »Good evening,« said that nobleman in a sprightly manner, laying his hand on the catch. »Chilly, ain't it?«

The man made no reply, but leaned more heavily, and breathed. He wore a rough coat and breeches, and his leggings were covered with manure.

»Seasonable, of course, what?« said Peter. »Good for the sheep, I daresay. Makes their wool curl, and so on.«

The man removed the straw and spat in the direction of Peter's right boot.

»Do you lose many animals in the bog?« went on Peter, carelessly unlatching the gate, and leaning upon it in the opposite direction. »I see you have a good wall all round the house. Must be a bit dangerous in the dark, what, if you're thinkin' of takin' a little evenin' stroll with a friend?«

The man spat again, pulled his hat over his forehead, and said briefly:
»What doost `a want?«

»Well,« said Peter, »I thought of payin' a little friendly call on Mr—on the owner of this farm, that is to say. Country neighbours, and all that. Lonely kind of country, don't you see. Is he in, d'ye think?«

The man grunted.

»I'm glad to hear it,« said Peter; »it's so uncommonly jolly findin' all you Yorkshire people so kind and hospitable, what? Never mind who you are, always a seat at the fireside and that kind of thing. Excuse me, but do you know you're leanin' on the gate so as I can't open it?  I'm sure it's a pure oversight, only you mayn't realize that just where you're standin' you get the maximum of leverage. What an awfully charmin' house this is, isn't it? All so jolly stark and grim and all the rest of it. No creepers or little rose-grown porches or anything suburban of that sort. Who lives in it?«

The man surveyed him up and down for some moments, and replied, »Mester Grimethorpe.«

»No, does he now?« said Lord Peter. »To think of that. Just the fellow I want to see. Model farmer, what? Wherever I go throughout the length and breadth of the North Riding I hear of Mr Grimethorpe.  »Grimethorpe's butter is the best«; »Grimethorpe's fleeces Never go to pieces«; »Grimethorpe's pork Melts on the fork«; »For Irish stews Take Grimethorpe's ewes«; »A tummy lined with Grimethorpe's beef, Never, never comes to grief«. It has been my life's ambition to see Mr  Grimethorpe in the flesh. And you no doubt are his sturdy henchman and right-hand man. You leap from bed before the breaking-day, To milk the kine amid the scented hay. You, when the shades of evening gather deep, Home from the mountain lead the mild-eyed sheep. You, by the ingle's red and welcoming blaze, Tell your sweet infants tales of olden days! A wonderful life, though a trifle monotonous p'raps in the winter. Allow me to clasp your honest hand.«

Whether the man was moved by this lyric outburst, or whether the failing light was not too dim to strike a pale sheen from the metal in Lord Peter's palm, at any rate he moved a trifle back from the gate.

»Thanks awfully, old bean,« said Peter, stepping briskly past him. »I take it I shall find Mr Grimethorpe in the house?«

The man said nothing till Wimsey had proceeded about a dozen yards up the flagged path, then he hailed him, but without turning round.

»Mester!«

»Yes, old thing?« said Peter affably, returning.

»Happen he'll set dog on tha.«

»You don't say so?« said Peter. »The faithful hound welcomes the return of the prodigal. Scene of family rejoicing. »My own long lost boy!« Sobs and speeches, beer all round for the delighted tenantry. Glees by the old fireside, till the rafters ring and all the smoked hams tumble down to join in the revelry. Good night, sweet Prince, until the cows come home and the dogs eat Jezebel in the portion of Jezreel when the hounds of spring are on winter's traces. I suppose,« he added to himself, »they will have finished tea.«

As Lord Peter approached the door of the farm his spirits rose. He enjoyed paying this kind of visit. Although he had taken to detecting as he might, with another conscience or constitution, have taken to Indian hemp—for its exhilarating properties—at a moment when life seemed dust and ashes, he had not primarily the detective temperament.  He expected next to nothing from inquiries at Grider's Hole, and, if he had, he might probably have extracted all the information he wanted by a judicious display of Treasury notes to the glum man at the gate. Parker would in all likelihood have done so; he was paid to detect and to do nothing else, and neither his natural gifts nor his education (at Barrow-in-Furness Grammar School) prompted him to stray into side-tracks at the beck of an ill-regulated imagination. But to Lord Peter the world presented itself as an entertaining labyrinth of side-issues. He was a respectable scholar in five or six languages, a musician of some skill and more understanding, something of an expert in toxicology, a collector of rare editions, an entertaining man-about-town, and a common sensationalist. He had been seen at half-past twelve on a Sunday morning walking in Hyde Park in a top-hat and frock-coat, reading the \textit{News of the World}. His passion for the unexplored led him to hunt up obscure pamphlets in the British Museum, to unravel the emotional history of income-tax collectors, and to find out where his own drains led to. In this case, the fascinating problem of a Yorkshire farmer who habitually set the dogs on casual visitors imperatively demanded investigation in a personal interview. The result was unexpected.

His first summons was unheeded, and he knocked again. This time there was a movement, and a surly male voice called out:

»Well, let `un in then, dang `un—and dang \textit{thee},« emphasized by the sound of something falling or thrown across the room.

The door was opened unexpectedly by a little girl of about seven, very dark and pretty, and rubbing her arm as though the missile had caught her there. She stood defensively, blocking the threshold, till the same voice growled impatiently:

»Well, who is it?«

»Good evening,« said Wimsey, removing his hat. »I hope you'll excuse me droppin' in like this. I'm livin' at Riddlesdale Lodge.«

»What of it?« demanded the voice. Above the child's head Wimsey saw the outline of a big, thick-set man smoking in the inglenook of an immense fireplace. There was no light but the firelight, for the window was small, and dusk had already fallen. It seemed to be a large room, but a high oak settle on the farther side of the chimney ran out across it, leaving a cavern of impenetrable blackness beyond.

»May I come in?« said Wimsey.

»If tha must,« said the man ungraciously. »Shoot door, lass; what art starin' at? Go to thi moother and bid her mend thi manners for thee.«

This seemed a case of the pot lecturing the kettle on cleanliness, but the child vanished hurriedly into the blackness behind the settle, and Peter walked in.

»Are you Mr Grimethorpe?« he asked politely.

»What if I am?« retorted the farmer. »\textit{I've} no call to be ashamed o' my name.«

»Rather not,« said Lord Peter, »nor of your farm. Delightful place, what? My name's Wimsey, by the way—Lord Peter Wimsey, in fact, the Duke of Denver's brother, y'know. I'm sure I hate interruptin' you—you must be busy with the sheep and all that—but I thought you wouldn't mind if I just ran over in a neighbourly way. Lonely sort of country, ain't it? I like to know the people next door, and all that sort of thing. I'm used to London, you see, where people live pretty thick on the ground. I suppose very few strangers ever pass this way?«

»None,« said Mr Grimethorpe, with decision.

»Well, perhaps it's as well,« pursued Lord Peter. »Makes one appreciate one's home circle more, what? Often think one sees too many strangers in town. Nothing like one's family when all's said and done—cosy, don't you know. You a married man, Mr Grimethorpe?«

»What the hell's that to you?« growled the farmer, rounding on him with such ferocity that Wimsey looked about quite nervously for the dogs before-mentioned.

»Oh, nothin',« he replied, »only I thought that charmin' little girl might be yours.«

»And if I thought she weren't,« said Mr Grimethorpe, »I'd strangle the bitch and her mother together. What hast got to say to that?«

As a matter of fact, the remark, considered as a conversational formula, seemed to leave so much to be desired that Wimsey's natural loquacity suffered a severe check. He fell back, however, on the usual resource of the male, and offered Mr Grimethorpe a cigar, thinking to himself as he did so:

»What a hell of a life the woman must lead.«

The farmer declined the cigar with a single word, and was silent.  Wimsey lit a cigarette for himself and became meditative, watching his companion. He was a man of about forty-five, apparently, rough, harsh, and weather-beaten, with great ridgy shoulders and short, thick thighs—a bull-terrier with a bad temper. Deciding that delicate hints would be wasted on such an organism, Wimsey adopted a franker method.

»To tell the truth, Mr Grimethorpe,« he said, »I didn't blow in without any excuse at all. Always best to provide oneself with an excuse for a call, what? Though it's so perfectly delightful to see you—I mean, no excuse might appear necessary. But fact is, I'm looking for a young man—a—an acquaintance of mine—who said he'd be roamin' about this neighbourhood some time or other about now. Only I'm afraid I may have missed him. You see, I've only just got over from Corsica—interestin' country and all that, Mr Grimethorpe, but a trifle out of the way—and from what my friend said I think he must have turned up here about a week ago and found me out. Just my luck.  But he didn't leave his card, so I can't be quite sure, you see. You didn't happen to come across him by any chance? Tall fellow with big feet on a motor-cycle with a side-car. I thought he might have come rootin' about here. Hullo! d'you know him?«

The farmer's face had become swollen and almost black with rage.

»What day sayst tha?« he demanded thickly.

»I should think last Wednesday night or Thursday morning,« said Peter, with a hand on his heavy malacca cane.

»I knew it,« growled Mr Grimethorpe. »—the slut, and all these dommed women wi' their dirty ways. Look here, mester. The tyke were a friend o' thine? Well, I wor at Stapley Wednesday and Thursday—tha knew that, didn't tha? And so did thi friend, didn't `un? An' if I hadn't, it'd `a' bin the worse for `un. He'd `a' been in Peter's Pot if I'd `a' cot `un, an' that's where tha'll be thesen in a minute, blast tha! And if I find `un sneakin' here again, I'll blast every boon in a's body and send `un to look for thee there.«

And with these surprising words he made for Peter's throat like a bull-dog.

»That won't do,« said Peter, disengaging himself with an ease which astonished his opponent, and catching his wrist in a grip of mysterious and excruciating agony. »'Tisn't wise, y'know—might murder a fellow like that. Nasty business, murder. Coroner's inquest and all that sort of thing. Counsel for the Prosecution askin' all sorts of inquisitive questions, and a feller puttin' a string round your neck. Besides, your method's a bit primitive. Stand still, you fool, or you'll break your arm. Feelin' better? That's right. Sit down. You'll get into trouble one of these days, behavin' like that when you're asked a civil question.«

»Get out o' t'house,« said Mr Grimethorpe sullenly.

»Certainly,« said Peter. »I have to thank you for a very entertainin' evenin', Mr Grimethorpe. I'm sorry you can give me no news of my friend\longdash«

Mr Grimethorpe sprang up with a blasphemous ejaculation, and made for the door, shouting »Jabez!« Lord Peter stared after him for a moment, and then stared round the room.

»Something fishy here,« he said. »Fellow knows somethin'. Murderous sort of brute. I wonder\longdash«

He peered round the settle, and came face to face with a woman—a dim patch of whiteness in the thick shadow.

»You?« she said, in a low, hoarse gasp. »You? You are mad to come here.  Quick, quick! He has gone for the dogs.«

She placed her two hands on his breast, thrusting him urgently back.  Then, as the firelight fell upon his face, she uttered a stifled shriek and stood petrified—a Medusa-head of terror.

Medusa was beautiful, says the tale, and so was this woman; a broad white forehead under massed, dusky hair, black eyes glowing under straight brows, a wide, passionate mouth—a shape so wonderful that even in that strenuous moment sixteen generations of feudal privilege stirred in Lord Peter's blood. His hands closed over hers instinctively, but she pulled herself hurriedly away and shrank back.

»Madam,« said Wimsey, recovering himself, »I don't quite\longdash«

A thousand questions surged up in his mind, but before he could frame them a long yell, and another, and then another came from the back of the house.

»Run, run!« she said. »The dogs! My God, my God, what will become of me? Go, if you don't want to see me killed. Go, go! Have pity!«

»Look here,« said Peter, »can't I stay and protect\longdash«

»You can stay and murder me,« said the woman. »Go!«

Peter cast Public School tradition to the winds, caught up his stick, and went. The brutes were at his heels as he fled. He struck the foremost with his stick, and it dropped back, snarling. The man was still leaning on the gate, and Grimethorpe's hoarse voice was heard shouting to him to seize the fugitive. Peter closed with him; there was a scuffle of dogs and men, and suddenly Peter found himself thrown bodily over the gate. As he picked himself up and ran, he heard the farmer cursing the man and the man retorting that he couldn't help it; then the woman's voice, uplifted in a frightened wail. He glanced over his shoulder. The man and the woman and a second man who had now joined the party, were beating the dogs back, and seemed to be persuading Grimethorpe not to let them through. Apparently their remonstrances had some effect, for the farmer turned moodily away, and the second man called the dogs off, with much whip-cracking and noise. The woman said something, and her husband turned furiously upon her and struck her to the ground.

Peter made a movement to go back, but a strong conviction that he could only make matters worse for her arrested him. He stood still, and waited till she had picked herself up and gone in, wiping the blood and dirt from her face with her shawl. The farmer looked round, shook his fist at him, and followed her into the house. Jabez collected the dogs and drove them back, and Peter's friend returned to lean over the gate.
Peter waited till the door had closed upon Mr and Mrs Grimethorpe; then he pulled out his handkerchief and, in the half-darkness, signalled cautiously to the man, who slipped through the gate and came slowly down to him.

»Thanks very much,« said Wimsey, putting money into his hand. »I'm afraid I've done unintentional mischief.«

The man looked at the money and at him.

»'Tes t' master's way wi' them as cooms t'look at t'missus,« he said.  »Tha's best keep away if so be tha wutna' have her blood on tha heid.«

»See here,« said Peter, »did you by any chance meet a young man with a motor-cycle wanderin' round here last Wednesday or thereabouts?«

»Naay. Wednesday? T'wod be day t'mester went to Stapley, Ah reckon, after machines. Naay, Ah seed nowt.«

»All right. If you find anybody who did, let me know. Here's my name, and I'm staying at Riddlesdale Lodge. Good night, many thanks.«

The man took the card from him and slouched back without a word of farewell.

\noindent\hfil\rule{0.5\textwidth}{.4pt}\hfil

Lord Peter walked slowly, his coat collar turned up and his hat pulled over his eyes. This cinematographic episode had troubled his logical faculty. With an effort he sorted out his ideas and arranged them in some kind of order.

»First item,« said he, »Mr Grimethorpe. A gentleman who will stick at nothing. Hefty. Unamiable. Inhospitable. Dominant characteristic—jealousy of his very astonishing wife. Was at Stapley last Wednesday and Thursday buying machinery. (Helpful gentleman at the gate corroborates this, by the way, so that at this stage of the proceedings one may allow it to be a sound alibi.) Did not, therefore, see our mysterious friend with the side-car, \textit{if} he was there. But is disposed to think he \textit{was} there, and has very little doubt about what he came for. Which raises an interestin' point. Why the side-car?  Awkward thing to tour about with. Very good. But if our friend came after Mrs G. he obviously didn't take her. Good again.«

»Second item, Mrs Grimethorpe. Very singular item. By Jove!« He paused meditatively to reconstruct a thrilling moment. »Let us at once admit that if № 10 came for the purpose suspected he had every excuse for it. Well! Mrs G. goes in terror of her husband, who thinks nothing of knocking her down on suspicion. I wish to God—but I'd only have made things worse. Only thing you can do for the wife of a brute like that is to keep away from her. Hope there won't be murder done. One's enough at a time. Where was I?«

»Yes—well, Mrs Grimethorpe knows something—and she knows somebody.  She took me for somebody who had every reason for not coming to Grider's Hole. Where was she, I wonder, while I was talking to Grimethorpe? She wasn't in the room. Perhaps the child warned her. No, that won't wash; I told the child who I was. Aha! wait a minute. Do I see light? She looked out of the window and saw a bloke in an aged Burberry. № 10 is a bloke in an aged Burberry. Now, let's suppose for a moment she takes me for № 10. What does she do? She sensibly keeps out of the way—can't think why I'm such a fool as to turn up.  Then, when Grimethorpe runs out shoutin' for the kennel-man, she nips down with her life in her hands to warn her—her—shall we say boldly her lover?—to get away. She finds it isn't her lover, but only a gaping ass of (I fear) a very comin'-on disposition. New compromisin' position. She tells the ass to save himself and herself by clearin' out. Ass clears—not too gracefully. The next instalment of this enthrallin' drama will be shown in this theatre—when? I'd jolly well like to know.«

He tramped on for some time.

»All the same,« he retorted upon himself, »all this throws no light on what № 10 was doing at Riddlesdale Lodge.«

At the end of his walk he had reached no conclusion.

»Whatever happens,« he said to himself, »and if it can be done without danger to her life, I must see Mrs Grimethorpe again.«



