%!TeX root=../cloudstop.tex 
\chapter{Mudstains and Bloodstains}

\epigraph{Other things are all very well in their way, but give me Blood.... We say, »There it is! that's Blood!« It is an actual matter of fact. We point it out. It admits of no doubt.... We must have Blood, you know.}{\textit{David Copperfield}}

\lettrine[lines=4,ante=`]{H}{itherto,}' said Lord Peter, as they picked their painful way through the little wood on the trail of Gent's № 10's, »I have always maintained that those obliging criminals who strew their tracks with little articles of personal adornment\allowbreak---\allowbreak here he is, on a squashed fungus\allowbreak---\allowbreak were an invention of detective fiction for the benefit of the author. I see that I have still something to learn about my job.«

»Well, you haven't been at it very long, have you?« said Parker.  »Besides, we don't know that the diamond cat is the criminal's. It may belong to a member of your own family, and have been lying here for days. It may belong to Mr What's-his-name in the States, or to the last tenant but one, and have been lying here for years. This broken branch may be our friend\allowbreak---\allowbreak I think it is.«

»I'll ask the family,« said Lord Peter, »and we could find out in the village if anyone's ever inquired for a lost cat. They're pukka stones. It ain't the sort of thing one would drop without making a fuss about\allowbreak---\allowbreak I've lost him altogether.«

»It's all right\allowbreak---\allowbreak I've got him. He's tripped over a root.«

»Serve him glad,« said Lord Peter viciously, straightening his back.  »I say, I don't think the human frame is very thoughtfully constructed for this sleuth-hound business. If one could go on all-fours, or had eyes in one's knees, it would be a lot more practical.«

»There are many difficulties inherent in a teleological view of creation,« said Parker placidly. »Ah! here we are at the park palings.«

»And here's where he got over,« said Lord Peter, pointing to a place where the \textit{chevaux de frise} on the top was broken away. »Here's the dent where his heels came down, and here's where he fell forward on hands and knees. Hum! Give us a back, old man, would you? Thanks.  An old break, I see. Mr Montague-now-in-the-States should keep his palings in better order. № 10 tore his coat on the spikes all the same; he left a fragment of Burberry behind him. What luck! Here's a deep, damp ditch on the other side, which I shall now proceed to fall into.«

A slithering crash proclaimed that he had carried out his intention.  Parker, thus callously abandoned, looked round, and, seeing that they were only a hundred yards or so from the gate, ran along and was let out, decorously, by Hardraw, the gamekeeper, who happened to be coming out of the Lodge.

»By the way,« said Parker to him, »did you ever find any signs of any poachers on Wednesday night after all?«

»Nay,« said the man, »not so much as a dead rabbit. I reckon t'lady wor mistaken, an 'twore the shot I heard as killed t'Captain.«

»Possibly,« said Parker. »Do you know how long the spikes have been broken off the palings over there?«

»A moonth or two, happen. They should 'a' bin put right, but the man's sick.«

»The gate's locked at night, I suppose?«

»Aye.«

»Anybody wishing to get in would have to waken you?«

»Aye, that he would.«

»You didn't see any suspicious character loitering about outside these palings last Wednesday, I suppose?«

»Nay, sir, but my wife may ha' done. Hey, lass!«

Mrs Hardraw, thus summoned, appeared at the door with a small boy clinging to her skirts.

»Wednesday?« said she. »Nay, I saw no loiterin' folks. I keep a look-out for tramps and such, as it be such a lonely place. Wednesday.  Eh, now, John, that wad be t'day t'young mon called wi' t'motor-bike.«

»Young man with a motor-bike?«

»I reckon 'twas. He said he'd had a puncture and asked for a bucket o' watter.«

»Was that all the asking he did?«

»He asked what were t'name o' t'place and whose house it were.«

»Did you tell him the Duke of Denver was living here?«

»Aye, sir, and he said he supposed a many gentlemen came up for t'shooting.«

»Did he say where he was going?«

»He said he'd coom oop fra' Weirdale an' were makin' a trip into Coomberland.«

»How long was he here?«

»Happen half an hour. An' then he tried to get his machine started, an' I see him hop-hoppitin' away towards King's Fenton.«

She pointed away to the right, where Lord Peter might be seen gesticulating in the middle of the road.

»What sort of a man was he?«

Like most people, Mrs Hardraw was poor at definition. She thought he was youngish and tallish, neither dark nor fair, in such a long coat as motor-bicyclists use, with a belt round it.

»Was he a gentleman?«

Mrs Hardraw hesitated, and Mr Parker mentally classed the stranger as »Not quite quite.«

»You didn't happen to notice the number of the bicycle?«

Mrs Hardraw had not. »But it had a side-car,« she added.

Lord Peter's gesticulations were becoming quite violent, and Mr Parker hastened to rejoin him.

»Come on, gossiping old thing,« said Lord Peter unreasonably. »This is a beautiful ditch.

\begin{verse}
From such a ditch as this,\\
When the soft wind did gently kiss the trees\\
And they did make no noise, from such a ditch\\
Our friend, methinks, mounted the Troyan walls,\\
And wiped his soles upon the greasy mud.\\
\end{verse}

Look at my trousers!«

»It's a bit of a climb from this side,« said Parker.

»It is. He stood here in the ditch, and put one foot into this place where the paling's broken away and one hand on the top, and hauled himself up. № 10 must have been a man of exceptional height, strength, and agility. I couldn't get my foot up, let alone reaching the top with my hand. I'm five foot nine. Could you?«

Parker was six foot, and could just touch the top of the wall with his hand.

»I \textit{might} do it\allowbreak---\allowbreak on one of my best days,« he said, »for an adequate object, or after adequate stimulant.«

»Just so,« said Lord Peter. »Hence we deduce № 10's exceptional height and strength.«

»Yes,« said Parker. »It's a bit unfortunate that we had to deduce his exceptional shortness and weakness just now, isn't it?«

»Oh!« said Peter. »Well\allowbreak---\allowbreak well, as you so rightly say, that \textit{is} a bit unfortunate.«

»Well, it may clear up presently. He didn't have a confederate to give him a back or a leg, I suppose?«

»Not unless the confederate was a being without feet or any visible means of support,« said Lord Peter, indicating the solitary print of a pair of patched 10's. »By the way, how did he make straight in the dark for the place where the spikes were missing? Looks as though he belonged to the neighborhood, or had reconnoitered previously.«

»Arising out of that reply,« said Parker, »I will now relate to you the entertaining »gossip« I have had with Mrs Hardraw.«

»Humph!« said Wimsey at the end of it. »That's interesting. We'd better make inquiries at Riddlesdale and King's Fenton. Meanwhile we know where № 10 came from; now where did he go after leaving Cathcart's body by the well?«

»The footsteps went into the preserve,« said Parker. »I lost them there. There is a regular carpet of dead leaves and bracken.«

»Well, but we needn't go through all that sleuth grind again,« objected his friend. »The fellow went in, and, as he presumably is not there still, he came out again. He didn't come out through the gate or Hardraw would have seen him; he didn't come out the same way he went in or he would have left some traces. Therefore he came out elsewhere.  Let's walk round the wall.«

»Then we'll turn to the left,« said Parker, »since that's the side of the preserve, and he apparently went through there.«

»True, O King; and as this isn't a church, there's no harm in going round it widdershins. Talking of church, there's Helen coming back. Get a move on, old thing.«

They crossed the drive, passed the cottage, and then, leaving the road, followed the paling across some open grass fields. It was not long before they found what they sought. From one of the iron spikes above them dangled forlornly a strip of material. With Parker's assistance Wimsey scrambled up in a state of almost lyric excitement.

»Here we are,« he cried. »The belt of a Burberry! No sort of precaution here. Here are the toe-prints of a fellow sprinting for his life. He tore off his Burberry; he made desperate leaps\allowbreak---\allowbreak one, two, three\allowbreak---\allowbreak at the palings. At the third leap he hooked it on to the spikes. He scrambled up, scoring long, scrabbling marks on the paling. He reached the top. Oh, here's a bloodstain run into this crack. He tore his hands. He dropped off. He wrenched the coat away, leaving the belt dangling\longdash«

»I wish you'd drop off,« grumbled Parker. »You're breaking my collar-bone.«

Lord Peter dropped off obediently, and stood there holding the belt between his fingers. His narrow grey eyes wandered restlessly over the field. Suddenly he seized Parker's arm and marched briskly in the direction of the wall on the farther side\allowbreak---\allowbreak a low erection of unmortared stone in the fashion of the country. Here he hunted along like a terrier, nose foremost, the tip of his tongue caught absurdly between his teeth, then jumped over, and, turning to Parker, said:

»Did you ever read \textit{The Lay of the Last Minstrel}?«

»I learnt a good deal of it at school,« said Parker. »Why?«

»Because there was a goblin page-boy in it,« said Lord Peter, »who was always yelling »Found! Found! Found!« at the most unnecessary moments.  I always thought him a terrible nuisance, but now I know how he felt.  See here.«

Close under the wall, and sunk heavily into the narrow and muddy lane which ran up here at right angles to the main road, was the track of a side-car combination.
»Very nice too,« said Mr Parker approvingly. »New Dunlop tire on the front wheel. Old tire on the back. Gaiter on the side-car tire.  Nothing could be better. Tracks come in from the road and go back to the road. Fellow shoved the machine in here in case anybody of an inquisitive turn of mind should pass on the road and make off with it, or take its number. Then he went round on shank's mare to the gap he'd spotted in the daytime and got over. After the Cathcart affair he took fright, bolted into the preserve, and took the shortest way to his bus, regardless. Well, now.«

He sat down on the wall, and, drawing out his note-book, began to jot down a description of the man from the data already known.

»Things begin to look a bit more comfortable for old Jerry,« said Lord Peter. He leaned on the wall and began whistling softly, but with great accuracy, that elaborate passage of Bach which begins »Let Zion's children«.

\noindent\hfil\rule{0.5\textwidth}{.4pt}\hfil

»I wonder,« said the Hon. Freddy Arbuthnot, »what damn silly fool invented Sunday afternoon.«

He shoveled coals onto the library fire with a vicious clatter, waking Colonel Marchbanks, who said, »Eh? Yes, quite right,« and fell asleep again instantly.

»Don't \textit{you} grumble, Freddy,« said Lord Peter, who had been occupied for some time in opening and shutting all the drawers of the writing-table in a thoroughly irritating manner, and idly snapping to and fro the catch of the French window. »Think how dull old Jerry must feel. 'Spose I'd better write him a line.«

He returned to the table and took a sheet of paper. »Do people use this room much to write letters in, do you know?«

»No idea,« said the Hon. Freddy. »Never write 'em myself. Where's the point of writin' when you can wire? Encourages people to write back, that's all. I think Denver writes here when he writes anywhere, and I saw the Colonel wrestlin' with pen and ink a day or two ago, didn't you, Colonel?« (The Colonel grunted, answering to his name like a dog that wags its tail in its sleep.) »What's the matter? Ain't there any ink?«

»I only wondered,« replied Peter placidly. He slipped a paper-knife under the top sheet of the blotting-pad and held it up to the light.  »Quite right, old man. Give you full marks for observation. Here's Jerry's signature, and the Colonel's, and a big, sprawly hand, which I should judge to be feminine.« He looked at the sheet again, shook his head, folded it up, and placed it in his pocket-book. »Doesn't seem to be anything there,« he commented, »but you never know. »Five something of fine something«---grouse, probably; »oe\allowbreak---\allowbreak is fou«---is found, I suppose. Well, it can't do any harm to keep it.« He spread out his paper and began:

\begin{quote}
\textsc{Dear Jerry},---Here I am, the family sleuth on the trail, and it's damned exciting---
\end{quote}

The Colonel snored.

Sunday afternoon. Parker had gone with the car to King's Fenton, with orders to look in at Riddlesdale on the way and inquire for a green-eyed cat, also for a young man with a side-car. The Duchess was lying down. Mrs Pettigrew-Robinson had taken her husband for a brisk walk. Upstairs, somewhere, Mrs Marchbanks enjoyed a perfect communion of thought with her husband.

Lord Peter's pen gritted gently over the paper, stopped, moved on again, stopped altogether. He leaned his long chin on his hands and stared out of the window, against which there came sudden little swishes of rain, and from time to time a soft, dead leaf. The Colonel snored; the fire tinkled; the Hon. Freddy began to hum and tap his fingers on the arms of his chair. The clock moved slothfully on to five o'clock, which brought teatime and the Duchess.

»How's Mary?« asked Lord Peter, coming suddenly into the firelight.

»I'm really worried about her,« said the Duchess. »She is giving way to her nerves in the strangest manner. It is so unlike her. She will hardly let anybody come near her. I have sent for Dr Thorpe again.«

»Don't you think she'd be better if she got up an' came downstairs a bit?« suggested Wimsey. »Gets broodin' about things all by herself, I shouldn't wonder. Wants a bit of Freddy's intellectual conversation to cheer her up.«

»You forget; poor girl,« said the Duchess, »she was engaged to Captain Cathcart. Everybody isn't as callous as you are.«

»Any more letters, your grace?« asked the footman, appearing with the post-bag.

»Oh, are you going down now?« said Wimsey. »Yes, here you are\allowbreak---\allowbreak and there's one other, if you don't mind waitin' a minute while I write it. Wish I could write at the rate people do on the cinema,« he added, scribbling rapidly as he spoke. »»\textsc{Dear Lilian},---Your father has killed Mr William Snooks, and unless you send me \textsterling 1,000 by bearer, I shall disclose all to your husband.---Sincerely, \textsc{Earl of Digglesbrake}.« That's the style; and all done in one scrape of the pen. Here you are, Fleming.«

The letter was addressed to her grace the Dowager Duchess of Denver.

\noindent\hfil\rule{0.5\textwidth}{.4pt}\hfil

From the \textit{Morning Post} of Monday, November ---, 19---:
\begin{quote}
\begin{center}
\textsc{Abandoned Motor-Cycle}
\end{center}

A singular discovery was made yesterday by a cattle-drover. He is accustomed to water his animals in a certain pond lying a little off the road about twelve miles south of Ripley. On this occasion he saw that one of them appeared to be in difficulties. On going to the rescue, he found the animal entangled in a motor-cycle, which had been driven into the pond and abandoned. With the assistance of a couple of workmen he extricated the machine. It is a Douglas, with dark-grey side-car. The number-plates and license-holder have been carefully removed. The pond is a deep one, and the outfit was entirely submerged. It seems probable, however, that it could not have been there for more than a week, since the pond is much used on Sundays and Mondays for the watering of cattle. The police are making search for the owner. The front tire of the bicycle is a new Dunlop, and the side-car tire has been repaired with a gaiter. The machine is a 1914 model, much worn.
\end{quote}

»That seems to strike a chord,« said Lord Peter musingly. He consulted a time-table for the time of the next train to Ripley, and ordered the car.

»And send Bunter to me,« he added.

That gentleman arrived just as his master was struggling into an overcoat.

»What was that thing in last Thursday's paper about a number-plate, Bunter?« inquired his lordship.

Mr Bunter produced, apparently by legerdemain, a cutting from an evening paper:

\begin{quote}
\begin{center}
\textsc{Number-Plate Mystery}
\end{center}

The Rev. Nathaniel Foulis, of St Simon's, North Fellcote, was stopped at six o'clock this morning for riding a motor-cycle without number-plates. The reverend gentleman seemed thunderstruck when his attention was called to the matter. He explained that he had been sent for in great haste at 4 \textsc{a.m.} to administer the Sacrament to a dying parishioner six miles away. He hastened out on his motor-cycle, which he confidingly left by the roadside while executing his sacred duties.  Mr Foulis left the house at 5:30 without noticing that anything was wrong. Mr Foulis is well known in North Fellcote and the surrounding country, and there seems little doubt that he has been the victim of a senseless practical joke. North Fellcote is a small village a couple of miles north of Ripley.
\end{quote}

»I'm going to Ripley, Bunter,« said Lord Peter.

»Yes, my lord. Does your lordship require me?«

»No,« said Lord Peter, »but\allowbreak---\allowbreak who has been lady's maiding my sister, Bunter?«

»Ellen, my lord\allowbreak---\allowbreak the housemaid.«

»Then I wish you'd exercise your powers of conversation on Ellen.«

»Very good, my lord.«

»Does she mend my sister's clothes, and brush her skirts, and all that?«

»I believe so, my lord.«

»Nothing she may think is of any importance, you know, Bunter.«

»I wouldn't suggest such a thing to a woman, my lord. It goes to their heads, if I may say so.«

»When did Mr Parker leave for town?«

»At six o'clock this morning, my lord.«

\noindent\hfil\rule{0.5\textwidth}{.4pt}\hfil

Circumstances favored Mr Bunter's inquiries. He bumped into Ellen as she was descending the back stairs with an armful of clothing. A pair of leather gauntlets was jerked from the top of the pile, and, picking them up, he apologetically followed the young woman into the servants' hall.

»There,« said Ellen, flinging her burden on the table, »and the work I've had to get them, I'm sure. Tantrums, that's what I call it, pretending you've got such a headache you can't let a person into the room to take your things down to brush, and, as soon as they're out of the way, 'opping out of bed and trapesing all over the place.  'Tisn't what I call a headache, would you, now? But there! I daresay you don't get them like I do. Regular fit to split, my head is sometimes\allowbreak---\allowbreak couldn't keep on my feet, not if the house was burning down.  I just have to lay down and keep laying\allowbreak---\allowbreak something cruel it is. And gives a person such wrinkles in one's forehead.«

»I'm sure I don't see any wrinkles,« said Mr Bunter, »but perhaps I haven't looked hard enough.« An interlude followed, during which Mr  Bunter looked hard enough and close enough to distinguish wrinkles.  »No,« said he, »wrinkles? I don't believe I'd see any if I was to take his lordship's big microscope he keeps up in town.«

»Lor' now, Mr Bunter,« said Ellen, fetching a sponge and a bottle of benzene from the cupboard, »what would his lordship be using a thing like that for, now?«

»Why, in our hobby, you see, Miss Ellen, which is criminal investigation, we might want to see something magnified extra big\allowbreak---\allowbreak as it might be handwriting in a forgery case, to see if anything's been altered or rubbed out, or if different kinds of ink have been used. Or we might want to look at the roots of a lock of hair, to see if it's been torn out or fallen out. Or take bloodstains, now; we'd want to know if it was animals' blood or human blood, or maybe only a glass of port.«

»Now is it really true, Mr Bunter,« said Ellen, laying a tweed skirt out upon the table and unstoppering the benzene, »that you and Lord Peter can find out all that?«

»Of course, we aren't analytical chemists,« Mr Bunter replied, »but his lordship's dabbled in a lot of things\allowbreak---\allowbreak enough to know when anything looks suspicious, and if we've any doubts we send to a very famous scientific gentleman.« (He gallantly intercepted Ellen's hand as it approached the skirt with a benzene-soaked sponge.) »For instance, now, here's a stain on the hem of this skirt, just at the bottom of the side-seam. Now, supposing it was a case of murder, we'll say, and the person that had worn this skirt was suspected, I should examine that stain.« (Here Mr Bunter whipped a lens out of his pocket.) »Then I might try it at one edge with a wet handkerchief.« (He suited the action to the word.) »And I should find, you see, that it came off red.  Then I should turn the skirt inside-out, I should see that the stain went right through, and I should take my scissors« (Mr Bunter produced a small, sharp pair) »and snip off a tiny bit of the inside edge of the seam, like this« (he did so) »and pop it into a little pill-box, so« (the pill-box appeared magically from an inner pocket), »and seal it up both sides with a wafer, and write on the top »Lady Mary Wimsey's skirt,« and the date. Then I should send it straight off to the analytical gentleman in London, and he'd look through his microscope, and tell me right off that it was rabbit's blood, maybe, and how many days it had been there, and that would be the end of that,« finished Mr Bunter triumphantly, replacing his nail-scissors and thoughtlessly pocketing the pill-box with its contents.

»Well, he'd be wrong, then,« said Ellen, with an engaging toss of the head, »because it's bird's blood, and not rabbit's at all, because her ladyship told me so; and wouldn't it be quicker just to go and ask the person than get fiddling round with your silly old microscope and things?«

»Well, I only mentioned rabbits for an example,« said Mr Bunter.  »Funny she should have got a stain down there. Must have regularly knelt in it.«

»Yes. Bled a lot, hasn't it, poor thing? Somebody must 'a' been shootin' careless-like. 'Twasn't his grace, nor yet the Captain, poor man. Perhaps it was Mr Arbuthnot. He shoots a bit wild sometimes.  It's a nasty mess, anyway, and it's so hard to clean off, being left so long. I'm sure I wasn't thinking about cleaning nothing the day the poor Captain was killed; and then the Coroner's inquest---'orrid, it was\allowbreak---\allowbreak and his grace being took off like that! Well, there, it upset me.  I suppose I'm a bit sensitive. Anyhow, we was all at sixes and sevens for a day or two, and then her ladyship shuts herself up in her room and won't let me go near the wardrobe. »Ow!« she says, »do leave that wardrobe door alone. Don't you know it squeaks, and my head's so bad and my nerves so bad I can't stand it,« she says. »I was only going to brush your skirts, my lady,« I says. »Bother my skirts,« says her ladyship, »and do go away, Ellen. I shall scream if I see you fidgeting about there. You get on my nerves,« she says. Well, I didn't see why I should go on, not after being spoken to like that. It's very nice to be a ladyship, and all your tempers coddled and called nervous prostration. I know I was dreadfully cut up about poor Bert, my young man what was killed in the war\allowbreak---\allowbreak nearly cried my eyes out, I did; but, law! Mr Bunter, I'd be ashamed to go on so. Besides, between you and I and the gate-post, Lady Mary wasn't that fond of the Captain. Never appreciated him, that's what I said to cook at the time, and she agreed with me. He had a way with him, the Captain had. Always quite the gentleman, of course, and never said anything as wasn't his place\allowbreak---\allowbreak I don't mean that\allowbreak---\allowbreak but I mean as it was a pleasure to do anything for him. Such a handsome man as he was, too, Mr Bunter.«

»Ah!« said Mr Bunter. »So on the whole her ladyship was a bit more upset than you expected her to be?«

»Well, to tell you the truth, Mr Bunter, I think it's just temper. She wanted to get married and away from home. Drat this stain! It's regular dried in. She and his grace never could get on, and when she was away in London during the war she had a rare old time, nursing officers, and going about with all kinds of queer people his grace didn't approve of.  Then she had some sort of a love-affair with some quite low-down sort of fellow, so cook says; I think he was one of them dirty Russians as wants to blow us all to smithereens\allowbreak---\allowbreak as if there hadn't been enough people blown up in the war already! Anyhow, his grace made a dreadful fuss, and stopped supplies, and sent for her ladyship home, and ever since then she's been just mad to be off with somebody. Full of notions, she is. Makes me tired, I can tell you. Now, I'm sorry for his grace. I can see what he thinks. Poor gentleman! And then to be taken up for murder and put in jail, just like one of them nasty tramps.  Fancy!«

Ellen, having exhausted her breath and finished cleaning off the bloodstains, paused and straightened her back.

»Hard work it is,« she said, »rubbing; I quite ache.«

»If you would allow me to help you,« said Mr Bunter, appropriating the hot water, the benzene bottle, and the sponge.

He turned up another breadth of the skirt.

»Have you got a brush handy,« he asked, »to take this mud off?«

»You're as blind as a bat, Mr Bunter,« said Ellen, giggling. »Can't you see it just in front of you?«

»Ah, yes,« said the valet. »But that's not as hard a one as I'd like.  Just you run and get me a real hard one, there's a dear good girl, and I'll fix this for you.«

»Cheek!« said Ellen. »But,« she added, relenting before the admiring gleam in Mr Bunter's eye, »I'll get the clothes-brush out of the hall for you. That's as hard as a brick-bat, that is.«

No sooner was she out of the room than Mr Bunter produced a pocket-knife and two more pill-boxes. In a twinkling of an eye he had scraped the surface of the skirt in two places and written two fresh labels:

»Gravel from Lady Mary's skirt, about 6 in. from hem.«

»Silver sand from hem of Lady Mary's skirt.«

He added the date, and had hardly pocketed the boxes when Ellen returned with the clothes-brush. The cleaning process continued for some time, to the accompaniment of desultory conversation. A third stain on the skirt caused Mr Bunter to stare critically.

»Hullo!« he said. »Her ladyship's been trying her hand at cleaning this herself.«

»What?« cried Ellen. She peered closely at the mark, which at one edge was smeared and whitened, and had a slightly greasy appearance.

»Well, I never,« she exclaimed, »so she has! Whatever's that for, I wonder? And her pretending to be so ill she couldn't raise her head off the pillow. She's a sly one, she is.«

»Couldn't it have been done before?« suggested Mr Bunter.

»Well, she might have been at it between the day the Captain was killed and the inquest,« agreed Ellen, »though you wouldn't think that was a time to choose to begin learning domestic work. \textit{She} ain't much hand at it, anyhow, for all her nursing. I never believed that came to anything.«

»She's used soap,« said Mr Bunter, benzening away resolutely. »Can she boil water in her bedroom?«

»Now, whatever should she do that for, Mr Bunter?« exclaimed Ellen, amazed. »You don't think she keeps a kettle? I bring up her morning tea. Ladyships don't want to boil water.«

»No,« said Mr Bunter, »and why didn't she get it from the bathroom?« He scrutinized the stain more carefully still. »Very amateurish,« he said; »distinctly amateurish. Interrupted, I fancy. An energetic young lady, but not ingenious.«

The last remarks were addressed in confidence to the benzene bottle.  Ellen had put her head out of the window to talk to the gamekeeper.

\noindent\hfil\rule{0.5\textwidth}{.4pt}\hfil

The Police Superintendent at Ripley received Lord Peter at first frigidly, and later, when he found out who he was, with a mixture of the official attitude to private detectives and the official attitude to a Duke's son.

»I've come to you,« said Wimsey, »because you can do this combin'-out business a sight better'n an amateur like myself. I suppose your fine organization's hard at work already, what?«

»Naturally,« said the Superintendent, »but it's not altogether easy to trace a motor-cycle without knowing the number. Look at the Bournemouth Murder.« He shook his head regretfully and accepted a Villar y Villar.

»We didn't think at first of connecting him with the number-plate business,« the Superintendent went on in a careless tone which somehow conveyed to Lord Peter that his own remarks within the last half-hour had established the connection in the official mind for the first time. »Of course, if he'd been seen going through Ripley \textit{without} a number-plate he'd have been noticed and stopped, whereas with Mr  Foulis's he was as safe as\allowbreak---\allowbreak as the Bank of England,« he concluded in a burst of originality.

»Obviously,« said Wimsey. »Very agitatin' for the parson, poor chap.  So early in the mornin', too. I suppose it was just taken to be a practical joke?«

»Just that,« agreed the Superintendent, »but, after hearing what you have to tell us, we shall use our best efforts to get the man. I expect his grace won't be any too sorry to hear he's found. You may rely on us, and if we find the man or the number-plates\longdash«

»Lord bless us and save us, man,« broke in Lord Peter with unexpected vivacity, »you're not goin' to waste your time lookin' for the number-plates. What d'you s'pose he'd pinch the curate's plates for if he wanted to advertise his own about the neighborhood? Once you drop on them you've got his name and address; s'long as they're in his trousers pocket you're up a gum-tree. Now, forgive me, Superintendent, for shovin' along with my opinion, but I simply can't bear to think of you takin' all that trouble for nothin'---draggin' ponds an' turnin' over rubbish-heaps to look for number-plates that ain't there. You just scour the railway-stations for a young man six foot one or two with a № 10 shoe, and dressed in a Burberry that's lost its belt, and with a deep scratch on one of his hands. And look here, here's my address, and I'll be very grateful if you'll let me know anything that turns up. So awkward for my brother, y'know, all this. Sensitive man; feels it keenly. By the way, I'm a very uncertain bird\allowbreak---\allowbreak always hoppin' about; you might wire me any news in duplicate, to Riddlesdale and to town\allowbreak---\allowbreak 110 Piccadilly. Always delighted to see you, by the way, if ever you're in town. You'll forgive me slopin' off now, won't you? I've got a lot to do.«

\noindent\hfil\rule{0.5\textwidth}{.4pt}\hfil

Returning to Riddlesdale, Lord Peter found a new visitor seated at the tea-table. At Peter's entry he rose into towering height, and extended a shapely, expressive hand that would have made an actor's fortune.  He was not an actor, but he found this hand useful, nevertheless, in the exploitation of dramatic moments. His magnificent build and the nobility of his head and mask were impressive; his features were flawless; his eyes ruthless. The Dowager Duchess had once remarked: »Sir Impey Biggs is the handsomest man in England, and no woman will ever care twopence for him.« He was, in fact, thirty-eight, and a bachelor, and was celebrated for his rhetoric and his suave but pitiless dissection of hostile witnesses. The breeding of canaries was his unexpected hobby, and besides their song he could appreciate no music but revue airs. He answered Wimsey's greeting in his beautiful, resonant, and exquisitely controlled voice. Tragic irony, cutting contempt, or a savage indignation were the emotions by which Sir Impey Biggs swayed court and jury; he prosecuted murderers of the innocent, defended in actions for criminal libel, and, moving others, was himself as stone. Wimsey expressed himself delighted to see him in a voice, by contrast, more husky and hesitant even than usual.

»You just come from Jerry?« he asked. »Fresh toast, please Fleming. How is he? Enjoyin' it? I never knew a fellow like Jerry for gettin' the least possible out of any situation. I'd rather like the experience myself, you know; only I'd hate bein' shut up and watchin' the other idiots bunglin' my case. No reflection on Murbles and you, Biggs. I mean myself\allowbreak---\allowbreak I mean the man who'd be me if I was Jerry. You follow me?«

»I was just saying to Sir Impey,« said the Duchess, »that he really must make Gerald say what he was doing in the garden at three in the morning. If only I'd been at Riddlesdale none of this would have happened. Of course, \textit{we} all know that he wasn't doing any harm, but we can't expect the jurymen to understand that. The lower orders are so prejudiced. It is absurd of Gerald not to realize that he must speak out. He has \textit{no} consideration.«

»I am doing my very best to persuade him, Duchess,« said Sir Impey, »but you must have patience. Lawyers enjoy a little mystery, you know.  Why, if everybody came forward and told the truth, the whole truth, and nothing but the truth straight out, we should all retire to the workhouse.«

»Captain Cathcart's death is very mysterious,« said the Duchess, »though when I think of the things that have come out about him it really seems quite providential, as far as my sister-in-law is concerned.«

»I s'pose you couldn't get 'em to bring it in »Death by the Visitation of God«, could you, Biggs?« suggested Lord Peter. »Sort of judgment for wantin' to marry into our family, what?«

»I have known less reasonable verdicts,« returned Biggs dryly. »It's wonderful what you can suggest to a jury if you try. I remember once at the Liverpool Assizes\longdash«

He steered skillfully away into a quiet channel of reminiscence. Lord Peter watched his statuesque profile against the fire; it reminded him of the severe beauty of the charioteer of Delphi and was about as communicative.

\noindent\hfil\rule{0.5\textwidth}{.4pt}\hfil

It was not until after dinner that Sir Impey opened his mind to Wimsey.  The Duchess had gone to bed, and the two men were alone in the library.  Peter, scrupulously in evening dress, had been valeted by Bunter, and had been more than usually rambling and cheerful all evening. He now took a cigar, retired to the largest chair, and effaced himself in a complete silence.

Sir Impey Biggs walked up and down for some half-hour, smoking. Then he came across with determination, brutally switched on a reading-lamp right into Peter's face, sat down opposite to him, and said:

»Now, Wimsey, I want to know all you know.«

»Do you, though?« said Peter. He got up, disconnected the reading-lamp, and carried it away to a side-table.

»No bullying of the witness, though,« he added, and grinned.

»I don't care so long as you wake up,« said Biggs, unperturbed. »Now then.«

Lord Peter removed his cigar from his mouth, considered it with his head on one side, turned it carefully over, decided that the ash could hang on to its parent leaf for another minute or two, smoked without speaking until collapse was inevitable, took the cigar out again, deposited the ash entire in the exact center of the ash-tray, and began his statement, omitting only the matter of the suit-case and Bunter's information obtained from Ellen.

Sir Impey Biggs listened with what Peter irritably described as a cross-examining countenance, putting a sharp question every now and again. He made a few notes, and, when Wimsey had finished, sat tapping his note-book thoughtfully.

»I think we can make a case out of this,« he said, »even if the police don't find your mysterious man. Denver's silence is an awkward complication, of course.« He hooded his eyes for a moment. »Did you say you'd put the police on to find the fellow?«
»Yes.«

»Have you a very poor opinion of the police?«

»Not for that kind of thing. That's in their line; they have all the facilities, and do it well.«

»Ah! You expect to find the man, do you?«

»I hope to.«

»Ah! What do you think is going to happen to my case if you \textit{do} find him, Wimsey?«

»What do I\longdash«

»See here, Wimsey,« said the barrister, »you are not a fool, and it's no use trying to look like a country policeman. You are really trying to find this man?«

»Certainly.«

»Just as you like, of course, but my hands are rather tied already. Has it ever occurred to you that perhaps he'd better not be found?«

Wimsey stared at the lawyer with such honest astonishment as actually to disarm him.

»Remember this,« said the latter earnestly, »that if once the police get hold of a thing or a person it's no use relying on my, or Murbles's, or anybody's professional discretion. Everything's raked out into the light of common day, and very common it is. Here's Denver accused of murder, and he refuses in the most categorical way to give me the smallest assistance.«

»Jerry's an ass. He doesn't realize\longdash«

»Do you suppose,« broke in Biggs, \enquote{I have not made it my business to \textit{make} him realize? All he says is, »They can't hang me; I didn't kill the man, though I think it's a jolly good thing he's dead. It's no business of theirs what I was doing in the garden.« Now I ask you, Wimsey, is that a reasonable attitude for a man in Denver's position to take up?}

Peter muttered something about »Never had any sense.«

»Had anybody told Denver about this other man?«

»Something vague was said about footsteps at the inquest, I believe.«

»That Scotland Yard man is your personal friend, I'm told?«

»Yes.«

»So much the better. He can hold his tongue.«

»Look here, Biggs, this is all damned impressive and mysterious, but what are you gettin' at? Why shouldn't I lay hold of the beggar if I can?«

»I'll answer that question by another.« Sir Impey leaned forward a little. »Why is Denver screening him?«

Sir Impey Biggs was accustomed to boast that no witness could perjure himself in his presence undetected. As he put the question, he released the other's eyes from his, and glanced down with finest cunning at Wimsey's long, flexible mouth and nervous hands. When he glanced up again a second later he met the eyes passing, guarded and inscrutable, through all the changes expressive of surprised enlightenment; but by that time it was too late; he had seen a little line at the corner of the mouth fade out, and the fingers relax ever so slightly. The first movement had been one of relief.

»B'Jove!« said Peter. »I never thought of that. What sleuths you lawyers are. If that's so, I'd better be careful, hadn't I? Always was a bit rash. My mother says\longdash«

»You're a clever devil, Wimsey,« said the barrister. »I may be wrong, then. Find your man by all means. There's just one other thing I'd like to ask. Whom are \textit{you} screening?«

»Look here, Biggs,« said Wimsey, »you're not paid to ask that kind of question here, you know. You can jolly well wait till you get into court. It's your job to make the best of the stuff we serve up to you, not to give us the third degree. Suppose I murdered Cathcart myself\longdash«

»You didn't.«

»I know I didn't, but if I did I'm not goin' to have you askin' questions and lookin' at me in that tone of voice. However, just to oblige you, I don't mind sayin' plainly that I don't know who did away with the fellow. When I do I'll tell you.«

»You will?«

»Yes, I will, but not till I'm sure. You people can make such a little circumstantial evidence go such a damn long way, you might hang me while I was only in the early stages of suspectin' myself.«

»H'm!« said Biggs. »Meanwhile, I tell you candidly, I am taking the line that they can't make out a case.«

»Not proven, eh? Well, anyhow, Biggs, I swear my brother shan't hang for lack of my evidence.«

»Of course not,« said Biggs, adding inwardly: »but you hope it won't come to that.«

A spurt of rain plashed down the wide chimney and sizzled on the logs.

\noindent\hfil\rule{0.5\textwidth}{.4pt}\hfil

\begin{flushright}
\textsc{Craven Hotel,}\\
\textsc{Strand, W.C.,}\\
\textsc{Tuesday.}
\end{flushright}


\textsc{My Dear Wimsey}---A line as I promised, to report progress, but it's precious little. On the journey up I sat next to Mrs  Pettigrew-Robinson, and opened and shut the window for her and looked after her parcels. She mentioned that when your sister roused the household on Thursday morning she went first to Mr Arbuthnot's room\allowbreak---\allowbreak a circumstance which the lady seemed to think odd, but which is natural enough when you come to think of it, the room being directly opposite the head of the staircase. It was Mr Arbuthnot who knocked up the Pettigrew-Robinsons, and Mr P. ran downstairs immediately.  Mrs P. then saw that Lady Mary was looking very faint, and tried to support her. Your sister threw her off\allowbreak---\allowbreak rudely, Mrs P. says\allowbreak---\allowbreak declined »in a most savage manner« all offers of assistance, rushed to her own room, and locked herself in. Mrs Pettigrew-Robinson listened at the door »to make sure,« as she says, »that everything was all right,« but, hearing her moving about and slamming cupboards, she concluded that she would have more chance of poking her finger into the pie downstairs, and departed.

If Mrs Marchbanks had told me this, I admit I should have thought the episode worth looking into, but I feel strongly that if I were dying I should still lock the door between myself and Mrs Pettigrew-Robinson.  Mrs P. was quite sure that at no time had Lady Mary anything in her hand. She was dressed as described at the inquest\allowbreak---\allowbreak a long coat over her pajamas (sleeping suit was Mrs P's expression), stout shoes, and a woolly cap, and she kept these garments on throughout the subsequent visit of the doctor. Another odd little circumstance is that Mrs  Pettigrew-Robinson (who was awake, you remember, from 2 \textsc{a.m.} onwards) is certain that just \textit{before} Lady Mary knocked on Mr Arbuthnot's door she heard a door slam somewhere in the passage. I don't know what to make of this\allowbreak---\allowbreak perhaps there's nothing in it, but I just mention it.

I've had a rotten time in town. Your brother-in-law elect was a model of discretion. His room at the Albany is a desert from a detecting point of view; no papers except a few English bills and receipts, and invitations. I looked up a few of his inviters, but they were mostly men who had met him at the club or knew him in the Army, and could tell me nothing about his private life. He is known at several night-clubs. I made the round of them last night\allowbreak---\allowbreak or, rather, this morning. General verdict: generous but impervious. By the way, poker seems to have been his great game. No suggestion of anything crooked.  He won pretty consistently on the whole, but never very spectacularly.

I think the information we want must be in Paris. I have written to the Sûreté and the Crédit Lyonnais to produce his papers, especially his account and check-book.

I'm pretty dead with yesterday's and today's work. Dancing all night on top of a journey is a jolly poor joke. Unless you want me, I'll wait here for the papers, or I may run over to Paris myself.

Cathcart's books here consist of a few modern French novels of the usual kind, and another copy of \textit{Manon} with what the catalogues call »curious« plates. He must have had a life somewhere, mustn't he?

The enclosed bill from a beauty specialist in Bond Street may interest you. I called on her. She says he came regularly every week when he was in England.

I drew quite blank at King's Fenton on Sunday\allowbreak---\allowbreak oh, but I told you that. I don't think the fellow ever went there. I wonder if he slunk off up into the moor. Is it worth rummaging about, do you think?  Rather like looking for a needle in a bundle of hay. It's odd about that diamond cat. You've got nothing out of the household, I suppose?  It doesn't seem to fit № 10, somehow\allowbreak---\allowbreak and yet you'd think somebody would have heard about it in the village if it had been lost. Well, so long,

\begin{flushright}
\textsc{Yours Ever,}\\
\textsc{Ch. Parker.}
\end{flushright}
