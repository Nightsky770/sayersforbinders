%!TeX root=../bellonatop.tex
\chapter{Lord Peter Leads Through Strength}

\lettrine[lines=4]{A}{t} eleven o'clock the next morning, Lord Peter Wimsey, unobtrusively attired in a navy-blue suit and dark-gray tie, suitable for a house of mourning, presented himself at the late Lady Dormer's house in Portman Square.

»Is Miss Dorland at home?«

»I will inquire, sir.«

»Kindly give her my card and ask if she can spare me a few moments.«

»Certainly, my lord. Will your lordship be good enough to take a seat?«

The man departed, leaving his lordship to cool his heels in a tall, forbidding room, with long crimson curtains, a dark red carpet and mahogany furniture of repellent appearance. After an interval of nearly fifteen minutes, he reappeared, bearing a note upon a salver. It was briefly worded:

»Miss Dorland presents her compliments to Lord Peter Wimsey, and regrets that she is not able to grant him an interview. If, as she supposes, Lord Peter has come to see her as the representative of Major and Captain Fentiman, Miss Dorland requests that he will address himself to Mr Pritchard, solicitor, of Lincoln's Inn, who is dealing, on her behalf, with all matters connected with the will of the late Lady Dormer.«

»Dear me,« said Wimsey to himself, »this looks almost like a snub. Very good for me, no doubt. Now I wonder\longdash« He read the note again. »Murbles must have been rather talkative. I suppose he told Pritchard he was putting me on to it. Very indiscreet of Murbles and not like him.«

The servant still stood mutely by, with an air of almost violently disassociating himself from all commentary.

»Thank you,« said Wimsey. »Would you be good enough to say to Miss Dorland that I am greatly obliged to her for this information.«

»Very good, my lord.«

»And perhaps you would kindly call me a taxi.«

»Certainly, my lord.«

Wimsey entered the taxi with all the dignity he could summon, and was taken to Lincoln's Inn.

Mr Pritchard was nearly as remote and snubbing in his manner as Miss Dorland. He kept Lord Peter waiting for twenty minutes and received him glacially, in the presence of a beady-eyed clerk.

»Oh, good morning,« said Wimsey, affably. »Excuse my callin' on you like this. More regular to do it through Murbles, I s'pose—nice old boy, Murbles, isn't he? But I always believe in goin' as direct to the point as may be. Saves time, what?«

Mr Pritchard bowed his head and asked how he might have the pleasure of serving his lordship.

»Well, it's about this Fentiman business. Survivorship and all that. Nearly said survival. Appropriate, what? You might call the old General a survival, eh?«

Mr Pritchard waited without moving.

»I take it Murbles told you I was lookin' into the business, what? Tryin' to check up on the timetable and all that?«

Mr Pritchard said neither yea nor nay, but placed his fingers together and sat patiently.

»It's a bit of a problem, you know. Mind if I smoke? Have one yourself?«

»I am obliged to you, I never smoke in business hours.«

»Very proper. Much more impressive. Puts the wind up the clients, what? Well, now, I just thought I'd let you know that it's likely to be a close-ish thing. Very difficult to tell to a minute or so, don't you know. May turn out one way—may turn out the other—may turn completely bafflin' and all that. You get me?«

»Indeed?«

»Oh, yes, absolutely. P'raps you'd like to hear how far I've got.« And Wimsey recounted the history of his researches at the Bellona, in so far as the evidence of the commissionaires and the hall-porter were concerned. He said nothing of his interview with Penberthy, nor of the odd circumstances connected with the unknown Oliver, confining himself to stressing the narrowness of the time-limits between which the General must be presumed to have arrived at the Club. Mr Pritchard listened without comment. Then he said:

»And what, precisely, have you come to suggest?«

»Well, what I mean to say is, don't you know, wouldn't it be rather a good thing if the parties could be got to come to terms? Give and take, you see—split the doings and share the proceeds? After all, half a million's a goodish bit of money—quite enough for three people to live on in a quiet way, don't you think? And it would save an awful lot of trouble and—ahem—lawyers' fees and things.«

»Ah!« said Mr Pritchard. »I may say that I have been expecting this. A similar suggestion was made to me earlier by Mr Murbles, and I then told him that my client preferred not to entertain the idea. You will permit me to add, Lord Peter, that the reiteration of this proposal by you, after your employment to investigate the facts of the case in the interests of the other party, has a highly suggestive appearance. You will excuse me, perhaps, if I warn you further that your whole course of conduct in this matter seems to me open to a very undesirable construction.«

Wimsey flushed.

»You will perhaps permit \textit{me}, Mr Pritchard, to inform you that I am not »employed« by anybody. I have been requested by Mr Murbles to ascertain the facts. They are rather difficult to ascertain, but I have learned one very important thing from you this afternoon. I am obliged to you for your assistance. Good morning.«

The beady-eyed clerk opened the door with immense politeness.

»Good morning,« said Mr Pritchard.

»Employed, indeed,« muttered his lordship, wrathfully. »Undesirable construction. I'll construct him. That old brute knows something, and if he knows something, that shows there's something to be known. Perhaps he knows Oliver; I shouldn't wonder. Wish I'd thought to spring the name on him and see what he said. Too late now. Never mind, we'll get Oliver. Bunter didn't have any luck with those `phone calls, apparently. I think I'd better get hold of Charles.«

He turned into the nearest telephone-booth and gave the number of Scotland Yard. Presently an official voice replied, of which Wimsey inquired whether Detective-Inspector Parker was available. A series of clicks proclaimed that he was being put through to Mr Parker, who presently said: »Hullo!«

»Hullo, Charles. This is Peter Wimsey. Look here, I want you to do something for me. It isn't a criminal job, but it's important. A man calling himself Oliver rang up a number in Mayfair at a little after nine on the night of November  10\textsuperscript{th}. Do you think you could get that call traced for me?«

»Probably. What was the number?«

Wimsey gave it.

»Right you are, old chap. I'll have it looked up and let you know. How goes it? Anything doing?«

»Yes—rather a cosy little problem—nothing for you people—as far as I know, that is. Come round one evening and I'll tell you about it, unofficially.«

»Thanks very much. Not for a day or two, though. We're run off our feet with this crate business.«

»Oh, I know—the gentleman who was sent from Sheffield to Euston in a crate, disguised as York hams. Splendid. Work hard and you will be happy. No, thanks, my child, I don't want another twopenn'orth—I'm spending the money on sweets. Cheerio, Charles!«

The rest of the day Wimsey was obliged to pass in idleness, so far as the Bellona Club affair was concerned. On the following morning he was rung up by Parker.

»I say—that `phone call you asked me to trace.«

»Yes?«

»It was put through at 9.13 \textsc{p.m.} from a public call-box at Charing Cross Underground Station.«

»Oh hell!—the operator didn't happen to notice the bloke, I suppose?«

»There isn't an operator. It's one of those automatic boxes.«

»Oh!—may the fellow who invented them fry in oil. Thanks frightfully, all the same. It gives us a line on the direction, anyhow.«

»Sorry I couldn't do better for you. Cheerio!«

»Oh, cheer-damnably-ho!« retorted Wimsey, crossly, slamming the receiver down. »What is it, Bunter?«

»A district messenger, with a note, my lord.«

»Ah,—from Mr Murbles. Good. This may be something. Yes. Tell the boy to wait, there's an answer.« He scribbled quickly. »Mr Murbles has got an answer to that cabman advertisement, Bunter. There are two men turning up at six o'clock, and I'm arranging to go down and interview them.«

»Very good, my lord.«

»Let's hope that means we get a move on. Get me my hat and coat—I'm running round to Dover Street for a moment.«

Robert Fentiman was there when Wimsey called, and welcomed him heartily.

»Any progress?«

»Possibly a little this evening. I've got a line on those cabmen. I just came round to ask if you could let me have a specimen of old Fentiman's fist.«

»Certainly. Pick what you like. He hasn't left much about. Not exactly the pen of a ready writer. There are a few interesting notes of his early campaigns, but they're rather antiques by this time.«

»I'd rather have something quite recent.«

»There's a bundle of cancelled cheques here, if that would do.«

»It would do particularly well—I want something with figures in it if possible. Many thanks. I'll take these.«

»How on earth is his handwriting going to tell you when he pegged out?«

»That's my secret, dash it all! Have you been down to Gatti's?«

»Yes. They seem to know Oliver fairly well by sight, but that's all. He lunched there fairly often, say once a week or so, but they don't remember seeing him since the eleventh. Perhaps he's keeping under cover. However, I'll haunt the place a bit and see if he turns up.«

»I wish you would. His call came from a public box, so that line of inquiry peters out.«

»Oh, bad luck!«

»You've found no mention of him in any of the General's papers?«

»Not a thing, and I've gone through every bit and scrap of writing in the place. By the way, have you seen George lately?«

»Night before last. Why?«

»He seems to me to be in rather a queer state. I went around last night and he complained of being spied on or something.«

»Spied on?«

»Followed about. Watched. Like the blighters in the `tec stories. Afraid all this business is getting on his nerves. I hope he doesn't go off his rocker or anything. It's bad enough for Sheila as it is. Decent little woman.«

»Thoroughly decent,« agreed Wimsey, »and very fond of him.«

»Yes. Works like Billy-oh to keep the home together and all that. Tell you the truth, I don't know how she puts up with George. Of course, married couples are always sparring and so on, but he ought to behave before other people. Dashed bad form, being rude to your wife in public. I'd like to give him a piece of my mind.«

»He's in a beastly galling position,« said Wimsey. »She's his wife and she's got to keep him, and I know he feels it very much.«

»Do you think so? Seems to me he takes it rather as a matter of course. And whenever the poor little woman reminds him of it, he thinks she's rubbing it in.«

»Naturally, he hates being reminded of it. And I've heard Mrs Fentiman say one or two sharp things to him.«

»I daresay. Trouble with George is, he can't control himself. He never could. A fellow ought to pull himself together and show a bit of gratitude. He seems to think that because Sheila has to work like a man she doesn't want the courtesy and—you know, tenderness and so on—that a woman ought to get.«

»It always gives me the pip,« said Wimsey, »to see how rude people are when they're married. I suppose it's inevitable. Women are funny. They don't seem to care half so much about a man's being honest and faithful—and I'm sure your brother's all that—as for their opening doors and saying thank you. I've noticed it lots of times.«

»A man ought to be just as courteous after marriage as he was before,« declared Robert Fentiman, virtuously.

»So he ought, but he never is. Possibly there's some reason we don't know about,« said Wimsey. »I've asked people, you know—my usual inquisitiveness—and they generally just grunt and say that their wives are sensible and take their affection for granted. But I don't believe women ever get sensible, not even through prolonged association with their husbands.«

The two bachelors wagged their heads, solemnly.

»Well, I think George is behaving like a sweep,« said Robert, »but perhaps I'm hard on him. We never did get on very well. And anyhow, I don't pretend to understand women. Still, this persecution-mania, or whatever it is, is another thing. He ought to see a doctor.«

»He certainly ought. We must keep an eye on him. If I see him at the Bellona I'll have a talk to him and try and get out of him what it's all about.«

»You won't find him at the Bellona. He's avoided it since all this unpleasantness started. I think he's out hunting for jobs. He said something about one of those motor people in Great Portland Street wanting a salesman. He can handle a car pretty well, you know.«

»I hope he gets it. Even if it doesn't pay very well it would do him a world of good to have something to do with himself. Well, I'd better be amblin' off. Many thanks, and let me know if you get hold of Oliver.«

»Oh, rather!«

Wimsey considered a few moments on the doorstep, and then drove straight down to New Scotland Yard, where he was soon ushered in to Detective-Inspector Parker's office.

Parker, a square-built man in the late thirties, with the nondescript features which lend themselves so excellently to detective purposes, was possibly Lord Peter's most intimate—in some ways his only intimate friend. The two men had worked out many cases together and each respected the other's qualities, though no two characters could have been more widely different. Wimsey was the Roland of the combination—quick, impulsive, careless and an artistic jack-of-all-trades. Parker was the Oliver—cautious, solid, painstaking, his mind a blank to art and literature and exercising itself, in spare moments, with Evangelical theology. He was the one person who was never irritated by Wimsey's mannerisms, and Wimsey repaid him with a genuine affection foreign to his usually detached nature.

»Well, how goes it?«

»Not so bad. I want you to do something for me.«

»Not really?«

»Yes, really, blast your eyes. Did you ever know me when I didn't? I want you to get hold of one of your handwriting experts to tell me if these two fists are the same.«

He put on the table, on the one hand the bundle of used cheques, and on the other the sheet of paper he had taken from the library at the Bellona Club.

Parker raised his eyebrows.

»That's a very pretty set of finger-prints you've been pulling up there. What is it? Forgery?«

»No, nothing of that sort. I just want to know whether the same bloke who wrote these cheques made the notes too.«

Parker rang a bell, and requested the attendance of Mr Collins.

»Nice fat sums involved, from the looks of it,« he went on, scanning the sheet of notes appreciatively. »\textsterling 150,000 to R., \textsterling 300,000 to G.—lucky G.—who's G? \textsterling 20,000 here and \textsterling 50,000 there. Who's your rich friend, Peter?«

»It's that long story I was going to tell you about when you'd finished your crate problem.«

»Oh, is it? Then I'll make a point of solving the crate without delay. As a matter of fact, I'm rather expecting to hear something about it before long. That's why I'm here, dancing attendance on the `phone. Oh, Collins, this is Lord Peter Wimsey, who wants very much to know whether these two handwritings are the same.«

The expert took up the paper and the cheques and looked them over attentively.

»Not a doubt about it, I should say, unless the forgery has been astonishingly well done. Some of the figures, especially, are highly characteristic. The fives, for instance and the threes, and the fours, made all of a piece with the two little loops. It's a very old-fashioned handwriting, and made by a very old man, in not too-good health, especially this sheet of notes. Is that the old Fentiman who died the other day?«

»Well, it is, but you needn't shout about it. It's just a private matter.«

»Just so. Well, I should say you need have no doubt about the authenticity of that bit of paper, if that's what you are thinking of.«

»Thanks. That's precisely what I do want to know. I don't think there's the slightest question of forgery or anything. In fact, it was just whether we could look on these rough notes as a guide to his wishes. Nothing more.«

»Oh, yes, if you rule out forgery, I'd answer for it any day that the same person wrote all these cheques and the notes.«

»That's fine. That checks up the results of the finger-print test too. I don't mind telling you, Charles,« he added, when Collins had departed, »that this case is getting damned interesting.«

At this point the telephone rang, and Parker, after listening for some time, ejaculated »Good work!« and then, turning to Wimsey,

»That's our man. They've got him. Excuse me if I rush off. Between you and me, we've pulled this off rather well. It may mean rather a big thing for me. Sure we can't do anything else for you? Because I've got to get to Sheffield. See you to-morrow or next day.«

He caught up his coat and hat and was gone. Wimsey made his own way out and sat for a long time at home, with Bunter's photographs of the Bellona Club before him, thinking.

At six o'clock, he presented himself at Mr Murbles' Chambers in Staple Inn. The two taxi-drivers had already arrived and were seated, well on the edges of their chairs, politely taking old sherry with the solicitor.

»Ah!« said Mr Murbles, »this is a gentleman who is interested in the inquiry we are making. Perhaps you would have the goodness to repeat to him what you have already told me. I have ascertained enough,« he added, turning to Wimsey, »to feel sure that these are the right drivers, but I should like you to put any questions you wish yourself. This gentleman's name is Swain, and his story should come first, I think.«

»Well, sir,« said Mr Swain, a stout man of the older type of driver, »you are wanting to know if anybody picked up an old gent in Portman Square the day before Ar\textit{mis}tice Day rahnd abaht the afternoon. Well, sir, I was goin' slow through the Square at `arf-past four, or it might be a quarter to five on that `ere day, when a footman comes out of a `ouse—I couldn't say the number for certain, but it was on the east side of the Square as might be abaht the middle—and `e makes a sign for me to stop. So I draws up, and presently a very old gent comes out. Very thin, `e was, an' muffled up, but I see `is legs and they was very thin and `e looked abaht a `undred an' two by `is face, and walked with a stick. `E was upright, for such a very old gent, but `e moved slow and rather feeble. An old milingtary gent, I thought `e might be—'e `ad that way of speakin', if you understand me, sir. So the footman tells me to drive `im to a number in `Arley Street.«

»Do you remember it?«

Swain mentioned a number which Wimsey recognized as Penberthy's.

»So I drives `im there. And `e asks me to ring the bell for `im, and when the young man comes to the door to ask if the doctor could please see General Fenton, or Fennimore or some such name, sir.«

»Was it Fentiman, do you think?«

»Well, yes, it might `ave been Fentiman. I think it was. So the young man comes back and says, yes, certingly, so I `elps the old gent aht. Very faint, `e seemed, and a very bad colour, sir, breathin' `eavy and blue-like abaht the lips. Pore old b\dots, I thinks, beggin' yer pardon, sir, `e won't be `ere long, I thinks. So we `elps him up the steps into the `ouse and `e gives me my fare and a shilling for myself, and that's the last I see of `im, sir.«

»That fits in all right with what Penberthy said,« agreed Wimsey. »The General felt the strain of his interview with his sister and went straight round to see him. Right. Now how about this other part of the business?«

»Well,« said Mr Murbles, »I think this gentleman, whose name is—let me see—Hinkins—yes. I think Mr Hinkins picked up the General when he left Harley Street.«

»Yes, sir,« agreed the other driver, a smartish-looking man with a keen profile and a sharp eye. »A very old gentleman, like what we've `eard described, took my taxi at this same number in `Arley Street at `alf past five. I remember the day very well, sir; November  10\textsuperscript{th} it was, and I remember it because, after I done taking him where I'm telling you, my magneto started to give trouble, and I didn't `ave the use of the `bus on Armistice Day, which was a great loss to me, because that's a good day as a rule. Well, this old military gentleman gets in, with his stick and all, just as Swain says, only I didn't notice him looking particular ill, though I see he was pretty old. Maybe the doctor would have given him something to make him better.«

»Very likely,« said Mr Murbles.

»Yes, sir. Well, he gets in, and he says, »Take me to Dover Street,« he says, but if you was to ask me the number, sir, I'm afraid I don't rightly remember, because, you see, we never went there after all.«

»Never went there?« cried Wimsey.

»No, sir. Just as we was comin' out into Cavendish Square, the old gentleman puts his head out and says, »Stop!« So I stops, and I see him wavin' his hand to a gentleman on the pavement. So this other one comes up, and they has a few words together and then the old\longdash«

»One moment. What was this other man like?«

»Dark and thin, sir, and looked about forty. He had on a gray suit and overcoat and a soft hat, with a dark handkerchief round his throat. Oh, yes, and he had a small black moustache. So the old gentleman says, »Cabman,« he says, just like that, »cabman, go back up to Regent's Park and drive round till I tell you to stop.« So the other gentleman gets in with him, and I goes back and drives round the Park, quiet-like, because I guessed they wanted to `ave a bit of a talk. So I goes twice round, and as we was going round the third time, the younger gentleman sticks `is `ed out and says, »Put me down at Gloucester Gate.« So I puts him down there, and the old gentleman says, »Good-bye, George, bear in mind what I have said.« So the gentleman says, »I will, sir,« and I see him cross the road, like as if he might be going up Park Street.«

Mr Murbles and Wimsey exchanged glances.

»And then where did you go?«

»Then, sir, the fare says to me, »Do you know the Bellona Club in Piccadilly?« he says. So I says, »Yes, sir.««

»The Bellona Club?«

»Yes, sir.«

»What time was that?«

»It might be getting on for half-past six, sir. I'd been driving very slow, as I tells you, sir. So I takes him to the Club, like he said, and in he goes, and that's the last I see of him, sir.«

»Thanks very much,« said Wimsey. »Did he seem to be at all upset or agitated when he was talking to the man he called George?«

»No, sir, I couldn't say that. But I thought he spoke a bit sharp-like. What you might call telling him off, sir.«

»I see. What time did you get to the Bellona?«

»I should reckon it was about twenty minutes to seven, sir, or just a little bit more. There was a tidy bit of traffic about. Between twenty and ten to seven, as near as I can recollect.«

»Excellent. Well, you have both been very helpful. That will be all to-day, but I'd like you to leave your names and addresses with Mr Murbles, in case we might want some sort of a statement from either of you later on. And—er\longdash«

A couple of Treasury notes crackled. Mr Swain and Mr Hinkins made suitable acknowledgment and departed, leaving their addresses behind them.

»So he went back to the Bellona Club. I wonder what for?«

»I think I know,« said Wimsey. »He was accustomed to do any writing or business there, and I fancy he went back to put down some notes as to what he meant to do with the money his sister was leaving him. Look at this sheet of paper, sir. That's the General's handwriting, as I've proved this afternoon, and those are his finger-prints. And the initials R and G probably stand for Robert and George, and these figures for the various sums he means to leave them.«

»That appears quite probable. Where did you find this?«

»In the end bay of the library at the Bellona, sir, tucked inside the blotting-paper.«

»The writing is very weak and straggly.«

»Yes—quite tails off, doesn't it. As though he had come over faint and couldn't go on. Or perhaps he was only tired. I must go down and find out if anybody saw him there that evening. But Oliver, curse him! is the man who knows. If only we could get hold of Oliver.«

»We've had no answer to our third question in the advertisement. I've had letters from several drivers who took old gentlemen to the Bellona that morning, but none of them corresponds with the General. Some had check overcoats, and some had whiskers and some had bowler hats or beards—whereas the General was never seen without his silk hat and had, of course, his old-fashioned long military moustache.«

»I wasn't hoping for very much from that. We might put in another ad. in case anybody picked him up from the Bellona on the evening or night of the  10\textsuperscript{th}, but I've got a feeling that this infernal Oliver probably took him away in his own car. If all else fails, we'll have to get Scotland Yard on to Oliver.«

»Make careful inquiries at the Club, Lord Peter. It now becomes more than possible that somebody saw Oliver there and noticed them leaving together.«

\enquote{Of course. I'll go along there at once. And I'll put the advertisement in as well. I don't think we'll rope in the \textsc{b.b.c.} It is so confoundedly public.}

»That,« said Mr Murbles, with a look of horror, \enquote{would be \textit{most} undesirable.}

Wimsey rose to go. The solicitor caught him at the door.

»Another thing we ought really to know,« he said, »is what General Fentiman was saying to Captain George.«

»I've not forgotten that,« said Wimsey, a little uneasily. »We shall have—oh, yes—certainly—of course, we shall have to know that.«