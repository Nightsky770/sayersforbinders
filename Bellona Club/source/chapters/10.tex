%!TeX root=../bellonatop.tex
\chapter{Lord Peter Forces a Card}

\lettrine[lines=4,ante=‘]{H}{ullo}!' 

\zz
»Is that you, Wimsey? Hullo! I say, is that Lord Peter Wimsey. Hullo! I must speak to Lord Peter Wimsey. Hullo!« 

\zz
»All right. I've said hullo. Who're you? And what's the excitement?«

\zz
»It's me. Major Fentiman. I say—\textit{is} that Wimsey?«

»Yes. Wimsey speaking. What's up?«

»I can't hear you.«

»Of course you can't if you keep on shouting. This is Wimsey. Good morning. Stand three inches from the mouth-piece and speak in an ordinary voice. Do not say hullo! To recall the operator, depress the receiver \textit{gently} two or three times.«

»Oh, shut up! don't be an ass. I've seen Oliver.«

»Have you, where?«

»Getting into a train at Charing Cross.«

»Did you speak to him?«

»No—it's maddening. I was just getting my ticket when I saw him passing the barrier. I tore down after him. Some people got in my way, curse them. There was a Circle train standing at the platform. He bolted in and they clanged the doors. I rushed on, waving and shouting, but the train went out. I cursed like anything.«

»I bet you did. How very sickening.«

»Yes, wasn't it? I took the next train\longdash«

»What for?«

»Oh, I don't know. I thought I might spot him on a platform somewhere.«

»What a hope! You didn't think to ask where he'd booked for?«

»No. Besides, he probably got the ticket from an automatic.«

»Probably. Well, it can't be helped, that's all. He'll probably turn up again. You're sure it was he?«

»Oh, dear, yes. I couldn't be mistaken. I'd know him anywhere. I thought I'd just let you know.«

»Thanks awfully. It encourages me extremely. Charing Cross seems to be a haunt of his. He `phoned from there on the evening of the tenth, you know.«

»So he did.«

»I'll tell you what we'd better do, Fentiman. The thing is getting rather serious. I propose that you should go and keep an eye on Charing Cross station. I'll get hold of a detective\longdash«

»A police detective?«

»Not necessarily. A private one would do. You and he can go along and keep watch on the station for, say a week. You must describe Oliver to the detective as best you can, and you can watch turn and turn about.«

»Hang it all, Wimsey—it'll take a lot of time. I've gone back to my rooms at Richmond. And besides, I've got my own duties to do.«

»Yes, well, while you're on duty the detective must keep watch.«

»It's a dreadful grind, Wimsey.« Fentiman's voice sounded dissatisfied.

»It's half a million of money. Of course, if you're not keen\longdash«

»I \textit{am} keen. But I don't believe anything will come of it.«

»Probably not; but it's worth trying. And in the meantime, I'll have another watch kept at Gatti's.«

»At Gatti's?«

»Yes. They know him there. I'll send a man down\longdash«

»But he never comes there now.«

»Oh, but he may come again. There's no reason why he shouldn't. We know now that he's in town, and not gone out of the country or anything. I'll tell the management that he's wanted for an urgent business matter, so as not to make unpleasantness.«

»They won't like it.«

»Then they'll have to lump it.«

»Well, all right. But, look here—\textit{I'll} do Gatti's.«

»That won't do. We want you to identify him at Charing Cross. The waiter or somebody can do the identifying at Gatti's. You say they know him.«

»Yes, of course they do. But\longdash«

»But what?—By the way, which waiter is it you spoke to. I had a talk with the head man there yesterday, and he didn't seem to know anything about it.«

»No—it wasn't the head waiter. One of the others. The plump, dark one.«

»All right. I'll find the right one. Now, will you see to the Charing Cross end?«

»Of course—if you really think it's any good.«

»Yes, I do. Right you are. I'll get hold of the `tec and send him along to you, and you can arrange with him.«

»Very well.«

»Cheerio!«

Lord Peter rang off and sat for a few moments, grinning to himself. Then he turned to Bunter.

»I don't often prophesy, Bunter, but I'm going to do it now. Your fortune told by hand or cards. Beware of the dark stranger. That sort of thing.«

»Indeed, my lord?«

»Cross the gypsy's palm with silver. I see Mr Oliver. I see him taking a journey in which he will cross water. I see trouble. I see the ace of spades—upside-down, Bunter.«

»And what then, my lord?«

»Nothing. I look into the future and I see a blank. The gypsy has spoken.«

»I will bear it in mind, my lord.«

»Do. If my prediction is not fulfilled, I will give you a new camera. And now I'm going round to see that fellow who calls himself Sleuths Incorporated, and get him to put a good man on to keep watch at Charing Cross. And after that, I'm going down to Chelsea and I don't quite know when I shall be back. You'd better take the afternoon off. Put me out some sandwiches or something, and don't wait up if I'm late.«

Wimsey disposed quickly of his business with Sleuths Incorporated, and then made his way to a pleasant little studio overlooking the river at Chelsea. The door, which bore a neat label »Miss Marjorie Phelps,« was opened by a pleasant-looking young woman with curly hair and a blue overall heavily smudged with clay.

»Lord Peter! How nice of you. Do come in.«

»Shan't I be in the way?«

»Not a scrap. You don't mind if I go on working.«

»Rather not.«

»You could put the kettle on and find some food if you liked to be really helpful. I just want to finish up this figure.«

»That's fine. I took the liberty of bringing a pot of Hybla honey with me.«

»What sweet ideas you have! I really think you are one of the nicest people I know. You don't talk rubbish about art, and you don't want your hand held, and your mind always turns on eating and drinking.«

»Don't speak too soon. I don't want my hand held, but I did come here with an object.«

»Very sensible of you. Most people come without any.«

»And stay interminably.«

»They do.«

Miss Phelps cocked her head on one side and looked critically at the little dancing lady she was modelling. She had made a line of her own in pottery figurines, which sold well and were worth the money.

»That's rather attractive,« said Wimsey.

»Rather pretty-pretty. But it's a special order, and one can't afford to be particular. I've done a Christmas present for you, by the way. You'd better have a look at it, and if you think it offensive we'll smash it together. It's in that cupboard.«

Wimsey opened the cupboard and extracted a little figure about nine inches high. It represented a young man in a flowing dressing-gown, absorbed in the study of a huge volume held on his knee. The portrait was life-like. He chuckled.

»It's damned good, Marjorie. A very fine bit of modelling. I'd love to have it. You aren't multiplying it too often, I hope? I mean, it won't be on sale at Selfridges?«

»I'll spare you that. I thought of giving one to your mother.«

»That'll please her no end. Thanks ever so. I shall look forward to Christmas, for once. Shall I make some toast?«

»Rather!«

Wimsey squatted happily down before the gasfire, while the modeller went on with her work. Tea and figurine were ready almost at the same moment, and Miss Phelps, flinging off her overall, threw herself luxuriously into a battered arm-chair by the hearth.

»And what can I do for you?«

»You can tell me all you know about Miss Ann Dorland.«

»Ann Dorland? Great heavens! You haven't fallen for Ann Dorland, have you? I've heard she's coming into a lot of money.«

»You have a perfectly disgusting mind, Miss Phelps. Have some more toast. Excuse me licking my fingers. I have not fallen for the lady. If I had, I'd manage my affairs without assistance. I haven't even seen her. What's she like?«

»To look at?«

»Among other things.«

»Well, she's rather plain. She has dark, straight hair, cut in a bang across the forehead and bobbed—like a Flemish page. Her forehead is broad and she has a square sort of face and a straight nose—quite good. Also, her eyes are good—gray, with nice heavy eyebrows, not fashionable a bit. But she has a bad skin and rather sticky-out teeth. And she's dumpy.«

»She's a painter, isn't she?«

»M'm—well! she paints.«

»I see. A well-off amateur with a studio.«

»Yes. I will say that old Lady Dormer was very decent to her. Ann Dorland, you know, is some sort of far-away distant cousin on the female side of the Fentiman family, and when Lady Dormer first got to hear of her she was an orphan and incredibly poverty-stricken. The old lady liked to have a bit of young life about the house, so she took charge of her, and the wonderful thing is that she didn't try to monopolize her. She let her have a big place for a studio and bring in any friends she liked and go about as she chose—in reason, of course.«

»Lady Dormer suffered a good deal from oppressive relations in her own youth,« said Wimsey.

»I know, but most old people seem to forget that. I'm sure Lady Dormer had time enough. She must have been rather unusual. Mind you, I didn't know her very well, and I don't really know a great deal about Ann Dorland. I've been there, of course. She gave parties—rather incompetently. And she comes round to some of our studios from time to time. But she isn't really one of us.«

»Probably one has to be really poor and hard-working to be that.«

»No. You, for instance, fit in quite well on the rare occasions when we have the pleasure. And it doesn't matter not being able to paint. Look at Bobby Hobart and his ghastly daubs—he's a perfect dear and everybody loves him. I think Ann Dorland must have a complex of some kind. Complexes explain so much, like the blessed word hippopotamus.«

Wimsey helped himself lavishly to honey and looked receptive.

»I think really,« went on Miss Phelps, »that Ann ought to have been something in the City. She has brains, you know. She'd run anything awfully well. But she isn't creative. And then, of course, so many of our little lot seem to be running love-affairs. And a continual atmosphere of hectic passion is very trying if you haven't got any of your own.«

»Has Miss Dorland a mind above hectic passion?«

»Well, no. I daresay she would quite have liked—but nothing ever came of it. Why are you interested in having Ann Dorland analysed?«

»I'll tell you some day. It isn't just vulgar curiosity.«

»No, you're very decent as a rule, or I wouldn't be telling you all this. I think, really, Ann has a sort of fixed idea that she couldn't ever possibly attract any one, and so she's either sentimental and tiresome, or rude and snubbing, and our crowd does hate sentimentality and simply can't bear to be snubbed. Ann's rather pathetic, really. As a matter of fact, I think she's gone off art a bit. Last time I heard about her, she had been telling some one she was going in for social service, or sick-nursing, or something of that kind. I think it's very sensible. She'd probably get along much better with the people who do that sort of thing. They're so much more solid and polite.«

»I see. Look here, suppose I ever wanted to run across Miss Dorland accidentally on purpose—where should I be likely to find her?«

»You \textit{do} seem thrilled about her! I think I should try the Rushworths. They go in rather for science and improving the submerged tenth and things like that. Of course, I suppose Ann's in mourning now, but I don't think that would necessarily keep her away from the Rushworth's. Their gatherings aren't precisely frivolous.«

»Thanks very much. You're a mine of valuable information. And, for a woman, you don't ask many questions.«

»Thank you for those few kind words, Lord Peter.«

»I am now free to devote my invaluable attention to \textit{your} concerns. What is the news? And who is in love with whom?«

»Oh, life is a perfect desert. Nobody is in love with me, and the Schlitzers have had a worse row than usual and separated.«

»No!«

»Yes. Only, owing to financial considerations, they've got to go on sharing the same studio—you know, that big room over the mews. It must be very awkward having to eat and sleep and work in the same room with somebody you're being separated from. They don't even speak, and it's very awkward when you call on one of them and the other has to pretend not to be able to see or hear you.«

»I shouldn't think one could keep it up under those circumstances.«

»It's difficult. I'd have had Olga here, only she is so dreadfully bad-tempered. Besides, neither of them will give up the studio to the other.«

»I see. But isn't there any third party in the case?«

»Yes—Ulric Fiennes, the sculptor, you know. But he can't have her at his place because his wife's there, and he's really dependent on his wife, because his sculpting doesn't pay. And besides, he's at work on that colossal group for the Exhibition and he can't move it; it weighs about twenty tons. And if he went off and took Olga away, his wife would lock him out of the place. It's very inconvenient being a sculptor. It's like playing the double-bass; one's so handicapped by one's baggage.«

»True. Whereas, when you run away with me, we'll be able to put all the pottery shepherds and shepherdesses in a handbag.«

»Of course. What fun it will be. Where shall we run to?«

»How about starting to-night and getting as far as Oddenino's and going on to a show—if you're not doing anything?«

»You are a loveable man, and I shall call you Peter. Shall we see »Betwixt and Between«?«

»The thing they had such a job to get past the censor? Yes, if you like. Is it particularly obscene?«

»No, epicene, I fancy.«

»Oh, I see. Well, I'm quite agreeable. Only I warn you that I shall make a point of asking you the meaning of all the risky bits in a very audible voice.«

»That's your idea of amusement, is it?«

»Yes. It does make them so wild. People say »Hush!« and giggle, and if I'm lucky I end up with a gorgeous row in the bar.«

»Then I won't risk it. No. I'll tell you what I'd really love. We'll go and see »George Barnwell« at the Elephant and have a fish-and-chips supper afterwards.«

This was agreed upon, and was voted in retrospect a most profitable evening. It finished up with grilled kippers at a friend's studio in the early hours. Lord Peter returned home to find a note upon the hall-table.

\begin{quote}
\noindent My lord,

The person from Sleuths Incorporated rang up to-day that he was inclined to acquiesce in your lordship's opinion, but that he was keeping his eye upon the party and would report further to-morrow. The sandwiches are on the dining-room table, if your lordship should require refreshment.

\begin{flushright}
Yours obediently,\\
\textsc{M. Bunter.}
\end{flushright}
\end{quote}

»Cross the gypsy's palm with silver,« said his lordship, happily, and rolled into bed.