%!TeX root=../bellonatop.tex
\chapter{Picture-cards}

\lettrine[lines=4,ante=‘]{S}{o} I've put a man in and had all the things in that cupboard taken away for examination,' said Parker.

\zz
Lord Peter shook his head.

\zz
»I wish I had been there,« he said, »I should have liked to see those paintings. However\longdash«

»They might have conveyed something to you,« said Parker, »you're artistic. You can come along and look at them any time, of course. But it's the time factor that's worrying me, you know. Supposing she gave the old boy digitalin in his B and S, why should it wait all that time before working? According to the books, it ought to have popped him off in about an hour's time. It was a biggish dose, according to Lubbock.«

»I know. I think you're up against a snag there. That's why I should have liked to see the pictures.«

Parker considered this apparent \textit{non sequitur} for a few moments and gave it up.

»George Fentiman\longdash« he began.

»Yes,« said Wimsey, »George Fentiman. I must be getting emotional in my old age, Charles, for I have an unconquerable dislike to examining the question of George Fentiman's opportunities.«

»Bar Robert,« pursued Parker, ruthlessly, »he was the last interested person to see General Fentiman.«

»Yes—by the way, we have only Robert's unsupported word for what happened in that last interview between him and the old man.«

»Come, Wimsey—you're not going to pretend that Robert had any interest in his grandfather's dying before Lady Dormer. On the contrary.«

»No—but he might have had some interest in his dying before he made a will. Those notes on that bit of paper. The larger share was to go to George. That doesn't entirely agree with what Robert said. And if there was no will, Robert stood to get everything.«

»So he did. But by killing the General then, he made sure of getting nothing at all.«

»That's the awkwardness. Unless he thought Lady Dormer was already dead. But I don't see how he could have thought that. Or unless\longdash«

»Well?«

»Unless he gave his grandfather a pill or something to be taken at some future time, and the old boy took it too soon by mistake.«

»That idea of a delayed-action pill is the most tiresome thing about this case. It makes almost anything possible.«

»Including, of course, the theory of its being given to him by Miss Dorland.«

»That's what I'm going to interview the nurse about, the minute I can get hold of her. But we've got away from George.«

»You're right. Let's face George. I don't want to, though. Like the lady in Maeterlinck who's running round the table while her husband tries to polish her off with a hatchet, I am not gay. George is the nearest in point of time. In fact he fits very well in point of time. He parted from General Fentiman at about half-past six, and Robert found Fentiman dead at about eight o'clock. So allowing that the stuff was given in a pill\longdash«

»Which it would have to be in a taxi,« interjected Parker.

»As you say—in a pill, which would take a bit longer to get working than the same stuff taken in solution—why then the General might quite well have been able to get to the Bellona and see Robert before collapsing.«

»Very nice. But how did George get the drug?«

»I know, that's the first difficulty.«

»And how did he happen to have it on him just at that time? He couldn't possibly have known that General Fentiman would run across him just at that moment. Even if he'd known of his being at Lady Dormer's, he couldn't be expecting him to go from there to Harley Street.«

»He might have been carrying the stuff about with him, waiting for a good opportunity to use it. And when the old man called him up and started jawing him about his conduct and all that, he thought he'd better do the job quick, before he was cut out of the will.«

»Um!—but why should George be such a fool, then, as to admit he'd never heard about Lady Dormer's will? If he had heard of it, we couldn't possibly suspect him. He'd only to say the General told him about it in the taxi.«

»I suppose it hadn't struck him in that light.«

»Then George is a bigger ass than I took him for.«

»Possibly he is,« said Parker, dryly. »At any rate, I have put a man on to make inquiries at his home.«

»Oh! have you? I say, do you know, I wish I'd left this case alone. What the deuce did it matter if old Fentiman was pushed painlessly off a bit before his time? He was simply indecently ancient.«

»We'll see if you say that in sixty years' time,« said Parker.

»By that time we shall, I hope, be moving in different circles. I shall be in the one devoted to murderers and you in the much lower and hotter one devoted for those who tempt others to murder them. I wash my hands of this case, Charles. There's nothing for me to do now you have come into it. It bores and annoys me. Let's talk about something else.«

Wimsey might wash his hands, but, like Pontius Pilate, he found society irrationally determined to connect him with an irritating and unsatisfactory case.

At midnight, the telephone bell rang.

He had just gone to bed, and cursed it.

»Tell them I'm out,« he shouted to Bunter, and cursed again on hearing the man assure the unknown caller that he would see whether his lordship had returned. Disobedience in Bunter spelt urgent necessity.

»Well?«

»It is Mrs George Fentiman, my lord; she appears to be in great distress. If your lordship wasn't in I was to beg you to communicate with her as soon as you arrived.«

»Punk! they're not on the 'phone.«

»No, my lord.«

»Did she say what the matter was?«

»She began by asking if Mr George Fentiman was here, my lord.«

»Oh, hades!«

Bunter advanced gently with his master's dressing-gown and slippers. Wimsey thrust himself into them savagely and padded away to the telephone.

»Hullo!«

»Is that Lord Peter?—Oh, \textit{good}!« The line sighed with relief—a harsh sound, like a death-rattle. »Do you know where George is?«

»No idea. Hasn't he come home?«

»No—and I'm frightened. Some people were here this morning....«

»The police.«

»Yes \textellipsis  George \textellipsis  they found something \textellipsis  I can't say it all over the 'phone \textellipsis  but George went off to Walmisley-Hubbard's with the car \textellipsis  and they say he never came back there \textellipsis  and \textellipsis  you remember that time he was so funny before \textellipsis  and got lost....«

»Your six minutes are up,« boomed the voice of the Exchange, »will you have another call?«

»Yes, please \textellipsis  oh, don't cut us off \textellipsis  wait \textellipsis  oh! I haven't any more pennies \textellipsis  Lord Peter....«

»I'll come round at once,« said Wimsey, with a groan.

»Oh, thank you—thank you so much!«

»I say—where's Robert?«

»Your six minutes are up,« said the voice, finally, and the line went dead with a metallic crash.

»Get me my clothes,« said Wimsey, bitterly—»give me those loathsome and despicable rags which I hoped to have put off forever. Get me a taxi. Get me a drink. Macbeth has murdered sleep. Oh! and get me Robert Fentiman, first.«

Major Fentiman was not in town, said Woodward. He had gone back to Richmond again. Wimsey tried to get through to Richmond. After a long time, a female voice, choked with sleep and fury, replied. Major Fentiman had not come home. Major Fentiman kept very late hours. Would she give Major Fentiman a message when he did come in? Indeed she would not. She had other things to do than to stay up all night answering the telephone calls and giving messages to Major Fentiman. This was the second time that night, and she had told the other party that she could not be responsible for telling Major Fentiman this, that and the other. Would she leave a note for Major Fentiman, asking him to go round to his brother's house at once? Well now, was it reasonable to expect her to sit up on a bitter cold night writing letters? Of course not, but this was a case of urgent illness. It would be a very great kindness. Just that—to go round to his brother's house and say the call came from Lord Peter Wimsey.

»Who?«

»Lord Peter Wimsey.«

»Very well, sir. I beg your pardon if I was a bit short, but really\longdash«

»You weren't, you snobby old cat, you were infernally long,« breathed his lordship inaudibly. He thanked her, and rang off.

Sheila Fentiman was anxiously waiting for him on the doorstep, so that he was saved the embarrassment of trying to remember which was the right number of rings to give. She clasped his hand eagerly as she drew him in.

»Oh! it is good of you. I'm so worried. I say, don't make a noise, will you? They complain, you know.« She spoke in a harassed whisper.

»Blast them, let them complain,« said Wimsey, cheerfully. »Why shouldn't you make a row when George is upset? Besides, if we whisper, they'll think we're no better than we ought to be. Now, my child, what's all this? You're as cold as a \textit{pêche Melba}. That won't do. Fire half out—where's the whisky?«

»Hush! I'm all right, really. George\longdash«

»You're not all right. Nor am I. As George Robey says, this getting up from my warm bed and going into the cold night air doesn't suit me.« He flung a generous shovelful of coals on the fire and thrust the poker between the bars. »And you've had no grub. No wonder you're feeling awful.«

Two places were set at the table—untouched—waiting for George. Wimsey plunged into the kitchen premises, followed by Sheila uttering agitated remonstrances. He found some disagreeable remnants—a watery stew, cold and sodden; a basin half-full of some kind of tinned soup; a chill suet pudding put away on a shelf.

»Does your woman cook for you? I suppose she does, as you're both out all day. Well, she can't cook, my child. No matter, here's some Bovril—she can't have hurt that. You go and sit down and I'll make you some.«

»Mrs Munns\longdash«

»Blow Mrs Munns!«

»But I must tell you about George.«

He looked at her, and decided that she really must tell him about George.

»I'm sorry. I didn't mean to bully. One has an ancestral idea that women must be treated like imbeciles in a crisis. Centuries of the »women-and-children-first« idea, I suppose. Poor devils!«

»Who, the women?«

»Yes. No wonder they sometimes lose their heads. Pushed into corners, told nothing of what's happening and made to sit quiet and do nothing. Strong men would go dotty in the circs. I suppose that's why we've always grabbed the privilege of rushing about and doing the heroic bits.«

»That's quite true. Give me the kettle.«

»No, no. I'll do that. You sit down and—I mean, sorry, \textit{take} the kettle. Fill it, light the gas, put it on. And tell me about George.«

The trouble, it seemed, had begun at breakfast. Ever since the story of the murder had come out, George had been very nervy and jumpy, and, to Sheila's horror, had »started muttering again.« »Muttering,« Wimsey remembered, had formerly been the prelude to one of George's »queer fits.« These had been a form of shell-shock, and they had generally ended in his going off and wandering about in a distraught manner for several days, sometimes with partial and occasionally with complete temporary loss of memory. There was the time when he had been found dancing naked in a field among a flock of sheep and singing to them. It had been the more ludicrously painful in that George was altogether tone-deaf, so that his singing, though loud, was like a hoarse and rumbling wind in the chimney. Then there was a dreadful time when George had deliberately walked into a bonfire. That was when they had been staying down in the country. George had been badly burnt, and the shock of the pain had brought him round. He never remembered afterwards why he had tried to do these things, and had only the faintest recollection of having done them at all. The next vagary might be even more disconcerting.

At any rate, George had been »muttering.«

They were at breakfast that morning, when they saw two men coming up the path. Sheila, who sat opposite the window, saw them first, and said carelessly: »Hullo! who are these? They look like plainclothes policemen.« George took one look, jumped up and rushed out of the room. She called to him to know what was the matter, but he did not answer, and she heard him »rummaging« in the back room, which was the bedroom. She was going to him, when she heard Mr Munns open the door to the policemen and then heard them inquiring for George. Mr Munns ushered them into the front room with a grim face on which »police« seemed written in capital letters. George—

At this point the kettle boiled. Sheila was taking it off the stove to make the Bovril, when Wimsey became aware of a hand on his coat-collar. He looked round into the face of a gentleman who appeared not to have shaved for several days.

»Now then,« said this apparition, »what's the meaning of this?«

»Which,« added an indignant voice from the door, »I thought as there was something behind all this talk of the Captain being missing. You didn't expect him to be missing, I suppose, ma'am. Oh, dear no! Nor your gentleman friend, neither, sneaking up in a taxi and you waiting at the door so's Munns and me shouldn't hear. But I'd have you know this is a respectable house, Lord Knows Who or whatever you call yourself—more likely one of these low-down confidence fellers, I expect, if the truth was known. With a monocle too, like that man we was reading about in the News of the World. And in my kitchen too, and drinking my Bovril in the middle of the night, the impudence! Not to speak of the goings-in-and-out all day, banging the front door, and that was the police come here this morning, you think I didn't know? Up to something, that's what they've been, the pair of them, and the captain as he says he is but that's as may be, I daresay he had his reasons for clearing off, and the sooner you goes after him my fine madam, the better I'll be pleased, I can tell you.«

»That's right,« said Mr Munns—»ow!«

Lord Peter had removed the intrusive hand from his collar with a sharp jerk which appeared to cause anguish out of all proportion to the force used.

»I'm glad you've come along,« he said. »In fact, I was just going to give you a call. Have you anything to drink in the house, by the way?«

»Drink?« cried Mrs Munns on a high note, »the impudence! And if I see you, Joe, giving drinks to thieves and worse in the middle of the night in my kitchen, you'll get a piece of my mind. Coming in here as bold as brass and the captain run away, and asking for drink\longdash«

»Because,« said Wimsey, fingering his note-case, »the public houses in this law-abiding neighbourhood are of course closed. Otherwise a bottle of Scotch\longdash«

Mr Munns appeared to hesitate.

»Call yourself a man!« said Mrs Munns.

»Of course,« said Mr Munns, »if I was to go in a friendly manner to Jimmy Rowe at the Dragon, and ask him to give me a bottle of Johnny Walker as a friend to a friend, and provided no money was to pass between him and me, that is\longdash«

»A good idea,« said Wimsey, cordially.

Mrs Munns gave a loud shriek.

»The ladies,« said Mr Munns, »gets nervous at times.« He shrugged his shoulders.

»I daresay a drop of Scotch wouldn't do Mrs Munns's nerves any harm,« said Wimsey.

»If you dare, Joe Munns,« said the landlady, »if you dare to go out at this time of night, hob-nobbing with Jimmy Rowe and making a fool of yourself with burglars and such\longdash«

Mr Munns executed a sudden volte-face.

»You shut up!« he shouted. »Always sticking your face in where you aren't wanted.«

»Are you speaking to me?«

»Yes. Shut up!«

Mrs Munns sat down suddenly on a kitchen chair and began to sniff.

»I'll just hop round to the »Dragon« now, sir,« said Mr Munns, »before old Jimmy goes to bed. And then we'll go into this here.«

He departed. Possibly he forgot what he had said about no money passing, for he certainly took the note which Wimsey absent-mindedly held out to him.

»Your drink's getting cold,« said Wimsey to Sheila.

She came across to him.

»Can't we get rid of these people?«

»In half a jiff. It's not good having a row with them. I'd do it like a shot, only, you see, you've got to stay on here for a bit, in case George comes back.«

»Of course. I'm sorry for all this upset, Mrs Munns,« she added, a little stiffly, »but I'm so worried about my husband.«

»Husband?« snorted Mrs Munns. »A lot husbands are to worry about. Look at that Joe. Off he goes to the Dragon, never mind what I say to him. They're dirt, that's what husbands are, the whole pack of them. And I don't care what anybody says.«

»Are they?« said Wimsey. »Well, I'm not one—yet—so you needn't mind what you say to me.«

»It's the same thing,« said the lady, viciously, »husbands and parricides, there's not a half-penny to choose between them. Only parricides aren't respectable—but then, they're easier got rid of.«

»Oh!« replied Wimsey, »but I'm not a parricide either—not Mrs Fentiman's parricide at any rate, I assure you. Hullo! here's Joe. Did you get the doings, old man? You did? Good work. Now, Mrs Munns, have just a spot with us. You'll feel all the better for it. And why shouldn't we go into the sitting-room where it's warmer?«

Mrs Munns complied. »Oh, well,« she said, »here's friends all round. But you'll allow it all looked a bit queer, now, didn't it? And the police this morning, asking all those questions, and emptying the dust-bin all over the back-yard.«

»Whatever did they want with the dust-bin?«

»Lord knows; and that Cummins woman looking on all the time over the wall. I can tell you, I was vexed. »Why, Mrs Munns,« she said, »have you been poisoning people?« she said. »I always told you,« she said »your cooking 'ud do for somebody one of these days.« The nasty cat.«

»What a rotten thing to say,« said Wimsey, sympathetically. »Just jealousy, I expect. But what did the police find in the dust-bin?«

»Find? Them find anything? I should like to see them finding things in my dust-bin. The less I see of their interfering ways the better I'm pleased. I told them so. I said, »If you want to come upsetting my dust-bin,« I said, »you'll have to come with a search-warrant,« I said. That's the law and they couldn't deny it. They said Mrs Fentiman had given them leave to look, so I told them Mrs Fentiman had no leave to give them. It was my dust-bin, I told them, not hers. So they went off with a flea in their ear.«

»That's the stuff to give 'em, Mrs Munns.«

»Not but what I'm respectable. If the police come to me in a right and lawful manner, I'll gladly give them any help they want. I don't want to get into trouble, not for any number of captains. But interference with a free-born woman and no search-warrant I will \textit{not} stand. And they can either come to me in a fitting way or they can go and whistle for their bottle.«

»What bottle?« asked Wimsey, quickly.

»The bottle they were looking for in my dust-bin, what the captain put there after breakfast.«

Sheila gave a faint cry.

»What bottle was that, Mrs Munns?«

»One of them little tablet bottles,« said Mrs Munns, »same as you have standing on the wash-hand stand, Mrs Fentiman. When I saw the Captain smashing it up in the yard with a poker\longdash«

»There now, Primrose,« said Mr Munns, »can't you see as Mrs Fentiman ain't well?«

»I'm quite all right,« said Sheila, hastily, pushing away the hair which clung damply to her forehead. »What was my husband doing?«

»I saw him,« said Mrs Munns, »run out into the back-yard—just after your breakfast it was, because I recollect Munns was letting the officers into the house at the time. Not that I knew then who it was, for, if you will excuse me mentioning of it, I was in the outside lavatory, and that was how I come to see the Captain. Which ordinarily, you can't see the dust-bin from the house, my lord I should say, I suppose, if you really are one, but you meet so many bad characters nowadays that one can't be too careful—on account of the lavatory standing out as you may say and hiding it.«

»Just so,« said Wimsey.

»So when I saw the captain breaking the bottle as I said, and throwing the bits into the dust-bin, »Hullo!« I said, »that's funny,« and I went to see what it was and I put it in an envelope, thinking, you see, as it might be something poisonous, and the cat such a dreadful thief as he is, I never can keep him out of that dust-bin. And when I came in, I found the police here. So after a bit, I found them poking about in the yard and I asked them what they were doing there. Such a mess as they'd made, you never would believe. So they showed me a little cap they'd found, same as it might be off that tablet-bottle. Did I know where the rest of it was? they said. And I said, what business had they got with the dust-bin at all. So they said\longdash«

»Yes, I know,« said Wimsey. »I think you acted very sensibly, Mrs Munns. And what did you do with the envelope and things?«

»I kept it,« replied Mrs Munns, nodding her head, »I kept it. Because, you see, if they \textit{did} return \textit{with} a warrant and I'd destroyed that bottle, where should \textit{I} be?«

»Quite right,« said Wimsey, with his eye on Sheila.

»Always keep on the right side of the law,« agreed Mr Munns, »and nobody can't interfere with you. That's what I say. I'm a Conservative, I am. I don't hold with these Socialist games. Have another.«

»Not just now,« said Wimsey. »And we really must not keep you and Mrs Munns up any longer. But, look here! You see, Captain Fentiman had shell-shock after the War, and he is liable to do these little odd things at times—break things up, I mean, and lose his memory and go wandering about. So Mrs Fentiman is naturally anxious about his not having turned up this evening.«

»Ay,« said Mr Munns, with relish. »I knew a fellow like that. Went clean off his rocker he did one night. Smashed up his family with a beetle—a pavior he was by profession, and that's how he came to have a beetle in the house—pounded 'em to a jelly, he did, his wife and five little children, and went off and drownded himself in the Regent's Canal. And, what's more, when they got him out, he didn't remember a word about it, not one word. So they sent him to—what's that place? Dartmoor? no, Broadmoor, that's it, where Ronnie True went to with his little toys and all.«

»Shut up, you fool,« said Wimsey, savagely.

»Haven't you got feelings?« demanded his wife.

Sheila got up, and made a blind effort in the direction of the door.

»Come and lie down,« said Wimsey, »you're worn out. Hullo! there's Robert, I expect. I left a message for him to come round as soon as he got home.«

Mr Munns went to answer the bell.

»We'd better get her to bed as quick as possible,« said Wimsey to the landlady. »Have you got such a thing as a hot-water bottle?«

Mrs Munns departed to fetch one, and Sheila caught Wimsey's hand.

»Can't you get hold of that bottle? Make her give it to you. You can. You can do anything. Make her.«

»Better not,« said Wimsey. »Look suspicious. Look here, Sheila, what \textit{is} the bottle?«

»My heart medicine. I missed it. It's something to do with digitalin.«

»Oh, lord,« said Wimsey, as Robert came in.

»It's all pretty damnable,« said Robert.

He thumped the fire gloomily; it was burning badly, the lower bars were choked with the ashes of a day and night.

»I've been having a talk with Frobisher,« he added. »All this talk in the Club—and the papers—naturally he couldn't overlook it.«

»Was he decent?«

»Very decent. But of course I couldn't explain the thing. I'm sending in my papers.«

Wimsey nodded. Colonel Frobisher could scarcely overlook an attempted fraud—not after things had been said in the papers.

»If I'd only let the old man alone. Too late now. He'd have been buried. Nobody would have asked questions.«

»I didn't \textit{want} to interfere,« said Wimsey, defending himself against the unspoken reproach.

»Oh, I know. I'm not blaming you. People \textellipsis  money oughtn't to depend on people's deaths \textellipsis  old people, with no use for their lives \textellipsis  it's a devil of a temptation. Look here, Wimsey, what are we to do about this woman?«

»The Munns female?«

»Yes. It's the devil and all she should have got hold of the stuff. If they find out what it's supposed to be, we shall be blackmailed for the rest of our lives.«

»No,« said Wimsey, »I'm sorry, old man, but the police have got to know about it.«

Robert sprang to his feet.

»My God!—you wouldn't\longdash«

»Sit down, Fentiman. Yes, I must. Don't you see I must? We can't suppress things. It always means trouble. It's not even as though they hadn't got their eyes on us already. They're suspicious\longdash«

»Yes, and why?« burst out Robert, violently. »Who put it into their heads?... For God's sake don't start talking about law and justice! Law and justice! You'd sell your best friend for the sake of making a sensational appearance in the witness-box, you infernal little police spy!«

»Chuck that, Fentiman!«

»I'll not chuck it! You'd go and give away a man to the police—when you know perfectly well he isn't responsible—just because you can't afford to be mixed up in anything unpleasant. I know you. Nothing's too dirty for you to meddle in, provided you can pose as the pious little friend of justice. You make me sick!«

»I tried to keep out of this\longdash«

»You tried!—don't be a blasted hypocrite! You get out of it now, and stay out—do you hear?«

»Yes, but listen a moment\longdash«

»Get out!« said Robert.

Wimsey stood up.

»I know how you feel, Fentiman\longdash«

»Don't stand there being righteous and forebearing, you sickening prig. For the last time—are you going to shut up, or are you going to trot round to your policeman friend and earn the thanks of a grateful country for splitting on George? Get on! Which is it to be?«

»You won't do George any good\longdash«

»Never mind that. Are you going to hold your tongue?«

»Be reasonable, Fentiman.«

»Reasonable be damned. Are you going to the police? No shuffling. Yes or no?«

»Yes.«

»You dirty little squirt,« said Robert, striking out passionately. Wimsey's return blow caught him neatly on the chin and landed him in the wastepaper basket.

»And now, look here,« said Wimsey, standing over him, hat and stick in hand. »It's no odds to me what you do or say. You think your brother murdered your grandfather. I don't know whether he did or not. But the worst thing you can do for him is to try and destroy evidence. And the worst thing you can possibly do for his wife is to make her a party to anything of the sort. And next time you try to smash anybody's face in, remember to cover up your chin. That's all. I can let myself out. Good-bye.«

He went round to 12 Great Ormond Street and rousted Parker out of bed.

Parker listened thoughtfully to what he had to say.

»I wish we'd stopped Fentiman before he bolted,« he said.

»Yes; why didn't you?«

»Well, Dykes seems to have muffed it rather. I wasn't there myself. But everything seemed all right. Fentiman looked a bit nervy, but many people do when they're interviewed by the police—think of their hideous pasts, I suppose, and wonder what's coming next. Or else it's just stage-fright. He stuck to the same tale he told you—said he was quite sure the old General hadn't taken any pills or anything in the taxi—didn't attempt to pretend he knew anything about Lady Dormer's will. There was nothing to detain him for. He said he had to get to his job in Great Portland Street. So they let him go. Dykes sent a man to follow him up, and he went along to Hubbard-Walmisley's all right. Dykes said, might he just have a look round the place before he went, and Mrs Fentiman said certainly. He didn't expect to find anything, really. Just happened to step into the back-yard, and saw a bit of broken glass. He then had a look round, and there was the cap of the tablet-bottle in the dust-bin. Well, then, of course, he started to get interested, and was just having a hunt through for the rest of it, when old mother Munns appeared and said the dust-bin was her property. So they had to clear out. But Dykes oughtn't to have let Fentiman go till they'd finished going over the place. He 'phoned through to Hubbard-Walmisley's at once, and heard that Fentiman had arrived and immediately gone out with the car, to visit a prospective customer in Herts. The fellow who was supposed to be trailing Fentiman got carburetor trouble just beyond St Albans, and by the time he was fixed, he'd lost Fentiman.«

»Did Fentiman go to the customer's house?«

»Not he. Disappeared completely. We shall find the car, of course—it's only a matter of time.«

»Yes,« said Wimsey. His voice sounded tired and constrained.

»This alters the look of things a bit,« said Parker, »doesn't it?«

»Yes.«

»What have you done to your face, old man?«

Wimsey glanced at the looking-glass, and saw that an angry red flush had come up on the cheekbone.

»Had a bit of a dust-up with Robert,« he said.

»Oh!«

Parker was aware of a thin veil of hostility, drawn between himself and the friend he valued. He knew that for the first time, Wimsey was seeing him as the police. Wimsey was ashamed and his shame made Parker ashamed too.

»You'd better have some breakfast,« said Parker. His voice sounded awkward to himself.

»No—no thanks, old man. I'll go home and get a bath and shave.«

»Oh, right-oh!«

There was a pause.

»Well, I'd better be going,« said Wimsey.

»Oh, yes,« said Parker again. »Right-oh!«

»Er—cheerio!« said Wimsey at the door.

»Cheerio!« said Parker.

The bedroom door shut. The flat door shut. The front door shut.

Parker pulled the telephone towards him and called up Scotland Yard.

The atmosphere of his own office was bracing to Parker when he got down there. For one thing, he was taken aside by a friend and congratulated in conspiratorial whispers.

»Your promotion's gone through,« said the friend. »Dead certainty. The Chief's no end pleased. Between you and me, of course. But you've got your Chief-Inspectorship all right. Damn good.«

Then, at ten o'clock, the news came through that the missing Walmisley-Hubbard had turned up. It had been abandoned in a remote Hertfordshire lane. It was in perfectly good order, the gear-lever in neutral and the tank full of petrol. Evidently, Fentiman had left it and wandered away somewhere, but he could not be far off. Parker made the necessary arrangements for combing out the neighbourhood. The bustle and occupation soothed his mind. Guilty or insane or both, George Fentiman had to be found; it was just a job to be done.

The man who had been sent to interview Mrs Munns (armed this time with a warrant) returned with the fragments of the bottle and tablets. Parker duly passed these along to the police analyst. One of the detectives who was shadowing Miss Dorland rang up to announce that a young woman had come to see her, and that the two had then come out carrying a suit-case and driven away in a taxi. Maddison, the other detective, was following them. Parker said, »All right; stay where you are for the present,« and considered this new development. The telephone rang again. He thought it would be Maddison, but it was Wimsey—a determinedly brisk and cheerful Wimsey this time.

»I say, Charles. I want something.«

»What?«

»I want to go and see Miss Dorland.«

»You can't. She's gone off somewhere. My man hasn't reported yet.«

»Oh! Well, never mind her. What I really want to see is her studio.«

»Yes? Well, there's no reason why you shouldn't.«

»Will they let me in?«

»Probably not. I'll meet you there and take you in with me. I was going out any way. I've got to interview the nurse. We've just got hold of her.«

»Thanks awfully. Sure you can spare the time?«

»Yes. I'd like your opinion.«

»I'm glad somebody wants it. I'm beginning to feel like a pelican in the wilderness.«

»Rot! I'll be round in ten minutes.«

»Of course,« explained Parker, as he ushered Wimsey into the studio, »we've taken away all the chemicals and things. There's not much to look at, really.«

»Well, you can deal best with all that. It's the books and paintings I want to look at. H'm! Books, you know, Charles, are like lobster shells, we surround ourselves with 'em, then we grow out of 'em and leave 'em behind, as evidence of our earlier stages of development.«

»That's a fact,« said Parker. »I've got rows of school-boy stuff at home—never touch it now, of course. And W. J. Locke—read everything he wrote once upon a time. And Le Queux, and Conan Doyle, and all that stuff.«

»And now you read theology. And what else?«

»Well, I read Hardy a good bit. And when I'm not too tired, I have a go at Henry James.«

»The refined self-examinations of the infinitely-sophisticated. 'M-m. Well now. Let's start with the shelves by the fireplace. Dorothy Richardson—Virginia Woolf—E. B. C. Jones—May Sinclair—Katherine Mansfield—the modern female writers are well represented, aren't they? Galsworthy. Yes. No J. D. Beresford—no Wells—no Bennett. Dear me, quite a row of D. H. Lawrence. I wonder if she reads him very often.«

He pulled down \textit{Women in Love} at random, and slapped the pages open and shut.

»Not kept very well dusted, are they? But they have been read. Compton Mackenzie—Storm Jameson—yes—I see.«

»The medical stuff is over here.«

»Oh!—a few text-books—first steps in chemistry. What's that tumbled down at the back of the book-case? Louis Berman, eh? \textit{The Personal Equation}. And here's \textit{Why We Behave Like Human Beings}. And Julian Huxley's essays. A determined effort at self-education here, what?«

»Girls seem to go in for that sort of thing nowadays.«

»Yes—hardly nice, is it? Hullo!«

»What?«

»Over here by the couch. This represents the latest of our lobster-shells, I fancy. Austin Freeman, Austin Freeman, Austin Freeman—bless me! she must have ordered him in wholesale. \textit{Through the Wall}—that's a good 'tec story, Charles—all about the third degree—Isabel Ostrander—three Edgar Wallaces—the girl's been indulging in an orgy of crime!«

»I shouldn't wonder,« said Parker, with emphasis. »That fellow Freeman is full of plots about poisonings and wills and survivorship, isn't he?«

»Yes«—Wimsey balanced \textit{A Silent Witness} gently in his hand, and laid it down again. »This one, for instance, is all about a bloke who murdered somebody and kept him in cold storage till he was ready to dispose of him. It would suit Robert Fentiman.«

Parker grinned.

»A bit elaborate for the ordinary criminal. But I daresay people do get ideas out of these books. Like to look at the pictures? They're pretty awful.«

»Don't try to break it gently. Show us the worst at once\textellipsis~. Oh, lord!«

»Well, it gives \textit{me} a pain,« said Parker. »But I thought perhaps that was my lack of artistic education.«

»It was your natural good taste. What vile colour, and viler drawing.«

»But nobody cares about drawing nowadays, do they?«

»Ah! but there's a difference between the man who can draw and won't draw, and the man who can't draw at all. Go on. Let's see the rest.«

Parker produced them, one after the other. Wimsey glanced quickly at each. He had picked up the brush and palette and was fingering them as he talked.

»These,« he said, »are the paintings of a completely untalented person, who is, moreover, trying to copy the mannerisms of a very advanced school. By the way, you have noticed, of course, that she has been painting within the last few days, but chucked it in sudden disgust. She has left the paints on the palette, and the brushes are still stuck in the turps, turning their ends up and generally ruining themselves. Suggestive, I fancy. The—stop a minute! Let's look at that again.«

Parker had brought forward the head of the sallow, squinting man which he had mentioned to Wimsey before.

»Put that up on the easel. That's very interesting. The others, you see, are all an effort to imitate other people's art, but this—this is an effort to imitate nature. Why?—it's very bad, but it's meant for somebody. And it's been worked on a lot. Now what was it made her do that?«

»Well, it wasn't for his beauty, I should think.«

»No?—but there must have been a reason. Dante, you may remember, once painted an angel. Do you know the limerick about the old man of Khartoum?«

»What did he do?«

»He kept two black sheep in his room. They remind me (he said) of two friends who are dead. But I cannot remember of whom.«

»If that reminds you of anybody you know, I don't care much for your friends. I never saw an uglier mug.«

»He's not beautiful. But I think the sinister squint is chiefly due to bad drawing. It's very difficult to get eyes looking the same way, when you can't draw. Cover up one eye, Charles—not yours, the portrait's.«

Parker did so.

Wimsey looked again, and shook his head.

»It escapes me for the moment,« he said. »Probably it's nobody I know after all. But, whoever it is, surely this room tells you something.«

»It suggests to me,« said Parker, »that the girl's been taking more interest in crimes and chemistry stuff than is altogether healthy in the circumstances.«

Wimsey looked at him for a moment.

»I wish I could think as you do.«

»What \textit{do} you think?« demanded Parker, impatiently.

»No,« said Wimsey. »I told you about that George business this morning, because glass bottles are facts, and one mustn't conceal facts. But I'm not obliged to tell you what I think.«

»You don't think, then, that Ann Dorland did the murder?«

»I don't know about that, Charles. I came here hoping that this room would tell me the same thing that it told you. But it hasn't. It's told me different. It's told me what I thought all along.«

»A penny for your thoughts, then,« said Parker, trying desperately to keep the conversation on a jocular footing.

»Not even thirty pieces of silver,« replied Wimsey, mournfully.

Parker stacked the canvasses away without another word.