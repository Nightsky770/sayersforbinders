%!TeX root=../bellonatop.tex
\chapter{Lord Peter Clears Trumps}
\lettrine[lines=4,ante=‘]{S}{leuths} Incorporated's' report, when it came, might be summed up as »Nothing doing and Major Fentiman convinced that there never will be anything doing; opinion shared by Sleuths Incorporated.« Lord Peter's reply was: »Keep on watching and something will happen before the week is out.«

His lordship was justified. On the fourth evening, »Sleuths Incorporated« reported again by telephone. The particular sleuth in charge of the case had been duly relieved by Major Fentiman at 6 \textsc{p.m.} and had gone to get his dinner. On returning to his post an hour later, he had been presented with a note left for him with the ticket-collector at the stair-head. It ran: »Just seen Oliver getting into taxi. Am following. Will communicate to refreshment-room. Fentiman.« The sleuth had perforce to return to the refreshment room and hang about waiting for a further message. »But all the while, my lord, the second man I put on as instructed by you, my lord, was a-following the Major unbeknownst.« Presently a call was put through from Waterloo. »Oliver is on the Southampton train. I am following.« The sleuth hurried down to Waterloo, found the train gone and followed on by the next. At Southampton he made inquiries and learned that a gentleman answering to Fentiman's description had made a violent disturbance as the Havre boat was just starting, and had been summarily ejected at the instance of an elderly man whom he appeared to have annoyed or attacked in some way. Further investigation among the Port authorities made it clear that Fentiman had followed this person down, made himself offensive on the train and been warned off by the guard, collared his prey again on the gangway and tried to prevent him from going aboard. The gentleman had produced his passport and pièces d'identité, showing him to be a retired manufacturer of the name of Postlethwaite, living at Kew. Fentiman had insisted that he was, on the contrary, a man called Oliver, address and circumstances unknown, whose testimony was wanted in some family matter. As Fentiman was unprovided with a passport and appeared to have no official authority for stopping and questioning travellers, and as his story seemed vague and his manner agitated, the local police had decided to detain Fentiman. Postlethwaite was allowed to proceed on his way, after leaving his address in England and his destination, which, as he contended, and as he produced papers and correspondence to prove, was Venice.

The sleuth went round to the police-station, where he found Fentiman, apoplectic with fury, threatening proceedings for false imprisonment. He was able to get him released, however, on bearing witness to Fentiman's identity and good faith, and after persuading him to give a promise to keep the peace. He had then reminded Fentiman that private persons were not entitled to assault or arrest peaceable people against whom no charge could be made, pointing out to him that his proper course, when Oliver denied being Oliver, would have been to follow on quietly and keep a watch on him, while communicating with Wimsey or Mr Murbles or Sleuths Incorporated. He added that he was himself now waiting at Southampton for further instructions from Lord Peter. Should he follow to Venice, or send his subordinate, or should he return to London? In view of the frank behaviour of Mr Postlethwaite, it seemed probable that a genuine mistake had been made as to identity, but Fentiman insisted that he was not mistaken.

Lord Peter, holding the trunk line, considered for a moment. Then he laughed.

»Where is Major Fentiman?« he asked.

»Returning to town, my lord. I have represented to him that I have now all the necessary information to go upon, and that his presence in Venice would only hamper my movements, now that he had made himself known to the party.«

»Quite so. Well, I think you might as well send your man on to Venice, just in case it's a true bill. And listen«\dots He gave some further instructions, ending with: »And ask Major Fentiman to come and see me as soon as he arrives.«

»Certainly, my lord.«

»What price the gypsy's warning now?« said Lord Peter, as he communicated this piece of intelligence to Bunter.

Major Fentiman came round to the flat that afternoon, in a whirl of apology and indignation.

»I'm sorry, old man. It was damned stupid of me, but I lost my temper. To hear that fellow calmly denying that he had ever seen me or poor old grandfather, and coming out with his bits of evidence so pat, put my bristles up. Of course, I see now that I made a mistake. I quite realize that I ought to have followed him up quietly. But how was I to know that he wouldn't answer to his name?«

»But you ought to have guessed when he didn't, that either you had made a mistake or that he had some very good motive for trying to get away,« said Wimsey.

»I wasn't accusing him of anything.«

»Of course not, but he seems to have thought you were.«

»But why?—I mean, when I first spoke to him, I just said, »Mr Oliver, I think?« And he said, »You are mistaken.« And I said, »Surely not. My name's Fentiman, and you knew my grandfather, old General Fentiman.« And he said he hadn't the pleasure. So I explained that we wanted to know where the old boy had spent the night before he died, and he looked at me as if I was a lunatic. That annoyed me, and I said I knew he was Oliver, and then he complained to the guard. And when I saw him just trying to hop off like that, without giving us any help, and when I thought about that half-million, it made me so mad I just collared him. »Oh, no, you don't,« I said—and that was how the fun began, don't you see.«

»I see perfectly,« said Wimsey. »But don't \textit{you} see, that if he really \textit{is} Oliver and has gone off in that elaborate manner, with false passports and everything, he must have something important to conceal.«

Fentiman's jaw dropped.

»You don't mean—you don't mean there's anything funny about the death? Oh! surely not.«

»There must be something funny about Oliver, anyway, mustn't there? On your own showing?«

»Well, if you look at it that way, I suppose there must. I tell you what, he's probably got into some bother or other and is clearing out. Debt, or a woman, or something. Of course that must be it. And I was beastly inconvenient popping up like that. So he pushed me off. I see it all now. Well, in that case, we'd better let him rip. We can't get him back, and I daresay he won't be able to tell us anything after all.«

»That's possible, of course. But when you bear in mind that he seems to have disappeared from Gatti's, where you used to see him, almost immediately after the General's death, doesn't it look rather as though he was afraid of being connected up with that particular incident?«

Fentiman wriggled uncomfortably.

»Oh, but, hang it all! What could he have to do with the old man's death?«

»I don't know. But I think we might try to find out.«

»How?«

»Well, we might apply for an exhumation order.«

»Dig him up!« cried Fentiman, scandalized.

»Yes. There was no post-mortem, you know.«

»No, but Penberthy knew all about it and gave the certificate.«

»Yes; but at that time there was no reason to suppose that anything was wrong.«

»And there isn't now.«

»There are a number of peculiar circumstances, to say the least.«

»There's only Oliver—and I may have been mistaken about him.«

»But I thought you were so sure?«

»So I was. But—this is preposterous, Wimsey! Besides, think what a scandal it would make!«

»Why should it? You are the executor. You can make a private application and the whole thing can be done quite privately.«

»Yes, but surely the Home Office would never consent, on such flimsy grounds.«

»I'll see that they do. They'll know I wouldn't be keen on anything flimsy. Little bits of fluff were never in my line.«

»Oh, do be serious. What reason can we give?«

»Quite apart from Oliver, we can give a very good one. We can say that we want to examine the contents of the viscera to see how soon the General died after taking his last meal. That might be of great assistance in solving the question of the survivorship. And the law, generally speaking, is nuts on what they call the orderly devolution of property.«

»Hold on! D'you mean to say you can tell when a bloke died just by looking inside his tummy?«

»Not exactly, of course. But one might get an idea. If we found, that is, that he'd only that moment swallowed his brekker, it would show that he'd died not very long after arriving at the Club.«

»Good lord!—that would be a poor look-out for me.«

»It might be the other way, you know.«

»I don't like it, Wimsey. It's very unpleasant. I wish to goodness we could compromise on it.«

»But the lady in the case won't compromise. You know that. We've got to get at the facts somehow. I shall certainly get Murbles to suggest the exhumation to Pritchard.«

»Oh, lord! What'll \textit{he} do?«

»Pritchard? If he's an honest man and his client's an honest woman, they'll support the application. If they don't, I shall fancy they've something to conceal.«

»I wouldn't put it past them. They're a low-down lot. But they can't do anything without my consent, can they?«

»Not exactly—at least, not without a lot of trouble and publicity. But if \textit{you're} an honest man, you'll give your consent. \textit{You've} nothing to conceal, I suppose?«

»Of course not. Still, it seems rather\longdash«

»They suspect us already of some kind of dirty work,« persisted Wimsey. »That brute Pritchard as good as told me so. I'm expecting every day to hear that he has suggested exhumation off his own bat. I'd rather we got in first with it.«

»If that's the case, I suppose we must do it. But I can't believe it'll do a bit of good, and it's sure to get round and make an upheaval. Isn't there some other way—you're so darned clever\longdash«

»Look here, Fentiman. Do you want to get at the facts? Or are you out to collar the cash by hook or by crook? You may as well tell me frankly which it is.«

»Of course I want to get at the facts.«

»Very well; I've told you the next step to take.«

»Damn it all,« said Fentiman, discontentedly; »I suppose it'll have to be done, then. But I don't know whom to apply to or how to do it.«

»Sit down, then, and I'll dictate the letter for you.«

From this there was no escape, and Robert Fentiman did as he was told, grumbling.

»There's George. I ought to consult him.«

»It doesn't concern George, except indirectly. That's right. Now write to Murbles, telling him what you're doing and instructing him to let the other party know.«

»Oughtn't we to consult about the whole thing with Murbles first?«

»I've already consulted Murbles, and he agrees it's the thing to do.«

»These fellows would agree to anything that means fees and trouble.«

»Just so. Still, solicitors are necessary evils. Is that finished?«

»Yes.«

»Give the letters to me; I'll see they're posted. Now you needn't worry any more about it. Murbles and I will see to it all, and the detective-wallah is looking after Oliver all right, so you can run away and play.«

»You\longdash«

»I'm sure you're going to say how good it is of me to take all this trouble. Delighted, I'm sure. It's of no consequence. A pleasure, in fact. Have a drink.«

The disconcerted major refused the drink rather shortly and prepared to depart.

»You mustn't think I'm not grateful, Wimsey, and all that. But it is rather unseemly.«

»With all your experience,« said Wimsey, »you oughtn't to be so sensitive about corpses. We've seen many things much unseemlier than a nice, quiet little resurrection in a respectable cemetery.«

»Oh, I don't care twopence about the corpse,« retorted the Major, »but the thing doesn't look well. That's all.«

»Think of the money,« grinned Wimsey, shutting the door of the flat upon him.

He returned to the library, balancing the two letters in his hand. »There's many a man now walking the streets of London,« said he, »through not clearing trumps. Take these letters to the post, Bunter. And Mr Parker will be dining here with me this evening. We will have a \textit{perdrix aux choux} and a savoury to follow, and you can bring up two bottles of the Chambertin.«

»Very good, my lord.«

Wimsey's next proceeding was to write a little confidential note to an official whom he knew very well at the Home Office. This done, he returned to the telephone and asked for Penberthy's number.

»That you, Penberthy?\dots Wimsey speaking\textellipsis. Look here, old man, you know that Fentiman business?\dots Yes, well, we're applying for an exhumation.«

»For a \textit{what}?«

»An exhumation. Nothing to do with your certificate. We know \textit{that's} all right. It's just by way of getting a bit more information about when the beggar died.«

He outlined his suggestion.

»Think there's something in it?«

»There might be, of course.«

»Glad to hear you say that. I'm a layman in these matters, but it occurred to me as a good idea.«

»Very ingenious.«

»I always was a bright lad. You'll have to be present, of course.«

»Am I to do the autopsy?«

»If you like. Lubbock will do the analysis.«

»Analysis of what?«

»Contents of the doings. Whether he had kidneys on toast or eggs and bacon and all that.«

»Oh, I see. I doubt if you'll get much from that, after all this time.«

»Possibly not, but Lubbock had better have a squint at it.«

»Yes, certainly. As I gave the certificate, it's better that my findings should be checked by somebody.«

»Exactly. I knew you'd feel that way. You quite understand about it?«

»Perfectly. Of course, if we'd had any idea there was going to be all this uncertainty, I'd have made a post-mortem at the time.«

»Naturally you would. Well, it can't be helped. All in the day's work. I'll let you know when it's to be. I suppose the Home Office will send somebody along. I thought I ought just to let you know about it.«

»Very good of you. Yes. I'm glad to know. Hope nothing unpleasant will come out.«

»Thinking of your certificate?«

»Oh, well—no—I'm not worrying much about that. Though you never know, of course. I was thinking of that rigour, you know. Seen Captain Fentiman lately?«

»Yes. I didn't mention\longdash«

»No. Better not, unless it becomes absolutely necessary. Well, I'll hear from you later, then?«

»That's the idea. Good-bye.«

That day was a day of incident.

About four o'clock a messenger arrived, panting, from Mr Murbles. (Mr Murbles refused to have his chambers desecrated by a telephone.) Mr Murbles' compliments, and would Lord Peter be good enough to read this note and let Mr Murbles have an immediate answer.

The note ran:

\begin{samepage}
\begin{quotation}
\noindent Dear Lord Peter,\\
\nopagebreak[4]
\indent In re Fentiman deceased. Mr Pritchard has called. He informs me that his client is now willing to compromise on a division of the money if the Court will permit. Before I consult my client, Major Fentiman, I should be greatly obliged by your opinion as to how the investigation stands at present.
\begin{flushright}
Yours faithfully,\\
\nopagebreak[4]
\textsc{Jno. Murbles.}
\end{flushright}
\end{quotation}
\end{samepage}

Lord Peter replied as follows:
\begin{samepage}
\begin{quotation}
\noindent Dear Mr Murbles,

\indent Re Fentiman deceased. Too late to compromise now, unless you are willing to be party to a fraud. I warned you, you know. Robert has applied for exhumation. Can you dine with me at 8?
\begin{flushright}
\textsc{P. W.}
\end{flushright}
\end{quotation}
\end{samepage}

Having sent this off his lordship rang for Bunter.

»Bunter, as you know, I seldom drink champagne. But I am inclined to do so now. Bring a glass for yourself as well.«

The cork popped merrily, and Lord Peter rose to his feet.

»Bunter,« said he, »I give you a toast. The triumph of Instinct over Reason!«