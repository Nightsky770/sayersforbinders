%!TeX root=../bellonatop.tex
\chapter{Parker Plays a Hand}

\lettrine[lines=4,ante=‘]{N}{ow}, Mrs Mitcham,' said Inspector Parker, affably. He was always saying »Now, Mrs Somebody,« and he always remembered to say it affably. It was part of the routine.

\zz
The late Lady Dormer's housekeeper bowed frigidly, to indicate that she would submit to questioning.

»We want just to get the exact details of every little thing that happened to General Fentiman the day before he was found dead. I am sure you will help us. Do you recollect exactly what time he got here?«

»It would be round about a quarter to four—not later; I am sure I could not say exactly to the minute.«

»Who let him in?«

»The footman.«

»Did you see him then?«

»Yes; he was shown into the drawing-room, and I came down to him and brought him upstairs to her ladyship's bedroom.«

»Miss Dorland did not see him then?«

»No; she was sitting with her ladyship. She sent her excuses by me, and begged General Fentiman to come up.«

»Did the General seem quite well when you saw him?«

»So far as I could say he seemed well—always bearing in mind that he was a very old gentleman and had heard bad news.«

»He was not bluish about the lips, or breathing very heavily, or anything of that kind?«

»Well, going up the stairs tried him rather.«

»Yes, of course it would.«

»He stood still on the landing for a few minutes to get his breath. I asked him whether he would like to take something, but he said no, he was all right.«

»Ah! I daresay it would have been a good thing if he had accepted your very wise suggestion, Mrs Mitcham.«

»No doubt he knew best,« replied the housekeeper, primly. She considered that in making observations the policeman was stepping out of his sphere.

»And then you showed him in. Did you witness the meeting between himself and Lady Dormer?«

»I did not.« (emphatically). »Miss Dorland got up and said »How do you do, General Fentiman?« and shook hands with him, and then I left the room, as it was my place to do.«

»Just so. Was Miss Dorland alone with Lady Dormer when General Fentiman was announced?«

»Oh, no—the nurse was there.«

»The nurse—yes, of course. Did Miss Dorland and the nurse stay in the room all the time that the General was there?«

»No. Miss Dorland came out again in about five minutes and came downstairs. She came to me in the housekeeper's room, and she looked rather sad. She said, »Poor old dears,«—just like that.«

»Did she say any more?«

»She said: »They quarrelled, Mrs Mitcham, ages and ages ago, when they were quite young, and they've never seen each other since.« Of course, I was aware of that, having been with her ladyship all these years, and so was Miss Dorland.«

»I expect it would seem very pitiful to a young lady like Miss Dorland?«

»No doubt; she is a young lady with feelings; not like some of those you see nowadays.«

Parker wagged his head sympathetically.

»And then?«

»Then Miss Dorland went away again, after a little talk with me, and presently Nellie came in—that's the housemaid.«

»How long after was that?«

»Oh, some time. I had just finished my cup of tea which I have at four o'clock. It would be about half past. She came to ask for some brandy for the General, as he was feeling badly. The spirits are kept in my room, you see, and I have the key.«

Parker showed nothing of his special interest in this piece of news.

»Did you see the General when you took the brandy?«

»I did not take it.« Mrs Mitcham's tone implied that fetching and carrying was not part of her duty. »I sent it by Nellie.«

»I see. So you did not see the General again before he left?«

»No. Miss Dorland informed me later that he had had a heart attack.«

»I am very much obliged to you, Mrs Mitcham. Now I should like just to ask Nellie a few questions.«

Mrs Mitcham touched a bell. A fresh-faced pleasant-looking girl appeared in answer.

»Nellie, this police-officer wants you to give him some information about that time General Fentiman came here. You must tell him what he wants to know, but remember he is busy and don't start your chattering. You can speak to Nellie here, officer.«

And she sailed out.

»A bit stiff, isn't she?« murmured Parker, in an awestruck whisper.

»She's one of the old-fashioned sort, I don't mind saying,« agreed Nellie with a laugh.

»She put the wind up me. Now, Nellie\longdash« he took up the old formula, »I hear you were sent to get some brandy for the old gentleman. Who told you about it?«

»Why, it was like this. After the General had been with Lady Dormer getting on for an hour, the bell rang in her ladyship's room. It was my business to answer that, so I went up, and Nurse Armstrong put her head out and said, »Get me a drop of brandy, Nellie, quick, and ask Miss Dorland to come here. General Fentiman's rather unwell.« So I went for the brandy to Mrs Mitcham, and on the way up with it, I knocked at the studio door where Miss Dorland was.«

»Where's that, Nellie?«

»It's a big room on the first floor—built over the kitchen. It used to be a billiard-room in the old days, with a glass roof. That's where Miss Dorland does her painting and messing about with bottles and things, and she uses it as a sitting-room, too.«

»Messing about with bottles?«

»Well, chemists' stuff and things. Ladies have to have their hobbies, you know, not having any work to do. It makes a lot to clear up.«

»I'm sure it does. Well, go on, Nellie—I didn't mean to interrupt.«

»Well, I gave Nurse Armstrong's message, and Miss Dorland said, »Oh, dear, Nellie,« she said, »poor old gentleman. It's been too much for him. Give me the brandy, I'll take it along. And run along and get Dr Penberthy on the telephone.« So I gave her the brandy and she took it upstairs.«

»Half a moment. Did you see her take it upstairs?«

»Well, no, I don't think I actually saw her go up—but I thought she did. But I was going down to the telephone, so I didn't exactly notice.«

»No—why should you?«

»I had to look Dr Penberthy's number up in the book, of course. There was two numbers, and when I got his private house, they told me he was in Harley Street. While I was trying to get the second number Miss Dorland called over the stairs to me. She said »Have you got the doctor, Nellie?« And I said, »No, miss, not yet. The doctor's round in Harley Street.« And she said, »Oh! well, when you get him, say General Fentiman's had a bad turn and he's coming round to see him at once.« So I said, »Isn't the doctor to come here, miss?« And she said, »No; the General's better now and he says he would rather go round there. Tell William to get a taxi.« So she went back, and just then I got through to the surgery and said to Dr Penberthy's man to expect General Fentiman at once. And then he came downstairs with Miss Dorland and Nurse Armstrong holding on to him, and he looked mortal bad, poor old gentleman. William—the footman, you know, came in then and said he'd got the taxi, and he put General Fentiman into it, and then Miss Dorland and Nurse went upstairs again, and that was the end of it.«

»I see. How long have you been here, Nellie?«

»Three years—sir.« The »sir« was a concession to Parker's nice manners and educated way of speech. »Quite the gentleman,« as Nellie remarked afterwards to Mrs Mitcham, who replied, »No, Nellie—gentlemanlike I will not deny, but a policeman is a person, and I will trouble you to remember it.«

»Three years? That's a long time as things go nowadays. Is it a comfortable place?«

»Not bad. There's Mrs Mitcham, of course, but I know how to keep the right side of her. And the old lady—well, she was a real lady in every way.«

»And Miss Dorland?«

»Oh, she gives no trouble, except clearing up after her. But she always speaks nicely and says please and thank you. I haven't any complaints.«

»Modified rapture,« thought Parker. Apparently Ann Dorland had not the knack of inspiring passionate devotion. »Not a very lively house, is it, for a young girl like yourself?«

»Dull as ditchwater,« agreed Nellie, frankly. »Miss Dorland would have what they called studio parties sometimes, but not at all smart and nearly all young ladies—artists and such-like.«

»And naturally it's been quieter still since Lady Dormer died. Was Miss Dorland very much distressed at her death?«

Nellie hesitated.

»She was very sorry, of course; her ladyship was the only one she had in the world. And then she was worried with all this lawyer's business—something about the will, I expect you know, sir?«

»Yes, I know about that. Worried, was she?«

»Yes, and that angry—you wouldn't believe. There was one day Mr Pritchard came, I remember particular, because I happened to be dusting the hall at the time, you see, and she was speaking that quick and loud I couldn't help hearing. »I'll fight it for all I'm worth,« that was what she said and »a \textellipsis  something—to defraud«—what would that be, now?«

»Plot?« suggested Parker.

»No—a—a conspiracy, that's it. A conspiracy to defraud. And then I didn't hear any more till Mr Pritchard came out, and he said to her, »Very well, Miss Dorland, we will make an independent inquiry.« And Miss Dorland looked so eager and angry, I was surprised. But it all seemed to wear off, like. She hasn't been the same person the last week or so.«

»How do you mean?«

»Well, don't you notice it yourself, sir? She seems so quiet and almost frightened-like. As if she'd had a shock. And she cries a dreadful lot. She didn't do that at first.«

»How long has she been so upset?«

»Well, I think it was when all this dreadful business came out about the poor old gentleman being murdered. It is awful, sir, isn't it? Do you think you'll catch the one as did it?«

»Oh, I expect so,« said Parker, cheerfully. »That came as a shock to Miss Dorland, did it?«

»Well, I should say so. There was a little bit in the paper, you know, sir, about Sir James Lubbock having found out about the poisoning, and when I called Miss Dorland in the morning I took leave to point it out. I said, »That's a funny thing, miss, isn't it, about General Fentiman being poisoned,« just like that, I said. And she said, »Poisoned, Nellie? You must be mistaken.« So I showed her the bit in the paper and she looked just dreadful.«

»Well, well,« said Parker, »it's a very horrid thing to hear about a person one knows. Anybody would be upset.«

»Yes, sir; me and Mrs Mitcham was quite overcome. »Poor old gentleman,« I said, »whatever should anybody want to do him in for? He must have gone off his head and made away with himself,« I said. Do you think that was it, sir?«

»It's quite possible, of course,« said Parker, genially.

»Cut up about his sister dying like that, don't you think? That's what I said to Mrs Mitcham. But she said a gentleman like General Fentiman wouldn't make away with himself and leave his affairs in confusion like he did. So I said, »Was his affairs in confusion then?« and she said, »They're not your affairs, Nellie, so you needn't be discussing them.« What do you think yourself, sir?«

»I don't think anything yet,« said Parker, »but you have been very helpful. Now, would you kindly run and ask Miss Dorland if she could spare me a few minutes?«

Ann Dorland received him in the back drawing-room. He thought what an unattractive girl she was, with her sullen manner and gracelessness of form and movement. She sat huddled on one end of the sofa, in a black dress which made the worst of her sallow, blotched complexion. She had certainly been crying Parker thought, and when she spoke to him, it was curtly, in a voice roughened and hoarse and curiously lifeless.

»I am sorry to trouble you again,« said Parker, politely.

»You can't help yourself, I suppose.« She avoided his eyes, and lit a fresh cigarette from the stump of the last.

»I just want to have any details you can give me about General Fentiman's visit to his sister. Mrs Mitcham brought him up to her bedroom, I understand.«

She gave a sulky nod.

»You were there?«

She made no answer.

»Were you with Lady Dormer?« he insisted, rather more sharply.

»Yes.«

»And the nurse was there too?«

»Yes.«

She would not help him at all.

»What happened?«

»Nothing happened. I took him up to the bed and said, »Auntie, here's General Fentiman.««

»Lady Dormer was conscious, then?«

»Yes.«

»Very weak, of course?«

»Yes.«

»Did she say anything?«

»She said »Arthur!« that's all. And he said, »Felicity!« And I said, »You'd like to be alone,« and went out.«

»Leaving the nurse there?«

»I couldn't dictate to the nurse. She had to look after her patient.«

»Quite so. Did she stay there throughout the interview?«

»I haven't the least idea.«

»Well,« said Parker, patiently, »you can tell me this. When you went in with the brandy, the nurse was in the bedroom then?«

»Yes, she was.«

»Now, about the brandy. Nellie brought that up to you in the studio, she tells me.«

»Yes.«

»Did she come into the studio?«

»I don't understand.«

»Did she come right into the room, or did she knock at the door and did you come out to her on the landing?«

This roused the girl a little. »Decent servants don't knock at doors,« she said, with a contemptuous rudeness; »she came in, of course.«

»I beg your pardon,« retorted Parker, stung. »I thought she might have knocked at the door of your private room.«

»No.«

»What did she say to you?«

»Can't you ask \textit{her} all these questions?«

»I have done so. But servants are not always accurate; I should like your corroboration.« Parker had himself in hand again now, and spoke pleasantly.

»She said that Nurse Armstrong had sent her for some brandy, because General Fentiman was feeling faint, and told her to call me. So I said she had better go and telephone Dr Penberthy while I took the brandy.«

All this was muttered hurriedly, and in such a low tone that the detective could hardly catch the words.

»And then did you take the brandy straight upstairs?«

»Yes, of course.«

»Taking it straight out of Nellie's hands? Or did she put it down on the table or anywhere?«

»How the hell should I remember?«

Parker disliked a swearing woman, but he tried hard not to let this prejudice him.

»You can't remember—at any rate, you know you went straight on up with it? You didn't wait to do anything else?«

She seemed to pull herself together and make an effort to remember.

»If it's so important as that, I think I stopped to turn down something that was boiling.«

»Boiling? On the fire?«

»On the gas-ring,« she said, impatiently.

»What sort of thing.«

»Oh, nothing—some stuff.«

»Tea or cocoa, or something like that, do you mean?«

»No—some chemical things,« she said, letting the words go reluctantly.

»Were you making chemical experiments?«

»Yes—I did a bit—just for fun—a hobby, you know—I don't do anything at it now. I took up the brandy\longdash«

Her anxiety to shelve the subject of chemistry seemed to be conquering her reluctance to get on with the story.

»You were making chemical experiments—although Lady Dormer was so ill?« said Parker, severely.

»It was just to occupy my mind,« she muttered.

»What was the experiment?«

»I don't remember.«

»You can't remember at all?«

»\textsc{no}!« she almost shouted at him.

»Never mind. You took the brandy upstairs?«

»Yes—at least, it isn't really upstairs. It's all on the same landing, only there are six steps up to Auntie's room. Nurse Armstrong met me at the door, and said »He's better now,« and I went in and saw General Fentiman sitting in a chair, looking very queer and gray. He was behind a screen where Auntie couldn't see him, or it would have been a great shock to her. Nurse said, »I've given him his drops and I think a little brandy will put him right again.« So we gave him the brandy—only a small dose, and after a bit, he got less deathly-looking and seemed to be breathing better. I told him we were sending for the doctor, and he said he'd rather go round to Harley Street. I thought it was rash, but Nurse Armstrong said he seemed really better, and it would be a mistake to worry him into doing what he didn't want. So I told Nellie to warn the doctor and send William for a taxi. General Fentiman seemed stronger then, so we helped him downstairs and he went off in the taxi.«

Out of this spate of words, Parker fixed on the one thing he had not heard before.

»What drops were those the Nurse gave him?«

»His own. He had them in his pocket.«

»Do you think she could possibly have given him too much? Was the quantity marked on the bottle?«

»I haven't the remotest idea. You'd better ask her.«

»Yes, I shall want to see her, if you will kindly tell me where to find her.«

»I've got the address upstairs. Is that all you want?«

»I should just like, if I may, to see Lady Dormer's room and the studio.«

»What for?«

»It's just a matter of routine. We are under orders to see everything there is to see,« replied Parker, reassuringly.

They went upstairs. A door on the first-floor landing immediately opposite the head of the staircase led into a pleasant, lofty room, with old-fashioned bedroom furniture in it.

»This is my aunt's room. She wasn't really my aunt, of course, but I called her so.«

»Quite. Where does that second door lead to?«

»That's the dressing-room. Nurse Armstrong slept there while Auntie was ill.«

Parker glanced in to the dressing-room, took in the arrangement of the bedroom and expressed himself satisfied.

She walked past him without acknowledgment while he held the door open. She was a sturdily-built girl, but moved with a languor distressing to watch—slouching, almost aggressively unalluring.

»You want to see the studio?«

»Please.«

She led the way down the six steps and along a short passage to the room which, as Parker already knew, was built out at the back over the kitchen premises. He mentally calculated the distance as he went.

The studio was large and well-lit by its glass roof. One end was furnished like a sitting-room; the other was left bare, and devoted to what Nellie called »mess.« A very ugly picture (in Parker's opinion) stood on an easel. Other canvases were stacked round the walls. In one corner was a table covered with American cloth, on which stood a gas-ring, protected by a tin plate, and a Bunsen burner.

»I'll look up that address,« said Miss Dorland, indifferently, »I've got it here somewhere.«

She began to rummage in an untidy desk. Parker strolled up to the business end of the room, and explored it with eyes, nose and fingers.

The ugly picture on the easel was newly-painted; the smell told him that, and the dabs of paint on the palette were still soft and sticky. Work had been done there within the last two days, he was sure. The brushes had been stuck at random into a small pot of turpentine. He lifted them out; they were still clogged with paint. The picture itself was a landscape, he thought, roughly drawn and hot and restless in colour. Parker was no judge of art; he would have liked to get Wimsey's opinion. He explored further. The table with the Bunsen burner was bare, but in a cupboard close by he discovered a quantity of chemical apparatus of the kind he remembered using at school. Everything had been tidily washed and stacked away. Nellie's job, he imagined. There were a number of simple and familiar chemical substances in jars and packages, occupying a couple of shelves. They would probably have to be analysed, he thought, to see if they were all they seemed. And what useless nonsense it all was, he thought to himself; anything suspicious would obviously have been destroyed weeks before. Still, there it was. A book in several volumes on the top shelf caught his attention: it was Quain's Dictionary of Medicine. He took down a volume in which he noticed a paper mark. Opening it at the marked place, his eye fell upon the words: »rigour Mortis,« and, a little later on—»action of certain poisons.« He was about to read more, when he heard Miss Dorland's voice just behind him.

»That's all nothing,« she said, »I don't do any of that muck now. It was just a passing craze. I paint, really. What do you think of this?« She indicated the unpleasant landscape.

Parker said it was very good.

»Are these your work, too?« he asked, indicating the other canvases.

»Yes,« she said.

He turned a few of them to the light, noticing at the same time how dusty they were. Nellie had scamped this bit of the work—or perhaps had been told not to touch. Miss Dorland showed a trifle more animation than she had done hitherto, while displaying her works. Landscape seemed to be rather a new departure; most of the canvases were figure-studies. Mr Parker thought that on the whole, the artist had done wisely to turn to landscape. He was not well acquainted with the modern school of thought in painting, and had difficulty in expressing his opinion of these curious figures, with their faces like eggs and their limbs like rubber.

»That is the Judgement of Paris,« said Miss Dorland.

»Oh, yes,« said Parker. »And this?«

»Oh, just a study of a woman dressing. It's not very good. I think this portrait of Mrs Mitcham is rather decent, though.«

Parker stared aghast; it might possibly be a symbolic representation of Mrs Mitcham's character, for it was very hard and spiky; but it looked more like a Dutch doll, with its triangular nose, like a sharp-edged block of wood, and its eyes mere dots in an expanse of liver-coloured cheek.

»It's not very like her,« he said, doubtfully.

»It's not meant to be.«

»This seems better—I mean, I like this better,« said Parker, turning the next picture up hurriedly.

»Oh, that's nothing—just a fancy head.«

Evidently this picture—the head of a rather cadaverous man, with a sinister smile and a slight cast in the eye—was despised—a Philistine backsliding, almost like a human being. It was put away, and Parker tried to concentrate his attention on a »Madonna and Child« which, to Parker's simple evangelical mind, seemed an abominable blasphemy.

Happily, Miss Dorland soon wearied, even of her paintings, and flung them all back into the corner.

»D'you want anything else?« she demanded abruptly. »Here's that address.«

Parker took it.

»Just one more question,« he said, looking her hard in the eyes. »Before Lady Dormer died—before General Fentiman came to see her—did you know what provision she had made for you and for him in her will?«

The girl stared back at him, and he saw panic come into her eyes. It seemed to flow all over her like a wave. She clenched her hands at her sides, and her miserable eyes dropped beneath his gaze, shifting as though looking for a way out.

»Well?« said Parker.

»No!« she said. »No! of course not. Why should I?« Then, surprisingly, a dull crimson flush flooded her sallow cheeks and ebbed away, leaving her looking like death.

»Go away,« she said, furiously, »you make me sick.«