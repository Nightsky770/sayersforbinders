%!TeX root=../bellonatop.tex
\chapter{Lord Peter Calls a Bluff}

\lettrine[lines=4,ante=‘]{I}{t} is new to me,' said Lord Peter, glancing from the back window of the taxi at the other taxi which was following them, »to be shadowed by the police, but it amuses them and doesn't hurt us.«

\zz
He was revolving ways and means of proof in his mind. Unhappily, all the evidence in favour of Ann Dorland was evidence against her as well—except, indeed, the letter to Pritchard. Damn Penberthy. The best that could be hoped for now was that the girl should escape from public inquiry with a verdict of »Not proven.« Even if acquitted—even if never charged with the murder—she would always be suspect. The question was not one which could be conveniently settled by a brilliant flash of deductive logic, or the discovery of a blood-stained thumb-mark. It was a case for lawyers to argue—for a weighing of the emotional situation by twelve good and lawful persons. Presumably the association could be proved—the couple had met and dined together; probably the quarrel could be proved—but what next? Would a jury believe in the cause of the quarrel? Would they think it a pre-arranged blind, or perhaps—or mistake it for the falling-out of rogues among themselves? What would they think of this plain, sulky, inarticulate girl, who had never had any real friends, and whose clumsy and tentative graspings after passion had been so obscure, so disastrous?

Penberthy, too—but Penberthy was easier to understand. Penberthy, cynical and bored with poverty, found himself in contact with this girl, who might be so well-off some day. And Penberthy, the physician, would not mistake the need for passion that made the girl such easy stuff to work on. So he carried on—bored with the girl, of course—keeping it all secret, till he saw which way the cat was going to jump. Then the old man—the truth about the will—the opportunity. And then, upsettingly, Robert\textellipsis Would the jury see it like that?

Wimsey leaned out of the cab window and told the driver to go to the Savoy. When they arrived, he handed the girl over to the cloak-room attendant. »I'm going up to change,« he added, and turning, had the pleasure of seeing his sleuth arguing with the porter in the entrance-hall.

Bunter, previously summoned by telephone, was already in attendance with his master's dress clothes. Having changed, Wimsey passed through the hall again. The sleuth was there, quietly waiting. Wimsey grinned at him, and offered him a drink.

»I can't help it, my lord,« said the detective.

»Of course not; you've sent for a bloke in a boiled shirt to take your place, I suppose?«

»Yes, my lord.«

»More power to his elbow. So long.«

He rejoined his charge and they went into the dining-room. Dressed in a green which did not suit her, she was undoubtedly plain. But she had character; he was not ashamed of her. He offered her the menu.

»What shall it be?« he asked. »Lobster and champagne?«

She laughed at him.

»Marjorie says you are an authority on food. I don't believe authorities on food ever take lobster and champagne. Anyway, I don't like lobster, much. Surely there's something they do best here, isn't there? Let's have that.«

»You show the right spirit,« said Wimsey. »I will compose a dinner for you.«

He called the head waiter, and went into the question scientifically.

»\textit{Huîtres Musgrave}—I am opposed on principle to the cooking of oysters—but it is a dish so excellent that one may depart from the rules in its favour. Fried in their shells, Miss Dorland, with little strips of bacon. Shall we try it?—The soup must be \textit{Tortue Vraie}, of course. The fish—oh! just a \textit{Filet de Sole}, the merest mouthful, a hyphen between the prologue and the main theme.«

»That all sounds delightful. And what is the main theme to be?«

»I think a \textit{Faisan Rôti} with \textit{Pommes Byron}. And a salad to promote digestion. And, waiter—be sure the salad is dry and perfectly crisp. A \textit{Soufflé Glace} to finish up with. And bring me the wine-list.«

They talked. When she was not on the defensive, the girl was pleasant enough in manner; a trifle downright and aggressive, perhaps, in her opinions, but needing only mellowing.

»What do you think of the \textit{Romanée Conti}?« he asked, suddenly.

»I don't know much about wine. It's good. Not sweet, like Sauterne. It's a little—well—harsh. But it's harsh without being thin—quite different from that horrid Chianti people always seem to drink at Chelsea parties.«

»You're right; it's rather unfinished, but it has plenty of body—it'll be a grand wine in ten years' time. It's 1915. Now, you see. Waiter, take this away and bring me a bottle of the 1908.«

He leaned towards his companion.

»Miss Dorland—may I be impertinent?«

»How? Why?«

»Not an artist, not a bohemian, and not a professional man;—a man of the world.«

»What \textit{do} you mean by those cryptic words?«

»For you. That is the kind of man who is going to like you very much. Look! that wine I've sent away—it's no good for a champagne-and-lobster sort of person, nor for very young people—it's too big and rough. But it's got the essential guts. So have you. It takes a fairly experienced palate to appreciate it. But you and it will come into your own one day. Get me?«

»Do you think so?«

»Yes. But your man won't be at all the sort of person you're expecting. You have always thought of being dominated by somebody, haven't you?«

»Well\longdash«

»But you'll find that yours will be the leading brain of the two. He will take great pride in the fact. And you will find the man reliable and kind, and it will turn out quite well.«

»I didn't know you were a prophet.«

»I am, though.«

Wimsey took the bottle of 1908 from the waiter and glanced over the girl's head at the door. A man in a boiled shirt was making his way in, accompanied by the manager.

»I am a prophet,« said Wimsey. »Listen. Something tiresome is going to happen—now, this minute. But don't worry. Drink your wine, and trust.«

The manager had brought the man to their table. It was Parker.

»Ah!« said Wimsey, brightly. »You'll forgive our starting without you, old man. Sit down. I think you know Miss Dorland.«

»Have you come to arrest me?« asked Ann.

»Just to ask you to come down to the Yard with me,« said Parker, smiling pleasantly and unfolding his napkin.

Ann looked palely at Wimsey, and took a gulp of the wine.

»Right,« said Wimsey. »Miss Dorland has quite a lot to tell you. After dinner will suit us charmingly. What will you have?«

Parker, who was not imaginative, demanded a grilled steak.

»Shall we find any other friends at the Yard?« pursued Wimsey.

»Possibly,« said Parker.

»Well, cheer up! You put me off my food, looking so grim. Hullo! Yes, waiter, what is it?«

»Excuse me, my lord; is this gentleman Detective-Inspector Parker?«

»Yes, yes,« said Parker, »what's the matter?«

»You're wanted on the `phone, sir.«

Parker departed.

»It's all right,« said Wimsey to the girl. »I know you're straight, and I'll damn well see you through.«

»What am I to do?«

»Tell the truth.«

»It sounds so silly.«

»They've heard lots of very much sillier stories than that.«

»But—I don't want to—to be the one to\longdash«

»You're still fond of him, then?«

»\textit{No!}—but I'd rather it wasn't me.«

»I'll be frank with you. I think it's going to be between you and him that suspicion will lie.«

»In that case«—she set her teeth—»he can have what's coming to him.«

»Thank the lord! I thought you were going to be noble and self-sacrificing and tiresome. You know. Like the people whose noble motives are misunderstood in chapter one and who get dozens of people tangled up in their miserable affairs till the family lawyer solves everything on the last page but two.«

Parker had come back from the telephone.

»Just a moment!« He spoke in Peter's ear.

»Hullo?«

»Look here; this is awkward. George Fentiman\longdash«

»Yes?«

»He's been found in Clerkenwell.«

»Clerkenwell?«

»Yes; must have wandered back by `bus or something. He's at the police-station; in fact he's given himself up.«

»Good lord!«

»For the murder of his grandfather.«

»The devil he has!«

»It's a nuisance; of course it must be looked into. I think perhaps I'd better put off interrogating Dorland and Penberthy. What are you doing with the girl, by the way?«

»I'll explain later. Look here—I'll take Miss Dorland back to Marjorie Phelps' place, and then come along and join you. The girl won't run away; I know that. And anyhow, you've got a man looking after her.«

»Yes, I rather wish you would come with me; Fentiman is pretty queer, by all accounts. We've sent for his wife.«

»Right. You buzz off, and I'll join you in—say in three quarters of an hour. What address? Oh, yes, righty-ho! Sorry you're missing your dinner.«

»It's all in the day's work,« growled Parker, and took his leave.

George Fentiman greeted them with a tired white smile.

»Hush!« he said. »I've told them all about it. \textit{He's} asleep; don't wake him.«

»Who's asleep, dearest?« said Sheila.

»I mustn't say the name,« said George, cunningly. »He'd hear it—even in his sleep—even if you whispered it. But he's tired, and he nodded off. So I ran in here and told them all about it while he snored.«

The police superintendent tapped his forehead significantly behind Sheila's back.

»Has he made any statement?« asked Parker.

»Yes, he insisted on writing it himself. Here it is. Of course\dots« the Superintendent shrugged his shoulders.

»That's all right,« said George. »I'm getting sleepy myself. I've been watching him for a day and a night, you know. I'm going to bed. Sheila—it's time to go to bed.«

»Yes, dear.«

»We'll have to keep him here to-night, I suppose,« muttered Parker. »Has the doctor seen him?«

»We've sent for him, sir.«

»Well, Mrs Fentiman, I think if you'd take your husband into the room the officer will show you, that would be the best way. And we'll send the doctor in to you when he arrives. Perhaps it would be as well that he should see his own medical man too. Whom would you like us to send for?«

»Dr Penberthy has vetted him from time to time, I think,« put in Wimsey, suddenly. »Why not send for him?«

Parker gasped involuntarily.

»He might be able to throw some light on the symptoms,« said Wimsey, in a rigid voice.

Parker nodded.

»A good idea,« he agreed. He moved to the telephone. George smiled as his wife put her arm about his shoulder.

»Tired,« he said, »very tired. Off to bed, old girl.«

A police-constable opened the door to them, and they started through it together; George leaned heavily on Sheila; his feet dragged.

»Let's have a look at his statement,« said Parker.

It was written in a staggering handwriting, much blotted and erased, with words left out and repeated here and there.

\begin{quote}
I am making this statement quickly while he is asleep, because if I wait he may wake up and stop me. You will say I was moved and seduced by instigation of but what they will not understand is that he is me and I am him. I killed my grandfather by giving him digitalin. I did not remember it till I saw the name on the bottle, but they have been looking for me ever since, so I know that he must have done it. That is why they began following me about, but he is very clever and misleads them. When he is awake. We were dancing all last night and that is why he is tired. He told me to smash the bottle so that you shouldn't find out, but they know I was the last person to see him. He is very cunning, but if you creep on him quickly now that he is asleep you will be able to bind him in chains and cast him into the pit and then I shall be able to sleep.
\begin{flushright}
\textsc{George Fentiman.}
\end{flushright}
\end{quote}

»Off his head, poor devil,« said Parker. »We can't pay much attention to this. What did he say to you, superintendent?«

»He just came in, sir, and said »I'm George Fentiman and I've come to tell you about how I killed my grandfather.« So I questioned him, and he rambled a good bit and then he asked for a pen and paper to make his statement. I thought he ought to be detained, and I rang up the Yard, sir.«

»Quite right,« said Parker.

The door opened and Sheila came out.

»He's fallen asleep,« she said. »It's the old trouble come back again. He thinks he's the devil, you know. He's been like that twice before,« she added, simply. »I'll go back to him till the doctors come.«

The police-surgeon arrived first and went in; then, after a wait of a quarter of an hour, Penberthy came. He looked worried, and greeted Wimsey abruptly. Then he, too, went into the inner room. The others stood vaguely about, and were presently joined by Robert Fentiman, whom an urgent summons had traced to a friend's house.

Presently the two doctors came out again.

»Nervous shock with well-marked delusions,« said the police-surgeon, briefly. »Probably be all right to-morrow. Sleeping it off now. Been this way before, I understand. Just so. A hundred years ago they'd have called it diabolic possession, but \textit{we} know better.«

»Yes,« said Parker, »but do you think he is under a delusion in saying he murdered his grandfather? Or did he actually murder him under the influence of this diabolical delusion? That's the point.«

»Can't say just at present. Might be the one—might be the other. Much better wait till the attack passes off. You'll be able to find out better then.«

»You don't think he's permanently—insane, then?« demanded Robert, with brusque anxiety.

»No—I don't. I think it's what you'd call a nerve-storm. That is your opinion, too, I believe?« he added, turning to Penberthy.

»Yes; that is my opinion.«

»And what do you think about this delusion, Dr Penberthy?« went on Parker. »Did he do this insane act?«

»He certainly thinks he did it,« said Penberthy; »I couldn't possibly say for certain whether he has any foundation for the belief. From time to time he undoubtedly gets these fits of thinking that the devil has taken hold of him, and of course it's hard to say what a man might or might not do under the influence of such a delusion.«

He avoided Robert's distressed eyes, and addressed himself exclusively to Parker.

»It seems to me,« said Wimsey, »if you'll excuse me pushin' my opinion forward and all that—it seems to me that's a question of fact that can be settled without reference to Fentiman and his delusions. There's only the one occasion on which the pill could have been administered—would it have produced the effect that was produced at that particular time, or wouldn't it? If it couldn't take effect at 8 o'clock, then it couldn't, and there's an end of it.«

He kept his eyes fixed on Penberthy, and saw him pass his tongue over his dry lips before speaking.

»I can't answer that off-hand,« he said.

»The pill might have been introduced into General Fentiman's stock of pills at some other time,« suggested Parker.

»So it might,« agreed Penberthy.

»Had it the same shape and appearance as his ordinary pills?« demanded Wimsey, again fixing his eyes on Penberthy.

»Not having seen the pill in question, I can't say,« said the latter.

»In any case,« said Wimsey, »the pill in question, which was one of Mrs Fentiman's, I understand, had strychnine in it as well as digitalin. The analysis of the stomach would no doubt have revealed strychnine if present. That can be looked into.«

»Of course,« said the police-surgeon. »Well, gentlemen, I don't think we can do much more to-night. I have written out a prescription for the patient, with Dr Penberthy's entire agreement«—he bowed; Penberthy bowed—»I will have it made up, and you will no doubt see that it is given to him. I shall be here in the morning.«

He looked interrogatively at Parker, who nodded.

»Thank you, doctor; we will ask you for a further report to-morrow morning. You'll see that Mrs Fentiman is properly looked after, Superintendent. If you wish to stay here and look after your brother and Mrs Fentiman, Major, of course you may, and the Superintendent will make you as comfortable as he can.«

Wimsey took Penberthy by the arm.

»Come round to the Club with me for a moment, Penberthy,« he said. »I want to have a word with you.«