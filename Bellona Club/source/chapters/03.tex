%!TeX root=../bellonatop.tex
\chapter{Hearts Count More Than Diamonds}
\lettrine[lines=4]{A}{bout} ten days after that notable Armistice Day, Lord Peter Wimsey was sitting in his library, reading a rare fourteenth century manuscript of Justinian. It gave him particular pleasure, being embellished with a large number of drawings in sepia, extremely delicate in workmanship, and not always equally so in subject. Beside him on a convenient table stood a long-necked decanter of priceless old port. From time to time he stimulated his interest with a few sips, pursing his lips thoughtfully, and slowly savouring the balmy after-taste.

A ring at the front door of the flat caused him to exclaim »Oh, hell!« and cock an attentive ear for the intruder's voice. Apparently the result was satisfactory, for he closed the Justinian and had assumed a welcoming smile when the door opened.

»Mr Murbles, my lord.«

The little elderly gentleman who entered was so perfectly the family solicitor as really to have no distinguishing personality at all, beyond a great kindliness of heart and a weakness for soda-mint lozenges.

»I am not disturbing you, I trust, Lord Peter.«

»Good lord, no, sir. Always delighted to see you. Bunter, a glass for Mr Murbles. Very glad you've turned up, sir. The Cockburn '80 always tastes a lot better in company—discernin' company, that is. Once knew a fellow who polluted it with a Trichinopoly. He was not asked again. Eight months later, he committed suicide. I don't say it was on that account. But he was earmarked for a bad end, what?«

»You horrify me,« said Mr Murbles, gravely. »I have seen many men sent to the gallows for crimes with which I could feel much more sympathy. Thank you, Bunter, thank you. You are quite well, I trust?«

»I am in excellent health, I am obliged to you, sir.«

»That's good. Been doing any photography lately?«

»A certain amount, sir. But merely of a pictorial description, if I may venture to call it so. Criminological material, sir, has been distressingly deficient of late.«

»Perhaps Mr Murbles has brought us something,« suggested Wimsey.

»No,« said Mr Murbles, holding the Cockburn '80 beneath his nostrils and gently agitating the glass to release the ethers, »no, I can't say I have, precisely. I will not disguise that I have come in the hope of deriving benefit from your trained habits of observation and deduction, but I fear—that is, I trust—in fact, I am confident—that nothing of an undesirable nature is involved. The fact is,« he went on, as the door closed upon the retreating Bunter, »a curious question has arisen with regard to the sad death of General Fentiman at the Bellona Club, to which, I understand, you were a witness.«

»If you understand that, Murbles,« said his lordship, cryptically, »you understand a damn sight more than I do. I did not witness the death—I witnessed the discovery of the death—which is a very different thing, by a long chalk.«

»By how long a chalk?« asked Mr Murbles, eagerly. »That is just what I am trying to find out.«

»That's very inquisitive of you,« said Wimsey. »I think perhaps it would be better ...« he lifted his glass and tilted it thoughtfully, watching the wine coil down in thin flower-petallings from rim to stem \textellipsis  »if you were to tell me exactly what you want to know \textellipsis  and why. After all \textellipsis  I'm a member of the Club \textellipsis  family associations chiefly, I suppose \textellipsis  but there it is.«

Mr Murbles looked up sharply, but Wimsey's attention seemed focussed upon the port.

»Quite so,« said the solicitor. »Very well. The facts of the matter are these. General Fentiman had, as you know, a sister Felicity, twelve years younger than himself. She was very beautiful and very wilful as a girl, and ought to have made a very fine match, but for the fact that the Fentimans, though extremely well-descended, were anything but well-off. As usual at that period, all the money there was went to educating the boy, buying him a commission in a crack regiment and supporting him there in the style which was considered indispensable for a Fentiman. Consequently there was nothing left to furnish a marriage-portion for Felicity, and that was rather disastrous for a young woman sixty years ago.

Well, Felicity got tired of being dragged through the social round in her darned muslins and gloves that had been to the cleaners—and she had the spirit to resent her mother's perpetual strategies in the match-making line. There was a dreadful, decrepit old viscount, eaten up with diseases and dissipations, who would have been delighted to totter to the altar with a handsome young creature of eighteen, and I am sorry to say that the girl's father and mother did everything they could to force her into accepting this disgraceful proposal. In fact, the engagement was announced and the wedding-day fixed, when, to the extreme horror of her family, Felicity calmly informed them one morning that she had gone out before breakfast and actually got married, in the most indecent secrecy and haste, to a middle-aged man called Dormer, very honest and abundantly wealthy, and—horrid to relate—a prosperous manufacturer. Buttons, in fact—made of papier mâché or something, with a patent indestructible shank—were the revolting antecedents to which this head-strong young Victorian had allied herself.

Naturally there was a terrible scandal, and the parents did their best—seeing that Felicity was a minor—to get the marriage annulled. However, Felicity checkmated their plans pretty effectually by escaping from her bedroom—I fear, indeed, that she actually climbed down a tree in the back-garden, crinoline and all—and running away with her husband. After which, seeing that the worst had happened—indeed, Dormer, a man of prompt action, lost no time in putting his bride in the family way—the old people put the best face they could on it in the grand Victorian manner. That is, they gave their consent to the marriage, forwarded their daughter's belongings to her new home in Manchester, and forbade her to darken their doors again.«

»Highly proper,« murmured Wimsey. »I'm determined never to be a parent. Modern manners and the break-up of the fine old traditions have simply ruined the business. I shall devote my life and fortune to the endowment of research on the best method of producin' human beings decorously and unobtrusively from eggs. All parental responsibility to devolve upon the incubator.«

»I hope not,« said Mr Murbles. »My own profession is largely supported by domestic entanglements. To proceed. Young Arthur Fentiman seems to have shared the family views. He was disgusted at having a brother-in-law in buttons, and the jests of his mess-mates did nothing to sweeten his feelings toward his sister. He became impenetrably military and professional, crusted over before his time, and refused to acknowledge the existence of anybody called Dormer. Mind you, the old boy was a fine soldier, and absolutely wrapped up in his Army associations. In due course he married—not well, for he had not the means to entitle himself to a noble wife, and he would not demean himself by marrying money, like the unspeakable Felicity. He married a suitable gentlewoman with a few thousand pounds. She died (largely, I believe, owing to the military regularity with which her husband ordained that she should perform her maternal functions), leaving a numerous but feeble family of children. Of these, the only one to attain maturity was the father of the two Fentimans you know—Major Robert and Captain George Fentiman.«

»I don't know Robert very well,« interjected Wimsey. »I've met him. Frightfully hearty and all that—regular army type.«

»Yes, he's of the old Fentiman stock. Poor George inherited a weakly strain from his grandmother, I'm afraid.«

»Well, nervous, anyhow,« said Wimsey, who knew better than the old solicitor the kind of mental and physical strain George Fentiman had undergone. The War pressed hardly upon imaginative men in responsible positions. »And then he was gassed and all that, you know,« he added, apologetically.

»Just so,« said Mr Murbles. »Robert, you know, is unmarried and still in the Army. He's not particularly well-off, naturally, for none of the Fentimans ever had a bean, as I believe one says nowadays; but he does very well. George\longdash«

»Poor old George! All right, sir, you needn't tell me about him. Usual story. Decentish job—imprudent marriage—chucks everything to join up in 1914—invalided out—job gone—health gone—no money—heroic wife keeping the home-fires burning—general fed-upness. Don't let's harrow our feelings. Take it as read.«

»Yes, I needn't go into that. Their father is dead, of course, and up till ten days ago there were just two surviving Fentimans of the earlier generation. The old General lived on the small fixed income which came to him through his wife and his retired pension. He had a solitary little flat in Dover Street and an elderly man-servant, and he practically lived at the Bellona Club. And there was his sister, Felicity.«

»How did she come to be Lady Dormer?«

»Why, that's where we come to the interesting part of the story. Henry Dormer\longdash«

»The button-maker?«

»The button-maker. He became an exceedingly rich man indeed—so rich, in fact, that he was able to offer financial assistance to certain exalted persons who need not be mentioned and so, in time, and in consideration of valuable services to the nation not very clearly specified in the Honours List, he became Sir Henry Dormer, Bart. His only child—a girl—had died, and there was no prospect of any further family, so there was, of course, no reason why he should not be made a baronet for his trouble.«

»Acid man you are,« said Wimsey. »No reverence, no simple faith or anything of that kind. Do lawyers ever go to heaven?«

»I have no information on that point,« said Mr Murbles, dryly. »Lady Dormer\longdash«

»Did the marriage turn out well otherwise?« inquired Wimsey.

»I believe it was perfectly happy,« replied the lawyer, »an unfortunate circumstance in one way, since it entirely precluded the possibility of any reconciliation with her relatives. Lady Dormer, who was a fine, generous-hearted woman, frequently made overtures of peace, but the General held sternly aloof. So did his son—partly out of respect for the old boy's wishes, but chiefly, I fancy, because he belonged to an Indian regiment and spent most of his time abroad. Robert Fentiman, however, showed the old lady a certain amount of attention, paying occasional visits and so forth, and so did George at one time. Of course they never let the General know a word about it, or he would have had a fit. After the War, George rather dropped his great-aunt—I don't know why.«

»I can guess,« said Wimsey. »No job—no money, y' know. Didn't want to look pointed. That sort of thing, what?«

»Possibly. Or there may have been some kind of quarrel. I don't know. Anyway, those are the facts. I hope I am not boring you, by the way?«

»I am bearing up,« said Wimsey, »waiting for the point where the Money comes in. There's a steely legal glitter in your eye, sir, which suggests that the thrill is not far off.«

»Quite correct,« said Mr Murbles. »I now come—thank you, well, yes—I will take just one more glass. I thank Providence I am not of a gouty constitution. Yes. Ah!—We now come to the melancholy event of November 11\textsuperscript{th} last, and I must ask you to follow me with the closest attention.«

»By all means,« said Wimsey, politely.

»Lady Dormer,« pursued Mr Murbles, leaning earnestly forward, and punctuating every sentence with sharp little jabs of his gold-mounted eye-glasses, held in his right finger and thumb, »was an old woman, and had been ailing for a very long time. However, she was still the same head-strong and vivacious personality that she had been as a girl, and on the fifth of November she was suddenly seized with a fancy to go out at night and see a display of fireworks at the Crystal Palace or some such place—it may have been Hampstead Heath or the White City—I forget, and it is of no consequence. The important thing is, that it was a raw, cold evening. She insisted on undertaking her little expedition nevertheless, enjoyed the entertainment as heartily as the youngest child, imprudently exposed herself to the night air and caught a severe cold which, in two days' time, turned to pneumonia. On November  10\textsuperscript{th} she was sinking fast, and scarcely expected to live out the night. Accordingly, the young lady who lived with her as her ward—a distant relative, Miss Ann Dorland—sent a message to General Fentiman that if he wished to see his sister alive, he should come immediately. For the sake of our common human nature, I am happy to say that this news broke down the barrier of pride and obstinacy that had kept the old gentleman away so long. He came, found Lady Dormer just conscious, though very feeble, stayed with her about half an hour and departed, still stiff as a ramrod, but visibly softened. This was about four o'clock in the afternoon. Shortly afterwards, Lady Dormer became unconscious, and, indeed, never moved or spoke again, passing peacefully away in her sleep at half-past ten the following morning.

Presumably the shock and nervous strain of the interview with his long-estranged sister had been too much for the old General's feeble system, for, as you know, he died at the Bellona Club at some time—not yet clearly ascertained—on the same day, the eleventh of November.

Now then, at last—and you have been very patient with my tedious way of explaining all this—we come to the point at which we want your help.«

Mr Murbles refreshed himself with a sip of port, and, looking a little anxiously at Wimsey, who had closed his eyes and appeared to be nearly asleep, he resumed.

»I have not mentioned, I think, how I come to be involved in this matter myself. My father was the Fentimans' family solicitor, a position to which I naturally succeeded when I took over the business at his death. General Fentiman, though he had little enough to leave, was not the sort of disorderly person who dies without making a proper testamentary disposition. His retired pension, of course, died with him, but his small private estate was properly disposed by will. There was a small legacy—fifty pounds—to his man-servant (a very attached and superior fellow); then one or two trifling bequests to old military friends and the servants at the Bellona Club (rings, medals, weapons and small sums of a few pounds each). Then came the bulk of his estate, about \textsterling 2,000, invested in sound securities, and bringing in an income of slightly over \textsterling 100 per annum. These securities, specifically named and enumerated, were left to Captain George Fentiman, the younger grandson, in a very proper clause, which stated that the testator intended no slight in thus passing over the elder one, Major Robert, but that, as George stood in the greater need of monetary help, being disabled, married, and so forth, whereas his brother had his profession and was without ties, George's greater necessity gave him the better claim to such money as there was. Robert was finally named as executor and residuary legatee, thus succeeding to all such personal effects and monies as were not specifically devised elsewhere. Is that clear?«

»Clear as a bell. Was Robert satisfied with that arrangement?«

»Oh dear, yes; perfectly. He knew all about the will beforehand and had agreed that it was quite fair and right.«

»Nevertheless,« said Wimsey, »it appears to be such a small matter, on the face of it, that you must be concealing something perfectly devastating up your sleeve. Out with it, man, out with it! Whatever the shock may be, I am braced to bear it.«

»The shock,« said Mr Murbles, »was inflicted on me, personally, last Friday by Lady Dormer's man of business—Mr Pritchard of Lincoln's Inn. He wrote to me, asking if I could inform him of the exact hour and minute of General Fentiman's decease. I replied, of course, that, owing to the peculiar circumstances under which the event took place, I was unable to answer his question as precisely as I could have wished, but that I understood Dr Penberthy to have given it as his opinion that the General had died some time in the forenoon of November  11\textsuperscript{th}. Mr Pritchard then asked if he might wait upon me without delay, as the matter he had to discuss was of the most urgent importance. Accordingly I appointed a time for the interview on Monday afternoon, and when Mr Pritchard arrived he informed me of the following particulars.

A good many years before her death, Lady Dormer—who, as I said before, was an eminently generous-minded woman—made a will. Her husband and her daughter were then dead. Henry Dormer had few relations, and all of them were fairly wealthy people. By his own will he had sufficiently provided for these persons, and had left the remainder of his property, amounting to something like seven hundred thousand pounds, to his wife, with the express stipulation that she was to consider it as her own, to do what she liked with, without any restriction whatsoever. Accordingly, Lady Dormer's will divided this very handsome fortune—apart from certain charitable and personal bequests with which I need not trouble you—between the people who, for one reason and another, had the greatest claims on her affection. Twelve thousand pounds were to go to Miss Ann Dorland. The whole of the remainder was to pass to her brother, General Fentiman, if he was still living at her death. If, on the other hand, he should pre-decease her, the conditions were reversed. In that case, the bulk of the money came to Miss Dorland, and fifteen thousand pounds were to be equally divided between Major Robert Fentiman and his brother George.«

Wimsey whistled softly.

»I quite agree with you,« said Mr Murbles. »It is a most awkward situation. Lady Dormer died at precisely 10:37 \textsc{a.m.} on November  11\textsuperscript{th}. General Fentiman died that same morning at some time, presumably after 10 o'clock, which was his usual hour for arriving at the Club, and certainly before 7 \textsc{p.m.} when his death was discovered. If he died immediately on his arrival, or at any time up to 10:36, then Miss Dorland is an important heiress, and my clients the Fentimans get only seven thousand pounds or so apiece. If, on the other hand, his death occurred even a few seconds after 10:37, Miss Dorland receives only twelve thousand pounds, George Fentiman is left with the small pittance bequeathed to him under his father's will—while Robert Fentiman, the residuary legatee, inherits a very considerable fortune of well over half a million.«

»And what,« said Wimsey, »do you want me to do about it?«

»Why,« replied the lawyer, with a slight cough, »it occurred to me that you, with your—if I may say so—remarkable powers of deduction and analysis might be able to solve the extremely difficult and delicate problem of the precise moment of General Fentiman's decease. You were in the Club when the death was discovered, you saw the body, you know the places and the persons involved, and you are, by your standing and personal character, exceptionally well fitted to carry out the necessary investigations without creating any—ahem!—public agitation or—er—scandal, or, in fact, notoriety, which would, I need hardly say, be extremely painful to all concerned.«

»It's awkward,« said Wimsey, »uncommonly awkward.«

»It is indeed,« said the lawyer with some warmth, »for as we are now situated, it is impossible to execute either will or—or in short do anything at all. It is most unfortunate that the circumstances were not fully understood at the time, when the—um—the body of General Fentiman was available for inspection. Naturally, Mr Pritchard was quite unaware of the anomalous situation, and as I knew nothing about Lady Dormer's will, I had no idea that anything beyond Dr Penberthy's certificate was, or ever could become, necessary.«

»Couldn't you get the parties to come to some agreement?« suggested Wimsey.

»If we are unable to reach any satisfactory conclusion about the time of the death, that will probably be the only way out of the difficulty. But at the moment there are certain obstacles\longdash«

»Somebody's being greedy, eh?—You'd rather not say more definitely, I suppose? No? H'm, well! From a purely detached point of view it's a very pleasin' and pretty little problem, you know.«

»You will undertake to solve it for us then, Lord Peter?«

Wimsey's fingers tapped out an intricate fugal passage on the arm of his chair.

»If I were you, Murbles, I'd try again to get a settlement.«

»Do you mean,« asked Mr Murbles, »that you think my clients have a losing case?«

»No—I can't say that. By the way, Murbles, who is your client—Robert or George?«

»Well, the Fentiman family in general. I know, naturally, that Robert's gain is George's loss. But none of the parties wishes anything but that the actual facts of the case should be determined.«

»I see. You'll put up with anything I happen to dig out?«

»Of course.«

»However favourable or unfavourable it may be?«

»I should not lend myself to any other course,« said Mr Murbles, rather stiffly.

»I know that, sir. But—well!—I only mean that—Look here, sir! when you were a boy, did you ever go about pokin' sticks and things into peaceful, mysterious lookin' ponds, just to see what was at the bottom?«

»Frequently,« replied Mr Murbles. »I was extremely fond of natural history and had a quite remarkable collection (if I may say so at this distance of time) of pond fauna.«

»Did you ever happen to stir up a deuce of a stink in the course of your researches?«

»My dear Lord Peter—you are making me positively uneasy.«

»Oh, I don't know that you need be. I am only giving you a general warning, you know. Of course, if you wish it, I'll investigate this business like a shot.«

»It's very good of you,« said Mr Murbles.

»Not at all. \textit{I} shall enjoy it all right. If anything odd comes of it, that's your funeral. You never know, you know.«

»If you decide that no satisfactory conclusion can be arrived at,« said Mr Murbles, »we can always fall back on the settlement. I am sure all parties wish to avoid litigation.«

»In case the estate vanishes in costs? Very wise. I hope it may be feasible. Have you made any preliminary inquiries?«

»None to speak of. I would rather you undertook the whole investigation from the beginning.«

»Very well. I'll start to-morrow and let you know how it gets on.«

The lawyer thanked him and took his departure. Wimsey sat pondering for a short time—then rang the bell for his man-servant.

»A new notebook, please, Bunter. Head it »Fentiman« and be ready to come round with me to the Bellona Club to-morrow, complete with camera and the rest of the outfit.«

»Very good, my lord. I take it your lordship has a new inquiry in hand?«

»Yes, Bunter—quite new.«

»May I venture to ask if it is a promising case, my lord?«

»It has its points. So has a porcupine. No matter. Begone, dull care! Be at great pains, Bunter, to cultivate a detached outlook on life. Take example by the bloodhound, who will follow up with equal and impartial zest the trail of a parricide or of a bottle of aniseed.«

»I will bear it in mind, my lord.«

Wimsey moved slowly across to the little black baby grand that stood in the corner of the library.

»Not Bach this evening,« he murmured to himself. »Bach for to-morrow when the gray matter begins to revolve.« A melody of Parry's formed itself crooningly under his fingers. »For man worketh in a vain shadow \textellipsis  he heapeth up riches and cannot tell who shall gather them.« He laughed suddenly, and plunged into an odd, noisy, and painfully inharmonious study by a modern composer in the key of seven sharps.