%!TeX root=../bellonatop.tex
\chapter{Shuffle the Cards and Deal Again}

\lettrine[lines=4]{A}{} hasty consultation with the powers that be at Scotland Yard put Detective-Inspector Parker in charge of the Fentiman case, and he promptly went into consultation with Wimsey.

\zz
»What put you on to this poison business?« he asked.

»Aristotle, chiefly,« replied Wimsey. »He says, you know, that one should always prefer the probable impossible to the improbable possible. It was possible, of course, that the General should have died off in that neat way at the most confusing moment. But how much nicer and more probable that the whole thing had been stage-managed. Even if it had seemed much more impossible I should have been dead nuts on murder. And there really was nothing impossible about it. Then there was Pritchard and the Dorland woman. Why should they have been so dead against compromise and so suspicious about things unless they had inside information from somewhere. After all, they hadn't seen the body as Penberthy and I did.«

»That leads on to the question of who did it. Miss Dorland is the obvious suspect, naturally.«

»She's got the biggest motive.«

»Yes. Well, let's be methodical. Old Fentiman was apparently as right as rain up till about half-past three when he started off for Portman Square, so that the drug must have been given him between then and eightish, when Robert Fentiman found him dead. Now who saw him between those two times?«

»Wait a sec. That's not absolutely accurate. He must have \textit{taken} the stuff between those two times, but might have been \textit{given} him earlier. Suppose, for instance, somebody had dropped a poisoned pill into his usual bottle of soda-mints or whatever he used to take. That could have been worked at any time.«

»Well—not too early on, Peter. Suppose he had died a lot too soon and Lady Dormer had heard about it.«

»It wouldn't have made any difference. She wouldn't need to alter her will, or anything. The bequest to Miss Dorland would just stand as before.«

»Quite right. I was being stupid. Well, then, we'd better find out if he did take anything of that kind regularly. If he did, who would have had the opportunity to drop the pill in?«

»Penberthy, for one.«

»The doctor?—yes, we must stick his name down as a possible, though he wouldn't have had the slightest motive. Still, we'll put him in the column headed Opportunity.«

»That's right, Charles. I do like your methodical ways.«

»Attraction of opposites,« said Parker, ruling a notebook into three columns. »Opportunity. Number 1, Dr~Penberthy. If the tablets or globules or whatever they were, were Penberthy's own prescription, he would have a specially good opportunity. Not so good, though, if they were the kind of things you get ready-made from the chemist in sealed bottles.«

»Oh, bosh! he could always have asked to have a squint at 'em to see if they were the right kind. I insist on having Penberthy in. Besides, he was one of the people who saw the General between the critical hours—during what we may call the administration period, so he had an extra amount of opportunity.«

»So he had. Well, I've put him down. Though there seems no reason for him\longdash«

»I'm not going to be put off by a trifling objection like that. He had the opportunity, so down he goes. Well, then, Miss Dorland comes next.«

»Yes. She goes down under opportunity and also under motive. She certainly had a big interest in polishing off the old man, she saw him during the period of administration and she very likely gave him something to eat or drink while he was in the house. So she is a very likely subject. The only difficulty with her is the difficulty of getting hold of the drug. You can't get digitalin just by asking for it, you know.«

»N—no. At least, not by itself. You can get it mixed up with other drugs quite easily. I saw an ad in the \textit{Daily Views} only this morning, offering a pill with half a grain of digitalin in it.«

»Did you? where?—oh, that! Yes, but it's got nux vomica in it too, which is supposed to be an antidote. At any rate, it bucks the heart up by stimulating the nerves, so as to counteract the slowing-down action of the digitalin.«

»H'm. Well, put down Miss Dorland under Means with a query-mark. Oh, of course, Penberthy has to go down under Means too. He is the one person who could get the stuff without any bother.«

»Right. Means: № 1, Dr~Penberthy. Opportunity: № 1, Dr~Penberthy, № 2, Miss Dorland. We'll have to put in the servants at Lady Dormer's too, shan't we? Any of them who brought him food or drink, at any rate?«

»Put 'em in, by all means. They might have been in collusion with Miss Dorland. And how about Lady Dormer herself?«

»Oh, come, Peter. There wouldn't be any sense in that.«

»Why not? She may have been planning revenge on her brother all these years, camouflaging her feelings under a pretense of generosity. It would be rather fun to leave a terrific legacy to somebody you loathed, and then, just when he was feelin' nice and grateful and all over coals of fire, poison him to make sure he didn't get it. We simply must have Lady Dormer. Stick her down under Opportunity and under Motive.«

»I refuse to do more than Opportunity and Motive (query?).«

»Have it your own way. Well now—there are our friends the two taxi-drivers.«

»I don't think you can be allowed those. It would be awfully hard work poisoning a fare, you know.«

»I'm afraid it would. I say! I've just got a rippin' idea for poisoning a taxi-man, though. You give him a dud half-crown, and when he bites it\longdash«

»He dies of lead poisoning. That one's got whiskers on it.«

»Juggins. You poison the half-crown with Prussic acid.«

»Splendid! And he falls down foaming at the mouth. That's frightfully brilliant. Do you mind giving your attention to the matter in hand?«

»You think we can leave out the taxi-drivers, then?«

»I think so.«

»Right-oh! I'll let you have them. That brings us, I'm sorry to say, to George Fentiman.«

»You've got rather a weakness for George Fentiman, haven't you?«

»Yes—I like old George. He's an awful pig in some ways, but I quite like him.«

»Well, I don't know George, so I shall firmly put him down. Opportunity № 3, he is.«

»He'll have to go down under Motive, too, then.«

»Why? What did he stand to gain by Miss Dorland's getting the legacy?«

»Nothing—if he knew about it. But Robert says emphatically that he didn't know. So does George. And if he didn't, don't you see, the General's death meant that he would immediately step into that two thousand quid which Dougal MacStewart was being so pressing about.«

»MacStewart?—oh, yes—the money-lender. That's one up to you, Peter; I'd forgotten him. That certainly does put George on the list of the possibles. He was pretty sore about things too, wasn't he?«

»Very. And I remember his saying one rather unguarded thing at least down at the Club on the very day the murder—or rather, the death—was discovered.«

»That's in his favour, if anything,« said Parker, cheerfully, »unless he's very reckless indeed.«

»It won't be in his favour with the police,« grumbled Wimsey.

»My dear man!«

»I beg your pardon. I was forgetting for the moment. I'm afraid you are getting a little above your job, Charles. So much intelligence will spell either a Chief-Commissionership or ostracism if you aren't careful.«

»I'll chance that. Come on—get on with it. Who else is there?«

»There's Woodward. Nobody could have a better opportunity of tampering with the General's pill-boxes.«

»And I suppose his little legacy might have been a motive.«

»Or he may have been in the enemy's pay. Sinister menservants so often are, you know. Look what a boom there has been lately in criminal butlers and thefts by perfect servants.«

»That's a fact. And now, how about the people at the Bellona?«

»There's Wetheridge. He's a disagreeable devil. And he has always cast covetous eyes at the General's chair by the fire. I've seen him.«

»Be serious, Peter.«

»I'm perfectly serious. I don't like Wetheridge. He annoys me. And then we mustn't forget to put down Robert.«

»Robert? Why, he's the one person we can definitely cross off. He knew it was to his interest to keep the old man alive. Look at the pains he took to cover up the death.«

»Exactly. He is the Most Unlikely Person, and that is why Sherlock Holmes would suspect him at once. He was, by his own admission, the last person to see General Fentiman alive. Suppose he had a row with the old man and killed him, and then discovered, afterwards, about the legacy.«

»You're scintillating with good plots to-day, Peter. If they'd quarreled, he might possibly have knocked his grandfather down—though I don't think he'd do such a rotten and unsportsmanlike thing—but he surely wouldn't have poisoned him.«

Wimsey sighed.

»There's something in what you say,« he admitted. »Still, you never know. Now then, is there any name we've thought of which appears in all three columns of our list?«

»No, not one. But several appear in two.«

»We'd better start on those, then. Miss Dorland is the most obvious, naturally, and after her, George, don't you think?«

»Yes. I'll have a round-up among all the chemists who may possibly have supplied her with the digitalin. Who's her family doctor?«

»Dunno. That's your pigeon. By the way, I'm supposed to be meeting the girl at a cocoa-party or something of the sort to-morrow. Don't pinch her before then if you can help it.«

»No; but it looks to me as though we might need to put a few questions. And I'd like to have a look round Lady Dormer's house.«

»For heaven's sake, don't be flat-footed about it, Charles. Use tact.«

»You can trust your father. And, I say, you might take me down to the Bellona in a tactful way. I'd like to ask a question or two there.«

Wimsey groaned.

»I shall be asked to resign if this goes on. Not that it's much loss. But it would please Wetheridge so much to see the back of me. Never mind. I'll make a Martha of myself. Come on.«

The entrance of the Bellona Club was filled with an unseemly confusion. Culyer was arguing heatedly with a number of men and three or four members of the committee stood beside him with brows as black as thunder. As Wimsey entered, one of the intruders caught sight of him with a yelp of joy.

»Wimsey—Wimsey, old man! Here, be a sport and get us in on this. We've got to have the story some day. You probably know all about it, you old blighter.«

It was Salcombe Hardy of the \textit{Daily Yell}, large and untidy and slightly drunk as usual. He gazed at Wimsey with child-like blue eyes. Barton of the \textit{Banner}, red-haired and pugnacious, faced round promptly.

»Ah, Wimsey, that's fine. Give us a line on this, can't you? Do explain that if we get a story we'll be good and go.«

»Good lord,« said Wimsey, »how do these things get into the papers?«

»I think it's rather obvious,« said Culyer, acidly.

»It wasn't me,« said Wimsey.

»No, no,« put in Hardy. »You mustn't think that. It was my stunt. In fact, I saw the whole show up at the Necropolis. I was on a family vault, pretending to be a recording angel.«

»You would be,« said Wimsey. »Just a moment, Culyer.« He drew the secretary aside. »See here, I'm damned annoyed about this, but it can't be helped. You can't stop these boys when they're after a story. And anyway, it's all got to come out. It's a police affair now. This is Detective-Inspector Parker of Scotland Yard.«

»But what's the matter?« demanded Culyer.

»Murder's the matter, I'm afraid.«

»Oh, hell!«

»Sorry and all that. But you'd better grin and bear it. Charles, give these fellows as much story as you think they ought to have and get on with it. And, Salcombe, if you'll call off your tripe-hounds, we'll let you have an interview and a set of photographs.«

»That's the stuff,« said Hardy.

»I'm sure,« agreed Parker, pleasantly, »that you lads don't want to get in the way, and I'll tell you all that's advisable. Show us a room, Captain Culyer, and I'll send out a statement and then you'll let us get to work.«

This was agreed, and, a suitable paragraph having been provided by Parker, the Fleet Street gang departed, bearing Wimsey away with them like a captured Sabine maiden to drink in the nearest bar, in the hope of acquiring picturesque detail.

»But I wish you'd kept out of it, Sally,« mourned Peter.

»Oh, God,« said Salcombe, »nobody loves us. It's a forsaken thing to be a poor bloody reporter.« He tossed a lank black lock of hair back from his forehead and wept.

Parker's first and most obvious move was to interview Penberthy, whom he caught at Harley Street, after surgery hours.

»Now I'm not going to worry you about that certificate, doctor,« he began, pleasantly. »We're all liable to make mistakes, and I understand that a death resulting from an over-dose of digitalin would look very like a death from heart-failure.«

»It would \textit{be} a death from heart-failure,« corrected the doctor, patiently. Doctors are weary of explaining that heart-failure is not a specific disease, like mumps or housemaid's knee. It is this incompatibility of outlook between the medical and the lay mind which involves counsel and medical witnesses in a fog of misunderstanding and mutual irritation.

»Just so,« said Parker. »Now, General Fentiman had got heart disease already, hadn't he? Is digitalin a thing one takes for heart disease?«

»Yes; in certain forms of heart disease, digitalin is a very valuable stimulant.«

»Stimulant? I thought it was a depressant.«

»It acts as a stimulant at first; in later stages it depresses the heart's action.«

»Oh, I see.« Parker did not see very well, since, like most people, he had a vague idea that each drug has one simple effect appropriate to it, and is, specifically, a cure for something or the other. »It first speeds up the heart and then slows it down.«

»Not exactly. It strengthens the heart's action by retarding the beat, so that the cavities can be more completely emptied and the pressure is relieved. We give it in certain cases of valvular disease—under proper safeguards, of course.«

»Were you giving it to General Fentiman?«

»I had given it to him from time to time.«

»On the afternoon of November 10th,—you remember that he came to you in consequence of a heart attack. Did you give him digitalin then?«

Dr~Penberthy appeared to hesitate painfully for a moment. Then he turned to his desk and extracted a large book.

»I had better be perfectly frank with you,« he said. »I did. When he came to me, the feebleness of the heart's action and the extreme difficulty in breathing suggested the urgent necessity of a cardiac stimulant. I gave him a prescription containing a small quantity of digitalin to relieve this condition. Here is the prescription. I will write it out for you.«

»A small quantity?« repeated Parker.

»Quite small, combined with other drugs to counteract the depressing after-effects.«

»It was not as large as the dose afterwards found in the body?«

»Good heavens, no—nothing like. In a case like General Fentiman's, digitalin is a drug to be administered with the greatest caution.«

»It would not be possible, I suppose, for you to have made a mistake in dispensing? To have given an over-dose by error?«

»That possibility occurred to me at once, but as soon as I heard Sir James Lubbock's figures, I realized that it was quite out of the question. The dose given was enormous; nearly two grains. But, to make quite certain, I have had my supply of the drug carefully checked, and it is all accounted for.«

»Who did that for you?«

»My trained nurse. I will let you have the books and chemists' receipts.«

»Thank you. Did your nurse make up the dose for General Fentiman?«

»Oh, no; it is a prescription I always keep by me, ready made up. If you'd like to see her, she will show it to you.«

»Thanks very much. Now, when General Fentiman came to see you, he had just had an attack. Could that have been caused by digitalin?«

»You mean, had he been poisoned before he came to me? Well, of course, digitalin is rather an uncertain drug.«

»How long would a big dose like that take to act?«

»I should expect it to take effect fairly quickly. In the ordinary way it would cause sickness and vertigo. But with a powerful cardiac stimulant like digitalin, the chief danger is that any sudden movement, such as springing suddenly to one's feet from a position of repose, is liable to cause sudden syncope and death. I should say that this was what occurred in General Fentiman's case.«

»And that might have happened at any time after the administration of the dose?«

»Just so.«

»Well, I'm very much obliged to you, Dr~Penberthy. I will just see your nurse and take copies of the entries in your books, if I may.«

This done, Parker made his way to Portman Square, still a little hazy in his mind as to the habits of the common foxglove when applied internally—a haziness which was in no way improved by a subsequent consultation of the \textit{Materia Medica}, the \textit{Pharmacopœia}, Dixon Mann, Taylor, Glaister, and others of those writers who have so kindly and helpfully published their conclusions on toxicology.