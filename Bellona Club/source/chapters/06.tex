%!TeX root=../bellonatop.tex
\chapter{A Card of Re-Entry}
\lettrine[lines=4]{T}{he} door of the little flat in Dover Street was opened by an elderly man-servant, whose anxious face bore signs of his grief at his master's death. He informed them that Major Fentiman was at home and would be happy to receive Lord Peter Wimsey. As he spoke, a tall, soldierly man of about forty-five came out from one of the rooms and hailed his visitor cheerily.

»That you, Wimsey? Murbles told me to expect you. Come in. Haven't seen you for a long time. Hear you're turning into a regular Sherlock. Smart bit of work that was you put in over your brother's little trouble. What's all this? Camera? Bless me, you're going to do our little job in the professional manner, eh? Woodward, see that Lord Peter's man has everything he wants. Have you had lunch? Well, you'll have a spot of something, I take it, before you start measuring up the footprints. Come along. We're a bit at sixes and sevens here, but you won't mind.«

He led the way into the small, austerely-furnished sitting-room.

»Thought I might as well camp here for a bit, while I get the old man's belongings settled up. It's going to be a deuce of a job, though, with all this fuss about the will. However, I'm his executor, so all this part of it falls to me in any case. It's very decent of you to lend us a hand. Queer old girl, Great-aunt Dormer. Meant well, you know, but made it damned awkward for everybody. How are you getting along?«

Wimsey explained the failure of his researches at the Bellona.

»Thought I'd better get a line on it at this end,« he added. »If we know exactly what time he left here in the morning, we ought to be able to get an idea of the time he got to the Club.«

Fentiman screwed his mouth into a whistle.

»But, my dear old egg, didn't Murbles tell you the snag?«

»He told me nothing. Left me to get on with it. What \textit{is} the snag?«

»Why, don't you see, the old boy never came home that night.«

»Never came home?—Where was he, then?«

»Dunno. That's the puzzle. All we know is \textellipsis  wait a minute, this is Woodward's story; he'd better tell you himself. Woodward!«

»Yes, sir.«

»Tell Lord Peter Wimsey the story you told me—about that telephone-call, you know.«

»Yes, sir. About nine o'clock....«

»Just a moment,« said Wimsey, »I do like a story to begin at the beginning. Let's start with the morning—the mornin' of November  10\textsuperscript{th}. Was the General all right that morning? Usual health and spirits and all that?«

»Entirely so, my lord. General Fentiman was accustomed to rise early, my lord, being a light sleeper, as was natural at his great age. He had his breakfast in bed at a quarter to eight—tea and buttered toast, with an egg lightly boiled, as he did every day in the year. Then he got up, and I helped him to dress—that would be about half-past eight to nine, my lord. Then he took a little rest, after the exertion of dressing, and at a quarter to ten I fetched his hat, overcoat, muffler and stick, and saw him start off to walk to the Club. That was his daily routine. He seemed in very good spirits—and in his usual health. Of course, his heart was always frail, my lord, but he seemed no different from ordinary.«

»I see. And in the ordinary way he'd just sit at the Club all day and come home—when, exactly?«

»I was accustomed to have his evening meal ready for him at half-past seven precisely, my lord.«

»Did he always turn up to time?«

»Invariably so, my lord. Everything as regular as on parade. That was the General's way. About three o'clock in the afternoon, there was a ring on the telephone. We had the telephone put in, my lord, on account of the General's heart, so that we could always call up a medical man in case of emergency.«

»Very right, too,« put in Robert Fentiman.

»Yes, sir. General Fentiman was good enough to say, sir, he did not wish me to have the heavy responsibility of looking after him alone in case of illness. He was a very kind, thoughtful gentleman.« The man's voice faltered.

»Just so,« said Wimsey. »I'm sure you must be very sorry to lose him, Woodward. Still, one couldn't expect otherwise, you know. I'm sure you looked after him splendidly. What was it happened about three o'clock?«

»Why, my lord, they rang up from Lady Dormer's to say as how her ladyship was very ill, and would General Fentiman please come at once if he wanted to see her alive. So I went down to the Club myself. I didn't like to telephone, you see, because General Fentiman was a little hard of hearing—though he had his faculties wonderful well for a gentleman of his age—and he never liked the telephone. Besides, I was afraid of the shock it might be to him, seeing his heart was so weak—which, of course, at his age you couldn't hardly expect otherwise—so that was why I went myself.«

»That was very considerate of you.«

»Thank you, my lord. Well, I see General Fentiman, and I give him the message—careful-like, and breaking it gently as you might say. I could see he was took aback a bit, but he just sits thinking for a few minutes, and then he says, 'very well, Woodward, I will go. It is certainly my duty to go.' So I wraps him up careful, and gets him a taxi, and he says. 'You needn't come with me, Woodward. I don't quite know how long I shall stay there. They will see that I get home quite safely.' So I told the man where to take him and came back to the flat. And that, my lord, was the last time I see him.«

Wimsey made a sympathetic clucking sound.

»Yes, my lord. When General Fentiman didn't return at his usual time, I thought he was maybe staying to dine at Lady Dormer's, and took no notice of it. However, at half-past eight, I began to be afraid of the night air for him; it was very cold that day, my lord, if you remember. At nine o'clock, I was thinking of calling up the household at Lady Dormer's to ask when he was to be expected home, when the 'phone rang.«

»At nine exactly?«

»About nine. It might have been a little later, but not more than a quarter-past at latest. It was a gentleman spoke to me. He said: »Is that General Fentiman's flat?« I said, »Yes, who is it, please?« And he said, »Is that Woodward?« giving my name, just like that. And I said »Yes.« And he said, »Oh, Woodward, General Fentiman wishes me to tell you not to wait up for him, as he is spending the night with me.« So I said, »Excuse me, sir, who is it speaking, please?« And he said, »Mr Oliver.« So I asked him to repeat the name, not having heard it before, and he said »Oliver«—it came over very plain, »Mr Oliver,« he said, »I'm an old friend of General Fentiman's, and he is staying to-night with me, as we have some business to talk over.« So I said, »Does the General require anything, sir?«—thinking, you know, my lord, as he might wish to have his sleeping-suit and his tooth-brush or something of that, but the gentleman said no, he had got everything necessary and I was not to trouble myself. Well, of course, my lord, as I explained to Major Fentiman, I didn't like to take upon myself to ask questions, being only in service, my lord; it might seem taking a liberty. But I was very much afraid of the excitement and staying up late being too much for the General, so I went so far as to say I hoped General Fentiman was in good health and not tiring of himself, and Mr Oliver laughed and said he would take very good care of him and send him to bed straight away. And I was just about to make so bold as to ask him where he lived, when he rang off. And that was all I knew till I heard next day of the General being dead, my lord.«

»There now,« said Robert Fentiman. »What do you think of that?«

»Odd,« said Wimsey, »and most unfortunate as it turns out. Did the General often stay out at night, Woodward?«

»Never, my lord. I don't recollect such a thing happening once in five or six years. In the old days, perhaps, he'd visit friends occasionally, but not of late.«

»And you'd never heard of this Mr Oliver?«

»No, my lord.«

»His voice wasn't familiar?«

»I couldn't say but what I might have heard it before, my lord, but I find it very difficult to recognize voices on the telephone. But I thought at the time it might be one of the gentlemen from the Club.«

»Do \textit{you} know anything about the man, Fentiman?«

»Oh yes—I've met him. At least, I suppose it's the same man. But I know nothing about him. I fancy I ran across him once in some frightful crush or other, a public dinner, or something of that kind, and he said he knew my grandfather. And I've seen him lunching at Gatti's and that sort of thing. But I haven't the remotest idea where he lives or what he does.«

»Army man?«

»No—something in the engineering line, I fancy.«

»What's he like?«

»Oh, tall, thin, gray hair and spectacles. About sixty-five to look at. He may be older—must be, if he's an old friend of grandfather's. I gathered he was retired from whatever it is he did, and lived in some suburb, but I'm hanged if I can remember which.«

»Not very helpful,« said Wimsey. »D'you know, occasionally I think there's quite a lot to be said for women.«

»What's that got to do with it?«

»Well, I mean, all this easy, uninquisitive way men have of makin' casual acquaintances is very fine and admirable and all that—but look how inconvenient it is! Here you are. You admit you've met this bloke two or three times, and all you know about him is that he is tall and thin and retired into some unspecified suburb. A woman, with the same opportunities, would have found out his address and occupation, whether he was married, how many children he had, with their names and what they did for a living, what his favourite author was, what food he liked best, the name of his tailor, dentist and bootmaker, when he knew your grandfather and what he thought of him—screeds of useful stuff!«

»So she would,« said Fentiman, with a grin. »That's why I've never married.«

»I quite agree,« said Wimsey, »but the fact remains that as a source of information you're simply a wash-out. Do, for goodness' sake, pull yourself together and try to remember something a bit more definite about the fellow. It may mean half a million to you to know what time grandpa set off in the morning from Tooting Bec or Finchley or wherever it was. If it was a distant suburb, it would account for his arriving rather late at the Club—which is rather in your favour, by the way.«

»I suppose it is. I'll do my best to remember. But I'm not sure that I ever knew.«

»It's awkward,« said Wimsey. »No doubt the police could find the man for us, but it's not a police case. And I don't suppose you particularly want to advertise.«

»Well—it may come to that. But naturally, we're not keen on publicity if we can avoid it. If only I could remember exactly what work he said he'd been connected with.«

»Yes—or the public dinner or whatever it was where you first met him. One might get hold of a list of the guests.«

»My dear Wimsey—that was two or three years ago!«

»Or maybe they know the blighter at Gatti's.«

»That's an idea. I've met him there several times. Tell you what, I'll go along there and make inquiries, and if they don't know him, I'll make a point of lunching there pretty regularly. He's almost bound to turn up again.«

»Right. You do that. And meanwhile, do you mind if I have a look round the flat?«

»Rather not. D'you want me? Or would you rather have Woodward? He really knows a lot more about things.«

»Thanks. I'll have Woodward. Don't mind me. I shall just be fussing about.«

»Carry on by all means. I've got one or two drawers full of papers to go through. If I come across anything bearing on the Oliver bloke I'll yell out to you.«

»Right.«

Wimsey went out, leaving him to it, and joined Woodward and Bunter, who were conversing in the next room. A glance told Wimsey that this was the General's bedroom.

On a table beside the narrow iron bedstead was an old-fashioned writing-desk. Wimsey took it up, weighed it in his hands a moment and then took it to Robert Fentiman in the other room. »Have you opened this?« he asked.

»Yes—only old letters and things.«

»You didn't come across Oliver's address, I suppose?«

»No. Of course I looked for that.«

»Looked anywhere else? Any drawers? Cupboards? That sort of thing?«

»Not so far,« said Fentiman, rather shortly.

»No telephone memorandum or anything—you've tried the telephone-book, I suppose?«

»Well, no—I can't very well ring up perfect strangers and\longdash«

»And sing 'em the Froth-Blowers' Anthem? Good God, man, anybody'd think you were chasing a lost umbrella, not half a million of money. The man rang you up, so he may very well be on the 'phone himself. Better let Bunter tackle the job. He has an excellent manner on the line; people find it a positive pleasure to be tr-r-roubled by him.«

Robert Fentiman greeted this feeble pleasantry with an indulgent grin, and produced the telephone directory, to which Bunter immediately applied himself. Finding two-and-a-half columns of Olivers, he removed the receiver and started to work steadily through them in rotation. Wimsey returned to the bedroom. It was in apple-pie order—the bed neatly made, the wash-hand apparatus set in order, as though the occupant might return at any moment, every speck of dust removed—a tribute to Woodward's reverent affection, but a depressing sight for an investigator. Wimsey sat down, and let his eye rove slowly from the hanging wardrobe, with its polished doors, over the orderly line of boots and shoes arranged on their trees on a small shelf, the dressing table, the washstand, the bed and the chest of drawers which, with the small bedside table and a couple of chairs, comprised the furniture.

»Did the General shave himself, Woodward?«

»No, my lord; not latterly. That was my duty, my lord.«

»Did he brush his own teeth, or dental plate or whatever it was?«

»Oh, yes, my lord. General Fentiman had an excellent set of teeth for his age.«

Wimsey fixed his powerful monocle into his eye, and carried the tooth-brush over to the window. The result of the scrutiny was unsatisfactory. He looked round again.

»Is that his walking-stick?«

»Yes, my lord.«

»May I see it?«

Woodward brought it across, carrying it, after the manner of a well-trained servant, by the middle. Lord Peter took it from him in the same manner, suppressing a slight, excited smile. The stick was a heavy malacca, with a thick crutch-handle of polished ivory, suitable for sustaining the feeble steps of old age. The monocle came into play again, and this time its owner gave a chuckle of pleasure.

»I shall want to take a photograph of this stick presently, Woodward. Will you be very careful to see that it is not touched by anybody beforehand?«

»Certainly, my lord.«

Wimsey stood the stick carefully in its corner again, and then, as though it had put a new train of ideas into his mind, walked across to the shoeshelf.

»Which were the shoes General Fentiman was wearing at the time of his death?«

»These, my lord.«

»Have they been cleaned since?«

Woodward looked a trifle stricken.

»Not to say cleaned, my lord. I just wiped them over with a duster. They were not very dirty, and somehow—I hadn't the heart—if you'll excuse me, my lord.«

»That's very fortunate.«

Wimsey turned them over and examined the soles very carefully, both with the lens and with the naked eye. With a small pair of tweezers, taken from his pocket, he delicately removed a small fragment of pile—apparently from a thick carpet—which was clinging to a projecting brad, and stored it carefully away in an envelope. Then, putting the right shoe aside, he subjected the left to a prolonged scrutiny, especially about the inner edge of the sole. Finally he asked for a sheet of paper, and wrapped the shoe up as tenderly as though it had been a piece of priceless Waterford glass.

»I should like to see all the clothes General Fentiman was wearing that day—the outer garments, I mean—hat, suit, overcoat and so on.«

The garments were produced, and Wimsey went over every inch of them with the same care and patience, watched by Woodward with flattering attention.

»Have they been brushed?«

»No, my lord—only shaken out.« This time Woodward offered no apology, having grasped dimly that polishing and brushing were not acts which called for approval under these unusual circumstances.

»You see,« said Wimsey, pausing for a moment to note an infinitesimally small ruffling of the threads on the left-hand trouser-leg, »we might be able to get some sort of a clew from the dust on the clothes, if any—to show us where the General spent the night. If—to take a rather unlikely example—we were to find a lot of sawdust, for instance, we might suppose that he had been visiting a carpenter. Or a dead leaf might suggest a garden or a common, or something of that sort. While a cobweb might mean a wine-cellar, or—or a potting-shed—and so on. You see?«

»Yes, my lord,« (rather doubtfully).

»You don't happen to remember noticing that little tear—well, it's hardly a tear—just a little roughness. It might have caught on a nail.«

»I can't say I recollect it, my lord. But I might have overlooked it.«

»Of course. It's probably of no importance. Well—lock the things up carefully. It's just possible I might have to have the dust extracted and analysed. Just a moment—Has anything been removed from these clothes? The pockets were emptied, I suppose?«

»Yes, my lord.«

»There was nothing unusual in them?«

»No, my lord. Nothing but what the General always took out with him. Just his handkerchief, keys, money and cigar-case.«

»H'm. How about the money?«

»Well, my lord—I couldn't say exactly as to that. Major Fentiman has got it all. There was two pound notes in his note-case, I remember. I believe he had two pounds ten when he went out, and some loose silver in the trouser pocket. He'd have paid his taxi-fare and his lunch at the Club out of the ten-shilling note.«

»That shows he didn't pay for anything unusual, then, in the way of train or taxis backwards and forwards, or dinner, or drinks.«

»No, my lord.«

»But naturally, this Oliver fellow would see to all that. Did the General have a fountain-pen?«

»No, my lord. He did very little writing, my lord. I was accustomed to write any necessary letters to tradesmen, and so on.«

»What sort of nib did he use, when he did write?«

»A J pen, my lord. You will find it in the sitting-room. But mostly I believe he wrote his letters at the Club. He had a very small private correspondence—it might be a letter or so to the Bank or to his man of business, my lord.«

»I see. Have you his check-book?«

»Major Fentiman has it, my lord.«

»Do you remember whether the General had it with him when he last went out?«

»No, my lord. It was kept in his writing-desk as a rule. He would write the checks for the household here, my lord, and give them to me. Or occasionally he might take the book down to the Club with him.«

»Ah! Well, it doesn't look as though the mysterious Mr Oliver was one of those undesirable blokes who demand money. Right you are, Woodward. You're perfectly certain that you removed nothing whatever from those clothes except what was in the pockets?«

»I am quite positive of that, my lord.«

»That's very odd,« said Wimsey, half to himself. »I'm not sure that it isn't the oddest thing about the case.«

»Indeed, my lord? Might I ask why?«

»Why,« said Wimsey, »I should have expected\longdash« he checked himself. Major Fentiman was looking in at the door.

»What's odd, Wimsey?«

»Oh, just a little thing struck me,« said Wimsey, vaguely. »I expected to find something among those clothes which isn't there. That's all.«

»Impenetrable sleuth,« said the major, laughing. »What are you driving at?«

»Work it out for yourself, my dear Watson,« said his lordship, grinning like a dog. »You have all the data. Work it out for yourself, and let me know the answer.«

Woodward, a trifle pained by this levity, gathered up the garments and put them away in the wardrobe.

»How's Bunter getting on with those calls?«

»No luck, at present.«

»Oh!—well, he'd better come in now and do some photographs. We can finish the telephoning at home. Bunter!—Oh, and, I say, Woodward—d'you mind if we take your finger-prints?«

»Finger-prints, my lord?«

»Good God, you're not trying to fasten anything on Woodward?«

»Fasten what?«

»Well—I mean, I thought it was only burglars and people who had finger-prints taken.«

»Not exactly. No—I want the General's finger-prints, really, to compare them with some others I got at the Club. There's a very fine set on that walking-stick of his, and I want Woodward's, just to make sure I'm not getting the two sets mixed up. I'd better take yours, too. It's just possible you might have handled the stick without noticing.«

»Oh, I get you, Steve. I don't think I've touched the thing, but it's as well to make sure, as you say. Funny sort of business, what? Quite the Scotland Yard touch. How d'you do it?«

»Bunter will show you.«

Bunter immediately produced a small inking-pad and roller, and a number of sheets of smooth, white paper. The fingers of the two candidates were carefully wiped with a clean cloth, and pressed first on the pad and then on the paper. The impressions thus obtained were labelled and put away in envelopes, after which the handle of the walking-stick was lightly dusted with gray powder, bringing to light an excellent set of prints of a right-hand set of fingers, superimposed here and there, but quite identifiable. Fentiman and Woodward gazed fascinated at this entertaining miracle.

»Are they all right?«

»Perfectly so, sir; they are quite unlike either of the other two specimens.«

»Then presumably they're the General's. Hurry up and get a negative.«

Bunter set up the camera and focussed it.

»Unless,« observed Major Fentiman, »they are Mr Oliver's. That would be a good joke, wouldn't it?«

»It would, indeed,« said Wimsey, a little taken aback. »A very good joke—on somebody. And for the moment, Fentiman, I'm not sure which of us would do the laughing.«

