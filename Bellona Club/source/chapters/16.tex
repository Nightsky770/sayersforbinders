%!TeX root=../bellonatop.tex
\chapter{Quadrille}

\lettrine[lines=4,ante=‘]{M}{rs} Rushworth, this is Lord Peter Wimsey. Naomi, this is Lord Peter. He's fearfully keen on glands and things, so I've brought him along. And Naomi, do tell me all about your news. Who is it? Do I know him?'

Mrs Rushworth was a long, untidy woman, with long, untidy hair wound into bell-pushes over her ears. She beamed short-sightedly at Peter.

»So glad to see you. So very wonderful about glands, isn't it? Dr Voronoff, you know, and those marvelous old sheep. Such a hope for all of us. Not that dear Walter is specially interested in rejuvenation. Perhaps life is long and difficult enough as it is, don't you think—so full of problems of one kind and another. And the insurance companies have quite set their faces against it, or so I understand. That's natural isn't it, when you come to think of it. But the effect on character is so interesting, you know. Are you devoted to young criminals by any chance?«

Wimsey said that they presented a very perplexing problem.

»How very true. So perplexing. And just to think that we have been quite wrong about them all these thousands of years. Flogging and bread-and-water, you know, and Holy Communion, when what they really needed was a little bit of rabbit-gland or something to make them just as good as gold. Quite terrible, isn't it? And all those poor freaks in sideshows, too—dwarfs and giants, you know—all pineal or pituitary, and they come right again. Though I daresay they make a great deal more money as they are, which throws such a distressing light on unemployment, does it not?«

Wimsey said that everything had the defects of its qualities.

»Yes, indeed,« agreed Mrs Rushworth. »But I think it is so infinitely more heartening to look at it from the opposite point of view. Everything has the qualities of its defects, too, has it not? It is so important to see these things in their true light. It will be such a joy for Naomi to be able to help dear Walter in this great work. I hope you will feel eager to subscribe to the establishment of the new Clinic.«

Wimsey asked, what new Clinic.

»Oh! hasn't Marjorie told you about it? The new Clinic to make everybody good by glands. That is what dear Walter is going to speak about. He is so keen and so is Naomi. It was such a joy to me when Naomi told me that they were really engaged, you know. Not that her old mother hadn't suspected something, of course,« added Mrs Rushworth, archly. »But young people are so odd nowadays and keep their affairs so much to themselves.«

Wimsey said that he thought both parties were heartily to be congratulated. And indeed, from what he had seen of Naomi Rushworth, he felt that she at least deserved congratulation, for she was a singularly plain girl, with a face like a weasel.

»You will excuse me if I run off and speak to some of these other people, won't you?« went on Mrs Rushworth. »I'm sure you will be able to amuse yourself. No doubt you have many friends in my little gathering.«

Wimsey glanced around and was about to felicitate himself on knowing nobody, when a familiar face caught his eye.

»Why,« said he, »there is Dr Penberthy.«

»Dear Walter!« cried Mrs Rushworth, turning hurriedly in the direction indicated. »I declare, so he is. Ah, well—now we shall be able to begin. He should have been here before, but a doctor's time is never his own.«

»Penberthy?« said Wimsey, half aloud, »good lord!«

»Very sound man,« said a voice beside him. »Don't think the worse of his work from seeing him in this crowd. Beggars in a good cause can't be choosers, as we parsons know too well.«

Wimsey turned to face a tall, lean man, with a handsome, humorous face, whom he recognized as a well-known slum padre.

»Father Whittington, isn't it?«

»The same. You're Lord Peter Wimsey, I know. We've got an interest in crime in common, haven't we? I'm interested in this glandular theory. It may throw a great light on some of our heart-breaking problems.«

»Glad to see there's no antagonism between religion and science,« said Wimsey.

»Of course not. Why should there be? We are all searching for Truth.«

»And all these?« asked Wimsey, indicating the curious crowd with a wave of the hand.

»In their way. They mean well. They do what they can, like the woman in the Gospels, and they are surprisingly generous. Here's Penberthy, looking for you, I fancy. Well, Dr Penberthy, I've come, you see, to hear you make mince-meat of original sin.«

»That's very open-minded of you,« said Penberthy, with a rather strained smile. »I hope you are not hostile. We've no quarrel with the Church, you know, if she'll stick to her business and leave us to ours.«

»My dear man, if you can cure sin with an injection, I shall be only too pleased. Only be sure you don't pump in something worse in the process. You know the parable of the swept and garnished house.«

»I'll be as careful as I can,« said Penberthy. »Excuse me one moment. I say, Wimsey, you've heard all about Lubbock's analysis, I suppose.«

»Yes. Bit of a startler, isn't it?«

»It's going to make things damnably awkward for me, Wimsey. I wish to God you'd given me a hint at the time. Such a thing never once occurred to me.«

»Why should it? You were expecting the old boy to pop off from heart, and he did pop off from heart. Nobody could possibly blame you.«

»Couldn't they? That's all you know about juries. I wouldn't have had this happen, just at this moment, for a fortune. It couldn't have chosen a more unfortunate time.«

»It'll blow over, Penberthy. That sort of mistake happens a hundred times a week. By the way, I gather I'm to congratulate you. When did this get settled? You've been very quiet about it.«

»I was starting to tell you up at that infernal exhumation business, only somebody barged in. Yes. Thanks very much. We fixed it up—oh! about a fortnight or three weeks ago. You have met Naomi?«

»Only for a moment this evening. My friend Miss Phelps carried her off to hear all about you.«

»Oh, yes. Well, you must come along and talk to her. She's a sweet girl, and very intelligent. The old lady's a bit of a trial, I don't mind saying, but her heart's in the right place. And there's no doubt she gets hold of people whom it's very useful to meet.«

»I didn't know you were such an authority on glands.«

»I only wish I could afford to be. I've done a certain amount of experimental work under Professor Sligo. It's the Science of the Future, as they say in the press. There really isn't any doubt about that. It puts biology in quite a new light. We're on the verge of some really interesting discoveries, no doubt about it. Only what with the anti-vivisectors and the parsons and the other old women, one doesn't make the progress one ought. Oh, lord—they're waiting for me to begin. See you later.«

»Half a jiff. I really came here—no, dash it, that's rude! but I'd no idea you were the lecturer till I spotted you. I originally came here (that sounds better) to get a look at Miss Dorland of Fentiman fame. But my trusty guide has abandoned me. Do you know Miss Dorland? Can you tell me which she is?«

»I know her to speak to. I haven't seen her this evening. She may not turn up, you know.«

»I thought she was very keen on—on glands and things.«

»I believe she is—or thinks she is. Anything does for these women, as long as it's new—especially if it's sexual. By the way, I don't intend to be sexual.«

»Bless you for that. Well, possibly Miss Dorland will show up later.«

»Perhaps. But—I say, Wimsey. She's in rather a queer position, isn't she? She may not feel inclined to face it. It's all in the papers, you know.«

»Dash it, don't I know it? That inspired tippler, Salcombe Hardy, got hold of it somehow. I think he bribes the cemetrey officials to give him advance news of exhumations. He's worth his weight in pound notes to the \textit{Yell}. Cheerio! Speak your bit nicely. You don't mind if I'm not in the front row, do you? I always take up a strategic position near the door that leads to the grub.«

Penberthy's paper struck Wimsey as being original and well-delivered. The subject was not altogether unfamiliar to him, for Wimsey had a number of distinguished scientific friends who found him a good listener, but some of the experiments mentioned were new and the conclusions suggestive. True to his principles, Wimsey made a bolt for the supper-room, while polite hands were still applauding. He was not the first, however. A large figure in a hard-worked looking dress-suit was already engaged with a pile of savoury sandwiches and a whisky-and-soda. It turned at his approach and beamed at him from its liquid and innocent blue eyes. Sally Hardy—never quite drunk and never quite sober—was on the job, as usual. He held out the sandwich-plate invitingly.

»Damn good, these are,« he said. »What are you doing here?«

»What are you, if it comes to that?« asked Wimsey.

Hardy laid a fat hand on his sleeve.

»Two birds with one stone,« he said, impressively. »Smart fellow, that Penberthy. Glands are news, you know. He knows it. He'll be one of these fashionable practitioners«—Sally repeated this phrase once or twice, as it seemed to have got mixed up with the soda—»before long. Doing us poor bloody journalists out of a job like\textellipsis  and\dots« (He mentioned two gentlemen whose signed contributions to popular dailies were a continual source of annoyance to the G.M.C.)

»Provided he doesn't damage his reputation over this Fentiman affair,« rejoined Wimsey, in a refined shriek which did duty for a whisper amid the noisy stampede which had followed them up to the refreshment-table.

»Ah! there you are,« said Hardy. »Penberthy's news in himself. He's a story, don't you see. We'll have to sit on the fence a bit, of course, till we see which way the cat jumps. I'll have a par. about it at the end, mentioning that he attended old Fentiman. Presently we'll be able to work up a little thing on the magazine page about the advisability of a p.m. in all cases of sudden death. You know—even experienced doctors may be deceived. If he comes off very badly in cross-examination, there can be something about specialists not always being trustworthy—a kind word for the poor down-trodden G.P. and all that. Anyhow, he's worth a story. It doesn't matter what you say about him, provided you say something. You couldn't do us a little thing—about eight hundred words, could you—about rigour mortis or something? Only make it snappy.«

»I could not,« said Wimsey. »I haven't time and I don't want the money. Why should I? I'm not a dean or an actress.«

»No, but you're news. You can give me the money, if you're so beastly flush. Look here, have you got a line on this case at all? That police friend of yours won't give anything away. I want to get something in before there's an arrest, because after that it's contempt. I suppose it's the girl you're after, isn't it? Can you tell me anything about her?«

»No—I came here to-night to get a look at her but she hasn't turned up. I wish you could dig up her hideous past for me. The Rushworths must know something about her, I should think. She used to paint or something. Can't you get on to that?«

Hardy's face lighted up.

»Waffles Newton will probably know something,« he said. »I'll see what I can dig out. Thanks very much, old man. That's given me an idea. We might get one of her pictures on the back pages. The old lady seems to have been a queer old soul. Odd will, wasn't it?«

»Oh, I can tell you all about that,« said Wimsey. »I thought you probably knew.«

He gave Hardy the history of Lady Dormer as he had heard it from Mr Murbles. The journalist was enthralled.

»Great stuff!« he said. »That'll get em. Romance there! This'll be a scoop for the \textit{Yell}. Excuse me. I want to `phone it through to `em before somebody else gets it. Don't hand it out to any of the other fellows.«

»They can get it from Robert or George Fentiman,« warned Wimsey.

»Not much, they won't,« said Salcombe Hardy, feelingly. »Robert Fentiman gave old Barton of the \textit{Banner} such a clip under the ear this morning that he had to go and see a dentist. And George has gone down to the Bellona, and they won't let anybody in. I'm all right on this. If there's anything I can do for you, I will, you bet. So long.«

He faded away. A hand was laid on Peter's arm.

»You're neglecting me shockingly,« said Marjorie Phelps. »And I'm frightfully hungry. I've been doing my best to find things out for you.«

»That's top-hole of you. Look here. Come and sit out in the hall; it's quieter. I'll scrounge some grub and bring it along.«

He secured a quantity of curious little stuffed buns, four \textit{petits-fours}, some dubious claret-cup and some coffee and brought them with him on a tray, snatched while the waitress's back was turned.

»Thanks,« said Marjorie. »I deserve all I can get for having talked to Naomi Rushworth. I cannot like that girl. She hints things.«

»What, particularly?«

»Well, I started to ask about Ann Dorland. So she said she wasn't coming. So I said, »Oh, why?« and she said, »She \textit{said} she wasn't well.««

»Who said?«

»Naomi Rushworth said Ann Dorland said she couldn't come because she wasn't well. But she said that was only an excuse, of course.«

»Who said?«

»Naomi said. So I said, was it? And she said yes, she didn't suppose she felt like facing people very much. So I said, »I thought you were such friends.« So she said, »Well, we are, but of course Ann always was a little abnormal, you see.« So I said that was the first I had heard of it. And she gave me one of her catty looks and said, »Well, there was Ambrose Ledbury, wasn't there? But of course you had other things to think of then, hadn't you?« The little beast. She meant Komski. And after all, everybody knows how obvious she's made herself over this man Penberthy.«

»I'm sorry, I've got mixed.«

»Well, I was rather fond of Komski. And I did almost promise to live with him, till I found that his last three women had all got fed up with him and left him, and I felt there must be something wrong with a man who continually got left, and I've discovered since that he was a dreadful bully when he dropped that touching lost-dog manner of his. So I was well out of it. Still, seeing that Naomi had been going about for the last year nearly, looking at Dr Penberthy like a female spaniel that thinks it's going to be whipped, I can't see why she need throw Komski in my face. And as for Ambrose Ledbury, anybody might have been mistaken in him.«

»Who was Ambrose Ledbury?«

»Oh, he was the man who had that studio over Boulter's Mews. Powerfulness was his strong suit, and being above worldly considerations. He was rugged and wore homespun and painted craggy people in bedrooms, but his colour was amazing. He really could paint and so we could excuse a lot, but he was a professional heart-breaker. He used to gather people up hungrily in his great arms, you know—that's always rather irresistible. But he had no discrimination. It was just a habit, and his affairs never lasted long. But Ann Dorland was really rather overcome, you know. She tried the craggy style herself, but it wasn't at all her line—she hasn't any colour-sense, so there was nothing to make up for the bad drawing.«

»I thought you said she didn't have any affairs.«

»It wasn't an affair. I expect Ledbury gathered her up at some time or other when there wasn't anybody else handy, but he did demand good looks for anything serious. He went off to Poland a year ago with a woman called Natasha somebody. After that, Ann Dorland began to chuck painting. The trouble was, she took things seriously. A few little passions would have put her right, but she isn't the sort of person a man can enjoy flirting with. Heavy-handed. I don't think she would have gone on worrying about Ledbury if he hadn't happened to be the one and only episode. Because, as I say, she did make a few efforts, but she couldn't bring `em off.«

»I see.«

»But that's no reason why Naomi should turn round like that. The fact is, the little brute's so proud of having landed a man—\textit{and} an engagement ring—for herself, that she's out to patronize everybody else.«

»Oh?«

»Yes; besides, everything is looked at from dear Walter's point of view now, and naturally Walter isn't feeling very loving towards Ann Dorland.«

»Why not?«

»My dear man, you're being very discreet, aren't you? Naturally, everybody's saying that she did it.«

»Are they?«

»Who else could they think did it?«

Wimsey realized, indeed, that everybody must be thinking it. He was exceedingly inclined to think it himself.

»Probably that's why she didn't turn up.«

»Of course it is. She's not a fool. She must know.«

»That's true. Look here, will you do something for me? Something more, I mean?«

»What?«

»From what you say, it looks as though Miss Dorland might find herself rather short of friends in the near future. If she comes to you\dots«

»I'm not going to spy on her. Not if she had poisoned fifty old generals.«

»I don't want you to. But I want you to keep an open mind, and tell me what you think. Because I don't want to make a mistake over this. And I'm prejudiced. I want Miss Dorland to be guilty. So I'm very likely to persuade myself she is when she isn't. See?«

»Why do you want her to be guilty?«

»I oughtn't to have mentioned that. Of course, I don't want her found guilty if she isn't really.«

»All right. I won't ask questions. And I'll try and see Ann. But I won't try to worm anything out of her. That's definite. I'm standing by Ann.«

»My dear girl,« said Wimsey, »you're not keeping an open mind. You think she did it.«

Marjorie Phelps flushed.

»I don't. Why do you think that?«

»Because you're so anxious not to worm anything out of her. Worming couldn't hurt an innocent person.«

»Peter Wimsey! You sit there looking a perfectly well-bred imbecile, and then in the most underhand way you twist people into doing things they ought to blush for. No wonder you detect things. I will \textit{not} do your worming for you!«

»Well, if you don't, I shall know your opinion, shan't I?«

The girl was silent for a moment. Then she said:

»It's all so beastly.«

»Poisoning is a beastly crime, don't you think?« said Wimsey.

He got up quickly. Father Whittington was approaching, with Penberthy.

»Well,« said Lord Peter, »have the altars reeled?«

»Dr Penberthy has just informed me that they haven't a leg to stand on,« replied the priest, smiling. »We have been spending a pleasant quarter of an hour abolishing good and evil. Unhappily, I understand his dogma as little as he understands mine. But I exercised myself in Christian humility. I said I was willing to learn.«

Penberthy laughed.

»You don't object, then, to my casting out devils with a syringe,« he said, »when they have proved obdurate to prayer and fasting?«

»Not at all. Why should I? So long as they \textit{are} cast out. And provided you are certain of your diagnosis.«

Penberthy crimsoned and turned away sharply.

»Oh, lord!« said Wimsey. »That was a nasty one. From a Christian priest, too!«

»What have I said?« cried Father Whittington, much disconcerted.

»You have reminded science,« said Wimsey, »that only the Pope is infallible.«