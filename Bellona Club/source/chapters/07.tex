%!TeX root=../bellonatop.tex
\chapter{The Curse of Scotland}

\lettrine[lines=4]{W}{hat} with telephone calls and the development of photographs, it appeared obvious that Bunter was booked for a busy afternoon. His master, therefore, considerately left him in possession of the flat in Piccadilly, and walked abroad to divert himself in his own peculiar way.

His first visit was to one of those offices which undertake to distribute advertisements to the press. Here he drew up an advertisement addressed to taxi-drivers and arranged for it to appear, at the earliest possible date, in all the papers which men of that profession might be expected to read. Three drivers were requested to communicate with Mr J. Murbles, Solicitor, of Staple Inn, who would recompense them amply for their time and trouble. First: any driver who remembered taking up an aged gentleman from Lady Dormer's house in Portman Square or the near vicinity on the afternoon of November  10\textsuperscript{th}. Secondly: any driver who recollected taking up an aged gentleman at or near Dr Penberthy's house in Harley Street at some time in the afternoon or evening of November  10\textsuperscript{th}. And thirdly: any driver who had deposited a similarly aged gentleman at the door of the Bellona Club between 10 and 12.30 in the morning of November  11\textsuperscript{th}.

»Though probably,« thought Wimsey, as he footed the bill for the insertions, to run for three days unless cancelled, »Oliver had a car and ran the old boy up himself. Still, it's just worth trying.«

He had a parcel under his arm, and his next proceeding was to hail a cab and drive to the residence of Sir James Lubbock, the well-known analyst. Sir James was fortunately at home and delighted to see Lord Peter. He was a square-built man, with a reddish face and strongly-curling gray hair, and received his visitor in his laboratory, where he was occupied in superintending a Marsh's test for arsenic.

»D'ye mind just taking a pew for a moment, while I finish this off?«

Wimsey took the pew and watched, interested, the flame from the Bunsen burner playing steadily upon the glass tube, the dark brown deposit slowly forming and deepening at the narrow end. From time to time, the analyst poured down the thistle-funnel a small quantity of a highly disagreeable-looking liquid from a stoppered phial; once his assistant came forward to add a few more drops of what Wimsey knew must be hydrochloric acid. Presently, the disagreeable liquid having all been transferred to the flask, and the deposit having deepened almost to black at its densest part, the tube was detached and taken away, and the burner extinguished, and Sir James Lubbock, after writing and signing a brief note, turned round and greeted Wimsey cordially.

»Sure I am not interrupting you, Lubbock?«

»Not a scrap. We've just finished. That was the last mirror. We shall be ready in good time for our appearance in Court. Not that there's much doubt about it. Enough of the stuff to kill an elephant. Considering the obliging care we take in criminal prosecutions to inform the public at large that two or three grains of arsenic will successfully account for an unpopular individual, however tough, it's surprising how wasteful people are with their drugs. You can't teach `em. An office-boy who was as incompetent as the average murderer would be sacked with a kick in the bottom. Well, now! and what's your little trouble?«

»A small matter,« said Wimsey, unrolling his parcel and producing General Fentiman's left boot, »it's cheek to come to you about it. But I want very much to know what this is, and as it's strictly a private matter, I took the liberty of bargin' round to you in a friendly way. Just along the inside of the sole, there—on the edge.«

»Blood?« suggested the analyst, grinning.

»Well, no—sorry to disappoint you. More like paint, I fancy.«

Sir James looked closely at the deposit with a powerful lens.

»Yes; some sort of brown varnish. Might be off a floor or a piece of furniture. Do you want an analysis?«

»If it's not too much trouble.«

»Not at all. I think we'll get Saunders to do it; he has made rather a speciality of this kind of thing. Saunders, would you scrape this off carefully and see what it is? Get a slide of it, and make an analysis of the rest, if you can. How soon is it wanted?«

»Well, I'd like it as soon as possible. I don't mean within the next five minutes.«

»Well, stay and have a spot of tea with us, and I dare say we can get something ready for you by then. It doesn't look anything out of the way. Knowing your tastes, I'm still surprised it isn't blood. Have you no blood in prospect?«

»Not that I know of. I'll stay to tea with pleasure, if you're certain I'm not being a bore.«

»Never that. Besides, while you're here, you might give me your opinion on those old medical books of mine. I don't suppose they're particularly valuable, but they're quaint. Come along.«

Wimsey passed a couple of hours agreeably with Lady Lubbock and crumpets and a dozen or so antiquated anatomical treatises. Presently Saunders returned with his report. The deposit was nothing more nor less than an ordinary brown paint and varnish of a kind well known to joiners and furniture-makers. It was a modern preparation, with nothing unusual about it; one might find it anywhere. It was not a floor-varnish—one would expect to meet it on a door or partition or something of that sort. The chemical formula followed.

»Not very helpful, I'm afraid,« said Sir James.

»You never know your luck,« replied Wimsey. »Would you be good enough to label the slide and sign your name to it, and to the analysis, and keep them both by you for reference in case they're wanted?«

»Sure thing. How do you want `em labelled?«

»Well—put down »Varnish from General Fentiman's left boot,« and »Analysis of varnish from General Fentiman's left boot,« and the date, and I'll sign it, and you and Saunders can sign it, and then I think we shall be all right.«

»Fentiman? Was that the old boy who died suddenly the other day?«

»It was. But it's no use looking at me with that child-like air of intelligent taking-notice, because I haven't got any gory yarn to spin. It's only a question of where the old man spent the night, if you \textit{must} know.«

»Curiouser and curiouser. Never mind, it's nothing to do with me. Perhaps when it's all over, you'll tell me what it's about. Meanwhile the labels shall go on. You, I take it, are ready to witness to the identity of the boot, and I can witness to having seen the varnish on the boot, and Saunders can witness that he removed the varnish from the boot and analysed it and that this is the varnish he analysed. All according to Cocker. Here you are. Sign here and here, and that will be eight-and-sixpence, please.«

»It might be cheap at eight-and-sixpence,« said Wimsey. »It might even turn out to be cheap at eight hundred and sixty quid—or eight thousand and sixty.«

Sir James Lubbock looked properly thrilled.

»You're only doing it to annoy, because you know it teases. Well, if you must be sphinx-like, you must. I'll keep these things under lock and key for you. Do you want the boot back?«

»I don't suppose the executor will worry. And a fellow looks such a fool carrying a boot about. Put it away with the other things till called for, there's a good man.«

So the boot was put away in a cupboard, and Lord Peter was free to carry on with his afternoon's entertainment.

His first idea was to go on up to Finsbury Park, to see the George Fentimans. He remembered in time, however, that Sheila would not yet be home from her work—she was employed as cashier in a fashionable tea-shop—and further (with a forethought rare in the well-to-do) that if he arrived too early he would have to be asked to supper, and that there would be very little supper and that Sheila would be worried about it and George annoyed. So he turned in to one of his numerous Clubs, and had a Sole Colbert very well cooked, with a bottle of Liebfraumilch; an Apple Charlotte and light savoury to follow, and black coffee and a rare old brandy to top up with—a simple and satisfactory meal which left him in the best of tempers.

The George Fentimans lived in two ground-floor rooms with use of kitchen and bathroom in a semi-detached house with a blue and yellow fanlight over the door and Madras muslin over the windows. They were really furnished apartments, but the landlady always referred to them as a flat, because that meant that tenants had to do their own work and provide their own service. The house felt stuffy as Lord Peter entered it, because somebody was frying fish in oil at no great distance, and a slight unpleasantness was caused at the start by the fact that he had rung only once, thus bringing up the person in the basement, whereas a better-instructed caller would have rung twice, to indicate that he wanted the ground floor.

Hearing explanations in the hall, George put his head out of the dining-room and said, »Oh! hullo!«

»Hullo,« said Wimsey, trying to find room for his belongings on an overladen hat-stand, and eventually disposing of them on the handle of a perambulator. »Thought I'd just come and look you up. Hope I'm not in the way.«

»Of course not. Jolly good of you to penetrate to this ghastly hole. Come in. Everything's in a beastly muddle as usual, but when you're poor you have to live like pigs. Sheila, here's Lord Peter Wimsey—you have met, haven't you?«

»Yes, of course. How nice of you to come round. Have you had dinner?«

»Yes, thanks.«

»Coffee?«

»No, thanks, really—I've only just had some.«

»Well,« said George, »there's only whisky to offer you.«

»Later on, perhaps, thanks, old man. Not just now. I've had a brandy. Never mix grape and grain.«

»Wise man,« said George, his brow clearing, since as a matter of fact, there was no whisky nearer than the public-house, and acceptance would have meant six-and-six, at least, besides the exertion of fetching it.

Sheila Fentiman drew an arm-chair forward, and herself sat down on a low pouffe. She was a woman of thirty-five or so, and would have been very good-looking but for an appearance of worry and ill-health that made her look older than her age.

»It's a miserable fire,« said George, gloomily, »is this all the coal there is?«

»I'm sorry,« said Sheila, »she didn't fill it up properly this morning.«

»Well, why can't you see that she does? It's always happening. If the scuttle isn't absolutely empty she seems to think she needn't bother about filling it up.«

»I'll get some.«

»No, it's all right. I'll go. But you ought to tell her about it.«

»I will—I'm always telling her.«

»The woman's no more sense than a hen. No—don't you go, Sheila—I won't have you carrying coal.«

»Nonsense,« said his wife, rather acidly. »What a hypocrite you are, George. It's only because there's somebody here that you're so chivalrous all at once.«

»Here, let me,« said Wimsey, desperately, »I like fetching coal. Always loved coal as a kid. Anything grubby or noisy. Where is it? Lead me to it!«

Mrs Fentiman released the scuttle, for which George and Wimsey politely struggled. In the end they all went out together to the inconvenient bin in the back-yard, Wimsey quarrying the coal, George receiving it in the scuttle and the lady lighting them with a long candle, insecurely fixed in an enamel candle-stick several sizes too large.

»And tell Mrs Crickett,« said George, irritably sticking to his grievance, »that she must fill that scuttle up properly every day.«

»I'll try. But she hates being spoken to. I'm always afraid she'll give warning.«

»Well, there are other charwomen, I suppose?«

»Mrs Crickett is very honest.«

»I know; but that's not everything. You could easily find one if you took the trouble.«

»Well, I'll see about it. But why don't \textit{you} speak to Mrs Crickett? I'm generally out before she gets here.«

»Oh, yes, I know. You needn't keep on rubbing it in about your having to go out to work. You don't suppose I enjoy it, do you? Wimsey can tell you how I feel about it.«

»Don't be so silly, George. Why is it, Lord Peter, that men are so cowardly about speaking to servants?«

»It's the woman's job to speak to servants,« said George, »no business of mine.«

»All right—I'll speak, and you'll have to put up with the consequences.«

»There won't \textit{be} any consequences, my dear, if you do it tactfully. I can't think why you want to make all this fuss.«

»Right-oh, I'll be as tactful as I can. You don't suffer from charladies, I suppose, Lord Peter?«

»Good lord, no!« interrupted George. »Wimsey lives decently. They don't know the dignified joys of hard-upness in Piccadilly.«

»I'm rather lucky,« said Wimsey, with that apologetic air which seems forced on anybody accused of too much wealth. »I have an extraordinarily faithful and intelligent man who looks after me like a mother.«

»Daresay he knows when he's well off,« said George, disagreeably.

»I dunno. I believe Bunter would stick to me whatever happened. He was my N.C.O. during part of the War, and we went through some roughish bits together, and after the whole thing was over I hunted him up and took him on. He was in service before that, of course, but his former master was killed and the family broken up, so he was quite pleased to come along. I don't know what I should do without Bunter now.«

»Is that the man who takes the photographs for you when you are on a crime-hunt?« suggested Sheila, hurriedly seizing on this, as she hoped, nonirritant topic.

»Yes. He's a great hand with a camera. Only drawback is that he's occasionally immured in the dark-room and I'm left to forage for myself. I've got a telephone extension through to him. »Bunter?«—»Yes, my lord!«—»Where are my dress studs?«—»In the middle section of the third small right-hand drawer of the dressing-cabinet, my lord.«—»Bunter!«—»Yes, my lord.«—»Where have I put my cigarette case?«—»I fancy I observed it last on the piano, my lord.«—»Bunter!«—»Yes, my lord!«—»I've got into a muddle with my white tie.«—»Indeed, my lord?«—»Well, can't you do anything about it?«—»Excuse me, my lord, I am engaged in the development of a plate.«—»To hell with the plate!«—»Very good, my lord.«—»Bunter—stop—don't be precipitate—finish the plate and then come and tie my tie.«—»Certainly, my lord.« And then I have to sit about miserably till the infernal plate is fixed, or whatever it is. Perfect slave in my own house—that's what I am.«

Sheila laughed.

»You look a very happy and well-treated slave. Are you investigating anything just now?«

»Yes. In fact—there you are again—Bunter has retired into photographic life for the evening. I haven't a roof to cover me. I have been wandering round like the what d'you call it bird, which has no feet\longdash«

»I'm sorry you were driven to such desperation as to seek asylum in our poverty-stricken hovel,« said George, with a sour laugh.

Wimsey began to wish he had not come. Mrs Fentiman looked vexed.

»You needn't answer that,« she said, with an effort to be light, »there \textit{is} no answer.«

»I'll send it to Aunt Judit of »Rosie's Weekly Bits«,« said Wimsey. »A makes a remark to which there is no answer. What is B to do?«

»Sorry,« said George, »my conversation doesn't seem to be up to standard. I'm forgetting all my civilized habits. You'd better go on and pay no attention to me.«

»What's the mystery on hand now?« asked Sheila, taking her husband at his word.

»Well, actually it's about this funny business of the old General's will,« said Wimsey. »Murbles suggested I should have a look into the question of the survivorship.«

»Oh, do you think you can really get it settled?«

»I hope so very much. But it's a very fine-drawn business—may resolve itself into a matter of seconds. By the way, Fentiman, were you in the Bellona smoking-room at all during the morning of Armistice Day?«

»So \textit{that's} what you've come about. Why didn't you say so? No, I wasn't. And what's more, I don't know anything at all about it. And why that infuriating old hag of a Dormer woman couldn't make a decent, sensible will while she was about it, I don't know. Where was the sense of leaving all those wads of money to the old man, when she knew perfectly well he was liable to peg out at any moment. And then, if he did die, handing the whole lot over to the Dorland girl, who hasn't an atom of claim on it? She might have had the decency to think about Robert and us a bit.«

»Considering how rude you were to her and Miss Dorland, George, I wonder she even left you the seven thousand.«

»What's seven thousand to her? Like a five-pound note to any ordinary person. An insult, I call it. I daresay I was rude to her, but I jolly well wasn't going to have her think I was sucking up to her for her money.«

»How inconsistent you are, George. If you didn't want the money, why grumble about not getting it?«

»You're always putting me in the wrong. You know I don't mean that. I \textit{didn't} want the money—but the Dorland girl was always hinting that I did, and I ticked her off. I didn't know anything about the confounded legacy, and I didn't want to. All I mean is, that if she did want to leave anything to Robert and me, she might have made it more than a rotten seven thousand apiece.«

»Well! don't grumble at it. It would be uncommonly handy at the moment.«

»I know—isn't that exactly what I'm saying? And now the old fool makes such a silly will that I don't know whether I'm to get it or not. I can't even lay hands on the old Governor's two thousand. I've got to sit here and twiddle my thumbs while Wimsey goes round with a tape measure and a tame photographer to see whether I'm entitled to my own grandfather's money!«

»I know it's frightfully trying, darling. But I expect it'll all come right soon. It wouldn't matter if it weren't for Dougal MacStewart.«

»Who's Dougal MacStewart?« inquired Wimsey, suddenly alert. »One of our old Scottish families, by the name. I fancy I have heard of him. Isn't he an obliging, helpful kind of chap, with a wealthy friend in the City?«

»Frightfully obliging,« said Sheila, grimly. »He simply forces his acquaintances on one. He\longdash«

»Shut up, Sheila,« interrupted her husband, rudely. »Lord Peter doesn't want to know all the sordid details of our private affairs.«

»Knowing Dougal,« said Wimsey, »I daresay I could give a guess at them. Some time ago you had a kind offer of assistance from our friend MacStewart. You accepted it to the mild tune of—what was it?«

»Five hundred,« said Sheila.

»Five hundred. Which turned out to be three-fifty in cash and the rest represented by a little honorarium to his friend in the City who advanced the money in so trustful a manner without security. When was that?«

»Three years ago—when I started that tea-shop in Kensington.«

»Ah, yes. And when you couldn't quite manage that sixty per cent per month or whatever it was, owing to trade depression, the friend in the City was obliging enough to add the interest to the principal, at great inconvenience to himself—and so forth. The MacStewart way is familiar to me. What's the demd total now, Fentiman, just out of curiosity?«

»Fifteen hundred by the thirtieth,« growled George, »if you must know.«

»I warned George\longdash« began Sheila, unwisely.

»Oh, you always know what's best! Anyhow, it was your tea business. I told you there was no money in it, but women always think they can run things on their own nowadays.«

»I know, George. But it was MacStewart's interest that swallowed up the profits. You know I wanted you to borrow the money from Lady Dormer.«

»Well, I wasn't going to, and that's flat. I told you so at the time.«

»Well, but look here,« said Wimsey, »you're perfectly all right about MacStewart's fifteen hundred, anyway, whichever way the thing goes. If General Fentiman died before his sister, you get seven thousand; if he died after her, you're certain of his two thousand, by the will. Besides, your brother will no doubt make a reasonable arrangement about sharing the money he gets as residuary legatee. Why worry?«

»Why? Because here's this infernal legal rigmarole tying the thing up and hanging it out till God knows when, and I can't touch anything.«

»I know, I know,« said Wimsey, patiently, »but all you've got to do is to go to Murbles and get him to advance you the money on your expectations. You can't get away with less than two thousand, whatever happens, so he'll be perfectly ready to do it. In fact, he's more or less bound to settle your just debts for you, if he's asked.«

»That's just what I've been telling you, George,« said Mrs Fentiman, eagerly.

»Of course, you \textit{would} be always telling me things. You never make mistakes, do you? And suppose the thing goes into Court and we get let in for thousands of pounds in fees and things, Mrs Clever, eh?«

»I should leave it to your brother to go into Court, if necessary,« said Wimsey, sensibly. »If he wins, he'll have plenty of cash for fees, and if he loses, you'll still have your seven thousand. You go to Murbles—he'll fix you up. Or, tell you what!—I'll get hold of friend MacStewart and see if I can't arrange to get the debt transferred to me. He won't consent, of course, if he knows it's me, but I can probably do it through Murbles. Then we'll threaten to fight him on the ground of extortionate interest and so on. We'll have some fun with it.«

»Dashed good of you, but I'd rather not, thanks.«

»Just as you like. But anyway, go to Murbles. He'll get it squared up for you. Anyhow, I don't think there will be any litigation about the will. If we can't get to the bottom of the survivorship question, I should think you and Miss Dorland would be far better advised to come to a settlement out of Court. It would probably be the fairest way in any case. Why don't you?«

»Why? Because the Dorland female wants her pound of flesh. That's why!«

»Does she? What kind of woman is she?«

»One of these modern, Chelsea women. Ugly as sin and hard as nails. Paints things—ugly, skinny prostitutes with green bodies and no clothes on. I suppose she thinks if she can't be a success as a woman she'll be a half-baked intellectual. No wonder a man can't get a decent job these days with these hard-mouthed, cigarette-smoking females all over the place, pretending they're geniuses and business women and all the rest of it.«

»Oh, come, George! Miss Dorland isn't doing anybody out of a job; she couldn't just sit there all day being Lady Dormer's companion. What's the harm in her painting things?«

»Why couldn't she be a companion? In the old days, heaps of unmarried women were companions, and let me tell you, my dear girl, they had a much better time than they have now, with all this jazzing and short skirts and pretending to have careers. The modern girl hasn't a scrap of decent feeling or sentiment about her. Money—money and notoriety, that's all she's after. That's what we fought the war for—and that's what we've come back to!«

»George, do keep to the point. Miss Dorland doesn't jazz\longdash«

»I am keeping to the point. I'm talking about modern women. I don't say Miss Dorland in particular. But you \textit{will} go taking everything personally. That's just like a woman. You can't argue about things in general—you always have to bring it down to some one little personal instance. You will sidetrack.«

»I wasn't side-tracking. We started to talk about Miss Dorland.«

»You said a person couldn't just be somebody's companion, and I said that in the old days plenty of nice women were companions and had a jolly good time\longdash«

»I don't know about that.«

»Well, I do. They did. And they learned to be decent companions to their husbands, too. Not always flying off to offices and clubs and parties like they are now. And if you think men like that sort of thing, I can tell you candidly, my girl, they don't. They hate it.«

»Does it matter? I mean, one doesn't have to bother so much about husband-hunting to-day.«

»Oh, no! Husbands don't matter at all, I suppose, to you advanced women. Any man will do, as long as he's got money\longdash«

»Why do you say »you« advanced women? I didn't say \textit{I} felt that way about it. I don't \textit{want} to go out to work\longdash«

»There you go. Taking everything to yourself. I \textit{know} you don't want to work. I know it's only because of the damned rotten position I'm in. You needn't keep on about it. I know I'm a failure. Thank your stars, Wimsey, that when you marry you'll be able to support your wife.«

»George, you've no business to speak like that. I didn't mean that at all. You said\longdash«

»I know what I said, but you took it all the wrong way. You always do. It's no good arguing with a woman. No—that's enough. For God's sake don't start all over again. I want a drink. Wimsey, you'll have a drink. Sheila, tell that girl of Mrs Munns's to go round for half a bottle of Johnny Walker.«

»Couldn't you get it yourself, dear? Mrs Munns doesn't like us sending her girl. She was frightfully disagreeable last time.«

»How can I go? I've taken my boots off. You do make such a fuss about nothing. What does it matter if old Mother Munns does kick up a shindy? She can't eat you.«

»No,« put in Wimsey. »But think of the corrupting influence of the jug-and-bottle department on Mrs Munns's girl. I approve of Mrs Munns. She has a motherly heart. I myself will be the St George to rescue Mrs Munns's girl from the Blue Dragon. Nothing shall stop me. No, don't bother to show me the way. I have a peculiar instinct about pubs. I can find one blindfold in a pea-souper with both hands tied behind me.«

Mrs Fentiman followed him to the front door.

»You mustn't mind what George says to-night. His tummy is feeling rotten and it makes him irritable. And it has been so worrying about this wretched money business.«

»That's all right,« said Wimsey. »I know exactly. You should see me when my tummy's upset. Took a young woman out the other night—lobster mayonnaise, meringues and sweet champagne—her choice—oh, lord!«

He made an eloquent grimace and departed in the direction of the public house.

When he returned, George Fentiman was standing on the doorstep.

»I say, Wimsey—I do apologize for being so bloody rude. It's my filthy temper. Rotten bad form. Sheila's gone up to bed in tears, poor kid. All my fault. If you knew how this damnable situation gets on my nerves—though I know there's no excuse\dots«

»'S quite all right,« said Wimsey. »Cheer up. It'll all come out in the wash.«

»My wife\longdash« began George again.

»She's damned fine, old man. But what it is, you both want a holiday.«

»We do, badly. Well, never say die. I'll see Murbles, as you suggest, Wimsey.«

Bunter received his master that evening with a prim smirk of satisfaction.

»Had a good day, Bunter?«

»Very gratifying indeed, I thank your lordship. The prints on the walking-stick are indubitably identical with those on the sheet of paper you gave me.«

»They are, are they? That's something. I'll look at `em to-morrow, Bunter—I've had a tiring evening.«