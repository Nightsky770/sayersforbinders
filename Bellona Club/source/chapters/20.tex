%!TeX root=../bellonatop.tex
\chapter{Ann Dorland Goes Misere}

\lettrine[lines=4]{T}{he} studio door was opened by a girl he did not know. She was not tall, but compactly and generously built. He noticed the wide shoulders and the strong swing of the thighs before he had taken in her face. The uncurtained window behind her threw her features into shadow, he was only aware of thick black hair, cut in a square bob, with a bang across the forehead.

»Miss Phelps is out.«

»Oh!—will she be long?«

»Don't know. She'll be in to supper.«

»Do you think I might come in and wait?«

»I expect so, if you're a friend of hers.«

The girl fell back from the doorway and let him pass. He laid his hat and stick on the table and turned to her. She took no notice of him, but walked over to the fireplace and stood with one hand on the mantelpiece. Unable to sit down, since she was still standing, Wimsey moved to the modeling-board, and raised the wet cloth that covered the little mound of clay.

He was gazing with an assumption of great interest at the half-modeled figure of an old flower-seller, when the girl said:

»I say!«

She had taken up Marjorie Phelps' figurine of himself, and was twisting it over in her fingers.

»Is this you?«

»Yes—rather good of me, don't you think?«

»What do you want?«

»Want?«

»You've come here to have a look at me, haven't you?«

»I came to see Miss Phelps.«

»I suppose the policeman at the corner comes to see Miss Phelps too.«

Wimsey glanced out of the window. There \textit{was} a man at the corner—an elaborately indifferent lounger.

»I am sorry,« said Wimsey, with sudden enlightenment. »I'm really awfully sorry to seem so stupid, and so intrusive. But honestly, I had no idea who you were till this moment.«

»Hadn't you? Oh, well, it doesn't matter.«

»Shall I go?«

»You can please yourself.«

»If you really mean that, Miss Dorland, I should like to stay. I've been wanting to meet you, you know.«

»That was nice of you,« she mocked. »First you wanted to defraud me, and now you're trying to\longdash«

»To what?«

She shrugged her wide shoulders.

»Yours is not a pleasant hobby, Lord Peter Wimsey.«

»Will you believe me,« said Wimsey, »when I assure you that I was never a party to the fraud. In fact, I showed it up. I did, really.«

»Oh, well. It doesn't matter now.«

»But do please believe that.«

»Very well. If you say so, I must believe it.«

She threw herself on the couch near the fire.

»That's better,« said Wimsey. »Napoleon or somebody said that you could always turn a tragedy into a comedy, by sittin' down. Perfectly true, isn't it? Let's talk about something ordinary till Miss Phelps comes in. Shall we?«

»What do you want to talk about?«

»Oh, well—that's rather embarrassin'. Books.« He waved a vague hand. »What have you been readin' lately?«

»Nothing much.«

»Don't know what I should do without books. Fact, I always wonder what people did in the old days. Just think of it. All sorts of bothers goin' on—matrimonial rows and love-affairs—prodigal sons and servants and worries—and no books to turn to.«

»People worked with their hands instead.«

»Yes—that's frightfully jolly for the people who can do it. I envy them myself. You paint, don't you?«

»I try to.«

»Portraits?«

»Oh, no—figure and landscape chiefly.«

»Oh!... A friend of mine—well, it's no use disguising it—he's a detective—you've met him, I think....«

»That man? Oh, yes. Quite a polite sort of detective.«

»He told me he'd seen some stuff of yours. It rather surprised him, I think. He's not exactly a modernist. He seemed to think your portraits were your best work.«

»There weren't many portraits. A few figure-studies....«

»They worried him a bit.« Wimsey laughed. »The only thing he understood, he said, was a man's head in oils....«

»Oh, that!—just an experiment—a fancy thing. My best stuff is some sketches I did of the Wiltshire Downs a year or two ago. Direct painting, without any preliminary sketch.«

She described a number of these works.

»They sound ever so jolly,« said Wimsey. »Great stuff. I wish I could do something of that kind. As I say, I have to fall back on books for my escape. Reading \textit{is} an escape to me. Is it to you?«

»How do you mean?«

»Well—it is to most people, I think. Servants and factory hands read about beautiful girls loved by dark, handsome men, all covered over with jewels and moving in scenes of gilded splendour. And passionate spinsters read Ethel M. Dell. And dull men in offices read detective stories. They wouldn't, if murder and police entered into their lives.«

»I don't know,« she said. »When Crippen and Le Neve were taken on the steamer, they were reading Edgar Wallace.« Her voice was losing its dull harshness; she sounded almost interested.

»Le Neve was reading it,« said Wimsey, »but I've never believed she knew about the murder. I think she was fighting desperately to know nothing about it—reading horrors, and persuading herself that nothing of that kind had happened, or could happen, to her. I think one might do that, don't you?«

»I don't know,« said Ann Dorland. »Of course, a detective story keeps your brain occupied. Rather like chess. Do you play chess?«

»No good at it. I like it—but I keep on thinking about the history of the various pieces, and the picturesqueness of the moves. So I get beaten. I'm not a player.«

»Nor am I. I wish I were.«

»Yes—that would keep one's mind off things with a vengeance. Draughts or dominoes or patience would be even better. No connection with anything. I remember,« added Wimsey, »one time when something perfectly grinding and hateful had happened to me. I played patience all day. I was in a nursing home—with shell-shock—and other things. I only played one game, the very simplest \textellipsis  the demon \textellipsis  a silly game with no ideas in it at all. I just went on laying it out and gathering it up \textellipsis  hundred times in an evening \textellipsis  so as to stop thinking.«

»Then you too....«

Wimsey waited; but she did not finish the sentence.

»It's a kind of drug, of course. That's an awfully trite thing to say, but it's quite true.«

»Yes, quite.«

»I read detective stories too. They were about the only thing I could read. All the others had the war in them—or love \textellipsis  or some damn thing I didn't want to think about.«

She moved restlessly.

»You've been through it, haven't you?« said Wimsey, gently.

»Me?... well \textellipsis  all this \textellipsis  it isn't pleasant, you know \textellipsis  the police \textellipsis  and \textellipsis  and everything.«

»You're not really worried about the police, are you?«

She had cause to be, if she only knew it, but he buried this knowledge at the bottom of his mind, defying it to show itself.

»Everything's pretty hateful, isn't it?«

»Something's hurt you \textellipsis  all right \textellipsis  don't talk about it if you don't want to \textellipsis  a man?«

»It usually is a man, isn't it?«

Her eyes were turned away from him, and she answered with a kind of shamefaced defiance.

»Practically always,« said Wimsey. »Fortunately, one gets over it.«

»Depends what it is.«

»One gets over everything,« repeated Wimsey, firmly. »Particularly if one tells somebody about it.«

»One can't always tell things.«

»I can't imagine anything really untellable.«

»Some things are so beastly.«

»Oh, yes—quite a lot of things. Birth is beastly—and death—and digestion, if it comes to that. Sometimes when I think of what's happening inside me to a beautiful \textit{suprème de sole}, with the caviare in boats, and the \textit{croûtons} and the jolly little twists of potato and all the gadgets—I could cry. But there it is, don't you know.«

Ann Dorland suddenly laughed.

»That's better,« said Wimsey. »Look here, you've been brooding over this and you're seeing it all out of proportion. Let's be practical and frightfully ordinary. Is it a baby?«

»Oh, no!«

»Well—that's rather a good thing, because babies, though no doubt excellent in their way, take a long time and come expensive. Is it blackmail?«

»Good heavens, no!«

»Good. Because blackmail is even longer and more expensive than babies. Is it Freudian, or sadistic, or any of those popular modern amusements?«

»I don't believe you'd turn a hair if it was.«

»Why should I?—I can't think of anything worse to suggest, except what Rose Macaulay refers to as »nameless orgies.« Or diseases, of course. It's not leprosy or anything?«

»What a mind you've got,« she said, beginning to laugh. »No, it isn't leprosy.«

»Well, what \textit{did} the blighter do?«

Ann Dorland smiled faintly: »It's nothing, really.«

»If only Heaven prevents Marjorie Phelps from coming in,« thought Wimsey, »I'm going to get it now\textellipsis~. It must have been something, to upset you like this,« he pursued aloud, »you're not the kind of woman to be upset about nothing.«

»You don't think I am?« She got up and faced him squarely. »He said \textellipsis  he said \textellipsis  I imagined things \textellipsis  he said \textellipsis  he said I had a mania about sex. I suppose you would call it Freudian, really,« she added hastily, flushing an ugly crimson.

»Is that all?« said Wimsey. »I know plenty of people who would take that as a compliment\textellipsis~. But obviously you don't. What exact form of mania did he suggest...?«

»Oh, the gibbering sort that hangs round church doors for curates,« she broke out, fiercely. »It's a lie. He did—he \textit{did}—pretend to—want me and all that. The beast!... I can't tell you the things he said \textellipsis  and I'd made such a fool of myself....«

She was back on the couch, crying, with large, ugly, streaming tears, and snorting into the cushions. Wimsey sat down beside her.

»Poor kid,« he said. This, then, was at the back of Marjorie's mysterious hints, and those scratchcat sneers of Naomi Rushworth's. The girl had wanted love-affairs, that was certain; imagined them perhaps. There had been Ambrose Ledbury. Between the normal and the abnormal, the gulf is deep, but so narrow that misrepresentation is made easy.

»Look here.« He put a comforting arm round Ann's heaving shoulders. »This fellow—was it Penberthy, by the way?«

»How did you know?«

»Oh!—the portrait, and lots of things. The things you liked once, and then wanted to hide away and forget. He's a rotter, anyway, for saying that kind of thing—even if it was true, which it isn't. You got to know him at the Rushworth's, I take it—when?«

»Nearly two years ago.«

»Were you keen on him then?«

»No. I—well, I was keen on somebody else. Only that was a mistake too. He—he was one of those people, you know.«

»They can't help themselves,« said Wimsey, soothingly. »When did the change-over happen?«

»The other man went away. And later on, Dr Penberthy—oh! I don't know! He walked home with me once or twice, and then he asked me to dine with him—in Soho.«

»Had you at that time told any one about this comic will of Lady Dormer's?«

»Of course not. How could I? I never knew anything about it till after she died.«

Her surprise sounded genuine enough.

»What did you think? Did you think the money would come to you?«

»I knew that some of it would; Auntie told me she would see me provided for.«

»There were the grandsons, of course.«

»Yes; I thought she would leave most of it to them. It's a pity she didn't, poor dear. Then there wouldn't have been all this dreadful bother.«

»People so often seem to lose their heads when they make wills. So you were a sort of dark horse at that time. H'm. Did this precious Penberthy ask you to marry him?«

»I thought he did. But he says he didn't. We talked about founding his clinic; I was to help him.«

»And that was when you chucked painting for books about medicine and first-aid classes. Did your aunt know about the engagement?«

»He didn't want her told. It was to be our secret, till he got a better position. He was afraid she might think he was after the money.«

»I daresay he was.«

»He made out he was fond of me,« she said, miserably.

»Of course, my dear child; your case is not unique. Didn't you tell any of your friends?«

»No.« Wimsey reflected that the Ledbury episode had probably left a scar. Besides—did women tell things to other women? He had long doubted it.

»You were still engaged when Lady Dormer died, I take it?«

»As engaged as we ever were. Of course, he told me that there was something funny about the body. He said you and the Fentimans were trying to defraud me of the money. I shouldn't have minded for myself—it was more money than I should have known what to do with. But it would have meant the clinic, you see.«

»Yes, you could start a pretty decent clinic with half a million. So that was why you shot me out of the house.«

He grinned—and then reflected a few moments.

»Look here,« he said, »I'm going to give you a bit of a shock, but it'll have to come sooner or later. Has it ever occurred to you that it was Penberthy who murdered General Fentiman?«

»I—wondered,« she said, slowly. »I couldn't think—who else—But you know they suspect \textit{me}?«

»Oh, well—\textit{cui bono} and all that—they couldn't overlook you. They have to suspect every possible person, you know.«

»I don't blame them at all. But I didn't, you know.«

»Of course not. It was Penberthy. I look at it like this. Penberthy wanted money; he was sick of being poor, and he knew you would be certain to get \textit{some} of Lady Dormer's money. He'd probably heard about the family quarrel with the General, and expected it would be the lot. So he started to make your acquaintance. But he was careful. He asked you to keep it quiet—just in case, you see. The money might be so tied up that you couldn't give it to him, or you might lose it if you married, or it might only be quite a small annuity, in which case he'd want to look for somebody richer.«

»We considered those points when we talked it over about the clinic.«

»Yes. Well, then, Lady Dormer fell ill. The General went round and heard about the legacy that was coming to him. And then he toddled along to Penberthy, feeling very groggy, and promptly told him all about it. You can imagine him saying: »You've got to patch me up long enough to get the money.« That must have been a nasty jar for Penberthy.«

»It was. You see, he didn't even hear about my twelve thousand.«

»Oh?«

»No. Apparently what the General said was, »If only I last out poor Felicity, all the money comes to me. Otherwise it goes to the girl and my boys only get seven thousand apiece.« That was why\longdash«

»Just a moment. When did Penberthy tell you about that?«

»Why, later—when he said I was to compromise with the Fentimans.«

»That explains it. I wondered why you gave in so suddenly. I thought, then, that you—Well, anyhow, Penberthy hears this, and gets the brilliant idea of putting General Fentiman out of the way. So he gives him a slow-working kind of a pill\longdash«

»Probably a powder in a very tough capsule that would take a long time to digest.«

»Good idea. Yes, very likely. And then the General, instead of heading straight for home, as he expected, goes off to the Club and dies there. And then Robert\longdash«

He explained in detail what Robert had done, and resumed.

»Well, now—Penberthy was in a bad fix. If he drew attention at the time to the peculiar appearance of the corpse, he couldn't reasonably give a certificate. In which case there would be a post-mortem and an analysis, and the digitalin would be found. If he kept quiet, the money might be lost and all his trouble would be wasted. Maddenin' for him, wasn't it? So he did what he could. He put the time of the death as early as he dared, and hoped for the best.«

»He told me he thought there would be some attempt to make it seem later than it really was. I thought it was \textit{you} who were trying to hush everything up. And I was so furious that of course I told Mr Pritchard to have a proper inquiry made and on no account to compromise.«

»Thank God you did,« said Wimsey.

»Why?«

»I'll tell you presently. But Penberthy now—I can't think why \textit{he} didn't persuade you to compromise. That would have made him absolutely safe.«

»But he did! That's what started our first quarrel. As soon as he heard about it, he said I was a fool not to compromise. I couldn't understand his saying that, since he himself had said there was something wrong. We had a fearful row. That was the time I mentioned the twelve thousand that was coming to me anyway.«

»What did he say?«

»»I didn't know that.« Just like that. And then he apologized and said that the law was so uncertain, it would be best to agree to divide the money anyhow. So I rang up Mr Pritchard and told him not to make any more fuss. And we were friends again.«

»Was it the day after that, that Penberthy—er—said things to you?«

»Yes.«

»Right. Then I can tell you one thing: he would never have been so brutal if he hadn't been in fear of his life. Do you know what had happened in between?«

She shook her head.

»I had been on the phone to him, and told him there was going to be an autopsy.«

»Oh!«

»Yes—listen—you needn't worry any more about it. He knew the poison would be discovered, and that if he was known to be engaged to you, he was absolutely bound to be suspected. So he hurried to cut the connection with you—purely in self-defence.«

»But why do it in that brutal way?«

»Because, my dear, he knew that that particular accusation would be the very last thing a girl of your sort would tell people about. He made it absolutely impossible for you to claim him publicly. And he bolstered it up by engaging himself to the Rushworth female.«

»He didn't care how \textit{I} suffered.«

»He was in a beast of a hole,« said Wimsey, apologetically. »Mind you, it was a perfectly diabolical thing to do. I daresay he's feeling pretty rotten about it.«

Ann Dorland clenched her hands.

»I've been so horribly ashamed\longdash«

»Well, you aren't any more, are you?«

»No—but\longdash« A thought seemed to strike her. »Lord Peter—I can't \textit{prove} a word of this. Everybody will think I was in league with him. And they'll think that our quarrel and his getting engaged to Naomi was just a put-up job between us to get us both out of a difficulty.«

»You've got brains,« said Wimsey, admiringly. »\textit{Now} you see why I thanked God you'd been so keen on an inquiry at first. Pritchard can make it pretty certain that you weren't an accessory before the fact, anyhow.«

»Of course—so he can. Oh, I'm so glad! I \textit{am} so glad.« She burst into excited sobs and clutched Wimsey's hand. »I wrote him a letter—right at the beginning—saying I'd read about a case in which they'd proved the time of somebody's death by looking into his stomach, and asking if General Fentiman couldn't be dug up.«

»Did you? Splendid girl! You \textit{have} got a head on your shoulders!... No, I observe that it's on my shoulders. Go on. Have a real, good howl—I feel rather like howling myself. I've been quite worried about it all. But it's all right now, isn't it?«

»I am a fool \textellipsis  but I'm so thankful you came.«

»So am I. Here, have a hanky. Poor old dear!... Hullo! there's Marjorie.«

He released her and went out to meet Marjorie Phelps at the door.

»Lord Peter! Good lord!«

»Thank you, Marjorie,« said Wimsey, gravely.

»No, but listen! Have you seen Ann?—I took her away. She's frightfully queer—and there's a policeman outside. But whatever she's done, I couldn't leave her alone in that awful house. You haven't come to—to\longdash«

»Marjorie!« said Wimsey, »don't you ever talk to me again about feminine intuition. You've been thinking all this time that that girl was suffering from guilty conscience. Well, she wasn't. It was a man, my child—a \textsc{man}!«

»How do you know?«

»My experienced eye told me as much at the first glance. It's all right now. Sorrow and sighing have fled away. I am going to take your young friend out to dinner.«

»But why didn't she tell me what it was all about?«

»Because,« said Wimsey, mincingly, »it wasn't the kind of thing one woman tells another.«