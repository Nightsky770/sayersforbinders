%!TeX root=../bodytop.tex
\chapter[Chapter \thechapter]{}
\lettrine[lines=4,ante=‘]{A}{} regular pea-souper, by Jove,' said Lord~Peter.

\zz
Parker grunted, and struggled irritably into an overcoat.

\zz
<It affords me, if I may say so, the greatest satisfaction,> continued the noble lord, <that in a collaboration like ours all the uninteresting and disagreeable routine work is done by you.>

Parker grunted again.

<Do you anticipate any difficulty about the warrant?> inquired Lord~Peter.

Parker grunted a third time.

<I suppose you've seen to it that all this business is kept quiet?>

<Of course.>

<You've muzzled the workhouse people?>

<Of course.>

<And the police?>

<Yes.>

<Because, if you haven't there'll probably be nobody to arrest.>

<My dear Wimsey, do you think I'm a fool?>

<I had no such hope.>

Parker grunted finally and departed.

Lord~Peter settled down to a perusal of his Dante. It afforded him no solace. Lord~Peter was hampered in his career as a private detective by a public-school education. Despite Parker's admonitions, he was not always able to discount it. His mind had been warped in its young growth by <Raffles> and <Sherlock Holmes,> or the sentiments for which they stand. He belonged to a family which had never shot a fox.

<I am an amateur,> said Lord~Peter.

Nevertheless, while communing with Dante, he made up his mind.

In the afternoon he found himself in Harley Street. Sir Julian Freke might be consulted about one's nerves from two till four on Tuesdays and Fridays. Lord~Peter rang the bell.

<Have you an appointment, sir?> inquired the man who opened the door.

<No,> said Lord~Peter, <but will you give Sir Julian my card? I think it possible he may see me without one.>

He sat down in the beautiful room in which Sir Julian's patients awaited his healing counsel. It was full of people. Two or three fashionably dressed women were discussing shops and servants together, and teasing a toy griffon. A big, worried-looking man by himself in a corner looked at his watch twenty times a minute. Lord~Peter knew him by sight. It was Wintrington, a millionaire, who had tried to kill himself a few months ago. He controlled the finances of five countries, but he could not control his nerves. The finances of five countries were in Sir Julian Freke's capable hands. By the fireplace sat a soldierly-looking young man, of about Lord~Peter's own age. His face was prematurely lined and worn; he sat bolt upright, his restless eyes darting in the direction of every slightest sound. On the sofa was an elderly woman of modest appearance, with a young girl. The girl seemed listless and wretched; the woman's look showed deep affection, and anxiety tempered with a timid hope. Close beside Lord~Peter was another younger woman, with a little girl, and Lord~Peter noticed in both of them the broad cheekbones and beautiful grey, slanting eyes of the Slav. The child, moving restlessly about, trod on Lord~Peter's patent-leather toe, and the mother admonished her in French before turning to apologize to Lord~Peter.

<Mais je vous en prie, madame,> said the young man, <it is nothing.>

<She is nervous, pauvre petite,> said the young woman.

<You are seeking advice for her?>

<Yes. He is wonderful, the doctor. Figure to yourself, monsieur, she cannot forget, poor child, the things she has seen.> She leaned nearer, so that the child might not hear. <We have escaped—from starving Russia—six months ago. I dare not tell you—she has such quick ears, and then, the cries, the tremblings, the convulsions—they all begin again. We were skeletons when we arrived—mon Dieu!—but that is better now. See, she is thin, but she is not starved. She would be fatter but for the nerves that keep her from eating. We who are older, we forget—enfin, on apprend à ne pas y penser—but these children! When one is young, monsieur, tout ça impressionne trop.>

Lord~Peter, escaping from the thraldom of British good form, expressed himself in that language in which sympathy is not condemned to mutism.

<But she is much better, much better,> said the mother, proudly; <the great doctor, he does marvels.>

<C'est un homme précieux,> said Lord~Peter.

<Ah, monsieur, c'est un saint qui opère des miracles! Nous prions pour lui, Natasha et moi, tous les jours. N'est-ce pas, chérie? And consider, monsieur, that he does it all, ce grand homme, cet homme illustre, for nothing at all. When we come here, we have not even the clothes upon our backs—we are ruined, famished. Et avec ça que nous sommes de bonne famille—mais hélas! monsieur, en Russie, comme vous savez, ça ne vous vaut que des insultes—des atrocités. Enfin! the great Sir Julian sees us, he says—<Madame, your little girl is very interesting to me. Say no more. I cure her for nothing—pour ses beaux yeux,> a-t-il ajouté en riant. Ah, monsieur, c'est un saint, un véritable saint! and Natasha is much, much better.>

<Madame, je vous en félicite.>

<And you, monsieur? You are young, well, strong—you also suffer? It is still the war, perhaps?>

<A little remains of shell-shock,> said Lord~Peter.

<Ah, yes. So many good, brave, young men\longdash>

<Sir Julian can spare you a few minutes, my lord, if you will come in now,> said the servant.

Lord~Peter bowed to his neighbour, and walked across the waiting-room. As the door of the consulting-room closed behind him, he remembered having once gone, disguised, into the staff-room of a German officer. He experienced the same feeling—the feeling of being caught in a trap, and a mingling of bravado and shame.

He had seen Sir Julian Freke several times from a distance, but never close. Now, while carefully and quite truthfully detailing the circumstances of his recent nervous attack, he considered the man before him. A man taller than himself, with immense breadth of shoulder, and wonderful hands. A face beautiful, impassioned and inhuman; fanatical, compelling eyes, bright blue amid the ruddy bush of hair and beard. They were not the cool and kindly eyes of the family doctor, they were the brooding eyes of the inspired scientist, and they searched one through.

<Well,> thought Lord~Peter, <I shan't have to be explicit, anyhow.>

<Yes,> said Sir Julian, <yes. You had been working too hard. Puzzling your mind. Yes. More than that, perhaps—troubling your mind, shall we say?>

<I found myself faced with a very alarming contingency.>

<Yes. Unexpectedly, perhaps.>

<Very unexpected indeed.>

<Yes. Following on a period of mental and physical strain.>

<Well—perhaps. Nothing out of the way.>

<Yes. The unexpected contingency was—personal to yourself?>

<It demanded an immediate decision as to my own actions—yes, in that sense it was certainly personal.>

<Quite so. You would have to assume some responsibility, no doubt.>

<A very grave responsibility.>

<Affecting others besides yourself?>

<Affecting one other person vitally, and a very great number indirectly.>

<Yes. The time was night. You were sitting in the dark?>

<Not at first. I think I put the light out afterwards.>

<Quite so—that action would naturally suggest itself to you. Were you warm?>

<I think the fire had died down. My man tells me that my teeth were chattering when I went in to him.>

<Yes. You live in Piccadilly?>

<Yes.>

<Heavy traffic sometimes goes past during the night, I expect.>

<Oh, frequently.>

<Just so. Now this decision you refer to—you had taken that decision.>

<Yes.>

<Your mind was made up?>

<Oh, yes.>

<You had decided to take the action, whatever it was.>

<Yes.>

<Yes. It involved perhaps a period of inaction.>

<Of comparative inaction—yes.>

<Of suspense, shall we say?>

<Yes—of suspense, certainly.>

<Possibly of some danger?>

<I don't know that that was in my mind at the time.>

<No—it was a case in which you could not possibly consider yourself.>

<If you like to put it that way.>

<Quite so. Yes. You had these attacks frequently in 1918?>

<Yes—I was very ill for some months.>

<Quite. Since then they have recurred less frequently?>

<Much less frequently.>

<Yes—when did the last occur?>

<About nine months ago.>

<Under what circumstances?>

<I was being worried by certain family matters. It was a question of deciding about some investments, and I was largely responsible.>

<Yes. You were interested last year, I think, in some police case?>

<Yes—in the recovery of Lord~Attenbury's emerald necklace.>

<That involved some severe mental exercise?>

<I suppose so. But I enjoyed it very much.>

<Yes. Was the exertion of solving the problem attended by any bad results physically?>

<None.>

<No. You were interested, but not distressed.>

<Exactly.>

<Yes. You have been engaged in other investigations of the kind?>

<Yes. Little ones.>

<With bad results for your health?>

<Not a bit of it. On the contrary. I took up these cases as a sort of distraction. I had a bad knock just after the war, which didn't make matters any better for me, don't you know.>

<Ah! you are not married?>

<No.>

<No. Will you allow me to make an examination? Just come a little nearer to the light. I want to see your eyes. Whose advice have you had till now?>

<Sir James Hodges'.>

<Ah! yes—he was a sad loss to the medical profession. A really great man—a true scientist. Yes. Thank you. Now I should like to try you with this little invention.>

<What's it do?>

<Well—it tells me about your nervous reactions. Will you sit here?>

The examination that followed was purely medical. When it was concluded, Sir Julian said:

<Now, Lord~Peter, I'll tell you about yourself in quite untechnical language\longdash>

<Thanks,> said Peter, <that's kind of you. I'm an awful fool about long words.>

<Yes. Are you fond of private theatricals, Lord~Peter?>

<Not particularly,> said Peter, genuinely surprised. <Awful bore as a rule. Why?>

<I thought you might be,> said the specialist, drily. <Well, now. You know quite well that the strain you put on your nerves during the war has left its mark on you. It has left what I may call old wounds in your brain. Sensations received by your nerve-endings sent messages to your brain, and produced minute physical changes there—changes we are only beginning to be able to detect, even with our most delicate instruments. These changes in their turn set up sensations; or I should say, more accurately, that sensations are the names we give to these changes of tissue when we perceive them: we call them horror, fear, sense of responsibility and so on.>

<Yes, I follow you.>

<Very well. Now, if you stimulate those damaged places in your brain again, you run the risk of opening up the old wounds. I mean, that if you get nerve-sensations of any kind producing the reactions which we call horror, fear, and sense of responsibility, they may go on to make disturbance right along the old channel, and produce in their turn physical changes which you will call by the names you were accustomed to associate with them—dread of German mines, responsibility for the lives of your men, strained attention and the inability to distinguish small sounds through the overpowering noise of guns.>

<I see.>

<This effect would be increased by extraneous circumstances producing other familiar physical sensations—night, cold or the rattling of heavy traffic, for instance.>

<Yes.>

<Yes. The old wounds are nearly healed, but not quite. The ordinary exercise of your mental faculties has no bad effect. It is only when you excite the injured part of your brain.>

<Yes, I see.>

<Yes. You must avoid these occasions. You must learn to be irresponsible, Lord~Peter.>

<My friends say I'm only too irresponsible already.>

<Very likely. A sensitive nervous temperament often appears so, owing to its mental nimbleness.>

<Oh!>

<Yes. This particular responsibility you were speaking of still rests upon you?>

<Yes, it does.>

<You have not yet completed the course of action on which you have decided?>

<Not yet.>

<You feel bound to carry it through?>

<Oh, yes—I can't back out of it now.>

<No. You are expecting further strain?>

<A certain amount.>

<Do you expect it to last much longer?>

<Very little longer now.>

<Ah! Your nerves are not all they should be.>

<No?>

<No. Nothing to be alarmed about, but you must exercise care while undergoing this strain, and afterwards you should take a complete rest. How about a voyage in the Mediterranean or the South Seas or somewhere?>

<Thanks. I'll think about it.>

<Meanwhile, to carry you over the immediate trouble I will give you something to strengthen your nerves. It will do you no permanent good, you understand, but it will tide you over the bad time. And I will give you a prescription.>

<Thank you.>

Sir Julian got up and went into a small surgery leading out of the consulting-room. Lord~Peter watched him moving about—boiling something and writing. Presently he returned with a paper and a hypodermic syringe.

<Here is the prescription. And now, if you will just roll up your sleeve, I will deal with the necessity of the immediate moment.>

Lord~Peter obediently rolled up his sleeve. Sir Julian Freke selected a portion of his forearm and anointed it with iodine.

<What's that you're goin' to stick into me. Bugs?>

The surgeon laughed.

<Not exactly,> he said. He pinched up a portion of flesh between his finger and thumb. <You've had this kind of thing before, I expect.>

<Oh, yes,> said Lord~Peter. He watched the cool fingers, fascinated, and the steady approach of the needle. <Yes—I've had it before—and, d'you know—I don't care frightfully about it.>

He had brought up his right hand, and it closed over the surgeon's wrist like a vice.

The silence was like a shock. The blue eyes did not waver; they burned down steadily upon the heavy white lids below them. Then these slowly lifted; the grey eyes met the blue—coldly, steadily—and held them.

When lovers embrace, there seems no sound in the world but their own breathing. So the two men breathed face to face.

<As you like, of course, Lord~Peter,> said Sir Julian, courteously.

<Afraid I'm rather a silly ass,> said Lord~Peter, <but I never could abide these little gadgets. I had one once that went wrong and gave me a rotten bad time. They make me a bit nervous.>

<In that case,> replied Sir Julian, <it would certainly be better not to have the injection. It might rouse up just those sensations which we are desirous of avoiding. You will take the prescription, then, and do what you can to lessen the immediate strain as far as possible.>

<Oh, yes—I'll take it easy, thanks,> said Lord~Peter. He rolled his sleeve down neatly. <I'm much obliged to you. If I have any further trouble I'll look in again.>

<Do—do\longdash> said Sir Julian, cheerfully. <Only make an appointment another time. I'm rather rushed these days. I hope your mother is quite well. I saw her the other day at that Battersea inquest. You should have been there. It would have interested you.>