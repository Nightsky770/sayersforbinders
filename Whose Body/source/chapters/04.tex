%!TeX root=../bodytop.tex
\chapter[Chapter \thechapter]{}
\lettrine[lines=4,ante=‘—]{S}{o} there it is, Parker,' said Lord Peter, pushing his coffee-cup aside and lighting his after-breakfast pipe; »you may find it leads you to something, though it don't seem to get me any further with my bathroom problem. Did you do anything more at that after I left?«

»No; but I've been on the roof this morning.«

»The deuce you have—what an energetic devil you are! I say, Parker, I think this co-operative scheme is an uncommonly good one. It's much easier to work on someone else's job than one's own—gives one that delightful feelin' of interferin' and bossin' about, combined with the glorious sensation that another fellow is takin' all one's own work off one's hands. You scratch my back and I'll scratch yours, what? Did you find anything?«

»Not very much. I looked for any footmarks of course, but naturally, with all this rain, there wasn't a sign. Of course, if this were a detective story, there'd have been a convenient shower exactly an hour before the crime and a beautiful set of marks which could only have come there between two and three in the morning, but this being real life in a London November, you might as well expect footprints in Niagara. I searched the roofs right along—and came to the jolly conclusion that any person in any blessed flat in the blessed row might have done it. All the staircases open on to the roof and the leads are quite flat; you can walk along as easy as along Shaftesbury Avenue. Still, I've got some evidence that the body did walk along there.«

»What's that?«

Parker brought out his pocketbook and extracted a few shreds of material, which he laid before his friend.

»One was caught in the gutter just above Thipps's bathroom window, another in a crack of the stone parapet just over it, and the rest came from the chimney-stack behind, where they had caught in an iron stanchion. What do you make of them?«

Lord Peter scrutinized them very carefully through his lens.

»Interesting,« he said, »damned interesting. Have you developed those plates, Bunter?« he added, as that discreet assistant came in with the post.

»Yes, my lord.«

»Caught anything?«

»I don't know whether to call it anything or not, my lord,« said Bunter, dubiously. »I'll bring the prints in.«

»Do,« said Wimsey. »Hallo! here's our advertisement about the gold chain in the \textit{Times}---very nice it looks: »Write, 'phone or call 110, Piccadilly.« Perhaps it would have been safer to put a box number, though I always think that the franker you are with people, the more you're likely to deceive 'em; so unused is the modern world to the open hand and the guileless heart, what?«

»But you don't think the fellow who left that chain on the body is going to give himself away by coming here and inquiring about it?«

»I don't, fathead,« said Lord Peter, with the easy politeness of the real aristocracy; »that's why I've tried to get hold of the jeweller who originally sold the chain. See?« He pointed to the paragraph. »It's not an old chain—hardly worn at all. Oh, thanks, Bunter. Now, see here, Parker, these are the finger-marks you noticed yesterday on the window-sash and on the far edge of the bath. I'd overlooked them; I give you full credit for the discovery, I crawl, I grovel, my name is Watson, and you need not say what you were just going to say, because I admit it all. Now we shall—Hullo, hullo, hullo!«

The three men stared at the photographs.

»The criminal,« said Lord Peter, bitterly, »climbed over the roofs in the wet and not unnaturally got soot on his fingers. He arranged the body in the bath, and wiped away all traces of himself except two, which he obligingly left to show us how to do our job. We learn from a smudge on the floor that he wore india rubber boots, and from this admirable set of finger-prints on the edge of the bath that he had the usual number of fingers and wore rubber gloves. That's the kind of man he is. Take the fool away, gentlemen.«

He put the prints aside, and returned to an examination of the shreds of material in his hand. Suddenly he whistled softly.

»Do you make anything of these, Parker?«

»They seemed to me to be ravellings of some coarse cotton stuff—a sheet, perhaps, or an improvised rope.«

»Yes,« said Lord Peter---»yes. It may be a mistake—it may be \textit{our} mistake. I wonder. Tell me, d'you think these tiny threads are long enough and strong enough to hang a man?«

He was silent, his long eyes narrowing into slits behind the smoke of his pipe.

»What do you suggest doing this morning?« asked Parker.

»Well,« said Lord Peter, »it seems to me it's about time I took a hand in your job. Let's go round to Park Lane and see what larks Sir Reuben Levy was up to in bed last night.«

»And now, Mrs Pemming, if you would be so kind as to give me a blanket,« said Mr Bunter, coming down into the kitchen, »and permit of me hanging a sheet across the lower part of this window, and drawing the screen across here, so—so as to shut off any reflections, if you understand me, we'll get to work.«

Sir Reuben Levy's cook, with her eye upon Mr Bunter's gentlemanly and well-tailored appearance, hastened to produce what was necessary. Her visitor placed on the table a basket, containing a water-bottle, a silver-backed hair-brush, a pair of boots, a small roll of linoleum, and the »Letters of a Self-made Merchant to His Son,« bound in polished morocco. He drew an umbrella from beneath his arm and added it to the collection. He then advanced a ponderous photographic machine and set it up in the neighbourhood of the kitchen range; then, spreading a newspaper over the fair, scrubbed surface of the table, he began to roll up his sleeves and insinuate himself into a pair of surgical gloves. Sir Reuben Levy's valet, entering at the moment and finding him thus engaged, put aside the kitchenmaid, who was staring from a front-row position, and inspected the apparatus critically. Mr Bunter nodded brightly to him, and uncorked a small bottle of grey powder.

»Odd sort of fish, your employer, isn't he?« said the valet, carelessly.

»Very singular, indeed,« said Mr Bunter. »Now, my dear,« he added, ingratiatingly, to the kitchen-maid, »I wonder if you'd just pour a little of this grey powder over the edge of the bottle while I'm holding it—and the same with this boot—here, at the top—thank you, Miss—what is your name? Price? Oh, but you've got another name besides Price, haven't you? Mabel, eh? That's a name I'm uncommonly partial to—that's very nicely done, you've a steady hand, Miss Mabel—see that? That's the finger marks—three there, and two here, and smudged over in both places. No, don't you touch 'em, my dear, or you'll rub the bloom off. We'll stand 'em up here till they're ready to have their portraits taken. Now then, let's take the hair-brush next. Perhaps, Mrs Pemming, you'd like to lift him up very carefully by the bristles.«

»By the bristles, Mr Bunter?«

»If you please, Mrs Pemming—and lay him here. Now, Miss Mabel, another little exhibition of your skill, \textit{if} you please. No—we'll try lamp-black this time. Perfect. Couldn't have done it better myself. Ah! there's a beautiful set. No smudges this time. That'll interest his lordship. Now the little book—no, I'll pick that up myself—with these gloves, you see, and by the edges—I'm a careful criminal, Mrs Pemming, I don't want to leave any traces. Dust the cover all over, Miss Mabel; now this side—that's the way to do it. Lots of prints and no smudges. All according to plan. Oh, please, Mr Graves, you mustn't touch it—it's as much as my place is worth to have it touched.«

»D'you have to do much of this sort of thing?« inquired Mr Graves, from a superior standpoint.

»Any amount,« replied Mr Bunter, with a groan calculated to appeal to Mr Graves's heart and unlock his confidence. »If you'd kindly hold one end of this bit of linoleum, Mrs Pemming, I'll hold up this end while Miss Mabel operates. Yes, Mr Graves, it's a hard life, valeting by day and developing by night—morning tea at any time from 6.30 to 11, and criminal investigation at all hours. It's wonderful, the ideas these rich men with nothing to do get into their heads.«

»I wonder you stand it,« said Mr Graves. »Now there's none of that here. A quiet, orderly, domestic life, Mr Bunter, has much to be said for it. Meals at regular hours; decent, respectable families to dinner—none of your painted women—and no valeting at night, there's \textit{much} to be said for it. I don't hold with Hebrews as a rule, Mr Bunter, and of course I understand that you may find it to your advantage to be in a titled family, but there's less thought of that these days, and I will say, for a self-made man, no one could call Sir Reuben vulgar, and my lady at any rate is county—Miss Ford, she was, one of the Hampshire Fords, and both of them always most considerate.«

»I agree with you, Mr Graves—his lordship and me have never held with being narrow-minded—why, yes, my dear, of course it's a footmark, this is the washstand linoleum. A good Jew can be a good man, that's what I've always said. And regular hours and considerate habits have a great deal to recommend them. Very simple in his tastes, now, Sir Reuben, isn't he? for such a rich man, I mean.«

»Very simple indeed,« said the cook; »the meals he and her ladyship have when they're by themselves with Miss Rachel—well, there now—if it wasn't for the dinners, which is always good when there's company, I'd be wastin' my talents and education here, if you understand me, Mr Bunter.«

Mr Bunter added the handle of the umbrella to his collection, and began to pin a sheet across the window, aided by the housemaid.

»Admirable,« said he. »Now, if I might have this blanket on the table and another on a towel-horse or something of that kind by way of a background—you're very kind, Mrs Pemming.... Ah! I wish his lordship never wanted valeting at night. Many's the time I've sat up till three and four, and up again to call him early to go off Sherlocking at the other end of the country. And the mud he gets on his clothes and his boots!«

»I'm sure it's a shame, Mr Bunter,« said Mrs Pemming, warmly. »Low, I calls it. In my opinion, police-work ain't no fit occupation for a gentleman, let alone a lordship.«

»Everything made so difficult, too,« said Mr Bunter nobly sacrificing his employer's character and his own feelings in a good cause; »boots chucked into a corner, clothes hung up on the floor, as they say\longdash«

»That's often the case with these men as are born with a silver spoon in their mouths,« said Mr Graves. »Now, Sir Reuben, he's never lost his good old-fashioned habits. Clothes folded up neat, boots put out in his dressing-room, so as a man could get them in the morning, everything made easy.«

»He forgot them the night before last, though.«

»The clothes, not the boots. Always thoughtful for others, is Sir Reuben. Ah! I hope nothing's happened to him.«

»Indeed, no, poor gentleman,« chimed in the cook, »and as for what they're sayin', that he'd 'ave gone out surrepshous-like to do something he didn't ought, well, I'd never believe it of him, Mr Bunter, not if I was to take my dying oath upon it.«

»Ah!« said Mr Bunter, adjusting his arc-lamps and connecting them with the nearest electric light, »and that's more than most of us could say of them as pays us.«

»Five foot ten,« said Lord Peter, »and not an inch more.« He peered dubiously at the depression in the bed clothes, and measured it a second time with the gentleman-scout's vade-mecum. Parker entered this particular in a neat pocketbook.

»I suppose,« he said, »a six-foot-two man \textit{might} leave a five-foot-ten depression if he curled himself up.«

»Have you any Scotch blood in you, Parker?« inquired his colleague, bitterly.

»Not that I know of,« replied Parker. »Why?«

»Because of all the cautious, ungenerous, deliberate and cold-blooded devils I know,« said Lord Peter, »you are the most cautious, ungenerous, deliberate and cold-blooded. Here am I, sweating my brains out to introduce a really sensational incident into your dull and disreputable little police investigation, and you refuse to show a single spark of enthusiasm.«

»Well, it's no good jumping at conclusions.«

»Jump? You don't even crawl distantly within sight of a conclusion. I believe if you caught the cat with her head in the cream-jug you'd say it was conceivable that the jug was empty when she got there.«

»Well, it would be conceivable, wouldn't it?«

»Curse you,« said Lord Peter. He screwed his monocle into his eye, and bent over the pillow, breathing hard and tightly through his nose. »Here, give me the tweezers,« he said presently. »Good heavens, man, don't blow like that, you might be a whale.« He nipped up an almost invisible object from the linen.

»What is it?« asked Parker.

»It's a hair,« said Wimsey grimly, his hard eyes growing harder. »Let's go and look at Levy's hats, shall we? And you might just ring for that fellow with the churchyard name, do you mind?«

Mr Graves, when summoned, found Lord Peter Wimsey squatting on the floor of the dressing-room before a row of hats arranged upside down before him.

»Here you are,« said that nobleman cheerfully. »Now, Graves, this is a guessin' competition—a sort of three-hat trick, to mix metaphors. Here are nine hats, including three top-hats. Do you identify all these hats as belonging to Sir Reuben Levy? You do? Very good. Now I have three guesses as to which hat he wore the night he disappeared, and if I guess right, I win; if I don't, you win. See? Ready? Go. I suppose you know the answer yourself, by the way?«

»Do I understand your lordship to be asking which hat Sir Reuben wore when he went out on Monday night, your lordship?«

»No, you don't understand a bit,« said Lord Peter. »I'm asking if \textit{you} know—don't tell me, I'm going to guess.«

»I do know, your lordship,« said Mr Graves, reprovingly.

»Well,« said Lord Peter, »as he was dinin' at the Ritz he wore a topper. Here are three toppers. In three guesses I'd be bound to hit the right one, wouldn't I? That don't seem very sportin'. I'll take one guess. It was this one.«

He indicated the hat next the window.

»Am I right, Graves—have I got the prize?«

»That \textit{is} the hat in question, my lord,« said Mr Graves, without excitement.

»Thanks,« said Lord Peter, »that's all I wanted to know. Ask Bunter to step up, would you?«

Mr Bunter stepped up with an aggrieved air, and his usually smooth hair ruffled by the focussing cloth.

»Oh, there you are, Bunter,« said Lord Peter; »look here\longdash«

»Here I am, my lord,« said Mr Bunter, with respectful reproach, »but if you'll excuse me saying so, downstairs is where I ought to be, with all those young women about—they'll be fingering the evidence, my lord.«

»I cry your mercy,« said Lord Peter, »but I've quarrelled hopelessly with Mr Parker and distracted the estimable Graves, and I want you to tell me what finger-prints you have found. I shan't be happy till I get it, so don't be harsh with me, Bunter.«

»Well, my lord, your lordship understands I haven't photographed them yet, but I won't deny that their appearance is interesting, my lord. The little book off the night table, my lord, has only the marks of one set of fingers—there's a little scar on the right thumb which makes them easy recognised. The hair-brush, too, my lord, has only the same set of marks. The umbrella, the toothglass and the boots all have two sets: the hand with the scarred thumb, which I take to be Sir Reuben's, my lord, and a set of smudges superimposed upon them, if I may put it that way, my lord, which may or may not be the same hand in rubber gloves. I could tell you better when I've got the photographs made, to measure them, my lord. The linoleum in front of the washstand is very gratifying indeed, my lord, if you will excuse my mentioning it. Besides the marks of Sir Reuben's boots which your lordship pointed out, there's the print of a man's naked foot—a much smaller one, my lord, not much more than a ten-inch sock, I should say if you asked me.«

Lord Peter's face became irradiated with almost a dim, religious light.

»A mistake,« he breathed, »a mistake, a little one, but he can't afford it. When was the linoleum washed last, Bunter?«

»Monday morning, my lord. The housemaid did it and remembered to mention it. Only remark she's made yet, and it's to the point. The other domestics\longdash«

His features expressed disdain.

»What did I say, Parker? Five-foot-ten and not an inch longer. And he didn't dare to use the hair-brush. Beautiful. But he \textit{had} to risk the top-hat. Gentleman can't walk home in the rain late at night without a hat, you know, Parker. Look! what do you make of it? Two sets of finger-prints on everything but the book and the brush, two sets of feet on the linoleum, and two kinds of hair in the hat!«

He lifted the top-hat to the light, and extracted the evidence with tweezers.

»Think of it, Parker—to remember the hair-brush and forget the hat—to remember his fingers all the time, and to make that one careless step on the tell-tale linoleum. Here they are, you see, black hair and tan hair—black hair in the bowler and the panama, and black and tan in last night's topper. And then, just to make certain that we're on the right track, just one little auburn hair on the pillow, on this pillow, Parker, which isn't quite in the right place. It almost brings tears to my eyes.«

»Do you mean to say\longdash« said the detective, slowly.

»I mean to say,« said Lord Peter, »that it was not Sir Reuben Levy whom the cook saw last night on the doorstep. I say that it was another man, perhaps a couple of inches shorter, who came here in Levy's clothes and let himself in with Levy's latchkey. Oh, he was a bold, cunning devil, Parker. He had on Levy's boots, and every stitch of Levy's clothing down to the skin. He had rubber gloves on his hands which he never took off, and he did everything he could to make us think that Levy slept here last night. He took his chances, and won. He walked upstairs, he undressed, he even washed and cleaned his teeth, though he didn't use the hair-brush for fear of leaving red hairs in it. He had to guess what Levy did with boots and clothes; one guess was wrong and the other right, as it happened. The bed must look as if it had been slept in, so he gets in, and lies there in his victim's very pyjamas. Then, in the morning sometime, probably in the deadest hour between two and three, he gets up, dresses himself in his own clothes that he has brought with him in a bag, and creeps downstairs. If anybody wakes, he is lost, but he is a bold man, and he takes his chance. He knows that people do not wake as a rule—and they don't wake. He opens the street door which he left on the latch when he came in—he listens for the stray passer-by or the policeman on his beat. He slips out. He pulls the door quietly to with the latchkey. He walks briskly away in rubber-soled shoes—he's the kind of criminal who isn't complete without rubber-soled shoes. In a few minutes he is at Hyde Park Corner. After that\longdash«

He paused, and added:

»He did all that, and unless he had nothing at stake, he had everything at stake. Either Sir Reuben Levy has been spirited away for some silly practical joke, or the man with the auburn hair has the guilt of murder upon his soul.«

»Dear me!« ejaculated the detective, »you're very dramatic about it.«

Lord Peter passed his hand rather wearily over his hair.

»My true friend,« he murmured in a voice surcharged with emotion, »You recall me to the nursery rhymes of my youth—the sacred duty of flippancy:
\begin{samepage}
\begin{verse}
There was an old man of Whitehaven\\
Who danced a quadrille with a raven,\\
\vin But they said: It's absurd\\
\vin To encourage that bird---\\
So they smashed that old man of Whitehaven.\\
\end{verse}
\end{samepage}

That's the correct attitude, Parker. Here's a poor old buffer spirited away—such a joke—and I don't believe he'd hurt a fly himself—that makes it funnier. D'you know, Parker, I don't care frightfully about this case after all.«

»Which, this or yours?«

»Both. I say, Parker, shall we go quietly home and have lunch and go to the Coliseum?«

»You can if you like,« replied the detective; »but you forget I do this for my bread and butter.«

»And I haven't even that excuse,« said Lord Peter; »well, what's the next move? What would you do in my case?«

»I'd do some good, hard grind,« said Parker. »I'd distrust every bit of work Sugg ever did, and I'd get the family history of every tenant of every flat in Queen Caroline Mansions. I'd examine all their box-rooms and rooftraps, and I would inveigle them into conversations and suddenly bring in the words »body« and »pince-nez,« and see if they wriggled, like those modern psyo-what's-his-names.«

»You would, would you?« said Lord Peter with a grin. »Well, we've exchanged cases, you know, so just you toddle off and do it. I'm going to have a jolly time at Wyndham's.«

Parker made a grimace.

»Well,« he said, »I don't suppose you'd ever do it, so I'd better. You'll never become a professional till you learn to do a little work, Wimsey. How about lunch?«

»I'm invited out,« said Lord Peter, magnificently. »I'll run around and change at the club. Can't feed with Freddy Arbuthnot in these bags; Bunter!«

»Yes, my lord.«

»Pack up if you're ready, and come round and wash my face and hands for me at the club.«

»Work here for another two hours, my lord. Can't do with less than thirty minutes' exposure. The current's none too strong.«

»You see how I'm bullied by my own man, Parker? Well, I must bear it, I suppose. Ta-ta!«

He whistled his way downstairs.

The conscientious Mr Parker, with a groan, settled down to a systematic search through Sir Reuben Levy's papers, with the assistance of a plate of ham sandwiches and a bottle of Bass.

Lord Peter and the Honourable Freddy Arbuthnot, looking together like an advertisement for gents' trouserings, strolled into the dining-room at Wyndham's.

»Haven't seen you for an age,« said the Honourable Freddy. »What have you been doin' with yourself?«

»Oh, foolin' about,« said Lord Peter, languidly.

»Thick or clear, sir?« inquired the waiter of the Honourable Freddy.

»Which'll you have, Wimsey?« said that gentleman, transferring the burden of selection to his guest. »They're both equally poisonous.«

»Well, clear's less trouble to lick out of the spoon,« said Lord Peter.

»Clear,« said the Honourable Freddy.

»Consommé Polonais,« agreed the waiter. »Very nice, sir.«

Conversation languished until the Honourable Freddy found a bone in the filleted sole, and sent for the head waiter to explain its presence. When this matter had been adjusted Lord Peter found energy to say:

»Sorry to hear about your gov'nor, old man.«

»Yes, poor old buffer,« said the Honourable Freddy; »they say he can't last long now. What? Oh! the Montrachet '08. There's nothing fit to drink in this place,« he added gloomily.

After this deliberate insult to a noble vintage there was a further pause, till Lord Peter said: »How's 'Change?«

»Rotten,« said the Honourable Freddy.

He helped himself gloomily to salmis of game.

»Can I do anything?« asked Lord Peter.

»Oh, no, thanks—very decent of you, but it'll pan out all right in time.«

»This isn't a bad salmis,« said Lord Peter.

»I've eaten worse,« admitted his friend.

»What about those Argentines?« inquired Lord Peter. »Here, waiter, there's a bit of cork in my glass.«

»Cork?« cried the Honourable Freddy, with something approaching animation; »you'll hear about this, waiter. It's an amazing thing a fellow who's paid to do the job can't manage to take a cork out of a bottle. What you say? Argentines? Gone all to hell. Old Levy bunkin' off like that's knocked the bottom out of the market.«

»You don't say so,« said Lord Peter. »What d'you suppose has happened to the old man?«

»Cursed if I know,« said the Honourable Freddy; »knocked on the head by the bears, I should think.«

»P'r'aps he's gone off on his own,« suggested Lord Peter. »Double life, you know. Giddy old blighters, some of these City men.«

»Oh, no,« said the Honourable Freddy, faintly roused; »no, hang it all, Wimsey, I wouldn't care to say that. He's a decent old domestic bird, and his daughter's a charmin' girl. Besides, he's straight enough—he'd \textit{do} you down fast enough, but he wouldn't \textit{let} you down. Old Anderson is badly cut up about it.«

»Who's Anderson?«

»Chap with property out there. He belongs here. He was goin' to meet Levy on Tuesday. He's afraid those railway people will get in now, and then it'll be all U. P.«

»Who's runnin' the railway people over here?« inquired Lord Peter.

»Yankee blighter, John P. Milligan. He's got an option, or says he has. You can't trust these brutes.«

»Can't Anderson hold on?«

»Anderson isn't Levy. Hasn't got the shekels. Besides, he's only one. Levy covers the ground—he could boycott Milligan's beastly railway if he liked. That's where he's got the pull, you see.«

»B'lieve I met the Milligan man somewhere,« said Lord Peter, thoughtfully. »Ain't he a hulking brute with black hair and a beard?«

»You're thinkin' of somebody else,« said the Honourable Freddy. »Milligan don't stand any higher than I do, unless you call five-feet-ten hulking—and he's bald, anyway.«

Lord Peter considered this over the Gorgonzola. Then he said: »Didn't know Levy had a charmin' daughter.«

»Oh, yes,« said the Honourable Freddy, with an elaborate detachment. »Met her and Mamma last year abroad. That's how I got to know the old man. He's been very decent. Let me into this Argentine business on the ground floor, don't you know?«

»Well,« said Lord Peter, »you might do worse. Money's money, ain't it? And Lady Levy is quite a redeemin' point. At least, my mother knew her people.«

»Oh, \textit{she's} all right,« said the Honourable Freddy, »and the old man's nothing to be ashamed of nowadays. He's self-made, of course, but he don't pretend to be anything else. No side. Toddles off to business on a 96 'bus every morning. »Can't make up my mind to taxis, my boy,« he says. »I had to look at every halfpenny when I was a young man, and I can't get out of the way of it now.« Though, if he's takin' his family out, nothing's too good. Rachel—that's the girl—always laughs at the old man's little economies.«

»I suppose they've sent for Lady Levy,« said Lord Peter.

»I suppose so,« agreed the other. »I'd better pop round and express sympathy or somethin', what? Wouldn't look well not to, d'you think? But it's deuced awkward. What am I to say?«

»I don't think it matters much what you say,« said Lord Peter, helpfully. »I should ask if you can do anything.«

»Thanks,« said the lover, »I will. Energetic young man. Count on me. Always at your service. Ring me up any time of the day or night. That's the line to take, don't you think?«

»That's the idea,« said Lord Peter.

Mr John P. Milligan, the London representative of the great Milligan railroad and shipping company, was dictating code cables to his secretary in an office in Lombard Street, when a card was brought up to him, bearing the simple legend:

\begin{center}
\textsc{Lord Peter Wimsey}\\
\textit{Marlborough Club}
\end{center}

Mr Milligan was annoyed at the interruption, but, like many of his nation, if he had a weak point, it was the British aristocracy. He postponed for a few minutes the elimination from the map of a modest but promising farm, and directed that the visitor should be shown up.

»Good-afternoon,« said that nobleman, ambling genially in, »it's most uncommonly good of you to let me come round wastin' your time like this. I'll try not to be too long about it, though I'm not awfully good at comin' to the point. My brother never would let me stand for the county, y'know—said I wandered on so nobody'd know what I was talkin' about.«

»Pleased to meet you, Lord Wimsey,« said Mr Milligan. »Won't you take a seat?«

»Thanks,« said Lord Peter, »but I'm not a peer, you know—that's my brother Denver. My name's Peter. It's a silly name, I always think, so old-world and full of homely virtue and that sort of thing, but my godfathers and godmothers in my baptism are responsible for that, I suppose, officially—which is rather hard on them, you know, as they didn't actually choose it. But we always have a Peter, after the third duke, who betrayed five kings somewhere about the Wars of the Roses, though come to think of it, it ain't anything to be proud of. Still, one has to make the best of it.«

Mr Milligan, thus ingeniously placed at that disadvantage which attends ignorance, manoeuvred for position, and offered his interrupter a Corona Corona.

»Thanks, awfully,« said Lord Peter, »though you really mustn't tempt me to stay here burblin' all afternoon. By Jove, Mr Milligan, if you offer people such comfortable chairs and cigars like these, I wonder they don't come an' live in your office.« He added mentally: »I wish to goodness I could get those long-toed boots off you. How's a man to know the size of your feet? And a head like a potato. It's enough to make one swear.«

»Say now, Lord Peter,« said Mr Milligan, »can I do anything for you?«

»Well, d'you know,« said Lord Peter, »I'm wonderin' if you would. It's damned cheek to ask you, but fact is, it's my mother, you know. Wonderful woman, but don't realize what it means, demands on the time of a busy man like you. We don't understand hustle over here, you know, Mr Milligan.«

»Now don't you mention that,« said Mr Milligan; »I'd be surely charmed to do anything to oblige the Duchess.«

He felt a momentary qualm as to whether a duke's mother were also a duchess, but breathed more freely as Lord Peter went on:

»Thanks—that's uncommonly good of you. Well, now, it's like this. My mother—most energetic, self-sacrificin' woman, don't you see, is thinkin' of gettin' up a sort of a charity bazaar down at Denver this winter, in aid of the church roof, y'know. Very sad case, Mr Milligan—fine old antique—early English windows and decorated angel roof, and all that—all tumblin' to pieces, rain pourin' in and so on—vicar catchin' rheumatism at early service, owin' to the draught blowin' in over the altar—you know the sort of thing. They've got a man down startin' on it—little beggar called Thipps—lives with an aged mother in Battersea—vulgar little beast, but quite good on angel roofs and things, I'm told.«

At this point, Lord Peter watched his interlocutor narrowly, but finding that this rigmarole produced in him no reaction more startling than polite interest tinged with faint bewilderment, he abandoned this line of investigation, and proceeded:

»I say, I beg your pardon, frightfully—I'm afraid I'm bein' beastly long-winded. Fact is, my mother is gettin' up this bazaar, and she thought it'd be an awfully interestin' side-show to have some lectures—sort of little talks, y'know—by eminent business men of all nations. »How I Did It« kind of touch, y'know---»A Drop of Oil with a Kerosene King«---»Cash Conscience and Cocoa« and so on. It would interest people down there no end. You see, all my mother's friends will be there, and we've none of us any money—not what you'd call money, I mean—I expect our incomes wouldn't pay your telephone calls, would they?---but we like awfully to hear about the people who can make money. Gives us a sort of uplifted feelin', don't you know. Well, anyway, I mean, my mother'd be frightfully pleased and grateful to you, Mr Milligan, if you'd come down and give us a few words as a representative American. It needn't take more than ten minutes or so, y'know, because the local people can't understand much beyond shootin' and huntin', and my mother's crowd can't keep their minds on anythin' more than ten minutes together, but we'd really appreciate it very much if you'd come and stay a day or two and just give us a little breezy word on the almighty dollar.«

»Why, yes,« said Mr Milligan, »I'd like to, Lord Peter. It's kind of the Duchess to suggest it. It's a very sad thing when these fine old antiques begin to wear out. I'll come with great pleasure. And perhaps you'd be kind enough to accept a little donation to the Restoration Fund.«

This unexpected development nearly brought Lord Peter up all standing. To pump, by means of an ingenious lie, a hospitable gentleman whom you are inclined to suspect of a peculiarly malicious murder, and to accept from him in the course of the proceedings a large cheque for a charitable object, has something about it unpalatable to any but the hardened Secret Service agent. Lord Peter temporized.

»That's awfully decent of you,« he said. »I'm sure they'd be no end grateful. But you'd better not give it to me, you know. I might spend it, or lose it. I'm not very reliable, I'm afraid. The vicar's the right person—the Rev. Constantine Throgmorton, St John-before-the-Latin-Gate Vicarage, Duke's Denver, if you like to send it there.«

»I will,« said Mr Milligan. »Will you write it out now for a thousand pounds, Scoot, in case it slips my mind later?«

The secretary, a sandy-haired young man with a long chin and no eyebrows, silently did as he was requested. Lord Peter looked from the bald head of Mr Milligan to the red head of the secretary, hardened his heart and tried again.

»Well, I'm no end grateful to you, Mr Milligan, and so'll my mother be when I tell her. I'll let you know the date of the bazaar—it's not quite settled yet, and I've got to see some other business men, don't you know. I thought of askin' someone from one of the big newspaper combines to represent British advertisin' talent, what?---and a friend of mine promises me a leadin' German financier—very interestin' if there ain't too much feelin' against it down in the country, and I'll have to find somebody or other to do the Hebrew point of view. I thought of askin' Levy, y'know, only he's floated off in this inconvenient way.«

»Yes,« said Mr Milligan, »that's a very curious thing, though I don't mind saying, Lord Peter, that it's a convenience to me. He had a cinch on my railroad combine, but I'd nothing against him personally, and if he turns up after I've brought off a little deal I've got on, I'll be happy to give him the right hand of welcome.«

A vision passed through Lord Peter's mind of Sir Reuben kept somewhere in custody till a financial crisis was over. This was exceedingly possible, and far more agreeable than his earlier conjecture; it also agreed better with the impression he was forming of Mr Milligan.

»Well, it's a rum go,« said Lord Peter, »but I daresay he had his reasons. Much better not inquire into people's reasons, y'know, what? Specially as a police friend of mine who's connected with the case says the old johnnie dyed his hair before he went.«

Out of the tail of his eye, Lord Peter saw the redheaded secretary add up five columns of figures simultaneously and jot down the answer.

»Dyed his hair, did he?« said Mr Milligan.

»Dyed it red,« said Lord Peter. The secretary looked up. »Odd thing is,« continued Wimsey, »they can't lay hands on the bottle. Somethin' fishy there, don't you think, what?«

The secretary's interest seemed to have evaporated. He inserted a fresh sheet into his looseleaf ledger, and carried forward a row of digits from the preceding page.

»I daresay there's nothin' in it,« said Lord Peter, rising to go. »Well, it's uncommonly good of you to be bothered with me like this, Mr Milligan—my mother'll be no end pleased. She'll write you about the date.«

»I'm charmed,« said Mr Milligan. »Very pleased to have met you.«

Mr Scoot rose silently to open the door, uncoiling as he did so a portentous length of thin leg, hitherto hidden by the desk. With a mental sigh Lord Peter estimated him at six-foot-four.

»It's a pity I can't put Scoot's head on Milligan's shoulders,« said Lord Peter, emerging into the swirl of the city. »And what \textit{will} my mother say?«