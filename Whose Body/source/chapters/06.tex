%!TeX root=../bodytop.tex
\chapter[Chapter \thechapter]{}
\lettrine[lines=4]{I}{t} was, in fact, inconvenient for Mr Parker to leave London. He had had to go and see Lady Levy towards the end of the morning, and subsequently his plans for the day had been thrown out of gear and his movements delayed by the discovery that the adjourned inquest of Mr Thipps's unknown visitor was to be held that afternoon, since nothing very definite seemed forthcoming from Inspector Sugg's inquiries. Jury and witnesses had been convened accordingly for three o'clock. Mr Parker might altogether have missed the event, had he not run against Sugg that morning at the Yard and extracted the information from him as one would a reluctant tooth. Inspector Sugg, indeed, considered Mr Parker rather interfering; moreover, he was hand-in-glove with Lord Peter Wimsey, and Inspector Sugg had no words for the interferingness of Lord Peter. He could not, however, when directly questioned, deny that there was to be an inquest that afternoon, nor could he prevent Mr Parker from enjoying the inalienable right of any interested British citizen to be present. At a little before three, therefore, Mr Parker was in his place, and amusing himself with watching the efforts of those persons who arrived after the room was packed to insinuate, bribe or bully themselves into a position of vantage. The Coroner, a medical man of precise habits and unimaginative aspect, arrived punctually, and looking peevishly round at the crowded assembly, directed all the windows to be opened, thus letting in a stream of drizzling fog upon the heads of the unfortunates on that side of the room. This caused a commotion and some expressions of disapproval, checked sternly by the Coroner, who said that with the influenza about again an unventilated room was a death-trap; that anybody who chose to object to open windows had the obvious remedy of leaving the court, and further, that if any disturbance was made he would clear the court. He then took a Formamint lozenge, and proceeded, after the usual preliminaries, to call up fourteen good and lawful persons and swear them diligently to inquire and a true presentment make of all matters touching the death of the gentleman with the pince-nez and to give a true verdict according to the evidence, so help them God. When an expostulation by a woman juror\allowbreak---\allowbreak an elderly lady in spectacles who kept a sweet-shop, and appeared to wish she was back there\allowbreak---\allowbreak had been summarily quashed by the Coroner, the jury departed to view the body. Mr Parker gazed round again and identified the unhappy Mr Thipps and the girl Gladys led into an adjoining room under the grim guard of the police. They were soon followed by a gaunt old lady in a bonnet and mantle. With her, in a wonderful fur coat and a motor bonnet of fascinating construction, came the Dowager Duchess of Denver, her quick, dark eyes darting hither and thither about the crowd. The next moment they had lighted on Mr Parker, who had several times visited the Dower House, and she nodded to him, and spoke to a policeman. Before long, a way opened magically through the press, and Mr Parker found himself accommodated with a front seat just behind the Duchess, who greeted him charmingly, and said: »What's happened to poor Peter?« Parker began to explain, and the Coroner glanced irritably in their direction. Somebody went up and whispered in his ear, at which he coughed, and took another Formamint.

»We came up by car,« said the Duchess---»so tiresome\allowbreak---\allowbreak such bad roads between Denver and Gunbury St Walters\allowbreak---\allowbreak and there were people coming to lunch\allowbreak---\allowbreak I had to put them off\allowbreak---\allowbreak I couldn't let the old lady go alone, could I? By the way, such an odd thing's happened about the Church Restoration Fund\allowbreak---\allowbreak the Vicar\allowbreak---\allowbreak oh, dear, here are these people coming back again; well, I'll tell you afterwards\allowbreak---\allowbreak do look at that woman looking shocked, and the girl in tweeds trying to look as if she sat on undraped gentlemen every day of her life\allowbreak---\allowbreak I don't mean that\allowbreak---\allowbreak corpses of course\allowbreak---\allowbreak but one finds oneself being so Elizabethan nowadays\allowbreak---\allowbreak what an awful little man the coroner is, isn't he? He's looking daggers at me\allowbreak---\allowbreak do you think he'll dare to clear me out of the court or commit me for what-you-may-call-it?«

The first part of the evidence was not of great interest to Mr Parker. The wretched Mr Thipps, who had caught cold in gaol, deposed in an unhappy croak to having discovered the body when he went in to take his bath at eight o'clock. He had had such a shock, he had to sit down and send the girl for brandy. He had never seen the deceased before. He had no idea how he came there.

Yes, he had been in Manchester the day before. He had arrived at St Pancras at ten o'clock. He had cloak-roomed his bag. At this point Mr Thipps became very red, unhappy and confused, and glanced nervously about the court.

»Now, Mr Thipps,« said the Coroner, briskly, »we must have your movements quite clear. You must appreciate the importance of the matter. You have chosen to give evidence, which you need not have done, but having done so, you will find it best to be perfectly explicit.«

»Yes,« said Mr Thipps faintly.

»Have you cautioned this witness, officer?« inquired the Coroner, turning sharply to Inspector Sugg.

The Inspector replied that he had told Mr Thipps that anything he said might be used agin' him at his trial. Mr Thipps became ashy, and said in a bleating voice that he 'adn't\allowbreak---\allowbreak hadn't meant to do anything that wasn't right.

This remark produced a mild sensation, and the Coroner became even more acidulated in manner than before.

»Is anybody representing Mr Thipps?« he asked, irritably. »No? Did you not explain to him that he could\allowbreak---\allowbreak that he \textit{ought} to be represented? You did not? Really, Inspector! Did you not know, Mr Thipps, that you had a right to be legally represented?«

Mr Thipps clung to a chair-back for support, and said, »No,« in a voice barely audible.

»It is incredible,« said the Coroner, »that so-called educated people should be so ignorant of the legal procedure of their own country. This places us in a very awkward position. I doubt, Inspector, whether I should permit the prisoner\allowbreak---\allowbreak Mr Thipps\allowbreak---\allowbreak to give evidence at all. It is a delicate position.«

The perspiration stood on Mr Thipps's forehead.

»Save us from our friends,« whispered the Duchess to Parker. »If that cough-drop-devouring creature had openly instructed those fourteen people\allowbreak---\allowbreak and what unfinished-looking faces they have\allowbreak---\allowbreak so characteristic, I always think, of the lower middle-class, rather like sheep, or calves' head (boiled, I mean), to bring in wilful murder against the poor little man, he couldn't have made himself plainer.«

»He can't let him incriminate himself, you know,« said Parker.

»Stuff!« said the Duchess. »How could the man incriminate himself when he never did anything in his life? You men never think of anything but your red tape.«

Meanwhile Mr Thipps, wiping his brow with a handkerchief, had summoned up courage. He stood up with a kind of weak dignity, like a small white rabbit brought to bay.

»I would rather tell you,« he said, »though it's reelly very unpleasant for a man in my position. But I reelly couldn't have it thought for a moment that I'd committed this dreadful crime. I assure you, gentlemen, I \textit{couldn't bear} that. No. I'd rather tell you the truth, though I'm afraid it places me in rather a\allowbreak---\allowbreak well, I'll tell you.«

»You fully understand the gravity of making such a statement, Mr Thipps,« said the Coroner.

»Quite,« said Mr Thipps. »It's all right\allowbreak---\allowbreak I---might I have a drink of water?«

»Take your time,« said the Coroner, at the same time robbing his remark of all conviction by an impatient glance at his watch.

»Thank you, sir,« said Mr Thipps. »Well, then, it's true I got to St Pancras at ten. But there was a man in the carriage with me. He'd got in at Leicester. I didn't recognise him at first, but he turned out to be an old school-fellow of mine.«

»What was this gentleman's name?« inquired the Coroner, his pencil poised.

Mr Thipps shrank together visibly.

»I'm afraid I can't tell you that,« he said. »You see\allowbreak---\allowbreak that is, you \textit{will} see\allowbreak---\allowbreak it would get him into trouble, and I couldn't do that\allowbreak---\allowbreak no, I reelly couldn't do that, not if my life depended on it. No!« he added, as the ominous pertinence of the last phrase smote upon him, »I'm sure I couldn't do that.«

»Well, well,« said the Coroner.

The Duchess leaned over to Parker again. »I'm beginning quite to admire the little man,« she said.

Mr Thipps resumed.

»When we got to St Pancras I was going home, but my friend said no. We hadn't met for a long time and we ought to\allowbreak---\allowbreak to make a night of it, was his expression. I fear I was weak, and let him overpersuade me to accompany him to one of his haunts. I use the word advisedly,« said Mr Thipps, »and I assure you, sir, that if I had known beforehand where we were going I never would have set foot in the place.«

»I cloak-roomed my bag, for he did not like the notion of our being encumbered with it, and we got into a taxicab and drove to the corner of Tottenham Court Road and Oxford Street. We then walked a little way, and turned into a side street (I do not recollect which) where there was an open door, with the light shining out. There was a man at a counter, and my friend bought some tickets, and I heard the man at the counter say something to him about »Your friend,« meaning me, and my friend said, »Oh, yes, he's been here before, haven't you, Alf?« (which was what they called me at school), though I assure you, sir«---here Mr Thipps grew very earnest---»I never had, and nothing in the world should induce me to go to such a place again.«

»Well, we went down into a room underneath, where there were drinks, and my friend had several, and made me take one or two\allowbreak---\allowbreak though I am an abstemious man as a rule\allowbreak---\allowbreak and he talked to some other men and girls who were there\allowbreak---\allowbreak a very vulgar set of people, I thought them, though I wouldn't say but what some of the young ladies were nice-looking enough. One of them sat on my friend's knee and called him a slow old thing, and told him to come on\allowbreak---\allowbreak so we went into another room, where there were a lot of people dancing all these up-to-date dances. My friend went and danced, and I sat on a sofa. One of the young ladies came up to me and said, didn't I dance, and I said »No,« so she said wouldn't I stand her a drink then. »You'll stand us a drink then, darling,« that was what she said, and I said, »Wasn't it after hours?« and she said that didn't matter. So I ordered the drink\allowbreak---\allowbreak a gin and bitters it was\allowbreak---\allowbreak for I didn't like not to, the young lady seemed to expect it of me and I felt it wouldn't be gentlemanly to refuse when she asked. But it went against my conscience\allowbreak---\allowbreak such a young girl as she was\allowbreak---\allowbreak and she put her arm round my neck afterwards and kissed me just like as if she was paying for the drink\allowbreak---\allowbreak and it reelly went to my 'eart,« said Mr Thipps, a little ambiguously, but with uncommon emphasis.

Here somebody at the back said, »Cheer-oh!« and a sound was heard as of the noisy smacking of lips.

»Remove the person who made that improper noise,« said the Coroner, with great indignation. »Go on, please, Mr Thipps.«

»Well,« said Mr Thipps, »about half-past twelve, as I should reckon, things began to get a bit lively, and I was looking for my friend to say good-night, not wishing to stay longer, as you will understand, when I saw him with one of the young ladies, and they seemed to be getting on altogether too well, if you follow me, my friend pulling the ribbons off her shoulder and the young lady laughing\allowbreak---\allowbreak and so on,« said Mr Thipps, hurriedly, »so I thought I'd just slip quietly out, when I heard a scuffle and a shout\allowbreak---\allowbreak and before I knew what was happening there were half-a-dozen policemen in, and the lights went out, and everybody stampeding and shouting\allowbreak---\allowbreak quite horrid, it was. I was knocked down in the rush, and hit my head a nasty knock on a chair\allowbreak---\allowbreak that was where I got that bruise they asked me about\allowbreak---\allowbreak and I was dreadfully afraid I'd never get away and it would all come out, and perhaps my photograph in the papers, when someone caught hold of me\allowbreak---\allowbreak I think it was the young lady I'd given the gin and bitters to\allowbreak---\allowbreak and she said, »This way,« and pushed me along a passage and out at the back somewhere. So I ran through some streets, and found myself in Goodge Street, and there I got a taxi and came home. I saw the account of the raid afterwards in the papers, and saw my friend had escaped, and so, as it wasn't the sort of thing I wanted made public, and I didn't want to get him into difficulties, I just said nothing. But that's the truth.«

»Well, Mr Thipps,« said the Coroner, »we shall be able to substantiate a certain amount of this story. Your friend's name\longdash«

»No,« said Mr Thipps, stoutly, »not on any account.«

»Very good,« said the Coroner. »Now, can you tell us what time you did get in?«

»About half-past one, I should think. Though reelly, I was so upset\longdash«

»Quite so. Did you go straight to bed?«

»Yes, I took my sandwich and glass of milk first. I thought it might settle my inside, so to speak,« added the witness, apologetically, »not being accustomed to alcohol so late at night and on an empty stomach, as you may say.«

»Quite so. Nobody sat up for you?«

»Nobody.«

»How long did you take getting to bed first and last?«

Mr Thipps thought it might have been half-an-hour.

»Did you visit the bathroom before turning in?«

»No.«

»And you heard nothing in the night?«

»No. I fell fast asleep. I was rather agitated, so I took a little dose to make me sleep, and what with being so tired and the milk and the dose, I just tumbled right off and didn't wake till Gladys called me.«

Further questioning elicited little from Mr Thipps. Yes, the bathroom window had been open when he went in in the morning, he was sure of that, and he had spoken very sharply to the girl about it. He was ready to answer any questions; he would be only too 'appy\allowbreak---\allowbreak happy to have this dreadful affair sifted to the bottom.

Gladys Horrocks stated that she had been in Mr Thipps's employment about three months. Her previous employers would speak to her character. It was her duty to make the round of the flat at night, when she had seen Mrs Thipps to bed at ten. Yes, she remembered doing so on Monday evening. She had looked into all the rooms. Did she recollect shutting the bathroom window that night? Well, no, she couldn't swear to it, not in particular, but when Mr Thipps called her into the bathroom in the morning it certainly \textit{was} open. She had not been into the bathroom before Mr Thipps went in. Well, yes, it had happened that she had left that window open before, when anyone had been 'aving a bath in the evening and 'ad left the blind down. Mrs Thipps 'ad 'ad a bath on Monday evening, Mondays was one of her regular bath nights. She was very much afraid she 'adn't shut the window on Monday night, though she wished her 'ead 'ad been cut off afore she'd been so forgetful.

Here the witness burst into tears and was given some water, while the Coroner refreshed himself with a third lozenge.

Recovering, witness stated that she had certainly looked into all the rooms before going to bed. No, it was quite impossible for a body to be 'idden in the flat without her seeing of it. She 'ad been in the kitchen all evening, and there wasn't 'ardly room to keep the best dinner service there, let alone a body. Old Mrs Thipps sat in the drawing-room. Yes, she was sure she'd been into the dining-room. How? Because she put Mr Thipps's milk and sandwiches there ready for him. There had been nothing in there\allowbreak---\allowbreak that she could swear to. Nor yet in her own bedroom, nor in the 'all. Had she searched the bedroom cupboard and the box-room? Well, no, not to say searched; she wasn't use to searchin' people's 'ouses for skelintons every night. So that a man might have concealed himself in the box-room or a wardrobe? She supposed he might.

In reply to a woman juror\allowbreak---\allowbreak well, yes, she was walking out with a young man. Williams was his name, Bill Williams,---well, yes, William Williams, if they insisted. He was a glazier by profession. Well, yes, he 'ad been in the flat sometimes. Well, she supposed you might say he was acquainted with the flat. Had she ever\allowbreak---\allowbreak no, she 'adn't, and if she'd thought such a question was going to be put to a respectable girl she wouldn't 'ave offered to give evidence. The vicar of St Mary's would speak to her character and to Mr Williams's. Last time Mr Williams was at the flat was a fortnight ago.

Well, no, it wasn't exactly the last time she 'ad seen Mr Williams. Well, yes, the last time was Monday\allowbreak---\allowbreak well, yes, Monday night. Well, if she must tell the truth, she must. Yes, the officer had cautioned her, but there wasn't any 'arm in it, and it was better to lose her place than to be 'ung, though it was a cruel shame a girl couldn't 'ave a bit of fun without a nasty corpse comin' in through the window to get 'er into difficulties. After she 'ad put Mrs Thipps to bed, she 'ad slipped out to go to the Plumbers' and Glaziers' Ball at the »Black Faced Ram.« Mr Williams 'ad met 'er and brought 'er back. 'E could testify to where she'd been and that there wasn't no 'arm in it. She'd left before the end of the ball. It might 'ave been two o'clock when she got back. She'd got the keys of the flat from Mrs Thipps's drawer when Mrs Thipps wasn't looking. She 'ad asked leave to go, but couldn't get it, along of Mr Thipps bein' away that night. She was bitterly sorry she 'ad be'aved so, and she was sure she'd been punished for it. She had 'eard nothing suspicious when she came in. She had gone straight to bed without looking round the flat. She wished she were dead.

No, Mr and Mrs Thipps didn't 'ardly ever 'ave any visitors; they kep' themselves very retired. She had found the outside door bolted that morning as usual. She wouldn't never believe any 'arm of Mr Thipps. Thank you, Miss Horrocks. Call Georgiana Thipps, and the Coroner thought we had better light the gas.

The examination of Mrs Thipps provided more entertainment than enlightenment, affording as it did an excellent example of the game called »cross questions and crooked answers.« After fifteen minutes' suffering, both in voice and temper, the Coroner abandoned the struggle, leaving the lady with the last word.

»You needn't try to bully me, young man,« said that octogenarian with spirit, »settin' there spoilin' your stomach with them nasty jujubes.«

At this point a young man arose in court and demanded to give evidence. Having explained that he was William Williams, glazier, he was sworn, and corroborated the evidence of Gladys Horrocks in the matter of her presence at the »Black Faced Ram« on the Monday night. They had returned to the flat rather before two, he thought, but certainly later than 1.30. He was sorry that he had persuaded Miss Horrocks to come out with him when she didn't ought. He had observed nothing of a suspicious nature in Prince of Wales Road at either visit.

Inspector Sugg gave evidence of having been called in at about half-past eight on Monday morning. He had considered the girl's manner to be suspicious and had arrested her. On later information, leading him to suspect that the deceased might have been murdered that night, he had arrested Mr Thipps. He had found no trace of breaking into the flat. There were marks on the bathroom window-sill which pointed to somebody having got in that way. There were no ladder marks or footmarks in the yard; the yard was paved with asphalt. He had examined the roof, but found nothing on the roof. In his opinion the body had been brought into the flat previously and concealed till the evening by someone who had then gone out during the night by the bathroom window, with the connivance of the girl. In that case, why should not the girl have let the person out by the door? Well, it might have been so. Had he found traces of a body or a man or both having been hidden in the flat? He found nothing to show that they might \textit{not} have been so concealed. What was the evidence that led him to suppose that the death had occurred that night?

At this point Inspector Sugg appeared uneasy, and endeavoured to retire upon his professional dignity. On being pressed, however, he admitted that the evidence in question had come to nothing.

\begin{dialogue}
\speak{One of the jurors} Was it the case that any finger-marks had been left by the criminal?
Some marks had been found on the bath, but the criminal had worn gloves.

\speak{The Coroner} Do you draw any conclusion from this fact as to the experience of the criminal?

\speak{Inspector Sugg} Looks as if he was an old hand, sir.

\speak{The Juror} Is that very consistent with the charge against Alfred Thipps, Inspector?

The Inspector was silent.

\speak{The Coroner} In the light of the evidence which you have just heard, do you still press the charge against Alfred Thipps and Gladys Horrocks?

\speak{Inspector Sugg} I consider the whole set-out highly suspicious. Thipps's story isn't corroborated, and as for the girl Horrocks, how do we know this Williams ain't in it as well?

\speak{William Williams} Now, you drop that. I can bring a 'undred witnesses---

\speak{The Coroner} Silence, if you please. I am surprised, Inspector, that you should make this suggestion in that manner. It is highly improper. By the way, can you tell us whether a police raid was actually carried out on the Monday night on any Night Club in the neighbourhood of St Giles's Circus?

\speak{Inspector Sugg} \direct{sulkily} I believe there was something of the sort.

\speak{The Coroner} You will, no doubt, inquire into the matter. I seem to recollect having seen some mention of it in the newspapers. Thank you, Inspector, that will do.
\end{dialogue}

Several witnesses having appeared and testified to the characters of Mr Thipps and Gladys Horrocks, the Coroner stated his intention of proceeding to the medical evidence.

»Sir Julian Freke.«

There was considerable stir in the court as the great specialist walked up to give evidence. He was not only a distinguished man, but a striking figure, with his wide shoulders, upright carriage and leonine head. His manner as he kissed the Book presented to him with the usual deprecatory mumble by the Coroner's officer, was that of a St Paul condescending to humour the timid mumbo-jumbo of superstitious Corinthians.

»So handsome, I always think,« whispered the Duchess to Mr Parker; »just exactly like William Morris, with that bush of hair and beard and those exciting eyes looking out of it\allowbreak---\allowbreak so splendid, these dear men always devoted to something or other\allowbreak---\allowbreak not but what I think socialism is a mistake\allowbreak---\allowbreak of course it works with all those nice people, so good and happy in art linen and the weather always perfect\allowbreak---\allowbreak Morris, I mean, you know\allowbreak---\allowbreak but so difficult in real life. Science is different\allowbreak---\allowbreak I'm sure if I had nerves I should go to Sir Julian just to look at him\allowbreak---\allowbreak eyes like that give one something to think about, and that's what most of these people want, only I never had any\allowbreak---\allowbreak nerves, I mean. Don't you think so?«

»You are Sir Julian Freke,« said the Coroner, »and live at St Luke's House, Prince of Wales Road, Battersea, where you exercise a general direction over the surgical side of St Luke's Hospital?«

Sir Julian assented briefly to this definition of his personality.

»You were the first medical man to see the deceased?«

»I was.«

»And you have since conducted an examination in collaboration with Dr Grimbold of Scotland Yard?«

»I have.«

»You are in agreement as to the cause of death?«

»Generally speaking, yes.«

»Will you communicate your impressions to the Jury?«

»I was engaged in research work in the dissecting room at St Luke's Hospital at about nine o'clock on Monday morning, when I was informed that Inspector Sugg wished to see me. He told me that the dead body of a man had been discovered under mysterious circumstances at 59 Queen Caroline Mansions. He asked me whether it could be supposed to be a joke perpetrated by any of the medical students at the hospital. I was able to assure him, by an examination of the hospital's books, that there was no subject missing from the dissecting room.«

»Who would be in charge of such bodies?«

»William Watts, the dissecting-room attendant.«

»Is William Watts present?« inquired the Coroner of the officer.

William Watts was present, and could be called if the Coroner thought it necessary.

»I suppose no dead body would be delivered to the hospital without your knowledge, Sir Julian?«

»Certainly not.«

»Thank you. Will you proceed with your statement?«

»Inspector Sugg then asked me whether I would send a medical man round to view the body. I said that I would go myself.«

»Why did you do that?«

»I confess to my share of ordinary human curiosity, Mr Coroner.«

Laughter from a medical student at the back of the room.

»On arriving at the flat I found the deceased lying on his back in the bath. I examined him, and came to the conclusion that death had been caused by a blow on the back of the neck, dislocating the fourth and fifth cervical vertebrae, bruising the spinal cord and producing internal haemorrhage and partial paralysis of the brain. I judged the deceased to have been dead at least twelve hours, possibly more. I observed no other sign of violence of any kind upon the body. Deceased was a strong, well-nourished man of about fifty to fifty-five years of age.«

»In your opinion, could the blow have been self-inflicted?«

»Certainly not. It had been made with a heavy, blunt instrument from behind, with great force and considerable judgment. It is quite impossible that it was self-inflicted.«

»Could it have been the result of an accident?«

»That is possible, of course.«

»If, for example, the deceased had been looking out of the window, and the sash had shut violently down upon him?«

»No; in that case there would have been signs of strangulation and a bruise upon the throat as well.«

»But deceased might have been killed through a heavy weight accidentally falling upon him?«

»He might.«

»Was death instantaneous, in your opinion?«

»It is difficult to say. Such a blow might very well cause death instantaneously, or the patient might linger in a partially paralyzed condition for some time. In the present case I should be disposed to think that deceased might have lingered for some hours. I base my decision upon the condition of the brain revealed at the autopsy. I may say, however, that Dr Grimbold and I are not in complete agreement on the point.«

»I understand that a suggestion has been made as to the identification of the deceased. \textit{You} are not in a position to identify him?«

»Certainly not. I never saw him before. The suggestion to which you refer is a preposterous one, and ought never to have been made. I was not aware until this morning that it had been made; had it been made to me earlier, I should have known how to deal with it, and I should like to express my strong disapproval of the unnecessary shock and distress inflicted upon a lady with whom I have the honour to be acquainted.«

\textsc{The Coroner}: It was not my fault, Sir Julian; I had nothing to do with it; I agree with you that it was unfortunate you were not consulted.

The reporters scribbled busily, and the court asked each other what was meant, while the jury tried to look as if they knew already.

»In the matter of the eyeglasses found upon the body, Sir Julian. Do these give any indication to a medical man?«

»They are somewhat unusual lenses; an oculist would be able to speak more definitely, but I will say for myself that I should have expected them to belong to an older man than the deceased.«

»Speaking as a physician, who has had many opportunities of observing the human body, did you gather anything from the appearance of the deceased as to his personal habits?«

»I should say that he was a man in easy circumstances, but who had only recently come into money. His teeth are in a bad state, and his hands shows signs of recent manual labour.«

»An Australian colonist, for instance, who had made money?«

»Something of that sort; of course, I could not say positively.«

»Of course not. Thank you, Sir Julian.«

Dr Grimbold, called, corroborated his distinguished colleague in every particular, except that, in his opinion, death had not occurred for several days after the blow. It was with the greatest hesitancy that he ventured to differ from Sir Julian Freke, and he might be wrong. It was difficult to tell in any case, and when he saw the body, deceased had been dead at least twenty-four hours, in his opinion.

Inspector Sugg, recalled. Would he tell the jury what steps had been taken to identify the deceased?

A description had been sent to every police station and had been inserted in all the newspapers. In view of the suggestion made by Sir Julian Freke, had inquiries been made at all the seaports? They had. And with no results? With no results at all. No one had come forward to identify the body? Plenty of people had come forward; but nobody had succeeded in identifying it. Had any effort been made to follow up the clue afforded by the eyeglasses? Inspector Sugg submitted that, having regard to the interests of justice, he would beg to be excused from answering that question. Might the jury see the eyeglasses? The eyeglasses were handed to the jury.

William Watts, called, confirmed the evidence of Sir Julian Freke with regard to dissecting-room subjects. He explained the system by which they were entered. They usually were supplied by the workhouses and free hospitals. They were under his sole charge. The young gentlemen could not possibly get the keys. Had Sir Julian Freke, or any of the house surgeons, the keys? No, not even Sir Julian Freke. The keys had remained in his possession on Monday night? They had. And, in any case, the inquiry was irrelevant, as there was no body missing, nor ever had been? That was the case.

The Coroner then addressed the jury, reminding them with some asperity that they were not there to gossip about who the deceased could or could not have been, but to give their opinion as to the cause of death. He reminded them that they should consider whether, according to the medical evidence, death could have been accidental or self-inflicted, or whether it was deliberate murder, or homicide. If they considered the evidence on this point insufficient, they could return an open verdict. In any case, their verdict could not prejudice any person; if they brought it in »murder,« all the whole evidence would have to be gone through again before the magistrate. He then dismissed them, with the unspoken adjuration to be quick about it.

Sir Julian Freke, after giving his evidence, had caught the eye of the Duchess, and now came over and greeted her.

»I haven't seen you for an age,« said that lady. »How are you?«

»Hard at work,« said the specialist. »Just got my new book out. This kind of thing wastes time. Have you seen Lady Levy yet?«

»No, poor dear,« said the Duchess. »I only came up this morning, for this. Mrs Thipps is staying with me\allowbreak---\allowbreak one of Peter's eccentricities, you know. Poor Christine! I must run round and see her. This is Mr Parker,« she added, »who is investigating that case.«

»Oh,« said Sir Julian, and paused. »Do you know,« he said in a low voice to Parker, »I am very glad to meet you. Have you seen Lady Levy yet?«

»I saw her this morning.«

»Did she ask you to go on with the inquiry?«

»Yes,« said Parker; »she thinks,« he added, »that Sir Reuben may be detained in the hands of some financial rival or that perhaps some scoundrels are holding him to ransom.«

»And is that \textit{your} opinion?« asked Sir Julian.

»I think it very likely,« said Parker, frankly.

Sir Julian hesitated again.

»I wish you would walk back with me when this is over,« he said.

»I should be delighted,« said Parker.

At this moment the jury returned and took their places, and there was a little rustle and hush. The Coroner addressed the foreman and inquired if they were agreed upon their verdict.

»We are agreed, Mr Coroner, that deceased died of the effects of a blow upon the spine, but how that injury was inflicted we consider that there is not sufficient evidence to show.«

Mr Parker and Sir Julian Freke walked up the road together.

»I had absolutely no idea until I saw Lady Levy this morning,« said the doctor, »that there was any idea of connecting this matter with the disappearance of Sir Reuben. The suggestion was perfectly monstrous, and could only have grown up in the mind of that ridiculous police officer. If I had had any idea what was in his mind I could have disabused him and avoided all this.«

»I did my best to do so,« said Parker, »as soon as I was called in to the Levy case\longdash«

»Who called you in, if I may ask?« inquired Sir Julian.

»Well, the household first of all, and then Sir Reuben's uncle, Mr Levy of Portman Square, wrote to me to go on with the investigation.«

»And now Lady Levy has confirmed those instructions?«

»Certainly,« said Parker in some surprise.

Sir Julian was silent for a little time.

»I'm afraid I was the first person to put the idea into Sugg's head,« said Parker, rather penitently. »When Sir Reuben disappeared, my first step, almost, was to hunt up all the street accidents and suicides and so on that had turned up during the day, and I went down to see this Battersea Park body as a matter of routine. Of course, I saw that the thing was ridiculous as soon as I got there, but Sugg froze on to the idea\allowbreak---\allowbreak and it's true there was a good deal of resemblance between the dead man and the portraits I've seen of Sir Reuben.«

»A strong superficial likeness,« said Sir Julian. »The upper part of the face is a not uncommon type, and as Sir Reuben wore a heavy beard and there was no opportunity of comparing the mouths and chins, I can understand the idea occurring to anybody. But only to be dismissed at once. I am sorry,« he added, »as the whole matter has been painful to Lady Levy. You may know, Mr Parker, that I am an old, though I should not call myself an intimate, friend of the Levys.«

»I understood something of the sort.«

»Yes. When I was a young man I\allowbreak---\allowbreak in short, Mr Parker, I hoped once to marry Lady Levy.« (Mr Parker gave the usual sympathetic groan.) »I have never married, as you know,« pursued Sir Julian. »We have remained good friends. I have always done what I could to spare her pain.«

»Believe me, Sir Julian,« said Parker, »that I sympathize very much with you and with Lady Levy, and that I did all I could to disabuse Inspector Sugg of this notion. Unhappily, the coincidence of Sir Reuben's being seen that evening in the Battersea Park Road\longdash«

»Ah, yes,« said Sir Julian. »Dear me, here we are at home. Perhaps you would come in for a moment, Mr Parker, and have tea or a whisky-and-soda or something.«

Parker promptly accepted this invitation, feeling that there were other things to be said.

The two men stepped into a square, finely furnished hall with a fireplace on the same side as the door, and a staircase opposite. The dining-room door stood open on their right, and as Sir Julian rang the bell a man-servant appeared at the far end of the hall.

»What will you take?« asked the doctor.

»After that dreadfully cold place,« said Parker, »what I really want is gallons of hot tea, if you, as a nerve specialist, can bear the thought of it.«

»Provided you allow of a judicious blend of China in it,« replied Sir Julian in the same tone, »I have no objection to make. Tea in the library at once,« he added to the servant, and led the way upstairs.

»I don't use the downstairs rooms much, except the dining-room,« he explained as he ushered his guest into a small but cheerful library on the first floor. »This room leads out of my bedroom and is more convenient. I only live part of my time here, but it's very handy for my research work at the hospital. That's what I do there, mostly. It's a fatal thing for a theorist, Mr Parker, to let the practical work get behindhand. Dissection is the basis of all good theory and all correct diagnosis. One must keep one's hand and eye in training. This place is far more important to me than Harley Street, and some day I shall abandon my consulting practice altogether and settle down here to cut up my subjects and write my books in peace. So many things in this life are a waste of time, Mr Parker.«

Mr Parker assented to this.

»Very often,« said Sir Julian, »the only time I get for any research work\allowbreak---\allowbreak necessitating as it does the keenest observation and the faculties at their acutest\allowbreak---\allowbreak has to be at night, after a long day's work and by artificial light, which, magnificent as the lighting of the dissecting room here is, is always more trying to the eyes than daylight. Doubtless your own work has to be carried on under even more trying conditions.«

»Yes, sometimes,« said Parker; »but then you see,« he added, »the conditions are, so to speak, part of the work.«

»Quite so, quite so,« said Sir Julian; »you mean that the burglar, for example, does not demonstrate his methods in the light of day, or plant the perfect footmark in the middle of a damp patch of sand for you to analyse.«

»Not as a rule,« said the detective, »but I have no doubt many of your diseases work quite as insidiously as any burglar.«

»They do, they do,« said Sir Julian, laughing, »and it is my pride, as it is yours, to track them down for the good of society. The neuroses, you know, are particularly clever criminals\allowbreak---\allowbreak they break out into as many disguises as\longdash«

»As Leon Kestrel, the Master-Mummer,« suggested Parker, who read railway-stall detective stories on the principle of the 'busman's holiday.

»No doubt,« said Sir Julian, who did not, »and they cover up their tracks wonderfully. But when you can really investigate, Mr Parker, and break up the dead, or for preference the living body with the scalpel, you always find the footmarks\allowbreak---\allowbreak the little trail of ruin or disorder left by madness or disease or drink or any other similar pest. But the difficulty is to trace them back, merely by observing the surface symptoms\allowbreak---\allowbreak the hysteria, crime, religion, fear, shyness, conscience, or whatever it may be; just as you observe a theft or a murder and look for the footsteps of the criminal, so I observe a fit of hysterics or an outburst of piety and hunt for the little mechanical irritation which has produced it.«

»You regard all these things as physical?«

»Undoubtedly. I am not ignorant of the rise of another school of thought, Mr Parker, but its exponents are mostly charlatans or self-deceivers. »\textit{Sie haben sich so weit darin eingeheimnisst}« that, like Sludge the Medium, they are beginning to believe their own nonsense. I should like to have the exploring of some of their brains, Mr Parker; I would show you the little faults and landslips in the cells\allowbreak---\allowbreak the misfiring and short-circuiting of the nerves, which produce these notions and these books. At least,« he added, gazing sombrely at his guest, »at least, if I could not quite show you today, I shall be able to do so tomorrow\allowbreak---\allowbreak or in a year's time\allowbreak---\allowbreak or before I die.«

He sat for some minutes gazing into the fire, while the red light played upon his tawny beard and struck out answering gleams from his compelling eyes.

Parker drank tea in silence, watching him. On the whole, however, he remained but little interested in the causes of nervous phenomena and his mind strayed to Lord Peter, coping with the redoubtable Crimplesham down in Salisbury. Lord Peter had wanted him to come: that meant, either that Crimplesham was proving recalcitrant or that a clue wanted following. But Bunter had said that tomorrow would do, and it was just as well. After all, the Battersea affair was not Parker's case; he had already wasted valuable time attending an inconclusive inquest, and he really ought to get on with his legitimate work. There was still Levy's secretary to see and the little matter of the Peruvian Oil to be looked into. He looked at his watch.

»I am very much afraid\allowbreak---\allowbreak if you will excuse me\longdash« he murmured.

Sir Julian came back with a start to the consideration of actuality.

»Your work calls you?« he said, smiling. »Well, I can understand that. I won't keep you. But I wanted to say something to you in connection with your present inquiry\allowbreak---\allowbreak only I hardly know\allowbreak---\allowbreak I hardly like\longdash«

Parker sat down again, and banished every indication of hurry from his face and attitude.

»I shall be very grateful for any help you can give me,« he said.

»I'm afraid it's more in the nature of hindrance,« said Sir Julian, with a short laugh. »It's a case of destroying a clue for you, and a breach of professional confidence on my side. But since\allowbreak---\allowbreak accidentally---a certain amount has come out, perhaps the whole had better do so.«

Mr Parker made the encouraging noise which, among laymen, supplies the place of the priest's insinuating, »Yes, my son?«

»Sir Reuben Levy's visit on Monday night was to me,« said Sir Julian.

»Yes?« said Mr Parker, without expression.

»He found cause for certain grave suspicions concerning his health,« said Sir Julian, slowly, as though weighing how much he could in honour disclose to a stranger. »He came to me, in preference to his own medical man, as he was particularly anxious that the matter should be kept from his wife. As I told you, he knew me fairly well, and Lady Levy had consulted me about a nervous disorder in the summer.«

»Did he make an appointment with you?« asked Parker.

»I beg your pardon,« said the other, absently.

»Did he make an appointment?«

»An appointment? Oh, no! He turned up suddenly in the evening after dinner when I wasn't expecting him. I took him up here and examined him, and he left me somewhere about ten o'clock, I should think.«

»May I ask what was the result of your examination?«

»Why do you want to know?«

»It might illuminate\allowbreak---\allowbreak well, conjecture as to his subsequent conduct,« said Parker, cautiously. This story seemed to have little coherence with the rest of the business, and he wondered whether coincidence was alone responsible for Sir Reuben's disappearance on the same night that he visited the doctor.

»I see,« said Sir Julian. »Yes. Well, I will tell you in confidence that I saw grave grounds of suspicion, but as yet, no absolute certainty of mischief.«

»Thank you. Sir Reuben left you at ten o'clock?«

»Then or thereabouts. I did not at first mention the matter as it was so very much Sir Reuben's wish to keep his visit to me secret, and there was no question of accident in the street or anything of that kind, since he reached home safely at midnight.«

»Quite so,« said Parker.

»It would have been, and is, a breach of confidence,« said Sir Julian, »and I only tell you now because Sir Reuben was accidentally seen, and because I would rather tell you in private than have you ferretting round here and questioning my servants, Mr Parker. You will excuse my frankness.«

»Certainly,« said Parker. »I hold no brief for the pleasantness of my profession, Sir Julian. I am very much obliged to you for telling me this. I might otherwise have wasted valuable time following up a false trail.«

»I am sure I need not ask you, in your turn, to respect this confidence,« said the doctor. »To publish the matter abroad could only harm Sir Reuben and pain his wife, besides placing me in no favourable light with my patients.«

»I promise to keep the thing to myself,« said Parker, »except of course,« he added hastily, »that I must inform my colleague.«

»You have a colleague in the case?«

»I have.«

»What sort of person is he?«

»He will be perfectly discreet, Sir Julian.«

»Is he a police officer?«

»You need not be afraid of your confidence getting into the records at Scotland Yard.«

»I see that you know how to be discreet, Mr Parker.«

»We also have our professional etiquette, Sir Julian.«

On returning to Great Ormond Street, Mr Parker found a wire awaiting him, which said: \textsc{Do not trouble to come. All well. Returning tomorrow. Wimsey.}