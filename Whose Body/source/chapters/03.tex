%!TeX root=../bodytop.tex
\chapter[Chapter \thechapter]{}
\lettrine[lines=4]{L}{ord} Peter finished a Scarlatti sonata, and sat looking thoughtfully at his own hands. The fingers were long and muscular, with wide, flat joints and square tips. When he was playing, his rather hard grey eyes softened, and his long, indeterminate mouth hardened in compensation. At no other time had he any pretensions to good looks, and at all times he was spoilt by a long, narrow chin, and a long, receding forehead, accentuated by the brushed-back sleekness of his tow-coloured hair. Labour papers, softening down the chin, caricatured him as a typical aristocrat.

»That's a wonderful instrument,« said Parker.

»It ain't so bad,« said Lord Peter, »but Scarlatti wants a harpsichord. Piano's too modern\allowbreak---\allowbreak all thrills and overtones. No good for our job, Parker. Have you come to any conclusion?«

»The man in the bath,« said Parker, methodically, »was \textit{not} a well-off man careful of his personal appearance. He was a labouring man, unemployed, but who had only recently lost his employment. He had been tramping about looking for a job when he met with his end. Somebody killed him and washed him and scented him and shaved him in order to disguise him, and put him into Thipps's bath without leaving a trace. Conclusion: the murderer was a powerful man, since he killed him with a single blow on the neck, a man of cool head and masterly intellect, since he did all that ghastly business without leaving a mark, a man of wealth and refinement, since he had all the apparatus of an elegant toilet handy, and a man of bizarre, and almost perverted imagination, as is shown in the two horrible touches of putting the body in the bath and of adorning it with a pair of pince-nez.«

»He is a poet of crime,« said Wimsey. »By the way, your difficulty about the pince-nez is cleared up. Obviously, the pince-nez never belonged to the body.«

»That only makes a fresh puzzle. One can't suppose the murderer left them in that obliging manner as a clue to his own identity.«

»We can hardly suppose that; I'm afraid this man possessed what most criminals lack\allowbreak---\allowbreak a sense of humour.«

»Rather macabre humour.«

»True. But a man who can afford to be humorous at all in such circumstances is a terrible fellow. I wonder what he did with the body between the murder and depositing it chez Thipps. Then there are more questions. How did he get it there? And why? Was it brought in at the door, as Sugg of our heart suggests? or through the window, as we think, on the not very adequate testimony of a smudge on the window-sill? Had the murderer accomplices? Is little Thipps really in it, or the girl? It don't do to put the notion out of court merely because Sugg inclines to it. Even idiots occasionally speak the truth accidentally. If not, why was Thipps selected for such an abominable practical joke? Has anybody got a grudge against Thipps? Who are the people in the other flats? We must find out that. Does Thipps play the piano at midnight over their heads or damage the reputation of the staircase by bringing home dubiously respectable ladies? Are there unsuccessful architects thirsting for his blood? Damn it all, Parker, there must be a motive somewhere. Can't have a crime without a motive, you know.«

»A madman\longdash« suggested Parker, doubtfully.

»With a deuced lot of method in his madness. He hasn't made a mistake\allowbreak---\allowbreak not one, unless leaving hairs in the corpse's mouth can be called a mistake. Well, anyhow, it's not Levy\allowbreak---\allowbreak you're right there. I say, old thing, neither your man nor mine has left much clue to go upon, has he? And there don't seem to be any motives knockin' about, either. And we seem to be two suits of clothes short in last night's work. Sir Reuben makes tracks without so much as a fig-leaf, and a mysterious individual turns up with a pince-nez, which is quite useless for purposes of decency. Dash it all! If only I had some good excuse for takin' up this body case officially\longdash«

The telephone bell rang. The silent Bunter, whom the other two had almost forgotten, padded across to it.

»It's an elderly lady, my lord,« he said. »I think she's deaf\allowbreak---\allowbreak I can't make her hear anything, but she's asking for your lordship.«

Lord Peter seized the receiver, and yelled into it a »Hullo!« that might have cracked the vulcanite. He listened for some minutes with an incredulous smile, which gradually broadened into a grin of delight. At length he screamed: »All right! all right!« several times, and rang off.

»By Jove!« he announced, beaming, »sportin' old bird! It's old Mrs Thipps. Deaf as a post. Never used the 'phone before. But determined. Perfect Napoleon. The incomparable Sugg has made a discovery and arrested little Thipps. Old lady abandoned in the flat. Thipps's last shriek to her: »Tell Lord Peter Wimsey.« Old girl undaunted. Wrestles with telephone book. Wakes up the people at the exchange. Won't take no for an answer (not bein' able to hear it), gets through, says: »Will I do what I can?« Says she would feel safe in the hands of a real gentleman. Oh, Parker, Parker! I could kiss her, I reely could, as Thipps says. I'll write to her instead\allowbreak---\allowbreak no, hang it, Parker, we'll go round. Bunter, get your infernal machine and the magnesium. I say, we'll all go into partnership\allowbreak---\allowbreak pool the two cases and work 'em out together. You shall see my body tonight, Parker, and I'll look for your wandering Jew tomorrow. I feel so happy, I shall explode. O Sugg, Sugg, how art thou suggified! Bunter, my shoes. I say, Parker, I suppose yours are rubber-soled. Not? Tut, tut, you mustn't go out like that. We'll lend you a pair. Gloves? Here. My stick, my torch, the lampblack, the forceps, knife, pill-boxes\allowbreak---\allowbreak all complete?«

»Certainly, my lord.«

»Oh, Bunter, don't look so offended. I mean no harm. I believe in you, I trust you\allowbreak---\allowbreak what money have I got? That'll do. I knew a man once, Parker, who let a world-famous poisoner slip through his fingers because the machine on the Underground took nothing but pennies. There was a queue at the booking office and the man at the barrier stopped him, and while they were arguing about accepting a five-pound-note (which was all he had) for a twopenny ride to Baker Street, the criminal had sprung into a Circle train, and was next heard of in Constantinople, disguised as an elderly Church of England clergyman touring with his niece. Are we all ready? Go!«

They stepped out, Bunter carefully switching off the lights behind them.

As they emerged into the gloom and gleam of Piccadilly, Wimsey stopped short with a little exclamation.

»Wait a second,« he said. »I've thought of something. If Sugg's there he'll make trouble. I must short-circuit him.«

He ran back, and the other two men employed the few minutes of his absence in capturing a taxi.

Inspector Sugg and a subordinate Cerberus were on guard at 59, Queen Caroline Mansions, and showed no disposition to admit unofficial inquirers. Parker, indeed, they could not easily turn away, but Lord Peter found himself confronted with a surly manner and what Lord Beaconsfield described as a masterly inactivity. It was in vain that Lord Peter pleaded that he had been retained by Mrs Thipps on behalf of her son.

»Retained!« said Inspector Sugg, with a snort. »\textit{She'll} be retained if she doesn't look out. Shouldn't wonder if she wasn't in it herself, only she's so deaf, she's no good for anything at all.«

»Look here, Inspector,« said Lord Peter, »what's the use of bein' so bally obstructive? You'd much better let me in\allowbreak---\allowbreak you know I'll get there in the end. Dash it all, it's not as if I was takin' the bread out of your children's mouths. Nobody paid me for finding Lord Attenbury's emeralds for you.«

»It's my duty to keep out the public,« said Inspector Sugg, morosely, »and it's going to stay out.«

»I never said anything about your keeping out of the public,« said Lord Peter, easily, sitting down on the staircase to thrash the matter out comfortably, »though I've no doubt pussyfoot's a good thing, on principle, if not exaggerated. The golden mean, Sugg, as Aristotle says, keeps you from bein' a golden ass. Ever been a golden ass, Sugg? I have. It would take a whole rose-garden to cure me, Sugg---

\begin{verse}
You are my garden of beautiful roses,\\
My own rose, my one rose, that's you!\\
\end{verse}«

»I'm not going to stay any longer talking to you,« said the harassed Sugg; »it's bad enough--- Hullo, drat that telephone. Here, Cawthorn, go and see what it is, if that old catamaran will let you into the room. Shutting herself up there and screaming,« said the Inspector, »it's enough to make a man give up crime and take to hedging and ditching.«

The constable came back:

»It's from the Yard, sir,« he said, coughing apologetically; »the Chief says every facility is to be given to Lord Peter Wimsey, sir. Um!« He stood apart noncommittally, glazing his eyes.

»Five aces,« said Lord Peter, cheerfully. »The Chief's a dear friend of my mother's. No go, Sugg, it's no good buckin'; you've got a full house. I'm goin' to make it a bit fuller.«

He walked in with his followers.

The body had been removed a few hours previously, and when the bathroom and the whole flat had been explored by the naked eye and the camera of the competent Bunter, it became evident that the real problem of the household was old Mrs Thipps. Her son and servant had both been removed, and it appeared that they had no friends in town, beyond a few business acquaintances of Thipps's, whose very addresses the old lady did not know. The other flats in the building were occupied respectively by a family of seven, at present departed to winter abroad, an elderly Indian colonel of ferocious manners, who lived alone with an Indian man-servant, and a highly respectable family on the third floor, whom the disturbance over their heads had outraged to the last degree. The husband, indeed, when appealed to by Lord Peter, showed a little human weakness, but Mrs Appledore, appearing suddenly in a warm dressing-gown, extricated him from the difficulties into which he was carelessly wandering.

»I am sorry,« she said, »I'm afraid we can't interfere in any way. This is a very unpleasant business, Mr.--- I'm afraid I didn't catch your name, and we have always found it better not to be mixed up with the police. Of course, \textit{if} the Thippses are innocent, and I am sure I hope they are, it is very unfortunate for them, but I must say that the circumstances seem to me most suspicious, and to Theophilus too, and I should not like to have it said that we had assisted murderers. We might even be supposed to be accessories. Of course you are young, Mr\longdash«

»This is Lord Peter Wimsey, my dear,« said Theophilus mildly.

She was unimpressed.

»Ah, yes,« she said, »I believe you are distantly related to my late cousin, the Bishop of Carisbrooke. Poor man! He was always being taken in by impostors; he died without ever learning any better. I imagine you take after him, Lord Peter.«

»I doubt it,« said Lord Peter. »So far as I know he is only a connection, though it's a wise child that knows its own father. I congratulate you, dear lady, on takin' after the other side of the family. You'll forgive my buttin' in upon you like this in the middle of the night, though, as you say, it's all in the family, and I'm sure I'm very much obliged to you, and for permittin' me to admire that awfully fetchin' thing you've got on. Now, don't you worry, Mr Appledore. I'm thinkin' the best thing I can do is to trundle the old lady down to my mother and take her out of your way, otherwise you might be findin' your Christian feelin's gettin' the better of you some fine day, and there's nothin' like Christian feelin's for upsettin' a man's domestic comfort. Good-night, sir\allowbreak---\allowbreak good-night, dear lady\allowbreak---\allowbreak it's simply rippin' of you to let me drop in like this.«

»Well!« said Mrs Appledore, as the door closed behind him.

And---

»\begin{verse}
I thank the goodness and the grace\\
That on my birth have smiled,\\
\end{verse}«

said Lord Peter, »and taught me to be bestially impertinent when I choose. Cat!«

Two \textsc{a.m.} saw Lord Peter Wimsey arrive in a friend's car at the Dower House, Denver Castle, in company with a deaf and aged lady and an antique portmanteau.

»It's very nice to see you, dear,« said the Dowager Duchess, placidly. She was a small, plump woman, with perfectly white hair and exquisite hands. In feature she was as unlike her second son as she was like him in character; her black eyes twinkled cheerfully, and her manners and movements were marked with a neat and rapid decision. She wore a charming wrap from Liberty's, and sat watching Lord Peter eat cold beef and cheese as though his arrival in such incongruous circumstances and company were the most ordinary event possible, which with him, indeed, it was.

»Have you got the old lady to bed?« asked Lord Peter.

»Oh, yes, dear. Such a striking old person, isn't she? And very courageous. She tells me she has never been in a motor-car before. But she thinks you a very nice lad, dear\allowbreak---\allowbreak that careful of her, you remind her of her own son. Poor little Mr Thipps\allowbreak---\allowbreak whatever made your friend the inspector think he could have murdered anybody?«

»My friend the inspector\allowbreak---\allowbreak no, no more, thank you, Mother\allowbreak---\allowbreak is determined to prove that the intrusive person in Thipps's bath is Sir Reuben Levy, who disappeared mysteriously from his house last night. His line of reasoning is: We've lost a middle-aged gentleman without any clothes on in Park Lane; we've found a middle-aged gentleman without any clothes on in Battersea. Therefore they're one and the same person, \textsc{q.e.d.}, and put little Thipps in quod.«

»You're very elliptical, dear,« said the Duchess, mildly. »Why should Mr Thipps be arrested even if they are the same?«

»Sugg must arrest somebody,« said Lord Peter, »but there is one odd little bit of evidence come out which goes a long way to support Sugg's theory, only that I know it to be no go by the evidence of my own eyes. Last night at about 9.15 a young woman was strollin' up the Battersea Park Road for purposes best known to herself, when she saw a gentleman in a fur coat and top-hat saunterin' along under an umbrella, lookin' at the names of all the streets. He looked a bit out of place, so, not bein' a shy girl, you see, she walked up to him, and said: »Good-evening.« »Can you tell me, please,« says the mysterious stranger, »whether this street leads into Prince of Wales Road?« She said it did, and further asked him in a jocular manner what he was doing with himself and all the rest of it, only she wasn't altogether so explicit about that part of the conversation, because she was unburdenin' her heart to Sugg, d'you see, and he's paid by a grateful country to have very pure, high-minded ideals, what? Anyway, the old boy said he couldn't attend to her just then as he had an appointment. »I've got to go and see a man, my dear,« was how she said he put it, and he walked on up Alexandra Avenue towards Prince of Wales Road. She was starin' after him, still rather surprised, when she was joined by a friend of hers, who said: »It's no good wasting your time with him\allowbreak---\allowbreak that's Levy\allowbreak---\allowbreak I knew him when I lived in the West End, and the girls used to call him Peagreen Incorruptible«---friend's name suppressed, owing to implications of story, but girl vouches for what was said. She thought no more about it till the milkman brought news this morning of the excitement at Queen Caroline Mansions; then she went round, though not likin' the police as a rule, and asked the man there whether the dead gentleman had a beard and glasses. Told he had glasses but no beard, she incautiously said: »Oh, then, it isn't him,« and the man said: »Isn't who?« and collared her. That's her story. Sugg's delighted, of course, and quodded Thipps on the strength of it.«

»Dear me,« said the Duchess, »I hope the poor girl won't get into trouble.«

»Shouldn't think so,« said Lord Peter. »Thipps is the one that's going to get it in the neck. Besides, he's done a silly thing. I got that out of Sugg, too, though he was sittin' tight on the information. Seems Thipps got into a confusion about the train he took back from Manchester. Said first he got home at 10.30. Then they pumped Gladys Horrocks, who let out he wasn't back till after 11.45. Then Thipps, bein' asked to explain the discrepancy, stammers and bungles and says, first, that he missed the train. Then Sugg makes inquiries at St Pancras and discovers that he left a bag in the cloakroom there at ten. Thipps, again asked to explain, stammers worse an' says he walked about for a few hours\allowbreak---\allowbreak met a friend\allowbreak---\allowbreak can't say who\allowbreak---\allowbreak didn't meet a friend\allowbreak---\allowbreak can't say what he did with his time\allowbreak---\allowbreak can't explain why he didn't go back for his bag\allowbreak---\allowbreak can't say what time he \textit{did} get in\allowbreak---\allowbreak can't explain how he got a bruise on his forehead. In fact, can't explain himself at all. Gladys Horrocks interrogated again. Says, this time, Thipps came in at 10.30. Then admits she didn't hear him come in. Can't say why she didn't hear him come in. Can't say why she said first of all that she \textit{did} hear him. Bursts into tears. Contradicts herself. Everybody's suspicion roused. Quod 'em both.«

»As you put it, dear,« said the Duchess, »it all sounds very confusing, and not quite respectable. Poor little Mr Thipps would be terribly upset by anything that wasn't respectable.«

»I wonder what he did with himself,« said Lord Peter thoughtfully. »I really don't think he was committing a murder. Besides, I believe the fellow has been dead a day or two, though it don't do to build too much on doctors' evidence. It's an entertainin' little problem.«

»Very curious, dear. But so sad about poor Sir Reuben. I must write a few lines to Lady Levy; I used to know her quite well, you know, dear, down in Hampshire, when she was a girl. Christine Ford, she was then, and I remember so well the dreadful trouble there was about her marrying a Jew. That was before he made his money, of course, in that oil business out in America. The family wanted her to marry Julian Freke, who did so well afterwards and was connected with the family, but she fell in love with this Mr Levy and eloped with him. He was very handsome, then, you know, dear, in a foreign-looking way, but he hadn't any means, and the Fords didn't like his religion. Of course we're all Jews nowadays, and they wouldn't have minded so much if he'd pretended to be something else, like that Mr Simons we met at Mrs Porchester's, who always tells everybody that he got his nose in Italy at the Renaissance, and claims to be descended somehow or other from La Bella Simonetta\allowbreak---\allowbreak so foolish, you know, dear\allowbreak---\allowbreak as if anybody believed it; and I'm sure some Jews are very good people, and personally I'd much rather they believed something, though of course it must be very inconvenient, what with not working on Saturdays and circumcising the poor little babies and everything depending on the new moon and that funny kind of meat they have with such a slang-sounding name, and never being able to have bacon for breakfast. Still, there it was, and it was much better for the girl to marry him if she was really fond of him, though I believe young Freke was really devoted to her, and they're still great friends. Not that there was ever a real engagement, only a sort of understanding with her father, but he's never married, you know, and lives all by himself in that big house next to the hospital, though he's very rich and distinguished now, and I know ever so many people have tried to get hold of him\allowbreak---\allowbreak there was Lady Mainwaring wanted him for that eldest girl of hers, though I remember saying at the time it was no use expecting a surgeon to be taken in by a figure that was all padding\allowbreak---\allowbreak they have so many opportunities of judging, you know, dear.«

»Lady Levy seems to have had the knack of makin' people devoted to her,« said Peter. »Look at the pea-green incorruptible Levy.«

»That's quite true, dear; she was a most delightful girl, and they say her daughter is just like her. I rather lost sight of them when she married, and you know your father didn't care much about business people, but I know everybody always said they were a model couple. In fact it was a proverb that Sir Reuben was as well loved at home as he was hated abroad. I don't mean in foreign countries, you know, dear\allowbreak---\allowbreak just the proverbial way of putting things\allowbreak---\allowbreak like »a saint abroad and a devil at home«---only the other way on, reminding one of the \textit{Pilgrim's Progress}.«

»Yes,« said Peter, »I daresay the old man made one or two enemies.«

»Dozens, dear\allowbreak---\allowbreak such a dreadful place, the City, isn't it? Everybody Ishmaels together\allowbreak---\allowbreak though I don't suppose Sir Reuben would like to be called that, would he? Doesn't it mean illegitimate, or not a proper Jew, anyway? I always did get confused with those Old Testament characters.«

Lord Peter laughed and yawned.

»I think I'll turn in for an hour or two,« he said. »I must be back in town at eight\allowbreak---\allowbreak Parker's coming to breakfast.«

The Duchess looked at the clock, which marked five minutes to three.

»I'll send up your breakfast at half-past six, dear,« she said. »I hope you'll find everything all right. I told them just to slip a hot-water bottle in; those linen sheets are so chilly; you can put it out if it's in your way.«
