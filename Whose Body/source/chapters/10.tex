%!TeX root=../bodytop.tex
\chapter[Chapter \thechapter]{}
\lettrine[lines=4]{M}{r} Parker, a faithful though doubting Thomas, had duly secured his medical student: a large young man like an overgrown puppy, with innocent eyes and a freckled face. He sat on the Chesterfield before Lord Peter's library fire, bewildered in equal measure by his errand, his surroundings and the drink which he was absorbing. His palate, though untutored, was naturally a good one, and he realized that even to call this liquid a drink—the term ordinarily used by him to designate cheap whisky, post-war beer or a dubious glass of claret in a Soho restaurant—was a sacrilege; this was something outside normal experience: a genie in a bottle.

The man called Parker, whom he had happened to run across the evening before in the public-house at the corner of Prince of Wales Road, seemed to be a good sort. He had insisted on bringing him round to see this friend of his, who lived splendidly in Piccadilly. Parker was quite understandable; he put him down as a government servant, or perhaps something in the City. The friend was embarrassing; he was a lord, to begin with, and his clothes were a kind of rebuke to the world at large. He talked the most fatuous nonsense, certainly, but in a disconcerting way. He didn't dig into a joke and get all the fun out of it; he made it in passing, so to speak, and skipped away to something else before your retort was ready. He had a truly terrible man-servant—the sort you read about in books—who froze the marrow in your bones with silent criticism. Parker appeared to bear up under the strain, and this made you think more highly of Parker; he must be more habituated to the surroundings of the great than you would think to look at him. You wondered what the carpet had cost on which Parker was carelessly spilling cigar ash; your father was an upholsterer—Mr Piggott, of Piggott \& Piggott, Liverpool—and you knew enough about carpets to know that you couldn't even guess at the price of this one. When you moved your head on the bulging silk cushion in the corner of the sofa, it made you wish you shaved more often and more carefully. The sofa was a monster—but even so, it hardly seemed big enough to contain you. This Lord Peter was not very tall—in fact, he was rather a small man, but he didn't look undersized. He looked right; he made you feel that to be six-foot-three was rather vulgarly assertive; you felt like Mother's new drawing-room curtains—all over great big blobs. But everybody was very decent to you, and nobody said anything you couldn't understand, or sneered at you. There were some frightfully deep-looking books on the shelves all round, and you had looked into a great folio Dante which was lying on the table, but your hosts were talking quite ordinarily and rationally about the sort of books you read yourself—clinking good love stories and detective stories. You had read a lot of those, and could give an opinion, and they listened to what you had to say, though Lord Peter had a funny way of talking about books, too, as if the author had confided in him beforehand, and told him how the story was put together, and which bit was written first. It reminded you of the way old Freke took a body to pieces.

»Thing I object to in detective stories,« said Mr Piggott, »is the way fellows remember every bloomin' thing that's happened to `em within the last six months. They're always ready with their time of day and was it rainin' or not, and what were they doin' on such an' such a day. Reel it all off like a page of poetry. But one ain't like that in real life, d'you think so, Lord Peter?« Lord Peter smiled, and young Piggott, instantly embarrassed, appealed to his earlier acquaintance. »You know what I mean, Parker. Come now. One day's so like another, I'm sure I couldn't remember—well, I might remember yesterday, p'r'aps, but I couldn't be certain about what I was doin' last week if I was to be shot for it.«

»No,« said Parker, \enquote{and evidence given in police statements sounds just as impossible. But they don't really get it like that, you know. I mean, a man doesn't just say, \enquote{Last Friday I went out at 10 \textsc{a.m.} to buy a mutton chop. As I was turning into Mortimer Street I noticed a girl of about twenty-two with black hair and brown eyes, wearing a green jumper, check skirt, Panama hat and black shoes, riding a Royal Sunbeam Cycle at about ten miles an hour turning the corner by the Church of St Simon and St Jude on the wrong side of the road riding towards the market place!} It amounts to that, of course, but it's really wormed out of him by a series of questions.}

»And in short stories,« said Lord Peter, »it has to be put in statement form, because the real conversation would be so long and twaddly and tedious, and nobody would have the patience to read it. Writers have to consider their readers, if any, y'see.«

»Yes,« said Mr Piggott, »but I bet you most people would find it jolly difficult to remember, even if you asked `em things. I should—of course, I know I'm a bit of a fool, but then, most people are, ain't they? You know what I mean. Witnesses ain't detectives, they're just average idiots like you and me.«

»Quite so,« said Lord Peter, smiling as the force of the last phrase sank into its unhappy perpetrator; »you mean, if I were to ask you in a general way what you were doin'---say, a week ago today, you wouldn't be able to tell me a thing about it offhand?«

»No—I'm sure I shouldn't.« He considered. »No. I was in at the Hospital as usual, I suppose, and, being Tuesday, there'd be a lecture on something or the other—dashed if I know what—and in the evening I went out with Tommy Pringle—no, that must have been Monday—or was it Wednesday? I tell you, I couldn't swear to anything.«

»You do yourself an injustice,« said Lord Peter gravely. »I'm sure, for instance, you recollect what work you were doing in the dissecting-room on that day, for example.«

»Lord, no! not for certain. I mean, I daresay it might come back to me if I thought for a long time, but I wouldn't swear to it in a court of law.«

»I'll bet you half-a-crown to sixpence,« said Lord Peter, »that you'll remember within five minutes.«

»I'm sure I can't.«

»We'll see. Do you keep a notebook of the work you do when you dissect? Drawings or anything?«

»Oh, yes.«

»Think of that. What's the last thing you did in it?«

»That's easy, because I only did it this morning. It was leg muscles.«

»Yes. Who was the subject?«

»An old woman of sorts; died of pneumonia.«

»Yes. Turn back the pages of your drawing book in your mind. What came before that?«

»Oh, some animals—still legs; I'm doing motor muscles at present. Yes. That was old Cunningham's demonstration on comparative anatomy. I did rather a good thing of a hare's legs and a frog's, and rudimentary legs on a snake.«

»Yes. Which day does Mr Cunningham lecture?«

»Friday.«

»Friday; yes. Turn back again. What comes before that?«

Mr Piggott shook his head.

»Do your drawings of legs begin on the right-hand page or the left-hand page? Can you see the first drawing?«

»Yes—yes—I can see the date written at the top. It's a section of a frog's hind leg, on the right-hand page.«

»Yes. Think of the open book in your mind's eye. What is opposite to it?«

This demanded some mental concentration.

»Something round—coloured—oh, yes—it's a hand.«

»Yes. You went on from the muscles of the hand and arm to leg- and foot-muscles?«

»Yes; that's right. I've got a set of drawings of arms.«

»Yes. Did you make those on the Thursday?«

»No; I'm never in the dissecting-room on Thursday.«

»On Wednesday, perhaps?«

»Yes; I must have made them on Wednesday. Yes; I did. I went in there after we'd seen those tetanus patients in the morning. I did them on Wednesday afternoon. I know I went back because I wanted to finish `em. I worked rather hard—for me. That's why I remember.«

»Yes; you went back to finish them. When had you begun them, then?«

»Why, the day before.«

»The day before. That was Tuesday, wasn't it?«

»I've lost count—yes, the day before Wednesday—yes, Tuesday.«

»Yes. Were they a man's arms or a woman's arms?«

“Oh, a man's arms.

»Yes; last Tuesday, a week ago today, you were dissecting a man's arms in the dissecting-room. Sixpence, please.«

»By Jove!«

»Wait a moment. You know a lot more about it than that. You've no idea how much you know. You know what kind of man he was.«

»Oh, I never saw him complete, you know. I got there a bit late that day, I remember. I'd asked for an arm specially, because I was rather weak in arms, and Watts—that's the attendant—had promised to save me one.«

»Yes. You have arrived late and found your arm waiting for you. You are dissecting it—taking your scissors and slitting up the skin and pinning it back. Was it very young, fair skin?«

»Oh, no—no. Ordinary skin, I think—with dark hairs on it—yes, that was it.«

»Yes. A lean, stringy arm, perhaps, with no extra fat anywhere?«

»Oh, no—I was rather annoyed about that. I wanted a good, muscular arm, but it was rather poorly developed and the fat got in my way.«

»Yes; a sedentary man who didn't do much manual work.«

»That's right.«

»Yes. You dissected the hand, for instance, and made a drawing of it. You would have noticed any hard calluses.«

»Oh, there was nothing of that sort.«

»No. But should you say it was a young man's arm? Firm young flesh and limber joints?«

»No—no.«

»No. Old and stringy, perhaps.«

»No. Middle-aged—with rheumatism. I mean, there was a chalky deposit in the joints, and the fingers were a bit swollen.«

»Yes. A man about fifty.«

»About that.«

»Yes. There were other students at work on the same body.«

»Oh, yes.«

»Yes. And they made all the usual sort of jokes about it.«

»I expect so—oh, yes!«

»You can remember some of them. Who is your local funny man, so to speak?«

»Tommy Pringle.«

»What was Tommy Pringle's doing?«

»Can't remember.«

»Whereabouts was Tommy Pringle working?«

»Over by the instrument cupboard—by sink C.«

»Yes. Get a picture of Tommy Pringle in your mind's eye.«

Piggott began to laugh.

»I remember now. Tommy Pringle said the old Sheeny\longdash«

»Why did he call him a Sheeny?«

»I don't know. But I know he did.«

»Perhaps he looked like it. Did you see his head?«

»No.«

»Who had the head?«

»I don't know—oh, yes, I do, though. Old Freke bagged the head himself, and little Bouncible Binns was very cross about it, because he'd been promised a head to do with old Scrooger.«

»I see. What was Sir Julian doing with the head?«

»He called us up and gave us a jaw on spinal haemorrhage and nervous lesions.«

»Yes. Well, go back to Tommy Pringle.«

Tommy Pringle's joke was repeated, not without some embarrassment.

»Quite so. Was that all?«

»No. The chap who was working with Tommy said that sort of thing came from over-feeding.«

»I deduce that Tommy Pringle's partner was interested in the alimentary canal.«

»Yes; and Tommy said, if he'd thought they'd feed you like that he'd go to the workhouse himself.«

»Then the man was a pauper from the workhouse?«

»Well, he must have been, I suppose.«

»Are workhouse paupers usually fat and well-fed?«

»Well, no—come to think of it, not as a rule.«

»In fact, it struck Tommy Pringle and his friend that this was something a little out of the way in a workhouse subject?«

»Yes.«

»And if the alimentary canal was so entertaining to these gentlemen, I imagine the subject had come by his death shortly after a full meal.«

»Yes—oh, yes—he'd have had to, wouldn't he?«

»Well, I don't know,« said Lord Peter. »That's in your department, you know. That would be your inference, from what they said.«

»Oh, yes. Undoubtedly.«

»Yes; you wouldn't, for example, expect them to make that observation if the patient had been ill for a long time and fed on slops.«

»Of course not.«

»Well, you see, you really know a lot about it. On Tuesday week you were dissecting the arm muscles of a rheumatic middle-aged Jew, of sedentary habits, who had died shortly after eating a heavy meal, of some injury producing spinal haemorrhage and nervous lesions, and so forth, and who was presumed to come from the workhouse?«

»Yes.«

»And you could swear to those facts, if need were?«

»Well, if you put it in that way, I suppose I could.«

»Of course you could.«

Mr Piggott sat for some moments in contemplation.

»I say,« he said at last, »I did know all that, didn't I?«

»Oh, yes—you knew it all right—like Socrates's slave.«

»Who's he?«

»A person in a book I used to read as a boy.«

“Oh—does he come in ‘The Last Days of Pompeii'?”

»No—another book—I daresay you escaped it. It's rather dull.«

»I never read much except Henty and Fenimore Cooper at school\textellipsis . But—have I got rather an extra good memory, then?«

»You have a better memory than you credit yourself with.«

»Then why can't I remember all the medical stuff? It all goes out of my head like a sieve.«

»Well, why can't you?« said Lord Peter, standing on the hearthrug and smiling down at his guest.

»Well,« said the young man, »the chaps who examine one don't ask the same sort of questions you do.«

»No?«

»No—they leave you to remember all by yourself. And it's beastly hard. Nothing to catch hold of, don't you know? But, I say—how did you know about Tommy Pringle being the funny man and\longdash«

»I didn't, till you told me.«

»No; I know. But how did you know he'd be there if you did ask? I mean to say—I say,« said Mr Piggott, who was becoming mellowed by influences themselves not unconnected with the alimentary canal---»I say, are you rather clever, or am I rather stupid?«

»No, no,« said Lord Peter, »it's me. I'm always askin' such stupid questions, everybody thinks I must mean somethin' by `em.«

This was too involved for Mr Piggott.

»Never mind,« said Parker, soothingly, “he's always like that. You mustn't take any notice. He can't help it. It's premature senile decay, often observed in the families of hereditary legislators. Go away, Wimsey, and play us the ‘Beggar's Opera,' or something.”

»That's good enough, isn't it?« said Lord Peter, when the happy Mr Piggott had been despatched home after a really delightful evening.

»I'm afraid so,« said Parker. »But it seems almost incredible.«

»There's nothing incredible in human nature,« said Lord Peter; »at least, in educated human nature. Have you got that exhumation order?«

»I shall have it tomorrow. I thought of fixing up with the workhouse people for tomorrow afternoon. I shall have to go and see them first.«

»Right you are; I'll let my mother know.«

»I begin to feel like you, Wimsey, I don't like this job.«

»I like it a deal better than I did.«

»You are really certain we're not making a mistake?«

Lord Peter had strolled across to the window. The curtain was not perfectly drawn, and he stood gazing out through the gap into lighted Piccadilly. At this he turned round:

»If we are,« he said, »we shall know tomorrow, and no harm will have been done. But I rather think you will receive a certain amount of confirmation on your way home. Look here, Parker, d'you know, if I were you I'd spend the night here. There's a spare bedroom; I can easily put you up.«

Parker stared at him.

»Do you mean—I'm likely to be attacked?«

»I think it very likely indeed.«

»Is there anybody in the street?«

»Not now; there was half-an-hour ago.«

»When Piggott left?«

»Yes.«

»I say—I hope the boy is in no danger.«

»That's what I went down to see. I don't think so. Fact is, I don't suppose anybody would imagine we'd exactly made a confidant of Piggott. But I think you and I are in danger. You'll stay?«

»I'm damned if I will, Wimsey. Why should I run away?«

»Bosh!« said Peter. »You'd run away all right if you believed me, and why not? You don't believe me. In fact, you're still not certain I'm on the right tack. Go in peace, but don't say I didn't warn you.«

»I won't; I'll dictate a message with my dying breath to say I was convinced.«

»Well, don't walk—take a taxi.«

»Very well, I'll do that.«

»And don't let anybody else get into it.«

»No.«

It was a raw, unpleasant night. A taxi deposited a load of people returning from the theatre at the block of flats next door, and Parker secured it for himself. He was just giving the address to the driver, when a man came hastily running up from a side street. He was in evening dress and an overcoat. He rushed up, signalling frantically.

»Sir—sir!---dear me! why, it's Mr Parker! How fortunate! If you would be so kind—summoned from the club—a sick friend—can't find a taxi—everybody going home from the theatre—if I might share your cab—you are returning to Bloomsbury? I want Russell Square—if I might presume—a matter of life and death.«

He spoke in hurried gasps, as though he had been running violently and far. Parker promptly stepped out of the taxi.

»Delighted to be of service to you, Sir Julian,« he said; »take my taxi. I am going down to Craven Street myself, but I'm in no hurry. Pray make use of the cab.«

»It's extremely kind of you,« said the surgeon. »I am ashamed\longdash«

»That's all right,« said Parker, cheerily. »I can wait.« He assisted Freke into the taxi. »What number? 24 Russell Square, driver, and look sharp.«

The taxi drove off. Parker remounted the stairs and rang Lord Peter's bell.

»Thanks, old man,« he said. »I'll stop the night, after all.«

»Come in,« said Wimsey.

»Did you see that?« asked Parker.

»I saw something. What happened exactly?«

Parker told his story. »Frankly,« he said, »I've been thinking you a bit mad, but now I'm not quite so sure of it.«

Peter laughed.

»Blessed are they that have not seen and yet have believed. Bunter, Mr Parker will stay the night.«

»Look here, Wimsey, let's have another look at this business. Where's that letter?«

Lord Peter produced Bunter's essay in dialogue. Parker studied it for a short time in silence.

»You know, Wimsey, I'm as full of objections to this idea as an egg is of meat.«

»So'm I, old son. That's why I want to dig up our Chelsea pauper. But trot out your objections.«

»Well\longdash«

»Well, look here, I don't pretend to be able to fill in all the blanks myself. But here we have two mysterious occurrences in one night, and a complete chain connecting the one with another through one particular person. It's beastly, but it's not unthinkable.«

»Yes, I know all that. But there are one or two quite definite stumbling-blocks.«

“Yes, I know. But, see here. On the one hand, Levy disappeared after being last seen looking for Prince of Wales Road at nine o'clock. At eight next morning a dead man, not unlike him in general outline, is discovered in a bath in Queen Caroline Mansions. Levy, by Freke's own admission, was going to see Freke. By information received from Chelsea workhouse a dead man, answering to the description of the Battersea corpse in its natural state, was delivered that same day to Freke. We have Levy with a past, and no future, as it were; an unknown vagrant with a future (in the cemetery) and no past, and Freke stands between their future and their past.”

»That looks all right\longdash«

»Yes. Now, further: Freke has a motive for getting rid of Levy—an old jealousy.«

»Very old—and not much of a motive.«

»People have been known to do that sort of thing.\footnote{Lord Peter was not without authority for his opinion: »With respect to the alleged motive, it is of great importance to see whether there was a motive for committing such a crime, or whether there was not, or whether there is an improbability of its having been committed so strong as not to be overpowered by positive evidence. But \textit{if there be any motive which can be assigned, I am bound to tell you that the inadequacy of that motive is of little importance}. We know, from the experience of criminal courts, that atrocious crimes of this sort have been committed from very slight motives; \textit{not merely from malice and revenge}, but to gain a small pecuniary advantage, and to drive off for a time pressing difficulties.«—L. C. J. Campbell, summing up in Reg. v. Palmer, Shorthand Report, p. 308 C. C. C., May, 1856, Sess. Pa. 5. (Italics mine. D. L. S.)}	You're thinking that people don't keep up old jealousies for twenty years or so. Perhaps not. Not just primitive, brute jealousy. That means a word and a blow. But the thing that rankles is hurt vanity. That sticks. Humiliation. And we've all got a sore spot we don't like to have touched. I've got it. You've got it. Some blighter said hell knew no fury like a woman scorned. Stickin' it on to women, poor devils. Sex is every man's loco spot—you needn't fidget, you know it's true—he'll take a disappointment, but not a humiliation. I knew a man once who'd been turned down—not too charitably—by a girl he was engaged to. He spoke quite decently about her. I asked what had become of her. »Oh,« he said, »she married the other fellow.« And then burst out—couldn't help himself. »Lord, yes!« he cried. »To think of it—jilted for a Scotchman!« I don't know why he didn't like Scots, but that was what got him on the raw. Look at Freke. I've read his books. His attacks on his antagonists are savage. And he's a scientist. Yet he can't bear opposition, even in his work, which is where any first-class man is most sane and open-minded. Do you think he's a man to take a beating from any man on a side-issue? On a man's most sensitive side-issue? People are opinionated about side-issues, you know. I see red if anybody questions my judgment about a book. And Levy—who was nobody twenty years ago—romps in and carries off Freke's girl from under his nose. It isn't the girl Freke would bother about—it's having his aristocratic nose put out of joint by a little Jewish nobody.«

»There's another thing. Freke's got another side-issue. He likes crime. In that criminology book of his he gloats over a hardened murderer. I've read it, and I've seen the admiration simply glaring out between the lines whenever he writes about a callous and successful criminal. He reserves his contempt for the victims or the penitents or the men who lose their heads and get found out. His heroes are Edmond de la Pommerais, who persuaded his mistress into becoming an accessory to her own murder, and George Joseph Smith of Brides-in-a-bath fame, who could make passionate love to his wife in the night and carry out his plot to murder her in the morning. After all, he thinks conscience is a sort of vermiform appendix. Chop it out and you'll feel all the better. Freke isn't troubled by the usual conscientious deterrent. Witness his own hand in his books. Now again. The man who went to Levy's house in his place knew the house: Freke knew the house; he was a red-haired man, smaller than Levy, but not much smaller, since he could wear his clothes without appearing ludicrous: you have seen Freke—you know his height—about five-foot-eleven, I suppose, and his auburn mane; he probably wore surgical gloves: Freke is a surgeon; he was a methodical and daring man: surgeons are obliged to be both daring and methodical. Now take the other side. The man who got hold of the Battersea corpse had to have access to dead bodies. Freke obviously had access to dead bodies. He had to be cool and quick and callous about handling a dead body. Surgeons are all that. He had to be a strong man to carry the body across the roofs and dump it in at Thipps's window. Freke is a powerful man and a member of the Alpine Club. He probably wore surgical gloves and he let the body down from the roof with a surgical bandage. This points to a surgeon again. He undoubtedly lived in the neighbourhood. Freke lives next door. The girl you interviewed heard a bump on the roof of the end house. That is the house next to Freke's. Every time we look at Freke, he leads somewhere, whereas Milligan and Thipps and Crimplesham and all the other people we've honoured with our suspicion simply led nowhere.«

»Yes; but it's not quite so simple as you make out. What was Levy doing in that surreptitious way at Freke's on Monday night?«

»Well, you have Freke's explanation.«

»Rot, Wimsey. You said yourself it wouldn't do.«

»Excellent. It won't do. Therefore Freke was lying. Why should he lie about it, unless he had some object in hiding the truth?«

»Well, but why mention it at all?«

»Because Levy, contrary to all expectation, had been seen at the corner of the road. That was a nasty accident for Freke. He thought it best to be beforehand with an explanation—of sorts. He reckoned, of course, on nobody's ever connecting Levy with Battersea Park.«

»Well, then, we come back to the first question: Why did Levy go there?«

»I don't know, but he was got there somehow. Why did Freke buy all those Peruvian Oil shares?«

»I don't know,« said Parker in his turn.

»Anyway,« went on Wimsey, »Freke expected him, and made arrangements to let him in himself, so that Cummings shouldn't see who the caller was.«

»But the caller left again at ten.«

»Oh, Charles! I did not expect this of you. This is the purest Suggery! Who saw him go? Somebody said »Good-night« and walked away down the street. And you believe it was Levy because Freke didn't go out of his way to explain that it wasn't.«

»D'you mean that Freke walked cheerfully out of the house to Park Lane, and left Levy behind—dead or alive—for Cummings to find?«

»We have Cummings's word that he did nothing of the sort. A few minutes after the steps walked away from the house, Freke rang the library bell and told Cummings to shut up for the night.«

»Then\longdash«

»Well—there's a side door to the house, I suppose—in fact, you know there is—Cummings said so—through the hospital.«

»Yes—well, where was Levy?«

»Levy went up into the library and never came down. You've been in Freke's library. Where would you have put him?«

»In my bedroom next door.«

»Then that's where he did put him.«

»But suppose the man went in to turn down the bed?«

»Beds are turned down by the housekeeper, earlier than ten o'clock.«

»Yes\textellipsis . But Cummings heard Freke about the house all night.«

»He heard him go in and out two or three times. He'd expect him to do that, anyway.«

»Do you mean to say Freke got all that job finished before three in the morning?«

»Why not?«

»Quick work.«

»Well, call it quick work. Besides, why three? Cummings never saw him again till he called him for eight o'clock breakfast.«

»But he was having a bath at three.«

»I don't say he didn't get back from Park Lane before three. But I don't suppose Cummings went and looked through the bathroom keyhole to see if he was in the bath.«

Parker considered again.

»How about Crimplesham's pince-nez?« he asked.

»That is a bit mysterious,« said Lord Peter.

»And why Thipps's bathroom?«

»Why, indeed? Pure accident, perhaps—or pure devilry.«

»Do you think all this elaborate scheme could have been put together in a night, Wimsey?«

»Far from it. It was conceived as soon as that man who bore a superficial resemblance to Levy came into the workhouse. He had several days.«

»I see.«

»Freke gave himself away at the inquest. He and Grimbold disagreed about the length of the man's illness. If a small man (comparatively speaking) like Grimbold presumes to disagree with a man like Freke, it's because he is sure of his ground.«

»Then—if your theory is sound—Freke made a mistake.«

»Yes. A very slight one. He was guarding, with unnecessary caution, against starting a train of thought in the mind of anybody—say, the workhouse doctor. Up till then he'd been reckoning on the fact that people don't think a second time about anything (a body, say) that's once been accounted for.«

»What made him lose his head?«

»A chain of unforeseen accidents. Levy's having been recognised—my mother's son having foolishly advertised in the \textit{Times} his connection with the Battersea end of the mystery—Detective Parker (whose photograph has been a little prominent in the illustrated press lately) seen sitting next door to the Duchess of Denver at the inquest. His aim in life was to prevent the two ends of the problem from linking up. And there were two of the links, literally side by side. Many criminals are wrecked by over-caution.«

Parker was silent.