%!TeX root=../bodytop.tex
\chapter[Chapter \thechapter]{}
\lettrine[lines=4]{M}{r} Parker was a bachelor, and occupied a Georgian but inconvenient flat at № 12A Great Ormond Street, for which he paid a pound a week. His exertions in the cause of civilization were rewarded, not by the gift of diamond rings from empresses or munificent cheques from grateful Prime Ministers, but by a modest, though sufficient, salary, drawn from the pockets of the British taxpayer. He awoke, after a long day of arduous and inconclusive labour, to the smell of burnt porridge. Through his bedroom window, hygienically open top and bottom, a raw fog was rolling slowly in, and the sight of a pair of winter pants, flung hastily over a chair the previous night, fretted him with a sense of the sordid absurdity of the human form. The telephone bell rang, and he crawled wretchedly out of bed and into the sitting-room, where Mrs~Munns, who did for him by the day, was laying the table, sneezing as she went.

Mr~Bunter was speaking.

<His lordship says he'd be very glad, sir, if you could make it convenient to step round to breakfast.>

If the odour of kidneys and bacon had been wafted along the wire, Mr~Parker could not have experienced a more vivid sense of consolation.

<Tell his lordship I'll be with him in half an hour,> he said, thankfully, and plunging into the bathroom, which was also the kitchen, he informed Mrs~Munns, who was just making tea from a kettle which had gone off the boil, that he should be out to breakfast.

<You can take the porridge home for the family,> he added, viciously, and flung off his dressing-gown with such determination that Mrs~Munns could only scuttle away with a snort.

A 19 'bus deposited him in Piccadilly only fifteen minutes later than his rather sanguine impulse had prompted him to suggest, and Mr~Bunter served him with glorious food, incomparable coffee, and the \textit{Daily Mail} before a blazing fire of wood and coal. A distant voice singing the <et iterum venturus est> from Bach's \textit{Mass in B minor} proclaimed that for the owner of the flat cleanliness and godliness met at least once a day, and presently Lord~Peter roamed in, moist and verbena-scented, in a bath-robe cheerfully patterned with unnaturally variegated peacocks.

<Mornin', old dear,> said that gentleman. <Beast of a day, ain't it? Very good of you to trundle out in it, but I had a letter I wanted you to see, and I hadn't the energy to come round to your place. Bunter and I've been makin' a night of it.>

<What's the letter?> asked Parker.

<Never talk business with your mouth full,> said Lord~Peter, reprovingly; <have some Oxford marmalade—and then I'll show you my Dante; they brought it round last night. What ought I to read this morning, Bunter?>

<Lord~Erith's collection is going to be sold, my lord. There is a column about it in the \textit{Morning Post}. I think your lordship should look at this review of Sir Julian Freke's new book on \textit{The Physiological Bases of the Conscience} in the \textit{Times Literary Supplement}. Then there is a very singular little burglary in the \textit{Chronicle}, my lord, and an attack on titled families in the \textit{Herald}—rather ill-written, if I may say so, but not without unconscious humour which your lordship will appreciate.>

<All right, give me that and the burglary,> said his lordship.

<I have looked over the other papers,> pursued Mr~Bunter, indicating a formidable pile, <and marked your lordship's after-breakfast reading.>

<Oh, pray don't allude to it,> said Lord~Peter; <you take my appetite away.>

There was silence, but for the crunching of toast and the crackling of paper.

<I see they adjourned the inquest,> said Parker presently.

<Nothing else to do,> said Lord~Peter; <but Lady~Levy arrived last night, and will have to go and fail to identify the body this morning for Sugg's benefit.>

<Time, too,> said Mr~Parker shortly.

Silence fell again.

<I don't think much of your burglary, Bunter,> said Lord~Peter. <Competent, of course, but no imagination. I want imagination in a criminal. Where's the \textit{Morning Post}?>

After a further silence, Lord~Peter said: <You might send for the catalogue, Bunter, that Apollonios Rhodios\footnote{Apollonios Rhodios. Lorenzobodi Alopa. Firenze. 1496. (4to.) The excitement attendant on the solution of the Battersea Mystery did not prevent Lord~Peter from securing this rare work before his departure for Corsica.} might be worth looking at. No, I'm damned if I'm going to stodge through that review, but you can stick the book on the library list if you like. His book on crime was entertainin' enough as far as it went, but the fellow's got a bee in his bonnet. Thinks God's a secretion of the liver—all right once in a way, but there's no need to keep on about it. There's nothing you can't prove if your outlook is only sufficiently limited. Look at Sugg.>

<I beg your pardon,> said Parker; <I wasn't attending. Argentines are steadying a little, I see.>

<Milligan,> said Lord~Peter.

<Oil's in a bad way. Levy's made a difference there. That funny little boom in Peruvians that came on just before he disappeared has died away again. I wonder if he was concerned in it. D'you know at all?>

<I'll find out,> said Lord~Peter. <What was it?>

<Oh, an absolutely dud enterprise that hadn't been heard of for years. It suddenly took a little lease of life last week. I happened to notice it because my mother got let in for a couple of hundred shares a long time ago. It never paid a dividend. Now it's petered out again.>

Wimsey pushed his plate aside and lit a pipe.

<Having finished, I don't mind doing some work,> he said. <How did you get on yesterday?>

<I didn't,> replied Parker. <I sleuthed up and down those flats in my own bodily shape and two different disguises. I was a gas-meter man and a collector for a Home for Lost Doggies, and I didn't get a thing to go on, except a servant in the top flat at the Battersea Bridge Road end of the row who said she thought she heard a bump on the roof one night. Asked which night, she couldn't rightly say. Asked if it was Monday night, she thought it very likely. Asked if it mightn't have been in that high wind on Saturday night that blew my chimney-pot off, she couldn't say but what it might have been. Asked if she was sure it was on the roof and not inside the flat, said to be sure they did find a picture tumbled down next morning. Very suggestible girl. I saw your friends, Mr~and Mrs~Appledore, who received me coldly, but could make no definite complaint about Thipps except that his mother dropped her h's, and that he once called on them uninvited, armed with a pamphlet about anti-vivisection. The Indian Colonel on the first floor was loud, but unexpectedly friendly. He gave me Indian curry for supper and some very good whisky, but he's a sort of hermit, and all \textit{he} could tell me was that he couldn't stand Mrs~Appledore.>

<Did you get nothing at the house?>

<Only Levy's private diary. I brought it away with me. Here it is. It doesn't tell one much, though. It's full of entries like: <Tom and Annie to dinner>; and <My dear wife's birthday; gave her an old opal ring>; <Mr~Arbuthnot dropped in to tea; he wants to marry Rachel, but I should like someone steadier for my treasure.> Still, I thought it would show who came to the house and so on. He evidently wrote it up at night. There's no entry for Monday.>

<I expect it'll be useful,> said Lord~Peter, turning over the pages. <Poor old buffer. I say, I'm not so certain now he was done away with.>

He detailed to Mr~Parker his day's work.

<Arbuthnot?> said Parker. <Is that the Arbuthnot of the diary?>

<I suppose so. I hunted him up because I knew he was fond of fooling round the Stock Exchange. As for Milligan, he \textit{looks} all right, but I believe he's pretty ruthless in business and you never can tell. Then there's the red-haired secretary—lightnin' calculator man with a face like a fish, keeps on sayin' nuthin'—got the Tarbaby in his family tree, I should think. Milligan's got a jolly good motive for, at any rate, suspendin' Levy for a few days. Then there's the new man.>

<What new man?>

<Ah, that's the letter I mentioned to you. Where did I put it? Here we are. Good parchment paper, printed address of solicitor's office in Salisbury, and postmark to correspond. Very precisely written with a fine nib by an elderly business man of old-fashioned habits.>

Parker took the letter and read:

\begin{a4}
	\clearpage
\end{a4}


\begin{quotation}
\begin{flushright}
\hfill
\begin{minipage}{0.5\linewidth}
\textsc{Crimplesham and Wicks,}\\
\vin \textit{Solicitors,}\\
\textsc{Milford Hill, Salisbury,}\\
17 November, 192—.
\end{minipage}
\end{flushright}

\noindent \textsc{Sir,}

With reference to your advertisement today in the personal column of \textit{The Times}, I am disposed to believe that the eyeglasses and chain in question may be those I lost on the \textsc{l.b.\&s.c.} Electric Railway while visiting London last Monday. I left Victoria by the 5.45 train, and did not notice my loss till I arrived at Balham. This indication and the optician's specification of the glasses, which I enclose, should suffice at once as an identification and a guarantee of my bona fides. If the glasses should prove to be mine, I should be greatly obliged to you if you would kindly forward them to me by registered post, as the chain was a present from my daughter, and is one of my dearest possessions.

Thanking you in advance for this kindness, and regretting the trouble to which I shall be putting you, I am,

\begin{flushright}
Yours very truly,\\
\textsc{Thos. Crimplesham}\\
\end{flushright}

\noindent Lord~Peter Wimsey,\\
\vin 110, Piccadilly, W\@.\\
\noindent (Encl.)
\end{quotation}

<Dear me,> said Parker, <this is what you might call unexpected.>

<Either it is some extraordinary misunderstanding,> said Lord~Peter, <or Mr~Crimplesham is a very bold and cunning villain. Or possibly, of course, they are the wrong glasses. We may as well get a ruling on that point at once. I suppose the glasses are at the Yard. I wish you'd just ring 'em up and ask 'em to send round an optician's description of them at once—and you might ask at the same time whether it's a very common prescription.>

<Right you are,> said Parker, and took the receiver off its hook.

<And now,> said his friend, when the message was delivered, <just come into the library for a minute.>

On the library table, Lord~Peter had spread out a series of bromide prints, some dry, some damp, and some but half-washed.

<These little ones are the originals of the photos we've been taking,> said Lord~Peter, <and these big ones are enlargements all made to precisely the same scale. This one here is the footmark on the linoleum; we'll put that by itself at present. Now these finger-prints can be divided into five lots. I've numbered 'em on the prints—see?—and made a list:

\textsc{A\@.} The finger-prints of Levy himself, off his little bedside book and his hair-brush—this and this—you can't mistake the little scar on the thumb.

\textsc{B\@.} The smudges made by the gloved fingers of the man who slept in Levy's room on Monday night. They show clearly on the water-bottle and on the boots—superimposed on Levy's. They are very distinct on the boots—surprisingly so for gloved hands, and I deduce that the gloves were rubber ones and had recently been in water.

Here's another interestin' point. Levy walked in the rain on Monday night, as we know, and these dark marks are mud-splashes. You see they lie \textit{over} Levy's finger-prints in every case. Now see: on this left boot we find the stranger's thumb-mark \textit{over} the mud on the leather above the heel. That's a funny place to find a thumb-mark on a boot, isn't it? That is, if Levy took off his own boots. But it's the place where you'd expect to see it if somebody forcibly removed his boots for him. Again, most of the stranger's finger-marks come \textit{over} the mud-marks, but here is one splash of mud which comes on top of them again. Which makes me infer that the stranger came back to Park Lane, wearing Levy's boots, in a cab, carriage or car, but that at some point or other he walked a little way—just enough to tread in a puddle and get a splash on the boots. What do you say?>

<Very pretty,> said Parker. <A bit intricate, though, and the marks are not all that I could wish a finger-print to be.>

<Well, I won't lay too much stress on it. But it fits in with our previous ideas. Now let's turn to:

\textsc{C\@.} The prints obligingly left by my own particular villain on the further edge of Thipps's bath, where you spotted them, and I ought to be scourged for not having spotted them. The left hand, you notice, the base of the palm and the fingers, but not the tips, looking as though he had steadied himself on the edge of the bath while leaning down to adjust something at the bottom, the pince-nez perhaps. Gloved, you see, but showing no ridge or seam of any kind—I say rubber, you say rubber. That's that. Now see here:

\textsc{D} and \textsc{E} come off a visiting-card of mine. There's this thing at the corner, marked F\@, but that you can disregard; in the original document it's a sticky mark left by the thumb of the youth who took it from me, after first removing a piece of chewing-gum from his teeth with his finger to tell me that Mr~Milligan might or might not be disengaged. D and E are the thumb-marks of Mr~Milligan and his red-haired secretary. I'm not clear which is which, but I saw the youth with the chewing-gum hand the card to the secretary, and when I got into the inner shrine I saw John P\@. Milligan standing with it in his hand, so it's one or the other, and for the moment it's immaterial to our purpose which is which. I boned the card from the table when I left.>

<Well, now, Parker, here's what's been keeping Bunter and me up till the small hours. I've measured and measured every way backwards and forwards till my head's spinnin', and I've stared till I'm nearly blind, but I'm hanged if I can make my mind up. Question 1. Is \textsc{C} identical with \textsc{B}? Question 2. Is \textsc{D} or \textsc{E} identical with \textsc{B}? There's nothing to go on but the size and shape, of course, and the marks are so faint—what do you think?>

Parker shook his head doubtfully.

<I think \textsc{E} might almost be put out of the question,> he said; <it seems such an excessively long and narrow thumb. But I think there is a decided resemblance between the span of \textsc{B} on the water-bottle and \textsc{C} on the bath. And I don't see any reason why \textsc{D} shouldn't be the same as \textsc{B}, only there's so little to judge from.>

<Your untutored judgment and my measurements have brought us both to the same conclusion—if you can call it a conclusion,> said Lord~Peter, bitterly.

<Another thing,> said Parker. <Why on earth should we try to connect \textsc{B} with \textsc{C}? The fact that you and I happen to be friends doesn't make it necessary to conclude that the two cases we happen to be interested in have any organic connection with one another. Why should they? The only person who thinks they have is Sugg, and he's nothing to go by. It would be different if there were any truth in the suggestion that the man in the bath was Levy, but we know for a certainty he wasn't. It's ridiculous to suppose that the same man was employed in committing two totally distinct crimes on the same night, one in Battersea and the other in Park Lane.>

<I know,> said Wimsey, <though of course we mustn't forget that Levy \textit{was} in Battersea at the time, and now we know he didn't return home at twelve as was supposed, we've no reason to think he ever left Battersea at all.>

<True. But there are other places in Battersea besides Thipps's bathroom. And he \textit{wasn't} in Thipps's bathroom. In fact, come to think of it, that's the one place in the universe where we know definitely that he wasn't. So what's Thipps's bath got to do with it?>

<I don't know,> said Lord~Peter. <Well, perhaps we shall get something better to go on today.>

He leaned back in his chair and smoked thoughtfully for some time over the papers which Bunter had marked for him.

<They've got you out in the limelight,> he said. <Thank Heaven, Sugg hates me too much to give me any publicity. What a dull Agony Column! <Darling Pipsey—Come back soon to your distracted Popsey>—and the usual young man in need of financial assistance, and the usual injunction to <Remember thy Creator in the days of thy youth.> Hullo! there's the bell. Oh, it's our answer from Scotland Yard.>

The note from Scotland Yard enclosed an optician's specification identical with that sent by Mr~Crimplesham, and added that it was an unusual one, owing to the peculiar strength of the lenses and the marked difference between the sight of the two eyes.

<That's good enough,> said Parker.

<Yes,> said Wimsey. <Then Possibility № 3 is knocked on the head. There remain Possibility № 1: Accident or Misunderstanding, and № 2: Deliberate Villainy, of a remarkably bold and calculating kind—of a kind, in fact, characteristic of the author or authors of our two problems. Following the methods inculcated at that University of which I have the honour to be a member, we will now examine severally the various suggestions afforded by Possibility № 2. This Possibility may be again subdivided into two or more Hypotheses. On Hypothesis 1 (strongly advocated by my distinguished colleague Professor Snupshed), the criminal, whom we may designate as X\@, is not identical with Crimplesham, but is using the name of Crimplesham as his shield, or aegis. This hypothesis may be further subdivided into two alternatives. Alternative A\@: Crimplesham is an innocent and unconscious accomplice, and X is in his employment. X writes in Crimplesham's name on Crimplesham's office-paper and obtains that the object in question, i.e., the eyeglasses, be despatched to Crimplesham's address. He is in a position to intercept the parcel before it reaches Crimplesham. The presumption is that X is Crimplesham's charwoman, office-boy, clerk, secretary or porter. This offers a wide field of investigation. The method of inquiry will be to interview Crimplesham and discover whether he sent the letter, and if not, who has access to his correspondence. Alternative B\@: Crimplesham is under X's influence or in his power, and has been induced to write the letter by (\textit{a}) bribery, (\textit{b}) misrepresentation or (\textit{c}) threats. X may in that case be a persuasive relation or friend, or else a creditor, blackmailer or assassin; Crimplesham, on the other hand, is obviously venal or a fool. The method of inquiry in this case, I would tentatively suggest, is again to interview Crimplesham, put the facts of the case strongly before him, and assure him in the most intimidating terms that he is liable to a prolonged term of penal servitude as an accessory after the fact in the crime of murder— Ah-hem! Trusting, gentlemen, that you have followed me thus far, we will pass to the consideration of Hypothesis № 2, to which I personally incline, and according to which X is identical with Crimplesham.>

<In this case, Crimplesham, who is, in the words of an English classic, a man-of-infinite-resource-and-sagacity, correctly deduces that, of all people, the last whom we shall expect to find answering our advertisement is the criminal himself. Accordingly, he plays a bold game of bluff. He invents an occasion on which the glasses may very easily have been lost or stolen, and applies for them. If confronted, nobody will be more astonished than he to learn where they were found. He will produce witnesses to prove that he left Victoria at 5.45 and emerged from the train at Balham at the scheduled time, and sat up all Monday night playing chess with a respectable gentleman well known in Balham. In this case, the method of inquiry will be to pump the respectable gentleman in Balham, and if he should happen to be a single gentleman with a deaf housekeeper, it may be no easy matter to impugn the alibi, since, outside detective romances, few ticket-collectors and 'bus-conductors keep an exact remembrance of all the passengers passing between Balham and London on any and every evening of the week.>

<Finally, gentlemen, I will frankly point out the weak point of all these hypotheses, namely: that none of them offers any explanation as to why the incriminating article was left so conspicuously on the body in the first instance.>

Mr~Parker had listened with commendable patience to this academic exposition.

<Might not X\@,> he suggested, <be an enemy of Crimplesham's, who designed to throw suspicion upon him?>

<He might. In that case he should be easy to discover, since he obviously lives in close proximity to Crimplesham and his glasses, and Crimplesham in fear of his life will then be a valuable ally for the prosecution.>

<How about the first possibility of all, misunderstanding or accident?>

<Well! Well, for purposes of discussion, nothing, because it really doesn't afford any data for discussion.>

<In any case,> said Parker, <the obvious course appears to be to go to Salisbury.>

<That seems indicated,> said Lord~Peter.

<Very well,> said the detective, <is it to be you or me or both of us?>

<It is to be me,> said Lord~Peter, <and that for two reasons. First, because, if (by Possibility № 2, Hypothesis 1, Alternative A) Crimplesham is an innocent catspaw, the person who put in the advertisement is the proper person to hand over the property. Secondly, because, if we are to adopt Hypothesis 2, we must not overlook the sinister possibility that Crimplesham-X is laying a careful trap to rid himself of the person who so unwarily advertised in the daily press his interest in the solution of the Battersea Park mystery.>

<That appears to me to be an argument for our both going,> objected the detective.

<Far from it,> said Lord~Peter. <Why play into the hands of Crimplesham-X by delivering over to him the only two men in London with the evidence, such as it is, and shall I say the wits, to connect him with the Battersea body?>

<But if we told the Yard where we were going, and we both got nobbled,> said Mr~Parker, <it would afford strong presumptive evidence of Crimplesham's guilt, and anyhow, if he didn't get hanged for murdering the man in the bath he'd at least get hanged for murdering us.>

<Well,> said Lord~Peter, <if he only murdered me you could still hang him—what's the good of wasting a sound, marriageable young male like yourself? Besides, how about old Levy? If you're incapacitated, do you think anybody else is going to find him?>

<But we could frighten Crimplesham by threatening him with the Yard.>

<Well, dash it all, if it comes to that, I can frighten him by threatening him with \textit{you}, which, seeing you hold what evidence there is, is much more to the point. And, then, suppose it's a wild-goose chase after all, you'll have wasted time when you might have been getting on with the case. There are several things that need doing.>

<Well,> said Parker, silenced but reluctant, <why can't I go, in that case?>

<Bosh!> said Lord~Peter. <I am retained (by old Mrs~Thipps, for whom I entertain the greatest respect) to deal with this case, and it's only by courtesy I allow you to have anything to do with it.>

Mr~Parker groaned.

<Will you at least take Bunter?> he said.

<In deference to your feelings,> replied Lord~Peter, <I will take Bunter, though he could be far more usefully employed taking photographs or overhauling my wardrobe. When is there a good train to Salisbury, Bunter?>

<There is an excellent train at 10.50, my lord.>

<Kindly make arrangements to catch it,> said Lord~Peter, throwing off his bath-robe and trailing away with it into his bedroom. <And, Parker—if you have nothing else to do you might get hold of Levy's secretary and look into that little matter of the Peruvian oil.>

Lord~Peter took with him, for light reading in the train, Sir Reuben Levy's diary. It was a simple, and in the light of recent facts, rather a pathetic document. The terrible fighter of the Stock Exchange, who could with one nod set the surly bear dancing, or bring the savage bull to feed out of his hand, whose breath devastated whole districts with famine or swept financial potentates from their seats, was revealed in private life as kindly, domestic, innocently proud of himself and his belongings, confiding, generous and a little dull. His own small economies were duly chronicled side by side with extravagant presents to his wife and daughter. Small incidents of household routine appeared, such as: <Man came to mend the conservatory roof,> or <The new butler (Simpson) has arrived, recommended by the Goldbergs. I think he will be satisfactory.> All visitors and entertainments were duly entered, from a very magnificent lunch to Lord~Dewsbury, the Minister for Foreign Affairs, and Dr~Jabez K\@. Wort, the American plenipotentiary, through a series of diplomatic dinners to eminent financiers, down to intimate family gatherings of persons designated by Christian names or nicknames. About May there came a mention of Lady~Levy's nerves, and further reference was made to the subject in subsequent months. In September it was stated that <Freke came to see my dear wife and advised complete rest and change of scene. She thinks of going abroad with Rachel.> The name of the famous nerve-specialist occurred as a diner or luncher about once a month, and it came into Lord~Peter's mind that Freke would be a good person to consult about Levy himself. <People sometimes tell things to the doctor,> he murmured to himself. <And, by Jove! if Levy was simply going round to see Freke on Monday night, that rather disposes of the Battersea incident, doesn't it?> He made a note to look up Sir Julian and turned on further. On September 18\textsuperscript{th}, Lady~Levy and her daughter had left for the south of France. Then suddenly, under the date October 5\textsuperscript{th}, Lord~Peter found what he was looking for: <Goldberg, Skriner and Milligan to dinner.>

There was the evidence that Milligan had been in that house. There had been a formal entertainment—a meeting as of two duellists shaking hands before the fight. Skriner was a well-known picture-dealer; Lord~Peter imagined an after-dinner excursion upstairs to see the two Corots in the drawing-room, and the portrait of the oldest Levy girl, who had died at the age of sixteen. It was by Augustus John, and hung in the bedroom. The name of the red-haired secretary was nowhere mentioned, unless the initial S\@., occurring in another entry, referred to him. Throughout September and October, Anderson (of Wyndham's) had been a frequent visitor.

Lord~Peter shook his head over the diary, and turned to the consideration of the Battersea Park mystery. Whereas in the Levy affair it was easy enough to supply a motive for the crime, if crime it were, and the difficulty was to discover the method of its carrying out and the whereabouts of the victim, in the other case the chief obstacle to inquiry was the entire absence of any imaginable motive. It was odd that, although the papers had carried news of the affair from one end of the country to the other and a description of the body had been sent to every police station in the country, nobody had as yet come forward to identify the mysterious occupant of Mr~Thipps's bath. It was true that the description, which mentioned the clean-shaven chin, elegantly cut hair and the pince-nez, was rather misleading, but on the other hand, the police had managed to discover the number of molars missing, and the height, complexion and other data were correctly enough stated, as also the date at which death had presumably occurred. It seemed, however, as though the man had melted out of society without leaving a gap or so much as a ripple. Assigning a motive for the murder of a person without relations or antecedents or even clothes is like trying to visualize the fourth dimension—admirable exercise for the imagination, but arduous and inconclusive. Even if the day's interview should disclose black spots in the past or present of Mr~Crimplesham, how were they to be brought into connection with a person apparently without a past, and whose present was confined to the narrow limits of a bath and a police mortuary?

<Bunter,> said Lord~Peter, <I beg that in the future you will restrain me from starting two hares at once. These cases are gettin' to be a strain on my constitution. One hare has nowhere to run from, and the other has nowhere to run to. It's a kind of mental D\@.T\@., Bunter. When this is over I shall turn pussyfoot, forswear the police news, and take to an emollient diet of the works of the late Charles Garvice.>

It was its comparative proximity to Milford Hill that induced Lord~Peter to lunch at the Minster Hotel rather than at the White Hart or some other more picturesquely situated hostel. It was not a lunch calculated to cheer his mind; as in all Cathedral cities, the atmosphere of the Close pervades every nook and corner of Salisbury, and no food in that city but seems faintly flavoured with prayer-books. As he sat sadly consuming that impassive pale substance known to the English as <cheese> unqualified (for there are cheeses which go openly by their names, as Stilton, Camembert, Gruyère, Wensleydale or Gorgonzola, but <cheese> is cheese and everywhere the same), he inquired of the waiter the whereabouts of Mr~Crimplesham's office.

The waiter directed him to a house rather further up the street on the opposite side, adding: <But anybody'll tell you, sir; Mr~Crimplesham's very well known hereabouts.>

<He's a good solicitor, I suppose?> said Lord~Peter.

<Oh, yes, sir,> said the waiter, <you couldn't do better than trust to Mr~Crimplesham, sir. There's folk say he's old-fashioned, but I'd rather have my little bits of business done by Mr~Crimplesham than by one of these fly-away young men. Not but what Mr~Crimplesham'll be retiring soon, sir, I don't doubt, for he must be close on eighty, sir, if he's a day, but then there's young Mr~Wicks to carry on the business, and he's a very nice, steady-like young gentleman.>

<Is Mr~Crimplesham really as old as that?> said Lord~Peter. <Dear me! He must be very active for his years. A friend of mine was doing business with him in town last week.>

<Wonderful active, sir,> agreed the waiter, <and with his game leg, too, you'd be surprised. But there, sir, I often think when a man's once past a certain age, the older he grows the tougher he gets, and women the same or more so.>

<Very likely,> said Lord~Peter, calling up and dismissing the mental picture of a gentleman of eighty with a game leg carrying a dead body over the roof of a Battersea flat at midnight. <He's tough, sir, tough, is old Joey Bagstock, tough and devilish sly,> he added, thoughtlessly.

<Indeed, sir?> said the waiter. <I couldn't say, I'm sure.>

<I beg your pardon,> said Lord~Peter; <I was quoting poetry. Very silly of me. I got the habit at my mother's knee and I can't break myself of it.>

<No, sir,> said the waiter, pocketing a liberal tip. <Thank you very much, sir. You'll find the house easy. Just afore you come to Penny-farthing Street, sir, about two turnings off, on the right-hand side opposite.>

<Afraid that disposes of Crimplesham-X\@,> said Lord~Peter. <I'm rather sorry; he was a fine sinister figure as I had pictured him. Still, his may yet be the brain behind the hands—the aged spider sitting invisible in the centre of the vibrating web, you know, Bunter.>

<Yes, my lord,> said Bunter. They were walking up the street together.

<There is the office over the way,> pursued Lord~Peter. <I think, Bunter, you might step into this little shop and purchase a sporting paper, and if I do not emerge from the villain's lair—say within three-quarters of an hour, you may take such steps as your perspicuity may suggest.>

Mr~Bunter turned into the shop as desired, and Lord~Peter walked across and rang the lawyer's bell with decision.

<The truth, the whole truth and nothing but the truth is my long suit here, I fancy,> he murmured, and when the door was opened by a clerk he delivered over his card with an unflinching air.

He was ushered immediately into a confidential-looking office, obviously furnished in the early years of Queen Victoria's reign, and never altered since. A lean, frail-looking old gentleman rose briskly from his chair as he entered and limped forward to meet him.

<My dear sir,> exclaimed the lawyer, <how extremely good of you to come in person! Indeed, I am ashamed to have given you so much trouble. I trust you were passing this way, and that my glasses have not put you to any great inconvenience. Pray take a seat, Lord~Peter.> He peered gratefully at the young man over a pince-nez obviously the fellow of that now adorning a dossier in Scotland Yard.

Lord~Peter sat down. The lawyer sat down. Lord~Peter picked up a glass paper-weight from the desk and weighed it thoughtfully in his hand. Subconsciously he noted what an admirable set of finger-prints he was leaving upon it. He replaced it with precision on the exact centre of a pile of letters.

<It's quite all right,> said Lord~Peter. <I was here on business. Very happy to be of service to you. Very awkward to lose one's glasses, Mr~Crimplesham.>

<Yes,> said the lawyer, <I assure you I feel quite lost without them. I have this pair, but they do not fit my nose so well—besides, that chain has a great sentimental value for me. I was terribly distressed on arriving at Balham to find that I had lost them. I made inquiries of the railway, but to no purpose. I feared they had been stolen. There were such crowds at Victoria, and the carriage was packed with people all the way to Balham. Did you come across them in the train?>

<Well, no,> said Lord~Peter, <I found them in rather an unexpected place. Do you mind telling me if you recognized any of your fellow-travellers on that occasion?>

The lawyer stared at him.

<Not a soul,> he answered. <Why do you ask?>

<Well,> said Lord~Peter, <I thought perhaps the—the person with whom I found them might have taken them for a joke.>

The lawyer looked puzzled.

<Did the person claim to be an acquaintance of mine?> he inquired. <I know practically nobody in London, except the friend with whom I was staying in Balham, Dr~Philpots, and I should be very greatly surprised at his practising a jest upon me. He knew very well how distressed I was at the loss of the glasses. My business was to attend a meeting of shareholders in Medlicott's Bank, but the other gentlemen present were all personally unknown to me, and I cannot think that any of them would take so great a liberty. In any case,> he added, <as the glasses are here, I will not inquire too closely into the manner of their restoration. I am deeply obliged to you for your trouble.>

Lord~Peter hesitated.

<Pray forgive my seeming inquisitiveness,> he said, <but I must ask you another question. It sounds rather melodramatic, I'm afraid, but it's this. Are you aware that you have any enemy—anyone, I mean, who would profit by your—er—decease or disgrace?>

Mr~Crimplesham sat frozen into stony surprise and disapproval.

<May I ask the meaning of this extraordinary question?> he inquired stiffly.

<Well,> said Lord~Peter, <the circumstances are a little unusual. You may recollect that my advertisement was addressed to the jeweller who sold the chain.>

<That surprised me at the time,> said Mr~Crimplesham, <but I begin to think your advertisement and your behaviour are all of a piece.>

<They are,> said Lord~Peter. <As a matter of fact I did not expect the owner of the glasses to answer my advertisement. Mr~Crimplesham, you have no doubt read what the papers have to say about the Battersea Park mystery. Your glasses are the pair that was found on the body, and they are now in the possession of the police at Scotland Yard, as you may see by this.> He placed the specification of the glasses and the official note before Crimplesham.

<Good God!> exclaimed the lawyer. He glanced at the paper, and then looked narrowly at Lord~Peter.

<Are you yourself connected with the police?> he inquired.

<Not officially,> said Lord~Peter. <I am investigating the matter privately, in the interests of one of the parties.>

Mr~Crimplesham rose to his feet.

<My good man,> he said, <this is a very impudent attempt, but blackmail is an indictable offence, and I advise you to leave my office before you commit yourself.> He rang the bell.

<I was afraid you'd take it like that,> said Lord~Peter. <It looks as though this ought to have been my friend Detective Parker's job, after all.> He laid Parker's card on the table beside the specification, and added: <If you should wish to see me again, Mr~Crimplesham, before tomorrow morning, you will find me at the Minster Hotel.>

Mr~Crimplesham disdained to reply further than to direct the clerk who entered to <show this person out.>

In the entrance Lord~Peter brushed against a tall young man who was just coming in, and who stared at him with surprised recognition. His face, however, aroused no memories in Lord~Peter's mind, and that baffled nobleman, calling out Bunter from the newspaper shop, departed to his hotel to get a trunk-call through to Parker.

Meanwhile, in the office, the meditations of the indignant Mr~Crimplesham were interrupted by the entrance of his junior partner.

<I say,> said the latter gentleman, <has somebody done something really wicked at last? Whatever brings such a distinguished amateur of crime on our sober doorstep?>

<I have been the victim of a vulgar attempt at blackmail,> said the lawyer; <an individual passing himself off as Lord~Peter Wimsey\longdash>

<But that \textit{is} Lord~Peter Wimsey,> said Mr~Wicks, <there's no mistaking him. I saw him give evidence in the Attenbury emerald case. He's a big little pot in his way, you know, and goes fishing with the head of Scotland Yard.>

<Oh, dear,> said Mr~Crimplesham.

Fate arranged that the nerves of Mr~Crimplesham should be tried that afternoon. When, escorted by Mr~Wicks, he arrived at the Minster Hotel, he was informed by the porter that Lord~Peter Wimsey had strolled out, mentioning that he thought of attending Evensong. <But his man is here, sir,> he added, <if you'd like to leave a message.>

Mr~Wicks thought that on the whole it would be well to leave a message. Mr~Bunter, on inquiry, was found to be sitting by the telephone, waiting for a trunk-call. As Mr~Wicks addressed him the bell rang, and Mr~Bunter, politely excusing himself, took down the receiver.

<Hullo!> he said. <Is that Mr~Parker? Oh, thanks! Exchange! Exchange! Sorry, can you put me through to Scotland Yard? Excuse me, gentlemen, keeping you waiting.—Exchange! all right—Scotland Yard—Hullo! Is that Scotland Yard?—Is Detective Parker round there?—Can I speak to him?—I shall have done in a moment, gentlemen.—Hullo! is that you, Mr~Parker? Lord~Peter would be much obliged if you could find it convenient to step down to Salisbury, sir. Oh, no, sir, he's in excellent health, sir—just stepped round to hear Evensong, sir—oh, no, I think tomorrow morning would do excellently, sir, thank you, sir.>