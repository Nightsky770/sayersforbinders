%!TeX root=../bodytop.tex
\chapter[Chapter \thechapter]{}
\lettrine[lines=4]{D}{ear} \textsc{Lord Peter}\allowbreak---\allowbreak When I was a young man I used to play chess with an old friend of my father’s. He was a very bad, and a very slow, player, and he could never see when a checkmate was inevitable, but insisted on playing every move out. I never had any patience with that kind of attitude, and I will freely admit now that the game is yours. I must either stay at home and be hanged or escape abroad and live in an idle and insecure obscurity. I prefer to acknowledge defeat.

If you have read my book on \textit{Criminal Lunacy,} you will remember that I wrote: \enquote{In the majority of cases, the criminal betrays himself by some abnormality attendant upon this pathological condition of the nervous tissues. His mental instability shows itself in various forms: an overweening vanity, leading him to brag of his achievement; a disproportionate sense of the importance of the offence, resulting from the hallucination of religion, and driving him to confession; egomania, producing the sense of horror or conviction of sin, and driving him to headlong flight without covering his tracks; a reckless confidence, resulting in the neglect of the most ordinary precautions, as in the case of Henry Wainwright, who left a boy in charge of the murdered woman’s remains while he went to call a cab, or on the other hand, a nervous distrust of apperceptions in the past, causing him to revisit the scene of the crime to assure himself that all traces have been as safely removed as \textit{his own judgment knows them to be}. I will not hesitate to assert that a perfectly sane man, not intimidated by religious or other delusions, could always render himself perfectly secure from detection, provided, that is, that the crime were sufficiently premeditated and that he were not pressed for time or thrown out in his calculations by purely fortuitous coincidence.}

You know as well as I do, how far I have made this assertion good in practice. The two accidents which betrayed me, I could not by any possibility have foreseen. The first was the chance recognition of Levy by the girl in the Battersea Park Road, which suggested a connection between the two problems. The second was that Thipps should have arranged to go down to Denver on the Tuesday morning, thus enabling your mother to get word of the matter through to you before the body was removed by the police and to suggest a motive for the murder out of what she knew of my previous personal history. If I had been able to destroy these two accidentally forged links of circumstance, I will venture to say that you would never have so much as suspected me, still less obtained sufficient evidence to convict.

Of all human emotions, except perhaps those of hunger and fear, the sexual appetite produces the most violent, and, under some circumstances, the most persistent reactions; I think, however, I am right in saying that at the time when I wrote my book, my original sensual impulse to kill Sir Reuben Levy had already become profoundly modified by my habits of thought. To the animal lust to slay and the primitive human desire for revenge, there was added the rational intention of substantiating my own theories for the satisfaction of myself and the world. If all had turned out as I had planned, I should have deposited a sealed account of my experiment with the Bank of England, instructing my executors to publish it after my death. Now that accident has spoiled the completeness of my demonstration, I entrust the account to you, whom it cannot fail to interest, with the request that you will make it known among scientific men, in justice to my professional reputation.

The really essential factors of success in any undertaking are money and opportunity, and as a rule, the man who can make the first can make the second. During my early career, though I was fairly well-off, I had not absolute command of circumstance. Accordingly I devoted myself to my profession, and contented myself with keeping up a friendly connection with Reuben Levy and his family. This enabled me to remain in touch with his fortunes and interests, so that, when the moment for action should arrive, I might know what weapons to use.

Meanwhile, I carefully studied criminology in fiction and fact\allowbreak---\allowbreak my work on \textit{Criminal Lunacy} was a side-product of this activity\allowbreak---\allowbreak and saw how, in every murder, the real crux of the problem was the disposal of the body. As a doctor, the means of death were always ready to my hand, and I was not likely to make any error in that connection. Nor was I likely to betray myself on account of any illusory sense of wrong-doing. The sole difficulty would be that of destroying all connection between my personality and that of the corpse. You will remember that Michael Finsbury, in Stevenson’s entertaining romance, observes: \enquote{What hangs people is the unfortunate circumstance of guilt.} It became clear to me that the mere leaving about of a superfluous corpse could convict nobody, provided that nobody was guilty in connection \textit{with that particular corpse}. Thus the idea of substituting the one body for the other was early arrived at, though it was not till I obtained the practical direction of St Luke’s Hospital that I found myself perfectly unfettered in the choice and handling of dead bodies. From this period on, I kept a careful watch on all the material brought in for dissection.

My opportunity did not present itself until the week before Sir Reuben’s disappearance, when the medical officer at the Chelsea workhouse sent word to me that an unknown vagrant had been injured that morning by the fall of a piece of scaffolding, and was exhibiting some very interesting nervous and cerebral reactions. I went round and saw the case, and was immediately struck by the man’s strong superficial resemblance to Sir Reuben. He had been heavily struck on the back of the neck, dislocating the fourth and fifth cervical vertebrae and heavily bruising the spinal cord. It seemed highly unlikely that he could ever recover, either mentally or physically, and in any case there appeared to me to be no object in indefinitely prolonging so unprofitable an existence. He had obviously been able to support life until recently, as he was fairly well nourished, but the state of his feet and clothing showed that he was unemployed, and under present conditions he was likely to remain so. I decided that he would suit my purpose very well, and immediately put in train certain transactions in the City which I had already sketched out in my own mind. In the meantime, the reactions mentioned by the workhouse doctor were interesting, and I made careful studies of them, and arranged for the delivery of the body to the hospital when I should have completed my preparations.

On the Thursday and Friday of that week I made private arrangements with various brokers to buy the stock of certain Peruvian Oil-fields, which had gone down almost to waste-paper. This part of my experiment did not cost me very much, but I contrived to arouse considerable curiosity, and even a mild excitement. At this point I was of course careful not to let my name appear. The incidence of Saturday and Sunday gave me some anxiety lest my man should after all die before I was ready for him, but by the use of saline injections I contrived to keep him alive and, late on Sunday night, he even manifested disquieting symptoms of at any rate a partial recovery.

On Monday morning the market in Peruvians opened briskly. Rumours had evidently got about that somebody knew something, and this day I was not the only buyer in the market. I bought a couple of hundred more shares in my own name, and left the matter to take care of itself. At lunch time I made my arrangements to run into Levy accidentally at the corner of the Mansion House. He expressed (as I expected) his surprise at seeing me in that part of London. I simulated some embarrassment and suggested that we should lunch together. I dragged him to a place a bit off the usual beat, and there ordered a good wine and drank of it as much as he might suppose sufficient to induce a confidential mood. I asked him how things were going on ’Change. He said, \enquote{Oh, all right,} but appeared a little doubtful, and asked me whether I did anything in that way. I said I had a little flutter occasionally, and that, as a matter of fact, I’d been put on to rather a good thing. I glanced round apprehensively at this point, and shifted my chair nearer to his.

\enquote{I suppose you don’t know anything about Peruvian Oil, do you?} he said.

I started and looked round again, and leaning across to him, said, dropping my voice:

\enquote{Well, I do, as a matter of fact, but I don’t want it to get about. I stand to make a good bit on it.}

\enquote{But I thought the thing was hollow,} he said; \enquote{it hasn’t paid a dividend for umpteen years.}

\enquote{No,} I said, \enquote{it hasn’t, but it’s going to. I’ve got inside information.} He looked a bit unconvinced, and I emptied off my glass, and edged right up to his ear.

\enquote{Look here,} I said, \enquote{I’m not giving this away to everyone, but I don’t mind doing you and Christine a good turn. You know, I’ve always kept a soft place in my heart for her, ever since the old days. You got in ahead of me that time, and now it’s up to me to heap coals of fire on you both.}

I was a little excited by this time, and he thought I was drunk.

\enquote{It’s very kind of you, old man,} he said, \enquote{but I’m a cautious bird, you know, always was. I’d like a bit of proof.}

And he shrugged up his shoulders and looked like a pawnbroker.

\enquote{I’ll give it to you,} I said, \enquote{but it isn’t safe here. Come round to my place tonight after dinner, and I’ll show you the report.}

\enquote{How d’you get hold of it?} said he.

\enquote{I’ll tell you tonight,} said I. \enquote{Come round after dinner\allowbreak---\allowbreak any time after nine, say.}

\enquote{To Harley Street?} he asked, and I saw that he meant coming.

\enquote{No,} I said, \enquote{to Battersea\allowbreak---\allowbreak Prince of Wales Road; I’ve got some work to do at the hospital. And look here,} I said, \enquote{don’t you let on to a soul that you’re coming. I bought a couple of hundred shares today, in my own name, and people are sure to get wind of it. If we’re known to be about together, someone’ll twig something. In fact, it’s anything but safe talking about it in this place.}

\enquote{All right,} he said, \enquote{I won’t say a word to anybody. I’ll turn up about nine o’clock. You’re sure it’s a sound thing?}

\enquote{It can’t go wrong,} I assured him. And I meant it.

We parted after that, and I went round to the workhouse. My man had died at about eleven o’clock. I had seen him just after breakfast, and was not surprised. I completed the usual formalities with the workhouse authorities, and arranged for his delivery at the hospital at about seven o’clock.

In the afternoon, as it was not one of my days to be in Harley Street, I looked up an old friend who lives close to Hyde Park, and found that he was just off to Brighton on some business or other. I had tea with him, and saw him off by the 5.35 from Victoria. On issuing from the barrier it occurred to me to purchase an evening paper, and I thoughtlessly turned my steps to the bookstall. The usual crowds were rushing to catch suburban trains home, and on moving away I found myself involved in a contrary stream of travellers coming up out of the Underground, or bolting from all sides for the 5.45 to Battersea Park and Wandsworth Common. I disengaged myself after some buffeting and went home in a taxi; and it was not till I was safely seated there that I discovered somebody’s gold-rimmed pince-nez involved in the astrakhan collar of my overcoat. The time from 6.15 to seven I spent concocting something to look like a bogus report for Sir Reuben.

At seven I went through to the hospital, and found the workhouse van just delivering my subject at the side door. I had him taken straight up to the theatre, and told the attendant, William Watts, that I intended to work there that night. I told him I would prepare the body myself\allowbreak---\allowbreak the injection of a preservative would have been a most regrettable complication. I sent him about his business, and then went home and had dinner. I told my man that I should be working in the hospital that evening, and that he could go to bed at 10.30 as usual, as I could not tell whether I should be late or not. He is used to my erratic ways. I only keep two servants in the Battersea house\allowbreak---\allowbreak the man-servant and his wife, who cooks for me. The rougher domestic work is done by a charwoman, who sleeps out. The servants’ bedroom is at the top of the house, overlooking Prince of Wales Road.

As soon as I had dined I established myself in the hall with some papers. My man had cleared dinner by a quarter past eight, and I told him to give me the syphon and tantalus; and sent him downstairs. Levy rang the bell at twenty minutes past nine, and I opened the door to him myself. My man appeared at the other end of the hall, but I called to him that it was all right, and he went away. Levy wore an overcoat with evening dress and carried an umbrella. \enquote{Why, how wet you are!} I said. \enquote{How did you come?} \enquote{By ’bus,} he said, \enquote{and the fool of a conductor forgot to put me down at the end of the road. It’s pouring cats and dogs and pitch-dark\allowbreak---\allowbreak I couldn’t see where I was.} I was glad he hadn’t taken a taxi, but I had rather reckoned on his not doing so. \enquote{Your little economies will be the death of you one of these days,} I said. I was right there, but I hadn’t reckoned on their being the death of me as well. I say again, I could not have foreseen it.

I sat him down by the fire, and gave him a whisky. He was in high spirits about some deal in Argentines he was bringing off the next day. We talked money for about a quarter of an hour and then he said:

\enquote{Well, how about this Peruvian mare’s-nest of yours?}

\enquote{It’s no mare’s-nest,} I said; \enquote{come and have a look at it.}

I took him upstairs into the library, and switched on the centre light and the reading lamp on the writing table. I gave him a chair at the table with his back to the fire, and fetched the papers I had been faking, out of the safe. He took them, and began to read them, poking over them in his short-sighted way, while I mended the fire. As soon as I saw his head in a favourable position I struck him heavily with the poker, just over the fourth cervical. It was delicate work calculating the exact force necessary to kill him without breaking the skin, but my professional experience was useful to me. He gave one loud gasp, and tumbled forward on to the table quite noiselessly. I put the poker back, and examined him. His neck was broken, and he was quite dead. I carried him into my bedroom and undressed him. It was about ten minutes to ten when I had finished. I put him away under my bed, which had been turned down for the night, and cleared up the papers in the library. Then I went downstairs, took Levy’s umbrella, and let myself out at the hall door, shouting \enquote{Good-night} loudly enough to be heard in the basement if the servants should be listening. I walked briskly away down the street, went in by the hospital side door, and returned to the house noiselessly by way of the private passage. It would have been awkward if anybody had seen me then, but I leaned over the back stairs and heard the cook and her husband still talking in the kitchen. I slipped back into the hall, replaced the umbrella in the stand, cleared up my papers there, went up into the library and rang the bell. When the man appeared I told him to lock up everything except the private door to the hospital. I waited in the library until he had done so, and about 10.30 I heard both servants go up to bed. I waited a quarter of an hour longer and then went through to the dissecting-room. I wheeled one of the stretcher tables through the passage to the house door, and then went to fetch Levy. It was a nuisance having to get him downstairs, but I had not liked to make away with him in any of the ground-floor rooms, in case my servant should take a fancy to poke his head in during the few minutes that I was out of the house, or while locking up. Besides, that was a flea-bite to what I should have to do later. I put Levy on the table, wheeled him across to the hospital and substituted him for my interesting pauper. I was sorry to have to abandon the idea of getting a look at the latter’s brain, but I could not afford to incur suspicion. It was still rather early, so I knocked down a few minutes getting Levy ready for dissection. Then I put my pauper on the table and trundled him over to the house. It was now five past eleven, and I thought I might conclude that the servants were in bed. I carried the body into my bedroom. He was rather heavy, but less so than Levy, and my Alpine experience had taught me how to handle bodies. It is as much a matter of knack as of strength, and I am, in any case, a powerful man for my height. I put the body into the bed\allowbreak---\allowbreak not that I expected anyone to look in during my absence, but if they should they might just as well see me apparently asleep in bed. I drew the clothes a little over his head, stripped, and put on Levy’s clothes, which were fortunately a little big for me everywhere, not forgetting to take his spectacles, watch and other oddments. At a little before half-past eleven I was in the road looking for a cab. People were just beginning to come home from the theatre, and I easily secured one at the corner of Prince of Wales Road. I told the man to drive me to Hyde Park Corner. There I got out, tipped him well, and asked him to pick me up again at the same place in an hour’s time. He assented with an understanding grin, and I walked on up Park Lane. I had my own clothes with me in a suitcase, and carried my own overcoat and Levy’s umbrella. When I got to No. 9A there were lights in some of the top windows. I was very nearly too early, owing to the old man’s having sent the servants to the theatre. I waited about for a few minutes, and heard it strike the quarter past midnight. The lights were extinguished shortly after, and I let myself in with Levy’s key.

It had been my original intention, when I thought over this plan of murder, to let Levy disappear from the study or the dining-room, leaving only a heap of clothes on the hearth-rug. The accident of my having been able to secure Lady Levy’s absence from London, however, made possible a solution more misleading, though less pleasantly fantastic. I turned on the hall light, hung up Levy’s wet overcoat and placed his umbrella in the stand. I walked up noisily and heavily to the bedroom and turned off the light by the duplicate switch on the landing. I knew the house well enough, of course. There was no chance of my running into the man-servant. Old Levy was a simple old man, who liked doing things for himself. He gave his valet little work, and never required any attendance at night. In the bedroom I took off Levy’s gloves and put on a surgical pair, so as to leave no tell-tale finger-prints. As I wished to convey the impression that Levy had gone to bed in the usual way, I simply went to bed. The surest and simplest method of making a thing appear to have been done is to do it. A bed that has been rumpled about with one’s hands, for instance, never looks like a bed that has been slept in. I dared not use Levy’s brush, of course, as my hair is not of his colour, but I did everything else. I supposed that a thoughtful old man like Levy would put his boots handy for his valet, and I ought to have deduced that he would fold up his clothes. That was a mistake, but not an important one. Remembering that well-thought-out little work of Mr Bentley’s, I had examined Levy’s mouth for false teeth, but he had none. I did not forget, however, to wet his tooth-brush.

At one o’clock I got up and dressed in my own clothes by the light of my own pocket torch. I dared not turn on the bedroom lights, as there were light blinds to the windows. I put on my own boots and an old pair of goloshes outside the door. There was a thick Turkey carpet on the stairs and hall-floor, and I was not afraid of leaving marks. I hesitated whether to chance the banging of the front door, but decided it would be safer to take the latchkey. (It is now in the Thames. I dropped it over Battersea Bridge the next day.) I slipped quietly down, and listened for a few minutes with my ear to the letter-box. I heard a constable tramp past. As soon as his steps had died away in the distance I stepped out and pulled the door gingerly to. It closed almost soundlessly, and I walked away to pick up my cab. I had an overcoat of much the same pattern as Levy’s, and had taken the precaution to pack an opera hat in my suitcase. I hoped the man would not notice that I had no umbrella this time. Fortunately the rain had diminished for the moment to a sort of drizzle, and if he noticed anything he made no observation. I told him to stop at 50 Overstrand Mansions, and I paid him off there, and stood under the porch till he had driven away. Then I hurried round to my own side door and let myself in. It was about a quarter to two, and the harder part of my task still lay before me.

My first step was so to alter the appearance of my subject as to eliminate any immediate suggestion either of Levy or of the workhouse vagrant. A fairly superficial alteration was all I considered necessary, since there was not likely to be any hue-and-cry after the pauper. He was fairly accounted for, and his deputy was at hand to represent him. Nor, if Levy was after all traced to my house, would it be difficult to show that the body in evidence was, as a matter of fact, not his. A clean shave and a little hair-oiling and manicuring seemed sufficient to suggest a distinct personality for my silent accomplice. His hands had been well washed in hospital, and though calloused, were not grimy. I was not able to do the work as thoroughly as I should have liked, because time was getting on. I was not sure how long it would take me to dispose of him, and moreover, I feared the onset of \textit{rigor mortis}, which would make my task more difficult. When I had him barbered to my satisfaction, I fetched a strong sheet and a couple of wide roller bandages, and fastened him up carefully, padding him with cotton wool wherever the bandages might chafe or leave a bruise.

Now came the really ticklish part of the business. I had already decided in my own mind that the only way of conveying him from the house was by the roof. To go through the garden at the back in this soft wet weather was to leave a ruinous trail behind us. To carry a dead man down a suburban street in the middle of the night seemed outside the range of practical politics. On the roof, on the other hand, the rain, which would have betrayed me on the ground, would stand my friend.

To reach the roof, it was necessary to carry my burden to the top of the house, past my servants’ room, and hoist him out through the trap-door in the box-room roof. Had it merely been a question of going quietly up there myself, I should have had no fear of waking the servants, but to do so burdened by a heavy body was more difficult. It would be possible, provided that the man and his wife were soundly asleep, but if not, the lumbering tread on the narrow stair and the noise of opening the trap-door would be only too plainly audible. I tiptoed delicately up the stair and listened at their door. To my disgust I heard the man give a grunt and mutter something as he moved in his bed.

I looked at my watch. My preparations had taken nearly an hour, first and last, and I dared not be too late on the roof. I determined to take a bold step and, as it were, bluff out an alibi. I went without precaution against noise into the bathroom, turned on the hot and cold water taps to the full and pulled out the plug.

My household has often had occasion to complain of my habit of using the bath at irregular night hours. Not only does the rush of water into the cistern disturb any sleepers on the Prince of Wales Road side of the house, but my cistern is afflicted with peculiarly loud gurglings and thumpings, while frequently the pipes emit a loud groaning sound. To my delight, on this particular occasion, the cistern was in excellent form, honking, whistling and booming like a railway terminus. I gave the noise five minutes’ start, and when I calculated that the sleepers would have finished cursing me and put their heads under the clothes to shut out the din, I reduced the flow of water to a small stream and left the bathroom, taking good care to leave the light burning and lock the door after me. Then I picked up my pauper and carried him upstairs as lightly as possible.

The box-room is a small attic on the side of the landing opposite to the servants’ bedroom and the cistern-room. It has a trap-door, reached by a short, wooden ladder. I set this up, hoisted up my pauper and climbed up after him. The water was still racing into the cistern, which was making a noise as though it were trying to digest an iron chain, and with the reduced flow in the bathroom the groaning of the pipes had risen almost to a hoot. I was not afraid of anybody hearing other noises. I pulled the ladder through on to the roof after me.

Between my house and the last house in Queen Caroline Mansions there is a space of only a few feet. Indeed, when the Mansions were put up, I believe there was some trouble about ancient lights, but I suppose the parties compromised somehow. Anyhow, my seven-foot ladder reached well across. I tied the body firmly to the ladder, and pushed it over till the far end was resting on the parapet of the opposite house. Then I took a short run across the cistern-room and the box-room roof, and landed easily on the other side, the parapet being happily both low and narrow.

The rest was simple. I carried my pauper along the flat roofs, intending to leave him, like the hunchback in the story, on someone’s staircase or down a chimney. I had got about half-way along when I suddenly thought, \enquote{Why, this must be about little Thipps’s place,} and I remembered his silly face, and his silly chatter about vivisection. It occurred to me pleasantly how delightful it would be to deposit my parcel with him and see what he made of it. I lay down and peered over the parapet at the back. It was pitch-dark and pouring with rain again by this time, and I risked using my torch. That was the only incautious thing I did, and the odds against being seen from the houses opposite were long enough. One second’s flash showed me what I had hardly dared to hope\allowbreak---\allowbreak an open window just below me.

I knew those flats well enough to be sure it was either the bathroom or the kitchen. I made a noose in a third bandage that I had brought with me, and made it fast under the arms of the corpse. I twisted it into a double rope, and secured the end to the iron stanchion of a chimney-stack. Then I dangled our friend over. I went down after him myself with the aid of a drain-pipe and was soon hauling him in by Thipps’s bathroom window.

By that time I had got a little conceited with myself, and spared a few minutes to lay him out prettily and make him shipshape. A sudden inspiration suggested that I should give him the pair of pince-nez which I had happened to pick up at Victoria. I came across them in my pocket while I was looking for a penknife to loosen a knot, and I saw what distinction they would lend his appearance, besides making it more misleading. I fixed them on him, effaced all traces of my presence as far as possible, and departed as I had come, going easily up between the drain-pipe and the rope.

I walked quietly back, re-crossed my crevasse and carried in my ladder and sheet. My discreet accomplice greeted me with a reassuring gurgle and thump. I didn’t make a sound on the stairs. Seeing that I had now been having a bath for about three-quarters of an hour, I turned the water off, and enabled my deserving domestics to get a little sleep. I also felt it was time I had a little myself.

First, however, I had to go over to the hospital and make all safe there. I took off Levy’s head, and started to open up the face. In twenty minutes his own wife could not have recognised him. I returned, leaving my wet goloshes and mackintosh by the garden door. My trousers I dried by the gas stove in my bedroom, and brushed away all traces of mud and brickdust. My pauper’s beard I burned in the library.

I got a good two hours’ sleep from five to seven, when my man called me as usual. I apologized for having kept the water running so long and so late, and added that I thought I would have the cistern seen to.

I was interested to note that I was rather extra hungry at breakfast, showing that my night’s work had caused a certain wear-and-tear of tissue. I went over afterwards to continue my dissection. During the morning a peculiarly thick-headed police inspector came to inquire whether a body had escaped from the hospital. I had him brought to me where I was, and had the pleasure of showing him the work I was doing on Sir Reuben Levy’s head. Afterwards I went round with him to Thipps’s and was able to satisfy myself that my pauper looked very convincing.

As soon as the Stock Exchange opened I telephoned my various brokers, and by exercising a little care, was able to sell out the greater part of my Peruvian stock on a rising market. Towards the end of the day, however, buyers became rather unsettled as a result of Levy’s death, and in the end I did not make more than a few hundreds by the transaction.

Trusting I have now made clear to you any point which you may have found obscure, and with congratulations on the good fortune and perspicacity which have enabled you to defeat me, I remain, with kind remembrances to your mother,

\begin{flushright}
Yours very truly,\\
\textsc{Julian Freke}
\end{flushright}

\textit{Post-Scriptum}: My will is made, leaving my money to St Luke’s Hospital, and bequeathing my body to the same institution for dissection. I feel sure that my brain will be of interest to the scientific world. As I shall die by my own hand, I imagine that there may be a little difficulty about this. Will you do me the favour, if you can, of seeing the persons concerned in the inquest, and obtaining that the brain is not damaged by an unskilful practitioner at the post-mortem, and that the body is disposed of according to my wish?

By the way, it may be of interest to you to know that I appreciated your motive in calling this afternoon. It conveyed a warning, and I am acting upon it in spite of the disastrous consequences to myself. I was pleased to realize that you had not underestimated my nerve and intelligence, and refused the injection. Had you submitted to it, you would, of course, never have reached home alive. No trace would have been left in your body of the injection, which consisted of a harmless preparation of strychnine, mixed with an almost unknown poison, for which there is at present no recognised test, a concentrated solution of sn---

At this point the manuscript broke off.

\enquote{Well, that’s all clear enough,} said Parker.

\enquote{Isn’t it queer?} said Lord Peter. \enquote{All that coolness, all those brains\allowbreak---\allowbreak and then he couldn’t resist writing a confession to show how clever he was, even to keep his head out of the noose.}

\enquote{And a very good thing for us,} said Inspector Sugg, \enquote{but Lord bless you, sir, these criminals are all alike.}

\enquote{Freke’s epitaph,} said Parker, when the Inspector had departed. \enquote{What next, Peter?}

\enquote{I shall now give a dinner party,} said Lord Peter, \enquote{to Mr John P. Milligan and his secretary and to Messrs. Crimplesham and Wicks. I feel they deserve it for not having murdered Levy.}

\enquote{Well, don’t forget the Thippses,} said Mr Parker.

\enquote{On no account,} said Lord Peter, \enquote{would I deprive myself of the pleasure of Mrs Thipps’s company. Bunter!}

\enquote{My lord?}

\enquote{The Napoleon brandy.}

