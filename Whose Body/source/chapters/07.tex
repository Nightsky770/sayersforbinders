%!TeX root=../bodytop.tex
\chapter[Chapter \thechapter]{}
\lettrine[lines=4]{O}{n} returning to the flat just before lunch-time on the following morning, after a few confirmatory researches in Balham and the neighbourhood of Victoria Station, Lord~Peter was greeted at the door by Mr~Bunter (who had gone straight home from Waterloo) with a telephone message and a severe and nursemaid-like eye.

»Lady~Swaffham rang up, my lord, and said she hoped your lordship had not forgotten you were lunching with her.«

»I have forgotten, Bunter, and I mean to forget. I trust you told her I had succumbed to lethargic encephalitis suddenly, no flowers by request.«

»Lady~Swaffham said, my lord, she was counting on you. She met the Duchess of Denver yesterday\longdash«

»If my sister-in-law's there I won't go, that's flat,« said Lord~Peter.

»I beg your pardon, my lord, the Dowager Duchess.«

»What's she doing in town?«

»I imagine she came up for the inquest, my lord.«

»Oh, yes—we missed that, Bunter.«

»Yes, my lord. Her Grace is lunching with Lady~Swaffham.«

»Bunter, I can't. I can't, really. Say I'm in bed with whooping cough, and ask my mother to come round after lunch.«

»Very well, my lord. Mrs~Tommy Frayle will be at Lady~Swaffham's, my lord, and Mr~Milligan\longdash«

»Mr~who?«

»Mr~John P. Milligan, my lord, and\longdash«

»Good God, Bunter, why didn't you say so before? Have I time to get there before he does? All right. I'm off. With a taxi I can just\longdash«

»Not in those trousers, my lord,« said Mr~Bunter, blocking the way to the door with deferential firmness.

»Oh, Bunter,« pleaded his lordship, »do let me—just this once. You don't know how important it is.«

»Not on any account, my lord. It would be as much as my place is worth.«

»The trousers are all right, Bunter.«

»Not for Lady~Swaffham's, my lord. Besides, your lordship forgets the man that ran against you with a milk-can at Salisbury.«

And Mr~Bunter laid an accusing finger on a slight stain of grease showing across the light cloth.

»I wish to God I'd never let you grow into a privileged family retainer, Bunter,« said Lord~Peter, bitterly, dashing his walking-stick into the umbrella-stand. »You've no conception of the mistakes my mother may be making.«

Mr~Bunter smiled grimly and led his victim away.

When an immaculate Lord~Peter was ushered, rather late for lunch, into Lady~Swaffham's drawing-room, the Dowager Duchess of Denver was seated on a sofa, plunged in intimate conversation with Mr~John P. Milligan of Chicago.

»I'm vurry pleased to meet you, Duchess,« had been that financier's opening remark, »to thank you for your exceedingly kind invitation. I assure you it's a compliment I deeply appreciate.«

The Duchess beamed at him, while conducting a rapid rally of all her intellectual forces.

»Do come and sit down and talk to me, Mr~Milligan,« she said. »I do so love talking to you great business men—let me see, is it a railway king you are or something about puss-in-the-corner—at least, I don't mean that exactly, but that game one used to play with cards, all about wheat and oats, and there was a bull and a bear, too—or was it a horse?---no, a bear, because I remember one always had to try and get rid of it and it used to get so dreadfully crumpled and torn, poor thing, always being handed about, one got to recognise it, and then one had to buy a new pack—so foolish it must seem to you, knowing the real thing, and dreadfully noisy, but really excellent for breaking the ice with rather stiff people who didn't know each other—I'm quite sorry it's gone out.«

Mr~Milligan sat down.

»Wal, now,« he said, »I guess it's as interesting for us business men to meet British aristocrats as it is for Britishers to meet American railway kings, Duchess. And I guess I'll make as many mistakes talking your kind of talk as you would make if you were tryin' to run a corner in wheat in Chicago. Fancy now, I called that fine lad of yours Lord~Wimsey the other day, and he thought I'd mistaken him for his brother. That made me feel rather green.«

This was an unhoped-for lead. The Duchess walked warily.

»Dear boy,« she said, »I am so glad you met him, Mr~Milligan. \textit{Both} my sons are a \textit{great} comfort to me, you know, though, of course, Gerald is more conventional—just the right kind of person for the House of Lords, you know, and a splendid farmer. I can't see Peter down at Denver half so well, though he is always going to all the right things in town, and very amusing sometimes, poor boy.«

»I was vurry much gratified by Lord~Peter's suggestion,« pursued Mr~Milligan, »for which I understand you are responsible, and I'll surely be very pleased to come any day you like, though I think you're flattering me too much.«

»Ah, well,« said the Duchess, »I don't know if you're the best judge of that, Mr~Milligan. Not that I know anything about business myself,« she added. »I'm rather old-fashioned for these days, you know, and I can't pretend to do more than know a nice \textit{man} when I see him; for the other things I rely on my son.«

The accent of this speech was so flattering that Mr~Milligan purred almost audibly, and said:

»Wal, Duchess, I guess that's where a lady with a real, beautiful, old-fashioned soul has the advantage of these modern young blatherskites—there aren't many men who wouldn't be nice—to her, and even then, if they aren't rock-bottom she can see through them.«

»But that leaves me where I was,« thought the Duchess. »I believe,« she said aloud, »that I ought to be thanking you in the name of the vicar of Duke's Denver for a very munificent cheque which reached him yesterday for the Church Restoration Fund. He was so delighted and astonished, poor dear man.«

»Oh, that's nothing,« said Mr~Milligan, »we haven't any fine old crusted buildings like yours over on our side, so it's a privilege to be allowed to drop a little kerosene into the worm-holes when we hear of one in the old country suffering from senile decay. So when your lad told me about Duke's Denver I took the liberty to subscribe without waiting for the Bazaar.«

»I'm sure it was very kind of you,« said the Duchess. »You are coming to the Bazaar, then?« she continued, gazing into his face appealingly.

»Sure thing,« said Mr~Milligan, with great promptness. »Lord~Peter said you'd let me know for sure about the date, but we can always make time for a little bit of good work anyway. Of course I'm hoping to be able to avail myself of your kind invitation to stop, but if I'm rushed, I'll manage anyhow to pop over and speak my piece and pop back again.«

»I hope so very much,« said the Duchess. »I must see what can be done about the date—of course, I can't promise\longdash«

»No, no,« said Mr~Milligan heartily. »I know what these things are to fix up. And then there's not only me—there's all the real big men of European eminence your son mentioned, to be consulted.«

The Duchess turned pale at the thought that any one of these illustrious persons might some time turn up in somebody's drawing-room, but by this time she had dug herself in comfortably, and was even beginning to find her range.

»I can't say how grateful we are to you,« she said; »it will be such a treat. Do tell me what you think of saying.«

»Wal\longdash« began Mr~Milligan.

Suddenly everybody was standing up and a penitent voice was heard to say:

»Really, most awfully sorry, y'know—hope you'll forgive me, Lady~Swaffham, what? Dear lady, could I possibly forget an invitation from you? Fact is, I had to go an' see a man down in Salisbury—absolutely true, 'pon my word, and the fellow wouldn't let me get away. I'm simply grovellin' before you, Lady~Swaffham. Shall I go an' eat my lunch in the corner?«

Lady~Swaffham gracefully forgave the culprit.

»Your dear mother is here,« she said.

»How do, Mother?« said Lord~Peter, uneasily.

»How are you, dear?« replied the Duchess. »You really oughtn't to have turned up just yet. Mr~Milligan was just going to tell me what a thrilling speech he's preparing for the Bazaar, when you came and interrupted us.«

Conversation at lunch turned, not unnaturally, on the Battersea inquest, the Duchess giving a vivid impersonation of Mrs~Thipps being interrogated by the Coroner.

»»Did you hear anything unusual in the night?« says the little man, leaning forward and screaming at her, and so crimson in the face and his ears sticking out so—just like a cherubim in that poem of Tennyson's—or is a cherub blue?---perhaps it's a seraphim I mean—anyway, you know what I mean, all eyes, with little wings on its head. And dear old Mrs~Thipps saying, »Of course I have, any time these eighty years,« and \textit{such} a sensation in court till they found out she thought he'd said, »Do you sleep without a light?« and everybody laughing, and then the Coroner said quite loudly, »Damn the woman,« and she heard that, I can't think why, and said: »Don't you get swearing, young man, sitting there in the presence of Providence, as you may say. I don't know what young people are coming to nowadays«---and he's sixty if he's a day, you know,« said the Duchess.

By a natural transition, Mrs~Tommy Frayle referred to the man who was hanged for murdering three brides in a bath.

»I always thought that was so ingenious,« she said, gazing soulfully at Lord~Peter, »and do you know, as it happened, Tommy had just made me insure my life, and I got so frightened, I gave up my morning bath and took to having it in the afternoon when he was in the House—I mean, when he was \textit{not} in the house—not at home, I mean.«

»Dear lady,« said Lord~Peter, reproachfully, »I have a distinct recollection that all those brides were thoroughly unattractive. But it was an uncommonly ingenious plan—the first time of askin'---only he shouldn't have repeated himself.«

»One demands a little originality in these days, even from murderers,« said Lady~Swaffham. »Like dramatists, you know—so much easier in Shakespeare's time, wasn't it? Always the same girl dressed up as a man, and even that borrowed from Boccaccio or Dante or somebody. I'm sure if I'd been a Shakespeare hero, the very minute I saw a slim-legged young page-boy I'd have said: »Odsbodikins! There's that girl again!««

»That's just what happened, as a matter of fact,« said Lord~Peter. »You see, Lady~Swaffham, if ever you want to commit a murder, the thing you've got to do is to prevent people from associatin' their ideas. Most people don't associate anythin'---their ideas just roll about like so many dry peas on a tray, makin' a lot of noise and goin' nowhere, but once you begin lettin' 'em string their peas into a necklace, it's goin' to be strong enough to hang you, what?«

»Dear me!« said Mrs~Tommy Frayle, with a little scream, »what a blessing it is none of my friends have any ideas at all!«

»Y'see,« said Lord~Peter, balancing a piece of duck on his fork and frowning, »it's only in Sherlock Holmes and stories like that, that people think things out logically. Or'nar'ly, if somebody tells you somethin' out of the way, you just say, »By Jove!« or »How sad!« an' leave it at that, an' half the time you forget about it, 'nless somethin' turns up afterwards to drive it home. F'r instance, Lady~Swaffham, I told you when I came in that I'd been down to Salisbury, 'n' that's true, only I don't suppose it impressed you much; 'n' I don't suppose it'd impress you much if you read in the paper tomorrow of a tragic discovery of a dead lawyer down in Salisbury, but if I went to Salisbury again next week 'n' there was a Salisbury doctor found dead the day after, you might begin to think I was a bird of ill omen for Salisbury residents; and if I went there again the week after, 'n' you heard next day that the see of Salisbury had fallen vacant suddenly, you might begin to wonder what took me to Salisbury, an' why I'd never mentioned before that I had friends down there, don't you see, an' you might think of goin' down to Salisbury yourself, an' askin' all kinds of people if they'd happened to see a young man in plum-coloured socks hangin' round the Bishop's Palace.«

»I daresay I should,« said Lady~Swaffham.

»Quite. An' if you found that the lawyer and the doctor had once upon a time been in business at Poggleton-on-the-Marsh when the Bishop had been vicar there, you'd begin to remember you'd once heard of me payin' a visit to Poggleton-on-the-Marsh a long time ago, an' you'd begin to look up the parish registers there an' discover I'd been married under an assumed name by the vicar to the widow of a wealthy farmer, who'd died suddenly of peritonitis, as certified by the doctor, after the lawyer'd made a will leavin' me all her money, and \textit{then} you'd begin to think I might have very good reasons for gettin' rid of such promisin' blackmailers as the lawyer, the doctor an' the bishop. Only, if I hadn't started an association in your mind by gettin' rid of 'em all in the same place, you'd never have thought of goin' to Poggleton-on-the-Marsh, 'n' you wouldn't even have remembered I'd ever been there.«

»\textit{Were} you ever there, Lord~Peter?« inquired Mrs~Tommy, anxiously.

»I don't think so,« said Lord~Peter; »the name threads no beads in my mind. But it might, any day, you know.«

»But if you were investigating a crime,« said Lady~Swaffham, »you'd have to begin by the usual things, I suppose—finding out what the person had been doing, and who'd been to call, and looking for a motive, wouldn't you?«

»Oh, yes,« said Lord~Peter, »but most of us have such dozens of motives for murderin' all sorts of inoffensive people. There's lots of people I'd like to murder, wouldn't you?«

»Heaps,« said Lady~Swaffham. »There's that dreadful—perhaps I'd better not say it, though, for fear you should remember it later on.«

»Well, I wouldn't if I were you,« said Peter, amiably. »You never know. It'd be beastly awkward if the person died suddenly tomorrow.«

»The difficulty with this Battersea case, I guess,« said Mr~Milligan, »is that nobody seems to have any associations with the gentleman in the bath.«

»So hard on poor Inspector~Sugg,« said the Duchess. »I quite felt for the man, having to stand up there and answer a lot of questions when he had nothing at all to say.«

Lord~Peter applied himself to the duck, having got a little behindhand. Presently he heard somebody ask the Duchess if she had seen Lady~Levy.

»She is in great distress,« said the woman who had spoken, a Mrs~Freemantle, »though she clings to the hope that he will turn up. I suppose you knew him, Mr~Milligan—know him, I should say, for I hope he's still alive somewhere.«

Mrs~Freemantle was the wife of an eminent railway director, and celebrated for her ignorance of the world of finance. Her \textit{faux pas} in this connection enlivened the tea parties of City men's wives.

»Wal, I've dined with him,« said Mr~Milligan, good-naturedly. »I think he and I've done our best to ruin each other, Mrs~Freemantle. If this were the States,« he added, »I'd be much inclined to suspect myself of having put Sir Reuben in a safe place. But we can't do business that way in your old country; no, ma'am.«

»It must be exciting work doing business in America,« said Lord~Peter.

»It is,« said Mr~Milligan. »I guess my brothers are having a good time there now. I'll be joining them again before long, as soon as I've fixed up a little bit of work for them on this side.«

»Well, you mustn't go till after my bazaar,« said the Duchess.

Lord~Peter spent the afternoon in a vain hunt for Mr~Parker. He ran him down eventually after dinner in Great Ormond Street.

Parker was sitting in an elderly but affectionate armchair, with his feet on the mantelpiece, relaxing his mind with a modern commentary on the Epistle to the Galatians. He received Lord~Peter with quiet pleasure, though without rapturous enthusiasm, and mixed him a whisky-and-soda. Peter took up the book his friend had laid down and glanced over the pages.

»All these men work with a bias in their minds, one way or other,« he said; »they find what they are looking for.«

»Oh, they do,« agreed the detective; »but one learns to discount that almost automatically, you know. When I was at college, I was all on the other side—Conybeare and Robertson and Drews and those people, you know, till I found they were all so busy looking for a burglar whom nobody had ever seen, that they couldn't recognise the footprints of the household, so to speak. Then I spent two years learning to be cautious.«

»Hum,« said Lord~Peter, »theology must be good exercise for the brain then, for you're easily the most cautious devil I know. But I say, do go on reading—it's a shame for me to come and root you up in your off-time like this.«

»It's all right, old man,« said Parker.

The two men sat silent for a little, and then Lord~Peter said:

»D'you like your job?«

The detective considered the question, and replied:

»Yes—yes, I do. I know it to be useful, and I am fitted to it. I do it quite well—not with inspiration, perhaps, but sufficiently well to take a pride in it. It is full of variety and it forces one to keep up to the mark and not get slack. And there's a future to it. Yes, I like it. Why?«

»Oh, nothing,« said Peter. »It's a hobby to me, you see. I took it up when the bottom of things was rather knocked out for me, because it was so damned exciting, and the worst of it is, I enjoy it—up to a point. If it was all on paper I'd enjoy every bit of it. I love the beginning of a job—when one doesn't know any of the people and it's just exciting and amusing. But if it comes to really running down a live person and getting him hanged, or even quodded, poor devil, there don't seem as if there was any excuse for me buttin' in, since I don't have to make my livin' by it. And I feel as if I oughtn't ever to find it amusin'. But I do.«

Parker gave this speech his careful attention.

»I see what you mean,« he said.

»There's old Milligan, f'r instance,« said Lord~Peter. »On paper, nothin' would be funnier than to catch old Milligan out. But he's rather a decent old bird to talk to. Mother likes him. He's taken a fancy to me. It's awfully entertainin' goin' and pumpin' him with stuff about a bazaar for church expenses, but when he's so jolly pleased about it and that, I feel a worm. S'pose old Milligan has cut Levy's throat and plugged him into the Thames. It ain't my business.«

»It's as much yours as anybody's,« said Parker; »it's no better to do it for money than to do it for nothing.«

»Yes, it is,« said Peter stubbornly. »Havin' to live is the only excuse there is for doin' that kind of thing.«

»Well, but look here!« said Parker. »If Milligan has cut poor old Levy's throat for no reason except to make himself richer, I don't see why he should buy himself off by giving \textsterling 1,000 to Duke's Denver church roof, or why he should be forgiven just because he's childishly vain, or childishly snobbish.«

»That's a nasty one,« said Lord~Peter.

»Well, if you like, even because he has taken a fancy to you.«

»No, but\longdash«

»Look here, Wimsey—do you think he \textit{has} murdered Levy?«

»Well, he may have.«

»But do you think he has?«

»I don't want to think so.«

»Because he has taken a fancy to you?«

»Well, that biases me, of course\longdash«

»I daresay it's quite a legitimate bias. You don't think a callous murderer would be likely to take a fancy to you?«

»Well—besides, I've taken rather a fancy to him.«

»I daresay that's quite legitimate, too. You've observed him and made a subconscious deduction from your observations, and the result is, you don't think he did it. Well, why not? You're entitled to take that into account.«

»But perhaps I'm wrong and he did do it.«

»Then why let your vainglorious conceit in your own power of estimating character stand in the way of unmasking the singularly cold-blooded murder of an innocent and lovable man?«

»I know—but I don't feel I'm playing the game somehow.«

»Look here, Peter,« said the other with some earnestness, »suppose you get this playing-fields-of-Eton complex out of your system once and for all. There doesn't seem to be much doubt that something unpleasant has happened to Sir Reuben Levy. Call it murder, to strengthen the argument. If Sir Reuben has been murdered, is it a game? and is it fair to treat it as a game?«

»That's what I'm ashamed of, really,« said Lord~Peter. »It \textit{is} a game to me, to begin with, and I go on cheerfully, and then I suddenly see that somebody is going to be hurt, and I want to get out of it.«

»Yes, yes, I know,« said the detective, »but that's because you're thinking about your attitude. You want to be consistent, you want to look pretty, you want to swagger debonairly through a comedy of puppets or else to stalk magnificently through a tragedy of human sorrows and things. But that's childish. If you've any duty to society in the way of finding out the truth about murders, you must do it in any attitude that comes handy. You want to be elegant and detached? That's all right, if you find the truth out that way, but it hasn't any value in itself, you know. You want to look dignified and consistent—what's that got to do with it? You want to hunt down a murderer for the sport of the thing and then shake hands with him and say, »Well played—hard luck—you shall have your revenge tomorrow!« Well, you can't do it like that. Life's not a football match. You want to be a sportsman. You can't be a sportsman. You're a responsible person.«

»I don't think you ought to read so much theology,« said Lord~Peter. »It has a brutalizing influence.«

He got up and paced about the room, looking idly over the bookshelves. Then he sat down again, filled and lit his pipe, and said:

»Well, I'd better tell you about the ferocious and hardened Crimplesham.«

He detailed his visit to Salisbury. Once assured of his bona fides, Mr~Crimplesham had given him the fullest details of his visit to town.

»And I've substantiated it all,« groaned Lord~Peter, »and unless he's corrupted half Balham, there's no doubt he spent the night there. And the afternoon was really spent with the bank people. And half the residents of Salisbury seem to have seen him off on Monday before lunch. And nobody but his own family or young Wicks seems to have anything to gain by his death. And even if young Wicks wanted to make away with him, it's rather far-fetched to go and murder an unknown man in Thipps's place in order to stick Crimplesham's eyeglasses on his nose.«

»Where was young Wicks on Monday?« asked Parker.

»At a dance given by the Precentor,« said Lord~Peter, wildly. »David—his name is David—dancing before the ark of the Lord~in the face of the whole Cathedral Close.«

There was a pause.

»Tell me about the inquest,« said Wimsey.

Parker obliged with a summary of the evidence.

»Do you believe the body could have been concealed in the flat after all?« he asked. »I know we looked, but I suppose we might have missed something.«

»We might. But Sugg looked as well.«

»Sugg!«

»You do Sugg an injustice,« said Lord~Peter; »if there had been any signs of Thipps's complicity in the crime, Sugg would have found them.«

»Why?«

»Why? Because he was looking for them. He's like your commentators on Galatians. He thinks that either Thipps, or Gladys Horrocks, or Gladys Horrocks's young man did it. Therefore he found marks on the window sill where Gladys Horrocks's young man might have come in or handed something in to Gladys Horrocks. He didn't find any signs on the roof, because he wasn't looking for them.«

»But he went over the roof before me.«

»Yes, but only in order to prove that there were no marks there. He reasons like this: Gladys Horrocks's young man is a glazier. Glaziers come on ladders. Glaziers have ready access to ladders. Therefore Gladys Horrocks's young man had ready access to a ladder. Therefore Gladys Horrocks's young man came on a ladder. Therefore there will be marks on the window sill and none on the roof. Therefore he finds marks on the window sill but none on the roof. He finds no marks on the ground, but he thinks he would have found them if the yard didn't happen to be paved with asphalt. Similarly, he thinks Mr~Thipps may have concealed the body in the box-room or elsewhere. Therefore you may be sure he searched the box-room and all the other places for signs of occupation. If they had been there he would have found them, because he was looking for them. Therefore, if he didn't find them it's because they weren't there.«

»All right,« said Parker, »stop talking. I believe you.«

He went on to detail the medical evidence.

»By the way,« said Lord~Peter, »to skip across for a moment to the other case, has it occurred to you that perhaps Levy was going out to see Freke on Monday night?«

»He was; he did,« said Parker, rather unexpectedly, and proceeded to recount his interview with the nerve-specialist.

»Humph!« said Lord~Peter. »I say, Parker, these are funny cases, ain't they? Every line of inquiry seems to peter out. It's awfully exciting up to a point, you know, and then nothing comes of it. It's like rivers getting lost in the sand.«

»Yes,« said Parker. »And there's another one I lost this morning.«

»What's that?«

»Oh, I was pumping Levy's secretary about his business. I couldn't get much that seemed important except further details about the Argentine and so on. Then I thought I'd just ask round in the City about those Peruvian Oil shares, but Levy hadn't even heard of them so far as I could make out. I routed out the brokers, and found a lot of mystery and concealment, as one always does, you know, when somebody's been rigging the market, and at last I found one name at the back of it. But it wasn't Levy's.«

»No? Whose was it?«

»Oddly enough, Freke's. It seems mysterious. He bought a lot of shares last week, in a secret kind of way, a few of them in his own name, and then quietly sold 'em out on Tuesday at a small profit—a few hundreds, not worth going to all that trouble about, you wouldn't think.«

»Shouldn't have thought he ever went in for that kind of gamble.«

»He doesn't as a rule. That's the funny part of it.«

»Well, you never know,« said Lord~Peter; »people do these things just to prove to themselves or somebody else that they could make a fortune that way if they liked. I've done it myself in a small way.«

He knocked out his pipe and rose to go.

»I say, old man,« he said suddenly, as Parker was letting him out, »does it occur to you that Freke's story doesn't fit in awfully well with what Anderson said about the old boy having been so jolly at dinner on Monday night? Would you be, if you thought you'd got anything of that sort?«

»No, I shouldn't,« said Parker; »but,« he added with his habitual caution, »some men will jest in the dentist's waiting-room. You, for one.«

»Well, that's true,« said Lord~Peter, and went downstairs.
