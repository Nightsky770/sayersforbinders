%!TeX root=../bodytop.tex
\chapter[Chapter \thechapter]{}
\lettrine[lines=4]{L}{ord} Peter reached home about midnight, feeling extraordinarily wakeful and alert. Something was jigging and worrying in his brain; it felt like a hive of bees, stirred up by a stick. He felt as though he were looking at a complicated riddle, of which he had once been told the answer but had forgotten it and was always on the point of remembering.

»Somewhere,« said Lord Peter to himself, »somewhere I've got the key to these two things. I know I've got it, only I can't remember what it is. Somebody said it. Perhaps I said it. I can't remember where, but I know I've got it. Go to bed, Bunter, I shall sit up a little. I'll just slip on a dressing-gown.«

Before the fire he sat down with his pipe in his mouth and his jazz-coloured peacocks gathered about him. He traced out this line and that line of investigation—rivers running into the sand. They ran out from the thought of Levy, last seen at ten o'clock in Prince of Wales Road. They ran back from the picture of the grotesque dead man in Mr Thipps's bathroom—they ran over the roof, and were lost—lost in the sand. Rivers running into the sand—rivers running underground, very far down—

\begin{verse}
Where Alph, the sacred river, ran\\
Through caverns measureless to man\\
Down to a sunless sea.\\
\end{verse}


By leaning his head down, it seemed to Lord Peter that he could hear them, very faintly, lipping and gurgling somewhere in the darkness. But where? He felt quite sure that somebody had told him once, only he had forgotten.

He roused himself, threw a log on the fire, and picked up a book which the indefatigable Bunter, carrying on his daily fatigues amid the excitements of special duty, had brought from the Times Book Club. It happened to be Sir Julian Freke's »Physiological Bases of the Conscience,« which he had seen reviewed two days before.

»This ought to send one to sleep,« said Lord Peter; »if I can't leave these problems to my subconscious I'll be as limp as a rag tomorrow.«

He opened the book slowly, and glanced carelessly through the preface.

»I wonder if that's true about Levy being ill,« he thought, putting the book down; »it doesn't seem likely. And yet—Dash it all, I'll take my mind off it.«

He read on resolutely for a little.

»I don't suppose Mother's kept up with the Levys much,« was the next importunate train of thought. »Dad always hated self-made people and wouldn't have `em at Denver. And old Gerald keeps up the tradition. I wonder if she knew Freke well in those days. She seems to get on with Milligan. I trust Mother's judgment a good deal. She was a brick about that bazaar business. I ought to have warned her. She said something once\longdash«

He pursued an elusive memory for some minutes, till it vanished altogether with a mocking flicker of the tail. He returned to his reading.

Presently another thought crossed his mind aroused by a photograph of some experiment in surgery.

»If the evidence of Freke and that man Watts hadn't been so positive,« he said to himself, »I should be inclined to look into the matter of those shreds of lint on the chimney.«

He considered this, shook his head and read with determination.

Mind and matter were one thing, that was the theme of the physiologist. Matter could erupt, as it were, into ideas. You could carve passions in the brain with a knife. You could get rid of imagination with drugs and cure an outworn convention like a disease. »The knowledge of good and evil is an observed phenomenon, attendant upon a certain condition of the brain-cells, which is removable.« That was one phrase; and again:

»Conscience in man may, in fact, be compared to the sting of a hive-bee, which, so far from conducing to the welfare of its possessor, cannot function, even in a single instance, without occasioning its death. The survival-value in each case is thus purely social; and if humanity ever passes from its present phase of social development into that of a higher individualism, as some of our philosophers have ventured to speculate, we may suppose that this interesting mental phenomenon may gradually cease to appear; just as the nerves and muscles which once controlled the movements of our ears and scalps have, in all save a few backward individuals, become atrophied and of interest only to the physiologist.«

»By Jove!« thought Lord Peter, idly, »that's an ideal doctrine for the criminal. A man who believed that would never\longdash«

And then it happened—the thing he had been half-unconsciously expecting. It happened suddenly, surely, as unmistakably, as sunrise. He remembered—not one thing, nor another thing, nor a logical succession of things, but everything—the whole thing, perfect, complete, in all its dimensions as it were and instantaneously; as if he stood outside the world and saw it suspended in infinitely dimensional space. He no longer needed to reason about it, or even to think about it. He knew it.

There is a game in which one is presented with a jumble of letters and is required to make a word out of them, as thus:

\begin{center}
\textsc{C O S S S S R I}
\end{center}

The slow way of solving the problem is to try out all the permutations and combinations in turn, throwing away impossible conjunctions of letters, as:

\begin{center}
\textsc{S S S I R C}
\end{center}

or

\begin{center}
\textsc{S C S R S O}
\end{center}

Another way is to stare at the inco-ordinate elements until, by no logical process that the conscious mind can detect, or under some adventitious external stimulus, the combination:

\begin{center}
\textsc{S C I S S O R S}
\end{center}

presents itself with calm certainty. After that, one does not even need to arrange the letters in order. The thing is done.

Even so, the scattered elements of two grotesque conundrums, flung higgledy-piggledy into Lord Peter's mind, resolved themselves, unquestioned henceforward. A bump on the roof of the end house—Levy in a welter of cold rain talking to a prostitute in the Battersea Park Road—a single ruddy hair—lint bandages—Inspector Sugg calling the great surgeon from the dissecting-room of the hospital—Lady Levy with a nervous attack—the smell of carbolic soap—the Duchess's voice---»not really an engagement, only a sort of understanding with her father«---shares in Peruvian Oil—the dark skin and curved, fleshy profile of the man in the bath—Dr Grimbold giving evidence, »In my opinion, death did not occur for several days after the blow«---india-rubber gloves—even, faintly, the voice of Mr Appledore, »He called on me, sir, with an anti-vivisectionist pamphlet«---all these things and many others rang together and made one sound, they swung together like bells in a steeple, with the deep tenor booming through the clamour:

»The knowledge of good and evil is a phenomenon of the brain, and is removable, removable, removable. The knowledge of good and evil is removable.«

Lord Peter Wimsey was not a young man who habitually took himself very seriously, but this time he was frankly appalled. »It's impossible,« said his reason, feebly; »\textit{credo quia impossibile,}« said his interior certainty with impervious self-satisfaction. »All right,« said conscience, instantly allying itself with blind faith, »what are you going to do about it?«

Lord Peter got up and paced the room: »Good Lord!« he said. »Good Lord!« He took down »Who's Who« from the little shelf over the telephone and sought comfort in its pages:

\begin{quote}
\textsc{Freke}, Sir Julian, Kt. \textit{cr.} 1916; G.C.V.O. \textit{cr.} 1919; K.C.V.O. 1917; K.C.B. 1918; M.D., F.R.C.P., F.R.C.S., Dr en Méd. Paris; D. Sci. Cantab.; Knight of Grace of the Order of S. John of Jerusalem; Consulting Surgeon of St Luke's Hospital, Battersea. \textit{b.} Gryllingham, 16 March, 1872, \textit{only son} of Edward Curzon Freke, Esq., of Gryll Court, Gryllingham. \textit{Educ.} Harrow and Trinity Coll., Cambridge; Col. A.M.S.; late Member of the Advisory Board of the Army Medical Service. \textit{Publications}: Some Notes on the Pathological Aspects of Genius, 1892; Statistical Contributions to the Study of Infantile Paralysis in England and Wales, 1894; Functional Disturbances of the Nervous System, 1899; Cerebro-Spinal Diseases, 1904; The Borderland of Insanity, 1906; An Examination into the Treatment of Pauper Lunacy in the United Kingdom, 1906; Modern Developments in Psycho-Therapy: A Criticism, 1910; Criminal Lunacy, 1914; The Application of Psycho-Therapy to the Treatment of Shell-Shock, 1917; An Answer to Professor Freud, with a Description of Some Experiments Carried Out at the Base Hospital at Amiens, 1919; Structural Modifications Accompanying the More Important Neuroses, 1920. \textit{Clubs}: White's; Oxford and Cambridge; Alpine, etc. Recreations: Chess, Mountaineering, Fishing. \textit{Address}: 282, Harley Street and St Luke's House, Prince of Wales Road, Battersea Park, S.W.11.
\end{quote}

He flung the book away. »Confirmation!« he groaned. »As if I needed it!«

He sat down again and buried his face in his hands. He remembered quite suddenly how, years ago, he had stood before the breakfast table at Denver Castle—a small, peaky boy in blue knickers, with a thunderously beating heart. The family had not come down; there was a great silver urn with a spirit lamp under it, and an elaborate coffee-pot boiling in a glass dome. He had twitched the corner of the tablecloth—twitched it harder, and the urn moved ponderously forward and all the teaspoons rattled. He seized the tablecloth in a firm grip and pulled his hardest—he could feel now the delicate and awful thrill as the urn and the coffee machine and the whole of a Sèvres breakfast service had crashed down in one stupendous ruin—he remembered the horrified face of the butler, and the screams of a lady guest.

A log broke across and sank into a fluff of white ash. A belated motor-lorry rumbled past the window.

Mr Bunter, sleeping the sleep of the true and faithful servant, was aroused in the small hours by a hoarse whisper, »Bunter!«

»Yes, my lord,« said Bunter, sitting up and switching on the light.

»Put that light out, damn you!« said the voice. »Listen—over there—listen—can't you hear it?«

»It's nothing, my lord,« said Mr Bunter, hastily getting out of bed and catching hold of his master; »it's all right, you get to bed quick and I'll fetch you a drop of bromide. Why, you're all shivering—you've been sitting up too late.«

»Hush! no, no—it's the water,« said Lord Peter with chattering teeth; »it's up to their waists down there, poor devils. But listen! can't you hear it? Tap, tap, tap—they're mining us—but I don't know where—I can't hear—I can't. Listen, you! There it is again—we must find it—we must stop it\textellipsis . Listen! Oh, my God! I can't hear—I can't hear anything for the noise of the guns. Can't they stop the guns?«

»Oh, dear!« said Mr Bunter to himself. »No, no—it's all right, Major—don't you worry.«

»But I hear it,« protested Peter.

»So do I,« said Mr Bunter stoutly; »very good hearing, too, my lord. That's our own sappers at work in the communication trench. Don't you fret about that, sir.«

Lord Peter grasped his wrist with a feverish hand.

»Our own sappers,« he said; »sure of that?«

»Certain of it,« said Mr Bunter, cheerfully.

»They'll bring down the tower,« said Lord Peter.

»To be sure they will,« said Mr Bunter, »and very nice, too. You just come and lay down a bit, sir—they've come to take over this section.«

»You're sure it's safe to leave it?« said Lord Peter.

»Safe as houses, sir,« said Mr Bunter, tucking his master's arm under his and walking him off to his bedroom.

Lord Peter allowed himself to be dosed and put to bed without further resistance. Mr Bunter, looking singularly un-Bunterlike in striped pyjamas, with his stiff black hair ruffled about his head, sat grimly watching the younger man's sharp cheekbones and the purple stains under his eyes.

»Thought we'd had the last of these attacks,« he said. »Been overdoin' of himself. Asleep?« He peered at him anxiously. An affectionate note crept into his voice. »Bloody little fool!« said Sergeant Bunter.