%!TeX root=../bodytop.tex
\chapter[Chapter \thechapter]{}
\lettrine[lines=4]{M}{r} Parker, summoned the next morning to 110 Piccadilly, arrived to find the Dowager Duchess in possession. She greeted him charmingly.

\zz
»I am going to take this silly boy down to Denver for the week-end,« she said, indicating Peter, who was writing and only acknowledged his friend's entrance with a brief nod. »He's been doing too much—running about to Salisbury and places and up till all hours of the night—you really shouldn't encourage him, Mr Parker, it's very naughty of you—waking poor Bunter up in the middle of the night with scares about Germans, as if that wasn't all over years ago, and he hasn't had an attack for ages, but there! Nerves are such funny things, and Peter always did have nightmares when he was quite a little boy—though very often of course it was only a little pill he wanted; but he was so dreadfully bad in 1918, you know, and I suppose we can't expect to forget all about a great war in a year or two, and, really, I ought to be very thankful with both my boys safe. Still, I think a little peace and quiet at Denver won't do him any harm.«

»Sorry you've been having a bad turn, old man,« said Parker, vaguely sympathetic; »you're looking a bit seedy.«

»Charles,« said Lord Peter, in a voice entirely void of expression, »I am going away for a couple of days because I can be no use to you in London. What has got to be done for the moment can be much better done by you than by me. I want you to take this«---he folded up his writing and placed it in an envelope---»to Scotland Yard immediately and get it sent out to all the workhouses, infirmaries, police stations, \textsc{y.m.c.a.}'s and so on in London. It is a description of Thipps's corpse as he was before he was shaved and cleaned up. I want to know whether any man answering to that description has been taken in anywhere, alive or dead, during the last fortnight. You will see Sir Andrew Mackenzie personally, and get the paper sent out at once, by his authority; you will tell him that you have solved the problems of the Levy murder and the Battersea mystery«---Mr Parker made an astonished noise to which his friend paid no attention---»and you will ask him to have men in readiness with a warrant to arrest a very dangerous and important criminal at any moment on your information. When the replies to this paper come in, you will search for any mention of St Luke's Hospital, or of any person connected with St Luke's Hospital, and you will send for me at once.«

»Meanwhile you will scrape acquaintance—I don't care how—with one of the students at St Luke's. Don't march in there blowing about murders and police warrants, or you may find yourself in Queer Street. I shall come up to town as soon as I hear from you, and I shall expect to find a nice ingenuous Sawbones here to meet me.« He grinned faintly.

»D'you mean you've got to the bottom of this thing?« asked Parker.

»Yes. I may be wrong. I hope I am, but I know I'm not.«

»You won't tell me?«

»D'you know,« said Peter, »honestly I'd rather not. I say I \textit{may} be wrong—and I'd feel as if I'd libelled the Archbishop of Canterbury.«

»Well, tell me—is it one mystery or two?«

»One.«

»You talked of the Levy murder. Is Levy dead?«

»God—yes!« said Peter, with a strong shudder.

The Duchess looked up from where she was reading the \textit{Tatler}.

»Peter,« she said, »is that your ague coming on again? Whatever you two are chattering about, you'd better stop it at once if it excites you. Besides, it's about time to be off.«

»All right, Mother,« said Peter. He turned to Bunter, standing respectfully in the door with an overcoat and suitcase. »You understand what you have to do, don't you?« he said.

»Perfectly, thank you, my lord. The car is just arriving, your Grace.«

»With Mrs Thipps inside it,« said the Duchess. »She'll be delighted to see you again, Peter. You remind her so of Mr Thipps. Good-morning, Bunter.«

»Good-morning, your Grace.«

Parker accompanied them downstairs.

When they had gone he looked blankly at the paper in his hand—then, remembering that it was Saturday and there was need for haste, he hailed a taxi.

»Scotland Yard!« he cried.

Tuesday morning saw Lord Peter and a man in a velveteen jacket swishing merrily through seven acres of turnip-tops, streaked yellow with early frosts. A little way ahead, a sinuous undercurrent of excitement among the leaves proclaimed the unseen yet ever-near presence of one of the Duke of Denver's setter pups. Presently a partridge flew up with a noise like a police rattle, and Lord Peter accounted for it very creditably for a man who, a few nights before, had been listening to imaginary German sappers. The setter bounded foolishly through the turnips, and fetched back the dead bird.

»Good dog,« said Lord Peter.

Encouraged by this, the dog gave a sudden ridiculous gambol and barked, its ear tossed inside out over its head.

»Heel,« said the man in velveteen, violently. The animal sidled up, ashamed.

»Fool of a dog, that,« said the man in velveteen; »can't keep quiet. Too nervous, my lord. One of old Black Lass's pups.«

»Dear me,« said Peter, »is the old dog still going?«

»No, my lord; we had to put her away in the spring.«

Peter nodded. He always proclaimed that he hated the country and was thankful to have nothing to do with the family estates, but this morning he enjoyed the crisp air and the wet leaves washing darkly over his polished boots. At Denver things moved in an orderly way; no one died sudden and violent deaths except aged setters—and partridges, to be sure. He sniffed up the autumn smell with appreciation. There was a letter in his pocket which had come by the morning post, but he did not intend to read it just yet. Parker had not wired; there was no hurry.

He read it in the smoking-room after lunch. His brother was there, dozing over the \textit{Times}---a good, clean Englishman, sturdy and conventional, rather like Henry \textsc{viii} in his youth; Gerald, sixteenth Duke of Denver. The Duke considered his cadet rather degenerate, and not quite good form; he disliked his taste for police-court news.

The letter was from Mr Bunter.

\clearpage


\begin{quotation}
\begin{flushright}
110, Piccadilly,\\
W.1.
\end{flushright}

\noindent My Lord:

I write 
\end{quotation}

(Mr Bunter had been carefully educated and knew that nothing is more vulgar than a careful avoidance of beginning a letter with the first person singular) 

\begin{quotation}
as your lordship directed, to inform you of the result of my investigations.

I experienced no difficulty in becoming acquainted with Sir Julian Freke's man-servant. He belongs to the same club as the Hon. Frederick Arbuthnot's man, who is a friend of mine, and was very willing to introduce me. He took me to the club yesterday (Sunday) evening, and we dined with the man, whose name is John Cummings, and afterwards I invited Cummings to drinks and a cigar in the flat. Your lordship will excuse me doing this, knowing that it is not my habit, but it has always been my experience that the best way to gain a man's confidence is to let him suppose that one takes advantage of one's employer.
\end{quotation}

(»I always suspected Bunter of being a student of human nature,« commented Lord Peter.)

\begin{quotation}
I gave him the best old port 
\end{quotation}

(»The deuce you did,« said Lord Peter), 

\begin{quotation}
having heard you and Mr Arbuthnot talk over it.
\end{quotation}

 (»Hum!« said Lord Peter.)

\begin{quotation}
Its effects were quite equal to my expectations as regards the principal matter in hand, but I very much regret to state that the man had so little understanding of what was offered to him that he smoked a cigar with it (one of your lordship's Villar Villars). You will understand that I made no comment on this at the time, but your lordship will sympathize with my feelings. May I take this opportunity of expressing my grateful appreciation of your lordship's excellent taste in food, drink and dress? It is, if I may say so, more than a pleasure—it is an education, to valet and buttle your lordship.
\end{quotation}

Lord Peter bowed his head gravely.

»What on earth are you doing, Peter, sittin' there noddin' an' grinnin' like a what-you-may-call-it?« demanded the Duke, coming suddenly out of a snooze. »Someone writin' pretty things to you, what?«

»Charming things,« said Lord Peter.

The Duke eyed him doubtfully.

»Hope to goodness you don't go and marry a chorus beauty,« he muttered inwardly, and returned to the \textit{Times}.

\begin{quotation}
Over dinner I had set myself to discover Cummings's tastes, and found them to run in the direction of the music-hall stage. During his first glass I drew him out in this direction, your lordship having kindly given me opportunities of seeing every performance in London, and I spoke more freely than I should consider becoming in the ordinary way in order to make myself pleasant to him. I may say that his views on women and the stage were such as I should have expected from a man who would smoke with your lordship's port.

With the second glass I introduced the subject of your lordship's inquiries. In order to save time I will write our conversation in the form of a dialogue, as nearly as possible as it actually took place.

\begin{dialogue}
\speak{Cummings} You seem to get many opportunities of seeing a bit of life, Mr Bunter.

\speak{Bunter} One can always make opportunities if one knows how.

\speak{Cummings} Ah, it's very easy for you to talk, Mr Bunter. You're not married, for one thing.

\speak{Bunter} I know better than that, Mr Cummings.

\speak{Cummings} So do I—\textit{now}, when it's too late. (He sighed heavily, and I filled up his glass.)

\speak{Bunter} Does Mrs Cummings live with you at Battersea?

\speak{Cummings} Yes, her and me we do for my governor. Such a life! Not but what there's a char comes in by the day. But what's a char? I can tell you it's dull all by ourselves in that d---d Battersea suburb.

\speak{Bunter} Not very convenient for the Halls, of course.

\speak{Cummings} I believe you. It's all right for you, here in Piccadilly, right on the spot as you might say. And I daresay your governor's often out all night, eh?

\speak{Bunter} Oh, frequently, Mr Cummings.

\speak{Cummings} And I daresay you take the opportunity to slip off yourself every so often, eh?

\speak{Bunter} Well, what do \textit{you} think, Mr Cummings?

\speak{Cummings} That's it; there you are! But what's a man to do with a nagging fool of a wife and a blasted scientific doctor for a governor, as sits up all night cutting up dead bodies and experimenting with frogs?

\speak{Bunter} Surely he goes out sometimes.

\speak{Cummings} Not often. And always back before twelve. And the way he goes on if he rings the bell and you ain't there. I give you \textit{my} word, Mr Bunter.

\speak{Bunter} Temper?

\speak{Cummings} No-o-o—but looking through you, nasty-like, as if you was on that operating table of his and he was going to cut you up. Nothing a man could rightly complain of, you understand, Mr Bunter, just nasty looks. Not but what I will say he's very correct. Apologizes if he's been inconsiderate. But what's the good of that when he's been and gone and lost you your night's rest?

\speak{Bunter} How does he do that? Keeps you up late, you mean?

\speak{Cummings} Not him; far from it. House locked up and household to bed at half-past ten. That's his little rule. Not but what I'm glad enough to go as a rule, it's that dreary. Still, when I \textit{do} go to bed I like to go to sleep.

\speak{Bunter} What does he do? Walk about the house?

\speak{Cummings} Doesn't he? All night. And in and out of the private door to the hospital.

\speak{Bunter} You don't mean to say, Mr Cummings, a great specialist like Sir Julian Freke does night work at the hospital?

\speak{Cummings} No, no; he does his own work—research work, as you may say. Cuts people up. They say he's very clever. Could take you or me to pieces like a clock, Mr Bunter, and put us together again.

\speak{Bunter} Do you sleep in the basement, then, to hear him so plain?

\speak{Cummings} No; our bedroom's at the top. But, Lord! what's that? He'll bang the door so you can hear him all over the house.

\speak{Bunter} Ah, many's the time I've had to speak to Lord Peter about that. And talking all night. And baths.

\speak{Cummings} Baths? You may well say that, Mr Bunter. Baths? Me and my wife sleep next to the cistern-room. Noise fit to wake the dead. All hours. When d'you think he chose to have a bath, no later than last Monday night, Mr Bunter?

\speak{Bunter} I've known them to do it at two in the morning, Mr Cummings.

\speak{Cummings} Have you, now? Well, this was at three. Three o'clock in the morning we was waked up. I give you \textit{my} word.

\speak{Bunter} You don't say so, Mr Cummings.

\speak{Cummings} He cuts up diseases, you see, Mr Bunter, and then he don't like to go to bed till he's washed the bacilluses off, if you understand me. Very natural, too, I daresay. But what I say is, the middle of the night's no time for a gentleman to be occupying his mind with diseases.

\speak{Bunter} These great men have their own way of doing things.

\speak{Cummings} Well, all I can say is, it isn't my way.

\direct{I could believe that, your lordship. Cummings has no signs of greatness about him, and his trousers are not what I would wish to see in a man of his profession.}

\speak{Bunter} Is he habitually as late as that, Mr Cummings?

\speak{Cummings} Well, no, Mr Bunter, I will say, not as a general rule. He apologized, too, in the morning, and said he would have the cistern seen to—and very necessary, in my opinion, for the air gets into the pipes, and the groaning and screeching as goes on is something awful. Just like Niagara, if you follow me, Mr Bunter, I give you \textit{my} word.

\speak{Bunter} Well, that's as it should be, Mr Cummings. One can put up with a great deal from a gentleman that has the manners to apologize. And, of course, sometimes they can't help themselves. A visitor will come in unexpectedly and keep them late, perhaps.

\speak{Cummings} That's true enough, Mr Bunter. Now I come to think of it, there \textit{was} a gentleman come in on Monday evening. Not that he came late, but he stayed about an hour, and may have put Sir Julian behindhand.

\speak{Bunter} Very likely. Let me give you some more port, Mr Cummings. Or a little of Lord Peter's old brandy.

\speak{Cummings} A little of the brandy, thank you, Mr Bunter. I suppose you have the run of the cellar here. \direct{He winked at me.}
\end{dialogue}

»Trust me for that,« I said, and I fetched him the Napoleon. I assure your lordship it went to my heart to pour it out for a man like that. However, seeing we had got on the right tack, I felt it wouldn't be wasted.

»I'm sure I wish it was always gentlemen that come here at night,« I said. (Your lordship will excuse me, I am sure, making such a suggestion.)
\end{quotation}

(»Good God,« said Lord Peter, »I wish Bunter was less thorough in his methods.«)

\begin{quotation}
\begin{dialogue}
\speak{Cummings} Oh, he's that sort, his lordship, is he? \direct{He chuckled and poked me. I suppress a portion of his conversation here, which could not fail to be as offensive to your lordship as it was to myself. He went on} No, it's none of that with Sir Julian. Very few visitors at night, and always gentlemen. And going early as a rule, like the one I mentioned.

\speak{Bunter} Just as well. There's nothing I find more wearisome, Mr Cummings, than sitting up to see visitors out.

\speak{Cummings} Oh, I didn't see this one out. Sir Julian let him out himself at ten o'clock or thereabouts. I heard the gentleman shout »Good-night« and off he goes.

\speak{Bunter} Does Sir Julian always do that?

\speak{Cummings} Well, that depends. If he sees visitors downstairs, he lets them out himself: if he sees them upstairs in the library, he rings for me.

\speak{Bunter} This was a downstairs visitor, then?

\speak{Cummings} Oh, yes. Sir Julian opened the door to him, I remember. He happened to be working in the hall. Though now I come to think of it, they went up to the library afterwards. That's funny. I know they did, because I happened to go up to the hall with coals, and I heard them upstairs. Besides, Sir Julian rang for me in the library a few minutes later. Still, anyway, we heard him go at ten, or it may have been a bit before. He hadn't only stayed about three-quarters of an hour. However, as I was saying, there was Sir Julian banging in and out of the private door all night, and a bath at three in the morning, and up again for breakfast at eight—it beats me. If I had all his money, curse me if I'd go poking about with dead men in the middle of the night. I'd find something better to do with my time, eh, Mr Bunter---

\end{dialogue}

I need not repeat any more of his conversation, as it became unpleasant and incoherent, and I could not bring him back to the events of Monday night. I was unable to get rid of him till three. He cried on my neck, and said I was the bird, and you were the governor for him. He said that Sir Julian would be greatly annoyed with him for coming home so late, but Sunday night was his night out and if anything was said about it he would give notice. I think he will be ill-advised to do so, as I feel he is not a man I could conscientiously recommend if I were in Sir Julian Freke's place. I noticed that his boot-heels were slightly worn down.

I should wish to add, as a tribute to the great merits of your lordship's cellar, that, although I was obliged to drink a somewhat large quantity both of the Cockburn '68 and the 1800 Napoleon I feel no headache or other ill effects this morning.

Trusting that your lordship is deriving real benefit from the country air, and that the little information I have been able to obtain will prove satisfactory, I remain.

With respectful duty to all the family,

\begin{flushright}
Obediently yours,\\
\textsc{Mervyn Bunter.}
\end{flushright}
\end{quotation}

»Y'know,« said Lord Peter thoughtfully to himself, »I sometimes think Mervyn Bunter's pullin' my leg. What is it, Soames?«

»A telegram, my lord.«

»Parker,« said Lord Peter, opening it. It said:

\textsc{Description recognised Chelsea Workhouse. Unknown vagrant injured street accident Wednesday week. Died workhouse Monday. Delivered St Luke's same evening by order Freke. Much puzzled. Parker.}

»Hurray!« said Lord Peter, suddenly sparkling. »I'm glad I've puzzled Parker. Gives me confidence in myself. Makes me feel like Sherlock Holmes. »Perfectly simple, Watson.« Dash it all, though! this is a beastly business. Still, it's puzzled Parker.«

»What's the matter?« asked the Duke, getting up and yawning.

»Marching orders,« said Peter, »back to town. Many thanks for your hospitality, old bird—I'm feelin' no end better. Ready to tackle Professor Moriarty or Leon Kestrel or any of 'em.«

»I do wish you'd keep out of the police courts,« grumbled the Duke. »It makes it so dashed awkward for me, havin' a brother makin' himself conspicuous.«

»Sorry, Gerald,« said the other; »I know I'm a beastly blot on the 'scutcheon.«

»Why can't you marry and settle down and live quietly, doin' something useful?« said the Duke, unappeased.

»Because that was a wash-out as you perfectly well know,« said Peter; »besides,« he added cheerfully, »I'm bein' no end useful. You may come to want me yourself, you never know. When anybody comes blackmailin' you, Gerald, or your first deserted wife turns up unexpectedly from the West Indies, you'll realize the pull of havin' a private detective in the family. »Delicate private business arranged with tact and discretion. Investigations undertaken. Divorce evidence a specialty. Every guarantee!« Come, now.«

»Ass!« said Lord Denver, throwing the newspaper violently into his armchair. »When do you want the car?«

»Almost at once. I say, Jerry, I'm taking Mother up with me.«

»Why should she be mixed up in it?«

»Well, I want her help.«

»I call it most unsuitable,« said the Duke.

The Dowager Duchess, however, made no objection.

»I used to know her quite well,« she said, »when she was Christine Ford. Why, dear?«

»Because,« said Lord Peter, »there's a terrible piece of news to be broken to her about her husband.«

»Is he dead, dear?«

»Yes; and she will have to come and identify him.«

»Poor Christine.«

»Under very revolting circumstances, Mother.«

»I'll come with you, dear.«

»Thank you, Mother, you're a brick. D'you mind gettin' your things on straight away and comin' up with me? I'll tell you about it in the car.«