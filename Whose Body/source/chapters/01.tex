%!TeX root=../bodytop.tex
\chapter[Chapter \thechapter]{}
\lettrine[lines=4,ante=‘]{O}{h}, damn!' said Lord Peter Wimsey at Piccadilly Circus. »Hi, driver!«

\zz
The taxi man, irritated at receiving this appeal while negotiating the intricacies of turning into Lower Regent Street across the route of a 19 `bus, a 38-B and a bicycle, bent an unwilling ear.

»I've left the catalogue behind,« said Lord Peter deprecatingly. »Uncommonly careless of me. D'you mind puttin' back to where we came from?«

»To the Savile Club, sir?«

»No—110 Piccadilly—just beyond—thank you.«

»Thought you was in a hurry,« said the man, overcome with a sense of injury.

»I'm afraid it's an awkward place to turn in,« said Lord Peter, answering the thought rather than the words. His long, amiable face looked as if it had generated spontaneously from his top hat, as white maggots breed from Gorgonzola.

The taxi, under the severe eye of a policeman, revolved by slow jerks, with a noise like the grinding of teeth.

The block of new, perfect and expensive flats in which Lord Peter dwelt upon the second floor, stood directly opposite the Green Park, in a spot for many years occupied by the skeleton of a frustrate commercial enterprise. As Lord Peter let himself in he heard his man's voice in the library, uplifted in that throttled stridency peculiar to well-trained persons using the telephone.

»I believe that's his lordship just coming in again—if your Grace would kindly hold the line a moment.«

»What is it, Bunter?«

»Her Grace has just called up from Denver, my lord. I was just saying your lordship had gone to the sale when I heard your lordship's latchkey.«

»Thanks,« said Lord Peter; »and you might find me my catalogue, would you? I think I must have left it in my bedroom, or on the desk.«

He sat down to the telephone with an air of leisurely courtesy, as though it were an acquaintance dropped in for a chat.

»Hullo, Mother—that you?«

»Oh, there you are, dear,« replied the voice of the Dowager Duchess. »I was afraid I'd just missed you.«

»Well, you had, as a matter of fact. I'd just started off to Brocklebury's sale to pick up a book or two, but I had to come back for the catalogue. What's up?«

»Such a quaint thing,« said the Duchess. »I thought I'd tell you. You know little Mr Thipps?«

»Thipps?« said Lord Peter. »Thipps? Oh, yes, the little architect man who's doing the church roof. Yes. What about him?«

»Mrs Throgmorton's just been in, in quite a state of mind.«

»Sorry, Mother, I can't hear. Mrs Who?«

»Throgmorton—Throgmorton—the vicar's wife.«

»Oh, Throgmorton, yes?«

»Mr Thipps rang them up this morning. It was his day to come down, you know.«

»Yes?«

»He rang them up to say he couldn't. He was so upset, poor little man. He'd found a dead body in his bath.«

»Sorry, Mother, I can't hear; found what, where?«

»A dead body, dear, in his bath.«

»What?---no, no, we haven't finished. Please don't cut us off. Hullo! Hullo! Is that you, Mother? Hullo!---Mother!---Oh, yes—sorry, the girl was trying to cut us off. What sort of body?«

»A dead man, dear, with nothing on but a pair of pince-nez. Mrs Throgmorton positively blushed when she was telling me. I'm afraid people do get a little narrow-minded in country vicarages.«

»Well, it sounds a bit unusual. Was it anybody he knew?«

»No, dear, I don't think so, but, of course, he couldn't give her many details. She said he sounded quite distracted. He's such a respectable little man—and having the police in the house and so on, really worried him.«

»Poor little Thipps! Uncommonly awkward for him. Let's see, he lives in Battersea, doesn't he?«

»Yes, dear; 59, Queen Caroline Mansions; opposite the Park. That big block just round the corner from the Hospital. I thought perhaps you'd like to run round and see him and ask if there's anything we can do. I always thought him a nice little man.«

»Oh, quite,« said Lord Peter, grinning at the telephone. The Duchess was always of the greatest assistance to his hobby of criminal investigation, though she never alluded to it, and maintained a polite fiction of its non-existence.

»What time did it happen, Mother?«

»I think he found it early this morning, but, of course, he didn't think of telling the Throgmortons just at first. She came up to me just before lunch—so tiresome, I had to ask her to stay. Fortunately, I was alone. I don't mind being bored myself, but I hate having my guests bored.«

»Poor old Mother! Well, thanks awfully for tellin' me. I think I'll send Bunter to the sale and toddle round to Battersea now an' try and console the poor little beast. So-long.«

»Good-bye, dear.«

»Bunter!«

»Yes, my lord.«

»Her Grace tells me that a respectable Battersea architect has discovered a dead man in his bath.«

»Indeed, my lord? That's very gratifying.«

»Very, Bunter. Your choice of words is unerring. I wish Eton and Balliol had done as much for me. Have you found the catalogue?«

»Here it is, my lord.«

»Thanks. I am going to Battersea at once. I want you to attend the sale for me. Don't lose time—I don't want to miss the Folio Dante\footnote{This is the first Florence edition, 1481, by Niccolo di Lorenzo. Lord Peter's collection of printed Dantes is worth inspection. It includes, besides the famous Aldine 8vo. of 1502, the Naples folio of 1477—»edizione rarissima«, according to Colomb. This copy has no history, and Mr Parker's private belief is that its present owner conveyed it away by stealth from somewhere or other. Lord Peter's own account is that he »picked it up in a little place in the hills«, when making a walking-tour through Italy.} nor the de Voragine—here you are—see? \textit{Golden Legend}---Wynkyn de Worde, 1493—got that?---and, I say, make a special effort for the Caxton folio of the \textit{Four Sons of Aymon}---it's the 1489 folio and unique. Look! I've marked the lots I want, and put my outside offer against each. Do your best for me. I shall be back to dinner.«

»Very good, my lord.«

»Take my cab and tell him to hurry. He may for you; he doesn't like me very much. Can I,« said Lord Peter, looking at himself in the eighteenth-century mirror over the mantelpiece, »can I have the heart to fluster the flustered Thipps further—that's very difficult to say quickly—by appearing in a top-hat and frock-coat? I think not. Ten to one he will overlook my trousers and mistake me for the undertaker. A grey suit, I fancy, neat but not gaudy, with a hat to tone, suits my other self better. Exit the amateur of first editions; new motive introduced by solo bassoon; enter Sherlock Holmes, disguised as a walking gentleman. There goes Bunter. Invaluable fellow—never offers to do his job when you've told him to do somethin' else. Hope he doesn't miss the \textit{Four Sons of Aymon}. Still, there \textit{is} another copy of that—in the Vatican.\footnote{Lord Peter's wits were wool-gathering. The book is in the possession of Earl Spencer. The Brocklebury copy is incomplete, the last five signatures being altogether missing, but is unique in possessing the colophon.} It might become available, you never know—if the Church of Rome went to pot or Switzerland invaded Italy—whereas a strange corpse doesn't turn up in a suburban bathroom more than once in a lifetime—at least, I should think not—at any rate, the number of times it's happened, \textit{with} a pince-nez, might be counted on the fingers of one hand, I imagine. Dear me! it's a dreadful mistake to ride two hobbies at once.«

He had drifted across the passage into his bedroom, and was changing with a rapidity one might not have expected from a man of his mannerisms. He selected a dark-green tie to match his socks and tied it accurately without hesitation or the slightest compression of his lips; substituted a pair of brown shoes for his black ones, slipped a monocle into a breast pocket, and took up a beautiful Malacca walking-stick with a heavy silver knob.

»That's all, I think,« he murmured to himself. »Stay—I may as well have you—you may come in useful—one never knows.« He added a flat silver matchbox to his equipment, glanced at his watch, and seeing that it was already a quarter to three, ran briskly downstairs, and, hailing a taxi, was carried to Battersea Park.

Mr Alfred Thipps was a small, nervous man, whose flaxen hair was beginning to abandon the unequal struggle with destiny. One might say that his only really marked feature was a large bruise over the left eyebrow, which gave him a faintly dissipated air incongruous with the rest of his appearance. Almost in the same breath with his first greeting, he made a self-conscious apology for it, murmuring something about having run against the dining-room door in the dark. He was touched almost to tears by Lord Peter's thoughtfulness and condescension in calling.

»I'm sure it's most kind of your lordship,« he repeated for the dozenth time, rapidly blinking his weak little eyelids. »I appreciate it very deeply, very deeply, indeed, and so would Mother, only she's so deaf, I don't like to trouble you with making her understand. It's been very hard all day,« he added, »with the policemen in the house and all this commotion. It's what Mother and me have never been used to, always living very retired, and it's most distressing to a man of regular habits, my lord, and reely, I'm almost thankful Mother doesn't understand, for I'm sure it would worry her terribly if she was to know about it. She was upset at first, but she's made up some idea of her own about it now, and I'm sure it's all for the best.«

The old lady who sat knitting by the fire nodded grimly in response to a look from her son.

»I always said as you ought to complain about that bath, Alfred,« she said suddenly, in the high, piping voice peculiar to the deaf, »and it's to be `oped the landlord'll see about it now; not but what I think you might have managed without having the police in, but there! you always were one to make a fuss about a little thing, from chicken-pox up.«

»There now,« said Mr Thipps apologetically, »you see how it is. Not but what it's just as well she's settled on that, because she understands we've locked up the bathroom and don't try to go in there. But it's been a terrible shock to me, sir—my lord, I should say, but there! my nerves are all to pieces. Such a thing has never `appened—happened to me in all my born days. Such a state I was in this morning—I didn't know if I was on my head or my heels—I reely didn't, and my heart not being too strong, I hardly knew how to get out of that horrid room and telephone for the police. It's affected me, sir, it's affected me, it reely has—I couldn't touch a bit of breakfast, nor lunch neither, and what with telephoning and putting off clients and interviewing people all morning, I've hardly known what to do with myself.«

»I'm sure it must have been uncommonly distressin',« said Lord Peter, sympathetically, »especially comin' like that before breakfast. Hate anything tiresome happenin' before breakfast. Takes a man at such a confounded disadvantage, what?«

»That's just it, that's just it,« said Mr Thipps, eagerly. »When I saw that dreadful thing lying there in my bath, mother-naked, too, except for a pair of eyeglasses, I assure you, my lord, it regularly turned my stomach, if you'll excuse the expression. I'm not very strong, sir, and I get that sinking feeling sometimes in the morning, and what with one thing and another I `ad—had to send the girl for a stiff brandy, or I don't know \textit{what} mightn't have happened. I felt so queer, though I'm anything but partial to spirits as a rule. Still, I make it a rule never to be without brandy in the house, in case of emergency, you know?«

»Very wise of you,« said Lord Peter, cheerfully. »You're a very far-seein' man, Mr Thipps. Wonderful what a little nip'll do in case of need, and the less you're used to it the more good it does you. Hope your girl is a sensible young woman, what? Nuisance to have women faintin' and shriekin' all over the place.«

»Oh, Gladys is a good girl,« said Mr Thipps, »very reasonable indeed. She was shocked, of course; that's very understandable. I was shocked myself, and it wouldn't be proper in a young woman not to be shocked under the circumstances, but she is reely a helpful, energetic girl in a crisis, if you understand me. I consider myself very fortunate these days to have got a good, decent girl to do for me and Mother, even though she is a bit careless and forgetful about little things, but that's only natural. She was very sorry indeed about having left the bathroom window open, she reely was, and though I was angry at first, seeing what's come of it, it wasn't anything to speak of, not in the ordinary way, as you might say. Girls will forget things, you know, my lord, and reely she was so distressed I didn't like to say too much to her. All I said was: »It might have been burglars,« I said, »remember that, next time you leave a window open all night; this time it was a dead man,« I said, »and that's unpleasant enough, but next time it might be burglars,« I said, »and all of us murdered in our beds.« But the police-inspector—Inspector Sugg, they called him, from the Yard—he was very sharp with her, poor girl. Quite frightened her, and made her think he suspected her of something, though what good a body could be to her, poor girl, I can't imagine, and so I told the Inspector. He was quite rude to me, my lord—I may say I didn't like his manner at all. »If you've got anything definite to accuse Gladys or me of, Inspector,« I said to him, »bring it forward, that's what you have to do,« I said, »but I've yet to learn that you're paid to be rude to a gentleman in his own `ouse—house.« Reely,« said Mr Thipps, growing quite pink on the top of his head, »he regularly roused me, regularly roused me, my lord, and I'm a mild man as a rule.«

»Sugg all over,« said Lord Peter. »I know him. When he don't know what else to say, he's rude. Stands to reason you and the girl wouldn't go collectin' bodies. Who'd want to saddle himself with a body? Difficulty's usually to get rid of `em. Have you got rid of this one yet, by the way?«

»It's still in the bathroom,« said Mr Thipps. »Inspector Sugg said nothing was to be touched till his men came in to move it. I'm expecting them at any time. If it would interest your lordship to have a look at it\longdash«

»Thanks awfully,« said Lord Peter. »I'd like to very much, if I'm not puttin' you out.«

»Not at all,« said Mr Thipps. His manner as he led the way along the passage convinced Lord Peter of two things—first, that, gruesome as his exhibit was, he rejoiced in the importance it reflected upon himself and his flat, and secondly, that Inspector Sugg had forbidden him to exhibit it to anyone. The latter supposition was confirmed by the action of Mr Thipps, who stopped to fetch the door-key from his bedroom, saying that the police had the other, but that he made it a rule to have two keys to every door, in case of accident.

The bathroom was in no way remarkable. It was long and narrow, the window being exactly over the head of the bath. The panes were of frosted glass; the frame wide enough to admit a man's body. Lord Peter stepped rapidly across to it, opened it and looked out.

The flat was the top one of the building and situated about the middle of the block. The bathroom window looked out upon the back-yards of the flats, which were occupied by various small outbuildings, coal-holes, garages, and the like. Beyond these were the back gardens of a parallel line of houses. On the right rose the extensive edifice of St Luke's Hospital, Battersea, with its grounds, and, connected with it by a covered way, the residence of the famous surgeon, Sir Julian Freke, who directed the surgical side of the great new hospital, and was, in addition, known in Harley Street as a distinguished neurologist with a highly individual point of view.

This information was poured into Lord Peter's ear at considerable length by Mr Thipps, who seemed to feel that the neighbourhood of anybody so distinguished shed a kind of halo of glory over Queen Caroline Mansions.

»We had him round here himself this morning,« he said, »about this horrid business. Inspector Sugg thought one of the young medical gentlemen at the hospital might have brought the corpse round for a joke, as you might say, they always having bodies in the dissecting-room. So Inspector Sugg went round to see Sir Julian this morning to ask if there was a body missing. He was very kind, was Sir Julian, very kind indeed, though he was at work when they got there, in the dissecting-room. He looked up the books to see that all the bodies were accounted for, and then very obligingly came round here to look at this«---he indicated the bath---»and said he was afraid he couldn't help us—there was no corpse missing from the hospital, and this one didn't answer to the description of any they'd had.«

»Nor to the description of any of the patients, I hope,« suggested Lord Peter casually.

At this grisly hint Mr Thipps turned pale.

»I didn't hear Inspector Sugg inquire,« he said, with some agitation. »What a very horrid thing that would be—God bless my soul, my lord, I never thought of it.«

»Well, if they had missed a patient they'd probably have discovered it by now,« said Lord Peter. »Let's have a look at this one.«

He screwed his monocle into his eye, adding: »I see you're troubled here with the soot blowing in. Beastly nuisance, ain't it? I get it, too—spoils all my books, you know. Here, don't you trouble, if you don't care about lookin' at it.«

He took from Mr Thipps's hesitating hand the sheet which had been flung over the bath, and turned it back.

The body which lay in the bath was that of a tall, stout man of about fifty. The hair, which was thick and black and naturally curly, had been cut and parted by a master hand, and exuded a faint violet perfume, perfectly recognisable in the close air of the bathroom. The features were thick, fleshy and strongly marked, with prominent dark eyes, and a long nose curving down to a heavy chin. The clean-shaven lips were full and sensual, and the dropped jaw showed teeth stained with tobacco. On the dead face the handsome pair of gold pince-nez mocked death with grotesque elegance; the fine gold chain curved over the naked breast. The legs lay stiffly stretched out side by side; the arms reposed close to the body; the fingers were flexed naturally. Lord Peter lifted one arm, and looked at the hand with a little frown.

»Bit of a dandy, your visitor, what?« he murmured. »Parma violet and manicure.« He bent again, slipping his hand beneath the head. The absurd eyeglasses slipped off, clattering into the bath, and the noise put the last touch to Mr Thipps's growing nervousness.

»If you'll excuse me,« he murmured, »it makes me feel quite faint, it reely does.«

He slipped outside, and he had no sooner done so than Lord Peter, lifting the body quickly and cautiously, turned it over and inspected it with his head on one side, bringing his monocle into play with the air of the late Joseph Chamberlain approving a rare orchid. He then laid the head over his arm, and bringing out the silver matchbox from his pocket, slipped it into the open mouth. Then making the noise usually written »Tut-tut,« he laid the body down, picked up the mysterious pince-nez, looked at it, put it on his nose and looked through it, made the same noise again, readjusted the pince-nez upon the nose of the corpse, so as to leave no traces of interference for the irritation of Inspector Sugg; rearranged the body; returned to the window and, leaning out, reached upwards and sideways with his walking-stick, which he had somewhat incongruously brought along with him. Nothing appearing to come of these investigations, he withdrew his head, closed the window, and rejoined Mr Thipps in the passage.

Mr Thipps, touched by this sympathetic interest in the younger son of a duke, took the liberty, on their return to the sitting-room, of offering him a cup of tea. Lord Peter, who had strolled over to the window and was admiring the outlook on Battersea Park, was about to accept, when an ambulance came into view at the end of Prince of Wales Road. Its appearance reminded Lord Peter of an important engagement, and with a hurried »By Jove!« he took his leave of Mr Thipps.

»My mother sent kind regards and all that,« he said, shaking hands fervently; »hopes you'll soon be down at Denver again. Good-bye, Mrs Thipps,« he bawled kindly into the ear of the old lady. »Oh, no, my dear sir, please don't trouble to come down.«

He was none too soon. As he stepped out of the door and turned towards the station, the ambulance drew up from the other direction, and Inspector Sugg emerged from it with two constables. The Inspector spoke to the officer on duty at the Mansions, and turned a suspicious gaze on Lord Peter's retreating back.

»Dear old Sugg,« said that nobleman, fondly, »dear, dear old bird! How he does hate me, to be sure.«