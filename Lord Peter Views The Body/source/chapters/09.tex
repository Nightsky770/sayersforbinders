%!TeX root=../viewsbodytop.tex
\addchap{The Learned Adventure of the Dragon's Head}

\lettrine[lines=4,ante=‘]{U}{ncle} Peter!'

\zz
»Half a jiff, Gherkins. No, I don't think I'll take the Catullus, Mr Ffolliott. After all, thirteen guineas is a bit steep without either the title or the last folio, what? But you might send me round the Vitruvius and the Satyricon when they come in; I'd like to have a look at them, anyhow. Well, old man, what is it?«

»Do come and look at these pictures, Uncle Peter. I'm sure it's an awfully old book.«

Lord Peter Wimsey sighed as he picked his way out of Mr Ffolliott's dark back shop, strewn with the flotsam and jetsam of many libraries. An unexpected outbreak of measles at Mr Bultridge's excellent preparatory school, coinciding with the absence of the Duke and Duchess of Denver on the Continent, had saddled his lordship with his ten-year-old nephew, Viscount St George, more commonly known as Young Jerry, Jerrykins, or Pickled Gherkins. Lord Peter was not one of those born uncles who delight old nurses by their fascinating »way with« children. He succeeded, however, in earning tolerance on honourable terms by treating the young with the same scrupulous politeness which he extended to their elders. He therefore prepared to receive Gherkins's discovery with respect, though a child's taste was not to be trusted, and the book might quite well be some horror of woolly mezzotints or an inferior modern reprint adorned with leprous electros. Nothing much better was really to be expected from the »cheap shelf« exposed to the dust of the street.

»Uncle! there's such a funny man here, with a great long nose and ears and a tail and dogs' heads all over his body. \textit{Monstrum hoc Cracoviæ}—that's a monster, isn't it? I should jolly well think it was. What's \textit{Cracoviæ}, Uncle Peter?«

»Oh,« said Lord Peter, greatly relieved, »the Cracow monster?« A portrait of that distressing infant certainly argued a respectable antiquity. »Let's have a look. Quite right, it's a very old book—Munster's \textit{Cosmographia Universalis}. I'm glad you know good stuff when you see it, Gherkins. What's the \textit{Cosmographia} doing out here, Mr Ffolliott, at five bob?«

»Well, my lord,« said the bookseller, who had followed his customers to the door, »it's in a very bad state, you see; covers loose and nearly all the double-page maps missing. It came in a few weeks ago—dumped in with a collection we bought from a gentleman in Norfolk—you'll find his name in it—Dr Conyers of Yelsall Manor. Of course, we might keep it and try to make up a complete copy when we get another example. But it's rather out of our line, as you know, classical authors being our speciality. So we just put it out to go for what it would fetch in the \textit{status quo}, as you might say.«

»Oh, look!« broke in Gherkins. »Here's a picture of a man being chopped up in little bits. What does it say about it?«

»I thought you could read Latin.«

»Well, but it's all full of sort of pothooks. What do they mean?«

»They're just contractions,« said Lord Peter patiently. »»\textit{Solent quoque hujus insulæ cultores}«—It is the custom of the dwellers in this island, when they see their parents stricken in years and of no further use, to take them down into the market-place and sell them to the cannibals, who kill them and eat them for food. This they do also with younger persons when they fall into any desperate sickness.«

»Ha, ha!« said Mr Ffolliott. »Rather sharp practice on the poor cannibals. They never got anything but tough old joints or diseased meat, eh?«

»The inhabitants seem to have had thoroughly advanced notions of business,« agreed his lordship.

The viscount was enthralled.

»I \textit{do} like this book,« he said; »could I buy it out of my pocket-money, please?«

»Another problem for uncles,« thought Lord Peter, rapidly ransacking his recollections of the \textit{Cosmographia} to determine whether any of its illustrations were indelicate; for he knew the duchess to be strait-laced. On consideration, he could only remember one that was dubious, and there was a sporting chance that the duchess might fail to light upon it.

»Well,« he said judicially, »in your place, Gherkins, I should be inclined to buy it. It's in a bad state, as Mr Ffolliott has honourably told you—otherwise, of course, it would be exceedingly valuable; but, apart from the lost pages, it's a very nice clean copy, and certainly worth five shillings to you, if you think of starting a collection.«

Till that moment, the viscount had obviously been more impressed by the cannibals than by the state of the margins, but the idea of figuring next term at Mr Bultridge's as a collector of rare editions had undeniable charm.

»None of the other fellows collect books,« he said; »they collect stamps, mostly. I think stamps are rather ordinary, don't you, Uncle Peter? I was rather thinking of giving up stamps. Mr Porter, who takes us for history, has got a lot of books like yours, and he is a splendid man at footer.«

Rightly interpreting this reference to Mr Porter, Lord Peter gave it as his opinion that book-collecting could be a perfectly manly pursuit. Girls, he said, practically never took it up, because it meant so much learning about dates and type-faces and other technicalities which called for a masculine brain.

»Besides,« he added, »it's a very interesting book in itself, you know. Well worth dipping into.«

»I'll take it, please,« said the viscount, blushing a little at transacting so important and expensive a piece of business; for the duchess did not encourage lavish spending by little boys, and was strict in the matter of allowances.

Mr Ffolliott bowed, and took the \textit{Cosmographia} away to wrap it up.

»Are you all right for cash?« enquired Lord Peter discreetly. »Or can I be of temporary assistance?«

»No, thank you, uncle; I've got Aunt Mary's half-crown and four shillings of my pocket-money, because, you see, with the measles happening, we didn't have our dormitory spread, and I was saving up for that.«

The business being settled in this gentlemanly manner, and the budding bibliophile taking personal and immediate charge of the stout, square volume, a taxi was chartered which, in due course of traffic delays, brought the \textit{Cosmographia} to 110A Piccadilly.

\noindent\hfil\rule{0.5\textwidth}{.4pt}\hfil 
\pagebreak[2]

»And who, Bunter, is Mr Wilberforce Pope?«

»I do not think we know the gentleman, my lord. He is asking to see your lordship for a few minutes on business.«

»He probably wants me to find a lost dog for his maiden aunt. What it is to have acquired a reputation as a sleuth! Show him in. Gherkins, if this good gentleman's business turns out to be private, you'd better retire into the dining-room.«

»Yes, Uncle Peter,« said the viscount dutifully. He was extended on his stomach on the library hearthrug, laboriously picking his way through the more exciting-looking bits of the \textit{Cosmographia}, with the aid of Messrs. Lewis \& Short, whose monumental compilation he had hitherto looked upon as a barbarous invention for the annoyance of upper forms.

Mr Wilberforce Pope turned out to be a rather plump, fair gentleman in the late thirties, with a prematurely bald forehead, horn-rimmed spectacles, and an engaging manner.

»You will excuse my intrusion, won't you?« he began. »I'm sure you must think me a terrible nuisance. But I wormed your name and address out of Mr Ffolliott. Not his fault, really. You won't blame him, will you? I positively badgered the poor man. Sat down on his doorstep and refused to go, though the boy was putting up the shutters. I'm afraid you will think me very silly when you know what it's all about. But you really mustn't hold poor Mr Ffolliott responsible, now, will you?«

»Not at all,« said his lordship. »I mean, I'm charmed and all that sort of thing. Something I can do for you about books? You're a collector, perhaps? Will you have a drink or anything?«

»Well, no,« said Mr Pope, with a faint giggle. »No, not exactly a collector. Thank you very much, just a spot—no, no, literally a spot. Thank you; no«—he glanced round the bookshelves, with their rows of rich old leather bindings—»certainly not a collector. But I happen to be er, interested—sentimentally interested—in a purchase you made yesterday. Really, such a very small matter. You will think it foolish. But I am told you are the present owner of a copy of Munster's \textit{Cosmographia}, which used to belong to my uncle, Dr Conyers.«

Gherkins looked up suddenly, seeing that the conversation had a personal interest for him.

»Well, that's not quite correct,« said Wimsey. »I was there at the time, but the actual purchaser is my nephew. Gerald, Mr Pope is interested in your \textit{Cosmographia}. My nephew, Lord St George.«

»How do you do, young man,« said Mr Pope affably. »I see that the collecting spirit runs in the family. A great Latin scholar, too, I expect, eh? Ready to decline \textit{jusjurandum} with the best of us? Ha, ha! And what are you going to do when you grow up? Be Lord Chancellor, eh? Now, I bet you think you'd rather be an engine-driver, what, what?«

»No, thank you,« said the viscount, with aloofness.

»What, not an engine-driver? Well, now, I want you to be a real business man this time. Put through a book deal, you know. Your uncle will see I offer you a fair price, what? Ha, ha! Now, you see, that picture-book of yours has a great value for me that it wouldn't have for anybody else. When \textit{I} was a little boy of your age it was one of my very greatest joys. I used to have it to look at on Sundays. Ah, dear! the happy hours I used to spend with those quaint old engravings, and the funny old maps with the ships and salamanders and »\textit{Hic dracones}«—you know what \textit{that} means, I dare say. What does it mean?«

»Here are dragons,« said the viscount, unwillingly but still politely.

»Quite right. I \textit{knew} you were a scholar.«

»It's a very attractive book,« said Lord Peter. »My nephew was quite entranced by the famous Cracow monster.«

»Ah yes—a glorious monster, isn't it?« agreed Mr Pope, with enthusiasm. »Many's the time I've fancied myself as Sir Lancelot or somebody on a white war horse, charging that monster, lance in rest, with the captive princess cheering me on. Ah! childhood! You're living the happiest days of your life, young man. You won't believe me, but you are.«

»Now what is it exactly you want my nephew to do?« enquired Lord Peter a little sharply.

»Quite right, quite right. Well now, you know, my uncle, Dr Conyers, sold his library a few months ago. I was abroad at the time, and it was only yesterday, when I went down to Yelsall on a visit, that I learnt the dear old book had gone with the rest. I can't tell you how distressed I was. I know it's not valuable—a great many pages missing and all that—but I can't bear to think of its being gone. So, purely from sentimental reasons, as I said, I hurried off to Ffolliott's to see if I could get it back. I was quite upset to find I was too late, and gave poor Mr Ffolliott no peace till he told me the name of the purchaser. Now, you see, Lord St George, I'm here to make you an offer for the book. Come, now, double what you gave for it. That's a good offer, isn't it, Lord Peter? Ha, ha! And you will be doing me a very great kindness as well.«

Viscount St George looked rather distressed, and turned appealingly to his uncle.

»Well, Gerald,« said Lord Peter, »it's your affair, you know. What do you say?«

The viscount stood first on one leg and then on the other. The career of a book-collector evidently had its problems, like other careers.

»If you please, Uncle Peter,« he said, with embarrassment, »may I whisper?«

»It's not usually considered the thing to whisper, Gherkins, but you could ask Mr Pope for time to consider his offer. Or you could say you would prefer to consult me first. That would be quite in order.«

»Then, if you don't mind, Mr Pope, I should like to consult my uncle first.«

»Certainly, certainly; ha, ha!« said Mr Pope. »Very prudent to consult a collector of greater experience, what? Ah! the younger generation, eh, Lord Peter? Regular little business men already.«

»Excuse us, then, for one moment,« said Lord Peter, and drew his nephew into the dining-room.

»I say, Uncle Peter,« said the collector breathlessly, when the door was shut, »\textit{need} I give him my book? I don't think he's a very nice man. I \textit{hate} people who ask you to decline nouns for them.«

»Certainly you needn't, Gherkins, if you don't want to. The book is yours, and you've a right to it.«

»What would \textit{you} do, uncle?«

Before replying, Lord Peter, in the most surprising manner, tiptoed gently to the door which communicated with the library and flung it suddenly open, in time to catch Mr Pope kneeling on the hearthrug intently turning over the pages of the coveted volume, which lay as the owner had left it. He started to his feet in a flurried manner as the door opened.

»Do help yourself, Mr Pope, won't you?« cried Lord Peter hospitably, and closed the door again.

»What is it, Uncle Peter?«

»If you want my advice, Gherkins, I should be rather careful how you had any dealings with Mr Pope. I don't think he's telling the truth. He called those wood-cuts engravings—though, of course, that may be just his ignorance. But I can't believe that he spent all his childhood's Sunday afternoons studying those maps and picking out the dragons in them, because, as you may have noticed for yourself, old Munster put very few dragons into his maps. They're mostly just plain maps—a bit queer to our ideas of geography, but perfectly straight-forward. That was why I brought in the Cracow monster, and, you see, he thought it was some sort of dragon.«

»Oh, I say, uncle! So you said that on purpose!«

»If Mr Pope wants the \textit{Cosmographia}, it's for some reason he doesn't want to tell us about. And, that being so, I wouldn't be in too big a hurry to sell, if the book were mine. See?«

»Do you mean there's something frightfully valuable about the book, which we don't know?«

»Possibly.«

»How exciting! It's just like a story in the \textit{Boys' Friend Library}. What am I to say to him, uncle?«

»Well, in your place I wouldn't be dramatic or anything. I'd just say you've considered the matter, and you've taken a fancy to the book and have decided not to sell. You thank him for his offer, of course.«

»Yes—er, won't you say it for me, uncle?«

»I think it would look better if you did it yourself.«

»Yes, perhaps it would. Will he be very cross?«

»Possibly,« said Lord Peter, »but, if he is, he won't let on. Ready?«

The consulting committee accordingly returned to the library. Mr Pope had prudently retired from the hearthrug and was examining a distant bookcase.

»Thank you very much for your offer, Mr Pope,« said the viscount, striding stoutly up to him, »but I have considered it, and I have taken a—a—a fancy for the book and decided not to sell.«

»Sorry and all that,« put in Lord Peter, »but my nephew's adamant about it. No, it isn't the price; he wants the book. Wish I could oblige you, but it isn't in my hands. Won't you take something else before you go? Really? Ring the bell, Gherkins. My man will see you to the lift. \textit{Good} evening.«

When the visitor had gone, Lord Peter returned and thoughtfully picked up the book.

»We were awful idiots to leave him with it, Gherkins, even for a moment. Luckily, there's no harm done.«

»You don't think he found out anything while we were away, do you, uncle?« gasped Gherkins, open-eyed.

»I'm sure he didn't.«

»Why?«

»He offered me fifty pounds for it on the way to the door. Gave the game away. H'm! Bunter.«

»My lord?«

»Put this book in the safe and bring me back the keys. And you'd better set all the burglar alarms when you lock up.«

»Oo—er!« said Viscount St George.

\noindent\hfil\rule{0.5\textwidth}{.4pt}\hfil 

On the third morning after the visit of Mr Wilberforce Pope, the viscount was seated at a very late breakfast in his uncle's flat, after the most glorious and soul-satisfying night that ever boy experienced. He was almost too excited to eat the kidneys and bacon placed before him by Bunter, whose usual impeccable manner was not in the least impaired by a rapidly swelling and blackening eye.

It was about two in the morning that Gherkins—who had not slept very well, owing to too lavish and grown-up a dinner and theatre the evening before—became aware of a stealthy sound somewhere in the direction of the fire-escape. He had got out of bed and crept very softly into Lord Peter's room and woken him up. He had said: »Uncle Peter, I'm sure there's burglars on the fire-escape.« And Uncle Peter, instead of saying, »Nonsense, Gherkins, hurry up and get back to bed,« had sat up and listened and said: »By Jove, Gherkins, I believe you're right.« And had sent Gherkins to call Bunter. And on his return, Gherkins, who had always regarded his uncle as a very top-hatted sort of person, actually saw him take from his handkerchief-drawer an undeniable automatic pistol.

It was at this point that Lord Peter was apotheosed from the state of Quite Decent Uncle to that of Glorified Uncle. He said:

»Look here, Gherkins, we don't know how many of these blighters there'll be, so you must be jolly smart and do anything I say sharp, on the word of command—even if I have to say `Scoot.' Promise?«

Gherkins promised, with his heart thumping, and they sat waiting in the dark, till suddenly a little electric bell rang sharply just over the head of Lord Peter's bed and a green light shone out.

»The library window,« said his lordship, promptly silencing the bell by turning a switch. »If they heard, they may think better of it. We'll give them a few minutes.«

They gave them five minutes, and then crept very quietly down the passage.

»Go round by the dining-room, Bunter,« said his lordship; »they may bolt that way.«

With infinite precaution, he unlocked and opened the library door, and Gherkins noticed how silently the locks moved.

A circle of light from an electric torch was moving slowly along the bookshelves. The burglars had obviously heard nothing of the counter-attack. Indeed, they seemed to have troubles enough of their own to keep their attention occupied. As his eyes grew accustomed to the dim light, Gherkins made out that one man was standing holding the torch, while the other took down and examined the books. It was fascinating to watch his apparently disembodied hands move along the shelves in the torch-light.

The men muttered discontentedly. Obviously the job was proving a harder one than they had bargained for. The habit of ancient authors of abbreviating the titles on the backs of their volumes, or leaving them completely untitled, made things extremely awkward. From time to time the man with the torch extended his hand into the light. It held a piece of paper, which they anxiously compared with the title-page of a book. Then the volume was replaced and the tedious search went on.

Suddenly some slight noise—Gherkins was sure \textit{he} did not make it; it may have been Bunter in the dining-room—seemed to catch the ear of the kneeling man.

»Wot's that?« he gasped, and his startled face swung round into view.

»Hands up!« said Lord Peter, and switched the light on.

The second man made one leap for the dining-room door, where a smash and an oath proclaimed that he had encountered Bunter. The kneeling man shot his hands up like a marionette.

»Gherkins,« said Lord Peter, »do you think you can go across to that gentleman by the bookcase and relieve him of the article which is so inelegantly distending the right-hand pocket of his coat? Wait a minute. Don't on any account get between him and my pistol, and mind you take the thing out \textit{very} carefully. There's no hurry. That's splendid. Just point it at the floor while you bring it across, would you? Thanks. Bunter has managed for himself, I see. Now run into my bedroom, and in the bottom of my wardrobe you will find a bundle of stout cord. Oh! I beg your pardon; yes, put your hands down by all means. It must be very tiring exercise.«

The arms of the intruders being secured behind their backs with a neatness which Gherkins felt to be worthy of the best traditions of Sexton Blake, Lord Peter motioned his captives to sit down and despatched Bunter for whisky-and-soda.

»Before we send for the police,« said Lord Peter, »you would do me a great personal favour by telling me what you were looking for, and who sent you. Ah! thanks, Bunter. As our guests are not at liberty to use their hands, perhaps you would be kind enough to assist them to a drink. Now then, say when.«

»Well, you're a gentleman, guv'nor,« said the First Burglar, wiping his mouth politely on his shoulder, the back of his hand not being available. »If we'd a known wot a job this wos goin' ter be, blow me if we'd a touched it. The bloke said, ses `e, »It's takin' candy from a baby,« `e ses. »The gentleman's a reg'lar softie,« `e ses, »one o' these `ere sersiety toffs wiv a maggot fer old books,« that's wot `e ses, »an' ef yer can find this `ere old book fer me,« `e ses, »there's a pony fer yer.« Well! Sech a job! `E didn't mention as `ow there'd be five `undred fousand bleedin' ole books all as alike as a regiment o' bleedin' dragoons. Nor as `ow yer kept a nice little machine-gun like that `andy by the bedside, \textit{nor} yet as `ow yer was so bleedin' good at tyin' knots in a bit o' string. No—'e didn't think ter mention them things.«

»Deuced unsporting of him,« said his lordship. »Do you happen to know the gentleman's name?«

»No—that was another o' them things wot `e didn't mention. `E's a stout, fair party, wiv `orn rims to `is goggles and a bald `ead. One o' these `ere philanthropists, I reckon. A friend o' mine, wot got inter trouble onct, got work froo `im, and the gentleman comes round and ses to `im, `e ses, »Could yer find me a couple o' lads ter do a little job?« `e ses, an' my friend, finkin' no `arm, you see, guv'nor, but wot it might be a bit of a joke like, `e gets `old of my pal an' me, an' we meets the gentleman in a pub dahn Whitechapel way. W'ich we was ter meet `im there again Friday night, us `avin' allowed that time fer ter git `old of the book.«

»The book being, if I may hazard a guess, the \textit{Cosmographia Universalis}?«

»Sumfink like that, guv'nor. I got its jaw-breakin' name wrote down on a bit o' paper, wot my pal `ad in `is `and. Wot did yer do wiv that `ere bit o' paper, Bill?«

»Well, look here,« said Lord Peter, »I'm afraid I must send for the police, but I think it likely, if you give us your assistance to get hold of your gentleman, whose name I strongly suspect to be Wilberforce Pope, that you will get off pretty easily. Telephone the police, Bunter, and then go and put something on that eye of yours. Gherkins, we'll give these gentlemen another drink, and then I think perhaps you'd better hop back to bed; the fun's over. No? Well, put a good thick coat on, there's a good fellow, because what your mother will say to me if you catch a cold I don't like to think.«

So the police had come and taken the burglars away, and now Detective-Inspector Parker, of Scotland Yard, a great personal friend of Lord Peter's, sat toying with a cup of coffee and listening to the story.

»But what's the matter with the jolly old book, anyhow, to make it so popular?« he demanded.

»I don't know,« replied Wimsey; »but after Mr Pope's little visit the other day I got kind of intrigued about it and had a look through it. I've got a hunch it may turn out rather valuable, after all. Unsuspected beauties and all that sort of thing. If only Mr Pope had been a trifle more accurate in his facts, he might have got away with something to which I feel pretty sure he isn't entitled. Anyway, when I'd seen—what I saw, I wrote off to Dr Conyers of Yelsall Manor, the late owner\longdash«

»Conyers, the cancer man?«

»Yes. He's done some pretty important research in his time, I fancy. Getting on now, though; about seventy-eight, I fancy. I hope he's more honest than his nephew, with one foot in the grave like that. Anyway, I wrote (with Gherkins's permission, naturally) to say we had the book and had been specially interested by something we found there, and would he be so obliging as to tell us something of its history. I also\longdash«

»But what did you find in it?«

»I don't think we'll tell him yet, Gherkins, shall we? I like to keep policemen guessing. As I was saying, when you so rudely interrupted me, I also asked him whether he knew anything about his good nephew's offer to buy it back. His answer has just arrived. He says he knows of nothing specially interesting about the book. It has been in the library untold years, and the tearing out of the maps must have been done a long time ago by some family vandal. He can't think why his nephew should be so keen on it, as he certainly never pored over it as a boy. In fact, the old man declares the engaging Wilberforce has never even set foot in Yelsall Manor to his knowledge. So much for the fire-breathing monsters and the pleasant Sunday afternoons.«

»Naughty Wilberforce!«

»M'm. Yes. So, after last night's little dust-up, I wired the old boy we were tooling down to Yelsall to have a heart-to-heart talk with him about his picture-book and his nephew.«

»Are you taking the book down with you?« asked Parker. »I can give you a police escort for it if you like.«

»That's not a bad idea,« said Wimsey. »We don't know where the insinuating Mr Pope may be hanging out, and I wouldn't put it past him to make another attempt.«

»Better be on the safe side,« said Parker. »I can't come myself, but I'll send down a couple of men with you.«

»Good egg,« said Lord Peter. »Call up your myrmidons. We'll get a car round at once. You're coming, Gherkins, I suppose? God knows what your mother would say. Don't ever be an uncle, Charles; it's frightfully difficult to be fair to all parties.«

\noindent\hfil\rule{0.5\textwidth}{.4pt}\hfil 

Yelsall Manor was one of those large, decaying country mansions which speak eloquently of times more spacious than our own. The original late Tudor construction had been masked by the addition of a wide frontage in the Italian manner, with a kind of classical portico surmounted by a pediment and approached by a semi-circular flight of steps. The grounds had originally been laid out in that formal manner in which grove nods to grove and each half duly reflects the other. A late owner, however, had burst out into the more eccentric sort of landscape gardening which is associated with the name of Capability Brown. A Chinese pagoda, somewhat resembling Sir William Chambers's erection in Kew Gardens, but smaller, rose out of a grove of laurustinus towards the eastern extremity of the house, while at the rear appeared a large artificial lake, dotted with numerous islands, on which odd little temples, grottos, tea-houses, and bridges peeped out from among clumps of shrubs, once ornamental, but now sadly overgrown. A boat-house, with wide eaves like the designs on a willow-pattern plate, stood at one corner, its landing-stage fallen into decay and wreathed with melancholy weeds.

»My disreputable old ancestor, Cuthbert Conyers, settled down here when he retired from the sea in 1732,« said Dr Conyers, smiling faintly. »His elder brother died childless, so the black sheep returned to the fold with the determination to become respectable and found a family. I fear he did not succeed altogether. There were very queer tales as to where his money came from. He is said to have been a pirate, and to have sailed with the notorious Captain Blackbeard. In the village, to this day, he is remembered and spoken of as Cut-throat Conyers. It used to make the old man very angry, and there is an unpleasant story of his slicing the ears off a groom who had been heard to call him »Old Cut-throat.« He was not an uncultivated person, though. It was he who did the landscape-gardening round at the back, and he built the pagoda for his telescope. He was reputed to study the Black Art, and there were certainly a number of astrological works in the library with his name on the fly-leaf, but probably the telescope was only a remembrance of his seafaring days.

Anyhow, towards the end of his life he became more and more odd and morose. He quarrelled with his family, and turned his younger son out of doors with his wife and children. An unpleasant old fellow.

On his deathbed he was attended by the parson—a good, earnest, God-fearing sort of man, who must have put up with a deal of insult in carrying out what he firmly believed to be the sacred duty of reconciling the old man to this shamefully treated son. Eventually, »Old Cut-throat« relented so far as to make a will, leaving to the younger son »My treasure which I have buried in Munster.« The parson represented to him that it was useless to bequeath a treasure unless he also bequeathed the information where to find it, but the horrid old pirate only chuckled spitefully, and said that, as he had been at the pains to collect the treasure, his son might well be at the pains of looking for it. Further than that he would not go, and so he died, and I dare say went to a very bad place.

Since then the family has died out, and I am the sole representative of the Conyers, and heir to the treasure, whatever and wherever it is, for it was never discovered. I do not suppose it was very honestly come by, but, since it would be useless now to try and find the original owners, I imagine I have a better right to it than anybody living.

You may think it very unseemly, Lord Peter, that an old, lonely man like myself should be greedy for a hoard of pirate's gold. But my whole life has been devoted to studying the disease of cancer, and I believe myself to be very close to a solution of one part at least of the terrible problem. Research costs money, and my limited means are very nearly exhausted. The property is mortgaged up to the hilt, and I do most urgently desire to complete my experiments before I die, and to leave a sufficient sum to found a clinic where the work can be carried on.

During the last year I have made very great efforts to solve the mystery of »Old Cut-throat's« treasure. I have been able to leave much of my experimental work in the most capable hands of my assistant, Dr Forbes, while I pursued my researches with the very slender clue I had to go upon. It was the more expensive and difficult that Cuthbert had left no indication in his will whether Münster in Germany or Munster in Ireland was the hiding-place of the treasure. My journeys and my search in both places cost money and brought me no further on my quest. I returned, disheartened, in August, and found myself obliged to sell my library, in order to defray my expenses and obtain a little money with which to struggle on with my sadly delayed experiments.«

»Ah!« said Lord Peter. »I begin to see light.«

The old physician looked at him enquiringly. They had finished tea, and were seated around the great fireplace in the study. Lord Peter's interested questions about the beautiful, dilapidated old house and estate had led the conversation naturally to Dr Conyers's family, shelving for the time the problem of the \textit{Cosmographia}, which lay on a table beside them.

»Everything you say fits into the puzzle,« went on Wimsey, »and I think there's not the smallest doubt what Mr Wilberforce Pope was after, though how he knew that you had the \textit{Cosmographia} here I couldn't say.«

»When I disposed of the library, I sent him a catalogue,« said Dr Conyers. »As a relative, I thought he ought to have the right to buy anything he fancied. I can't think why he didn't secure the book then, instead of behaving in this most shocking fashion.«

Lord Peter hooted with laughter.

»Why, because he never tumbled to it till afterwards,« he said. »And oh, dear, how wild he must have been! I forgive him everything. Although,« he added, »I don't want to raise your hopes too high, sir, for, even when we've solved old Cuthbert's riddle, I don't know that we're very much nearer to the treasure.«

»To the \textit{treasure}?«

»Well, now, sir. I want you first to look at this page, where there's a name scrawled in the margin. Our ancestors had an untidy way of signing their possessions higgledy-piggledy in margins instead of in a decent, Christian way in the fly-leaf. This is a handwriting of somewhere about Charles \textsc{i}'s reign: »Jac: Coniers.« I take it that goes to prove that the book was in the possession of your family at any rate as early as the first half of the seventeenth century, and has remained there ever since. Right. Now we turn to page 1099, where we find a description of the discoveries of Christopher Columbus. It's headed, you see, by a kind of map, with some of Mr Pope's monsters swimming about in it, and apparently representing the Canaries, or, as they used to be called, the Fortunate Isles. It doesn't look much more accurate than old maps usually are, but I take it the big island on the right is meant for Lanzarote, and the two nearest to it may be Teneriffe and Gran Canaria.«

»But what's that writing in the middle?«

»That's just the point. The writing is later than »Jac: Coniers's« signature; I should put it about 1700—but, of course, it may have been written a good deal later still. I mean, a man who was elderly in 1730 would still use the style of writing he adopted as a young man, especially if, like your ancestor the pirate, he had spent the early part of his life in outdoor pursuits and hadn't done much writing.«

»Do you mean to say, Uncle Peter,« broke in the viscount excitedly, »that that's »Old Cut-throat's« writing?«

»I'd be ready to lay a sporting bet it is. Look here, sir, you've been scouring round Münster in Germany and Munster in Ireland—but how about good old Sebastian Munster here in the library at home?«

»God bless my soul! Is it possible?«

»It's pretty nearly certain, sir. Here's what he says, written, you see, round the head of that sort of sea-dragon:
\begin{quote}
Hic in capite draconis ardet perpetuo Sol.\\
Here the sun shines perpetually upon the Dragon's Head.
\end{quote}«

\begin{center}\bfseries
\textsc{The Dragon's Head}\\
Liber V.\\
1099\\
\textsc{De Novis Insvlis,}\\
quomodo, quando, \& per quem\\
illæ inuentæ sint.\\
Christophorus Columbus natione Genuensis, cùm diu in aula regis Hispanorum deuersarus fuisset, animum induxit, ut hactenus inacceslias orbis partes peragraret. Pet à rege, utuoto suo non deesset, futurum sibi \& toti Hisp
\end{center}

\noindent\hfil\rule{0.5\textwidth}{.4pt}\hfil 

»Rather doggy Latin—sea-dog Latin, you might say, in fact.«

»I'm afraid,« said Dr Conyers, »I must be very stupid, but I can't see where that leads us.«

»No; »Old Cut-throat« was rather clever. No doubt he thought that, if anybody read it, they'd think it was just an allusion to where it says, further down, that »the islands were called \textit{Fortunatæ} because of the wonderful temperature of the air and the clemency of the skies.« But the cunning old astrologer up in his pagoda had a meaning of his own. Here's a little book published in 1678—Middleton's \textit{Practical Astrology}—just the sort of popular handbook an amateur like »Old Cut-throat« would use. Here you are: »If in your figure you find Jupiter or Venus or \textit{Dragon's head}, you may be confident there is Treasure in the place supposed.... If you find \textit{Sol} to be the significator of the hidden Treasure, you may conclude there is Gold, or some jewels.« You know, sir, I think we may conclude it.«

»Dear me!« said Dr Conyers. »I believe, indeed, you must be right. And I am ashamed to think that if anybody had suggested to me that it could ever be profitable to me to learn the terms of astrology, I should have replied in my vanity that my time was too valuable to waste on such foolishness. I am deeply indebted to you.«

»Yes,« said Gherkins, »but where \textit{is} the treasure, uncle?«

»That's just it,« said Lord Peter. »The map is very vague; there is no latitude or longitude given; and the directions, such as they are, seem not even to refer to any spot on the islands, but to some place in the middle of the sea. Besides, it is nearly two hundred years since the treasure was hidden, and it may already have been found by somebody or other.«

Dr Conyers stood up.

»I am an old man,« he said, »but I still have some strength. If I can by any means get together the money for an expedition, I will not rest till I have made every possible effort to find the treasure and to endow my clinic.«

»Then, sir, I hope you'll let me give a hand to the good work,« said Lord Peter.

\noindent\hfil\rule{0.5\textwidth}{.4pt}\hfil 


Dr Conyers had invited his guests to stay the night, and, after the excited viscount had been packed off to bed, Wimsey and the old man sat late, consulting maps and diligently reading Munster's chapter »\textit{De Novis Insulis},« in the hope of discovering some further clue. At length, however, they separated, and Lord Peter went upstairs, the book under his arm. He was restless, however, and, instead of going to bed, sat for a long time at his window, which looked out upon the lake. The moon, a few days past the full, was riding high among small, windy clouds, and picked out the sharp eaves of the Chinese tea-houses and the straggling tops of the unpruned shrubs. »Old Cut-throat« and his landscape-gardening! Wimsey could have fancied that the old pirate was sitting now beside his telescope in the preposterous pagoda, chuckling over his riddling testament and counting the craters of the moon. »If \textit{Luna}, there is silver.« The water of the lake was silver enough; there was a great smooth path across it, broken by the sinister wedge of the boat-house, the black shadows of the islands, and, almost in the middle of the lake, a decayed fountain, a writhing Celestial dragon-shape, spiny-backed and ridiculous.

Wimsey rubbed his eyes. There was something strangely familiar about the lake; from moment to moment it assumed the queer unreality of a place which one recognises without having ever known it. It was like one's first sight of the Leaning Tower of Pisa—too like its picture to be quite believable. Surely, thought Wimsey, he knew that elongated island on the right, shaped rather like a winged monster, with its two little clumps of buildings. And the island to the left of it, like the British Isles, but warped out of shape. And the third island, between the others, and nearer. The three formed a triangle, with the Chinese fountain in the centre, the moon shining steadily upon its dragon head. »\textit{Hic in capite draconis ardet perpetuo}\longdash«

Lord Peter sprang up with a loud exclamation, and flung open the door into the dressing-room. A small figure wrapped in an eiderdown hurriedly uncoiled itself from the window-seat.

»I'm sorry, Uncle Peter,« said Gherkins. »I was so \textit{dreadfully} wide awake, it wasn't any good staying in bed.«

»Come here,« said Lord Peter, »and tell me if I'm mad or dreaming. Look out of the window and compare it with the map—Old Cut-throat's »New Islands.« He made `em, Gherkins; he put `em here. Aren't they laid out just like the Canaries? Those three islands in a triangle, and the fourth down here in the corner? And the boat-house where the big ship is in the picture? And the dragon fountain where the dragon's head is? Well, my son, that's where your hidden treasure's gone to. Get your things on, Gherkins, and damn the time when all good little boys should be in bed! We're going for a row on the lake, if there's a tub in that boat-house that'll float.«

»Oh, Uncle Peter! This is a \textit{real} adventure!«

»All right,« said Wimsey. »Fifteen men on the dead man's chest, and all that! Yo-ho-ho, and a bottle of Johnny Walker! Pirate expedition fitted out in dead of night to seek hidden treasure and explore the Fortunate Isles! Come on, crew!«

\noindent\hfil\rule{0.5\textwidth}{.4pt}\hfil 

Lord Peter hitched the leaky dinghy to the dragon's knobbly tail and climbed out carefully, for the base of the fountain was green and weedy.

»I'm afraid it's your job to sit there and bail, Gherkins,« he said. »All the best captains bag the really interesting jobs for themselves. We'd better start with the head. If the old blighter said head, he probably meant it.« He passed an arm affectionately round the creature's neck for support, while he methodically pressed and pulled the various knobs and bumps of its anatomy. »It seems beastly solid, but I'm sure there's a spring somewhere. You won't forget to bail, will you? I'd simply hate to turn round and find the boat gone. Pirate chief marooned on island and all that. Well, it isn't its back hair, anyhow. We'll try its eyes. I say, Gherkins, I'm sure I felt something move, only it's frightfully stiff. We might have thought to bring some oil. Never mind; it's dogged as does it. It's coming. It's coming. Booh! Pah!«

A fierce effort thrust the rusted knob inwards, releasing a huge spout of water into his face from the dragon's gaping throat. The fountain, dry for many years, soared rejoicingly heavenwards, drenching the treasure-hunters, and making rainbows in the moonlight.

»I suppose this is »Old Cut-throat's« idea of humour,« grumbled Wimsey, retreating cautiously round the dragon's neck. »And now I can't turn it off again. Well, dash it all, let's try the other eye.«

He pressed for a few moments in vain. Then, with a grinding clang, the bronze wings of the monster clapped down to its sides, revealing a deep square hole, and the fountain ceased to play.

»Gherkins!« said Lord Peter, »we've done it. (But don't neglect bailing on that account!) There's a box here. And it's beastly heavy. No; all right, I can manage. Gimme the boat-hook. Now I do hope the old sinner really did have a treasure. What a bore if it's only one of his little jokes. Never mind—hold the boat steady. There. Always remember, Gherkins, that you can make quite an effective crane with a boat-hook and a stout pair of braces. Got it? That's right. Now for home and beauty.... Hullo! what's all that?«

As he paddled the boat round, it was evident that something was happening down by the boat-house. Lights were moving about, and a sound of voices came across the lake.

»They think we're burglars, Gherkins. Always misunderstood. Give way, my hearties\longdash«

\begin{quote}
A-roving, a-roving, since roving's been my ru-i-in,\\
I'll go no more a-roving with you, fair maid.
\end{quote}

»Is that you, my lord?« said a man's voice as they drew in to the boat-house.

»Why, it's our faithful sleuths!« cried his lordship. »What's the excitement?«

»We found this fellow sneaking round the boat-house,« said the man from Scotland Yard. »He says he's the old gentleman's nephew. Do you know him, my lord?«

»I rather fancy I do,« said Wimsey. »Mr Pope, I think. Good evening. Were you looking for anything? Not a treasure, by any chance? Because we've just found one. Oh! don't say that. \textit{Maxima reverentia}, you know. Lord St George is of tender years. And, by the way, thank you so much for sending your delightful friends to call on me last night. Oh, yes, Thompson, I'll charge him all right. You there, doctor? Splendid. Now, if anybody's got a spanner or anything handy, we'll have a look at Great-grandpapa Cuthbert. And if he turns out to be old iron, Mr Pope, you'll have had an uncommonly good joke for your money.«

An iron bar was produced from the boat-house and thrust under the hasp of the chest. It creaked and burst. Dr Conyers knelt down tremulously and threw open the lid.

There was a little pause.

»The drinks are on you, Mr Pope,« said Lord Peter. »I think, doctor, it ought to be a jolly good hospital when it's finished.«