%!TeX root=../viewsbodytop.tex
\addchap{The Unprincipled Affair of the Practical Joker}

\lettrine[lines=4]{T}{he} \textit{Zambesi}, they said, was expected to dock at six in the morning. Mrs~Ruyslaender booked a bedroom at the Magnifical, with despair in her heart. A bare nine hours and she would be greeting her husband. After that would begin the sickening period of waiting—it might be days, it might be weeks, possibly even months—for the inevitable discovery.

The reception-clerk twirled the register towards her. Mechanically, as she signed it, she glanced at the preceding entry:

<Lord~Peter Wimsey and valet—London—Suite 24.>

Mrs~Ruyslaender's heart seemed to stop for a second. Was it possible that, even now, God had left a loophole? She expected little from Him—all her life He had shown Himself a sufficiently stern creditor. It was fantastic to base the frailest hope on this signature of a man she had never even seen.

Yet the name remained in her mind while she dined in her own room. She dismissed her maid presently, and sat for a long time looking at her own haggard reflection in the mirror. Twice she rose and went to the door—then turned back, calling herself a fool. The third time she turned the handle quickly and hurried down the corridor, without giving herself time to think.

A large golden arrow at the corner directed her to Suite 24. It was 11 o'clock, and nobody was within view. Mrs~Ruyslaender gave a sharp knock on Lord~Peter Wimsey's door and stood back, waiting, with the sort of desperate relief one experiences after hearing a dangerous letter thump the bottom of the pillar-box. Whatever the adventure, she was committed to it.

The manservant was of the imperturbable sort. He neither invited nor rejected, but stood respectfully upon the threshold.

<Lord~Peter Wimsey?> murmured Mrs~Ruyslaender.

<Yes, madam.>

<Could I speak to him for a moment?>

<His Lordship has just retired, madam. If you will step in, I will enquire.>

Mrs~Ruyslaender followed him into one of those palatial sitting-rooms which the Magnifical provides for the wealthy pilgrim.

<Will you take a seat, madam?>

The man stepped noiselessly to the bedroom door and passed in, shutting it behind him. The lock, however, failed to catch, and Mrs~Ruyslaender caught the conversation.

<Pardon me, my lord, a lady has called. She mentioned no appointment, so I considered it better to acquaint your lordship.>

<Excellent discretion,> said a voice. It had a slow, sarcastic intonation, which brought a painful flush to Mrs~Ruyslaender's cheek. <I never make appointments. Do I know the lady?>

<No, my lord. But—hem—I know her by sight, my lord. It is Mrs~Ruyslaender.>

<Oh, the diamond-merchant's wife. Well, find out tactfully what it's all about, and, unless it's urgent, ask her to call to-morrow.>

The valet's next remark was inaudible, but the reply was:

<Don't be coarse, Bunter.>

The valet returned.

<His lordship desires me to ask you, madam, in what way he can be of service to you?>

<Will you say to him that I have heard of him in connection with the Attenbury diamond case, and am anxious to ask his advice.>

<Certainly, madam. May I suggest that, as his lordship is greatly fatigued, he would be better able to assist you after he has slept.>

<If to-morrow would have done, I would not have thought of disturbing him to-night. Tell him, I am aware of the trouble I am giving\longdash>

<Excuse me one moment, madam.>

This time the door shut properly. After a short interval Bunter returned to say, <His lordship will be with you immediately, madam,> and to place a decanter of wine and a box of Sobranies beside her.

Mrs~Ruyslaender lit a cigarette, but had barely sampled its flavour when she was aware of a soft step beside her. Looking round, she perceived a young man, attired in a mauve dressing-gown of great splendour, from beneath the hem of which peeped coyly a pair of primrose silk pyjamas.

<You must think it very strange of me, thrusting myself on you at this hour,> she said, with a nervous laugh.

Lord~Peter put his head on one side.

<Don't know the answer to that,> he said. <If I say, <Not at all,> it sounds abandoned. If I say, <Yes, very,> it's rude. Supposin' we give it a miss, what? and you tell me what I can do for you.>

Mrs~Ruyslaender hesitated. Lord~Peter was not what she had expected. She noted the sleek, straw-coloured hair, brushed flat back from a rather sloping forehead, the ugly, lean, arched nose, and the faintly foolish smile, and her heart sank within her.

<I—I'm afraid it's ridiculous of me to suppose you can help me,> she began.

<Always my unfortunate appearance,> moaned Lord~Peter, with such alarming acumen as to double her discomfort. <Would it invite confidence more, d'you suppose, if I dyed my hair black an' grew a Newgate fringe? It's very tryin', you can't think, always to look as if one's name was Algy.>

<I only meant,> said Mrs~Ruyslaender, <that I don't think \textit{anybody} could possibly help. But I saw your name in the hotel book, and it seemed just a chance.>

Lord~Peter filled the glasses and sat down.

<Carry on,> he said cheerfully; <it sounds interestin'.>

Mrs~Ruyslaender took the plunge.

<My husband,> she explained, <is Henry Ruyslaender, the diamond merchant. We came over from Kimberley ten years ago, and settled in England. He spends several months in Africa every year on business, and I am expecting him back on the \textit{Zambesi} to-morrow morning. Now, this is the trouble. Last year he gave me a magnificent diamond necklace of a hundred and fifteen stones\longdash>

<The Light of Africa—I know,> said Wimsey.

She looked a little surprised, but assented. <The necklace has been stolen from me, and I can't hope to conceal the loss from him. No duplicate would deceive him for an instant.>

She paused, and Lord~Peter prompted gently:

<You have come to me, I presume, because it is not to be a police matter. Will you tell me quite frankly why?>

<The police would be useless. I know who took it.>

<Yes?>

<There is a man we both know slightly—a man called Paul Melville.>

Lord~Peter's eyes narrowed. <M'm, yes, I fancy I've seen him about the clubs. New Army, but transferred himself into the Regulars. Dark. Showy. Bit of an ampelopsis, what?>

<Ampelopsis?>

<Surburban plant that climbs by suction. \textit{You} know—first year, tender little shoots—second year, fine show—next year, all over the shop. Now tell me I am rude.>

Mrs~Ruyslaender giggled. <Now you mention it, he is \textit{exactly} like an ampelopsis. What a relief to be able to think of him as that.... Well, he is some sort of distant relation of my husband's. He called one evening when I was alone. We talked about jewels, and I brought down my jewel-box and showed him the Light of Africa. He knows a good deal about stones. I was in and out of the room two or three times, but didn't think to lock up the box. After he left, I was putting the things away, and I opened the jeweller's case the diamonds were in—and they had gone!>

<H'm—pretty bare-faced. Look here, Mrs~Ruyslaender, you agree he's an ampelopsis, but you won't call in the police. Honestly, now—forgive me; you're askin' my advice, you know—is he worth botherin' about?>

<It's not that,> said the woman, in a low tone. <Oh, no! But he took something else as well. He took—a portrait—a small painting set with diamonds.>

<Oh!>

<Yes. It was in a secret drawer in the jewel-box. I can't imagine how he knew it was there, but the box was an old casket, belonging to my husband's family, and I fancy he must have known about the drawer and—well, thought that investigation might prove profitable. Anyway, the evening the diamonds went the portrait went too, and he knows I daren't try to get the necklace back because they'd both be found together.>

<Was there something more than just the portrait, then? A portrait in itself isn't necessarily hopeless of explanation. It was given you to take care of, say.>

<The names were on it—and—and an inscription which nothing, \textit{nothing} could ever explain away. A—a passage from Petronius.>

<Oh, dear!> said Lord~Peter, <dear me, yes. Rather a lively author.>

<I was married very young,> said Mrs~Ruyslaender, <and my husband and I have never got on well. Then one year, when he was in Africa, it all happened. We were wonderful—and shameless. It came to an end. I was bitter. I wish I had not been. He left me, you see, and I couldn't forgive it. I prayed day and night for revenge. Only now—I don't want it to be through me!>

<Wait a moment,> said Wimsey, <you mean that, if the diamonds are found and the portrait is found too, all this story is bound to come out.>

<My husband would get a divorce. He would never forgive me—or him. It is not so much that I mind paying the price myself, but\longdash>

She clenched her hands.

<I have cursed him again and again, and the clever girl who married him. She played her cards so well. This would ruin them both.>

<But if \textit{you} were the instrument of vengeance,> said Wimsey gently, <you would hate yourself. And it would be terrible to you because he would hate you. A woman like you couldn't stoop to get your own back. I see that. If God makes a thunderbolt, how awful and satisfying—if you help to make a beastly row, what a rotten business it would be.>

<You seem to understand,> said Mrs~Ruyslaender. <How unusual.>

<I understand perfectly. Though let me tell you,> said Wimsey, with a wry little twist of the lips, <that it's sheer foolishness for a woman to have a sense of honour in such matters. It only gives her excruciating pain, and nobody expects it, anyway. Look here, don't let's get all worked up. You certainly shan't have your vengeance thrust upon you by an ampelopsis. Why should you? Nasty fellow. We'll have him up—root, branch, and little suckers. Don't worry. Let's see. My business here will only take a day. Then I've got to get to know Melville—say a week. Then I've got to get the doings—say another week, provided he hasn't sold them yet, which isn't likely. Can you hold your husband off 'em for a fortnight, d'you think?>

<Oh, yes. I'll say they're in the country, or being cleaned, or something. But do you really think you can—?>

<I'll have a jolly good try, anyhow, Mrs~Ruyslaender. Is the fellow hard up, to start stealing diamonds?>

<I fancy he has got into debt over horses lately. And possibly poker.>

<Oh! Poker player, is he? That makes an excellent excuse for gettin' to know him. Well, cheer up—we'll get the goods, even if we have to buy 'em. But we won't, if we can help it. Bunter!>

<My lord?> The valet appeared from the inner room.

<Just go an' give the <All Clear,> will you?>

Mr~Bunter accordingly stepped into the passage, and, having seen an old gentleman safely away to the bathroom and a young lady in a pink kimono pop her head out of an adjacent door and hurriedly pop it back on beholding him, blew his nose with a loud, trumpeting sound.

<Good night,> said Mrs~Ruyslaender, <and thank you.>

She slipped back to her room unobserved.

<Whatever has induced you, my dear boy,> said Colonel Marchbanks, <to take up with that very objectionable fellow Melville?>

<Diamonds,> said Lord~Peter. <Do you find him so, really?>

<Perfectly dreadful man,> said the Hon. Frederick Arbuthnot. <Hearts. What did you want to go and get him a room here for? This used to be a quite decent club.>

<Two clubs?> said Sir Impey Biggs, who had been ordering a whisky, and had only caught the last word.

<No, no, one heart.>

<I beg your pardon. Well, partner, how about spades? Perfectly good suit.>

<Pass,> said the Colonel. <I don't know what the Army's coming to nowadays.>

<No trumps,> said Wimsey. <It's all right, children. Trust your Uncle Pete. Come on, Freddy, how many of those hearts are you going to shout for?>

<None, the Colonel havin' let me down so 'orrid,> said the Hon. Freddy.

<Cautious blighter. All content? Righty-ho! Bring out your dead, partner. Oh, very pretty indeed. We'll make it a slam this time. I'm rather glad to hear that expression of opinion from you, Colonel, because I particularly want you and Biggy to hang on this evening and take a hand with Melville and me.>

<What happens to me?> enquired the Hon. Freddy.

<You have an engagement and go home early, dear old thing. I've specially invited friend Melville to meet the redoubtable Colonel Marchbanks and our greatest criminal lawyer. Which hand am I supposed to be playin' this from? Oh, yes. Come on, Colonel—you've got to hike that old king out some time, why not now?>

<It's a plot,> said Mr~Arbuthnot, with an exaggerated expression of mystery. <Carry on, don't mind me.>

<I take it you have your own reasons for cultivating the man,> said Sir Impey.

<The rest are mine, I fancy. Well, yes, I have. You and the Colonel would really do me a favour by letting Melville cut in to-night.>

<If you wish it,> growled the Colonel, <but I hope the impudent young beggar won't presume on the acquaintance.>

<I'll see to that,> said his lordship. <Your cards, Freddy. Who had the ace of hearts? Oh! I had it myself, of course. Our honours.... Hullo! Evenin', Melville.>

The ampelopsis was rather a good-looking creature in his own way. Tall and bronzed, with a fine row of very persuasive teeth. He greeted Wimsey and Arbuthnot heartily, the Colonel with a shade too much familiarity, and expressed himself delighted to be introduced to Sir Impey Biggs.

<You're just in time to hold Freddy's hand,> said Wimsey; <he's got a date. Not his little paddy-paw, I don't mean—but the dam' rotten hand he generally gets dealt him. Joke.>

<Oh, well,> said the obedient Freddy, rising, <I s'pose I'd better make a noise like a hoop and roll away. Night, night, everybody.>

Melville took his place, and the game continued with varying fortunes for two hours, at the end of which time Colonel Marchbanks, who had suffered much under his partner's eloquent theory of the game, was beginning to wilt visibly.

Wimsey yawned.

<Gettin' a bit bored, Colonel? Wish they'd invent somethin' to liven this game up a bit.>

<Oh, Bridge is a one-horse show, anyway,> said Melville. <Why not have a little flutter at poker, Colonel? Do you all the good in the world. What d'you say, Biggs?>

Sir Impey turned on Wimsey a thoughtful eye, accustomed to the sizing-up of witnesses. Then he replied:

<I'm quite willing, if the others are.>

<Damn good idea,> said Lord~Peter. <Come now, Colonel, be a sport. You'll find the chips in that drawer, I think. I always lose money at poker, but what's the odds so long as you're happy. Let's have a new pack.>

<Any limit?>

<What do \textit{you} say, Colonel?>

The Colonel proposed a twenty-shilling limit. Melville, with a grimace, amended this to one-tenth of the pool. The amendment was carried and the cards cut, the deal falling to the Colonel.

Contrary to his own prophecy, Wimsey began by winning considerably, and grew so garrulously imbecile in the process that even the experienced Melville began to wonder whether this indescribable fatuity was the cloak of ignorance or the mask of the hardened poker-player. Soon, however, he was reassured. The luck came over to his side, and he found himself winning hands down, steadily from Sir Impey and the Colonel, who played cautiously and took little risk—heavily from Wimsey, who appeared reckless and slightly drunk, and was staking foolishly on quite impossible cards.

<I never knew such luck as yours, Melville,> said Sir Impey, when that young man had scooped in the proceeds from a handsome straight-flush.

<My turn to-night, yours to-morrow,> said Melville, pushing the cards across to Biggs, whose deal it was.

Colonel Marchbanks required one card. Wimsey laughed vacantly and demanded an entirely fresh hand; Biggs asked for three; and Melville, after a pause for consideration, took one.

It seemed as though everybody had something respectable this time—though Wimsey was not to be depended upon, frequently going the limit upon a pair of jacks in order, as he expressed it, to keep the pot a-boiling. He became peculiarly obstinate now, throwing his chips in with a flushed face, in spite of Melville's confident air.

The Colonel got out, and after a short time Biggs followed his example. Melville held on till the pool mounted to something under a hundred pounds, when Wimsey suddenly turned restive and demanded to see him.

<Four kings,> said Melville.

<Blast you!> said Lord~Peter, laying down four queens. <No holdin' this feller to-night, is there? Here, take the ruddy cards, Melville, and give somebody else a look in, will you.>

He shuffled them as he spoke, and handed them over. Melville dealt, satisfied the demands of the other three players, and was in the act of taking three new cards for himself, when Wimsey gave a sudden exclamation, and shot a swift hand across the table.

<Hullo! Melville,> he said, in a chill tone which bore no resemblance to his ordinary speech, <what exactly does this mean?>

He lifted Melville's left arm clear of the table and, with a sharp gesture, shook it. From the sleeve something fluttered to the table and glided away to the floor. Colonel Marchbanks picked it up, and in a dreadful silence laid the joker on the table.

<Good God!> said Sir Impey.

<You young blackguard!> gasped the Colonel, recovering speech.

<What the hell do you mean by this?> gasped Melville, with a face like chalk. <How dare you! This is a trick—a plant—> A horrible fury gripped him. <You dare to say that I have been cheating. You liar! You filthy sharper. You put it there. I tell you, gentlemen,< he cried, looking desperately round the table, >he must have put it there.>

<Come, come,> said Colonel Marchbanks, <no good carryin' on that way, Melville. Dear me, no good at all. Only makes matters worse. We all saw it, you know. Dear, dear, I don't know what the Army's coming to.>

<Do you mean you believe it?> shrieked Melville. <For God's sake, Wimsey, is this a joke or what? Biggs—you've got a head on your shoulders—are you going to believe this half-drunk fool and this doddering old idiot who ought to be in his grave?>

<That language won't do you any good, Melville,> said Sir Impey. <I'm afraid we all saw it clearly enough.>

<I've been suspectin' this some time, y'know,> said Wimsey. <That's why I asked you two to stay to-night. We don't want to make a public row, but\longdash>

<Gentlemen,> said Melville more soberly, <I swear to you that I am absolutely innocent of this ghastly thing. Can't you believe me?>

<I can believe the evidence of my own eyes, sir,> said the Colonel, with some heat.

<For the good of the club,> said Wimsey, <this couldn't go on, but—also for the good of the club—I think we should all prefer the matter to be quietly arranged. In the face of what Sir Impey and the Colonel can witness, Melville, I'm afraid your protestations are not likely to be credited.>

Melville looked from the soldier's face to that of the great criminal lawyer.

<I don't know what your game is,> he said sullenly to Wimsey, <but I can see you've laid a trap and pulled it off all right.>

<I think, gentlemen,> said Wimsey, <that, if I might have a word in private with Melville in his own room, I could get the thing settled satisfactorily, without undue fuss.>

<He'll have to resign his commission,> growled the Colonel.

<I'll put it to him in that light,> said Peter. <May we go to your room for a minute, Melville?>

With a lowering brow, the young soldier led the way. Once alone with Wimsey, he turned furiously on him.

<What do you want? What do you mean by making this monstrous charge? I'll take action for libel!>

<Do,> said Wimsey coolly, <if you think anybody is likely to believe your story.>

He lit a cigarette, and smiled lazily at the angry young man.

<Well, what's the meaning of it, anyway?>

<The meaning,> said Wimsey, <is simply that you, an officer and a member of this club, have been caught red-handed cheating at cards while playing for money, the witnesses being Sir Impey Biggs, Colonel Marchbanks, and myself. Now, I suggest to you, Captain Melville, that your best plan is to let me take charge of Mrs~Ruyslaender's diamond necklace and portrait, and then just to trickle away quiet-like from these halls of dazzlin' light—without any questions asked.>

Melville leapt to his feet.

<My God!> he cried. <I can see it now. It's blackmail.>

<You may certainly call it blackmail, and theft too,> said Lord~Peter, with a shrug. <But why use ugly names? I hold five aces, you see. Better chuck in your hand.>

<Suppose I say I never heard of the diamonds?>

<It's a bit late now, isn't it?> said Wimsey affably. <But, in that case, I'm beastly sorry and all that, of course, but we shall have to make to-night's business public.>

<Damn you!> muttered Melville, <you sneering devil.>

He showed all his white teeth, half springing, with crouched shoulders. Wimsey waited quietly, his hands in his pockets.

The rush did not come. With a furious gesture, Melville pulled out his keys and unlocked his dressing-case.

<Take them,> he growled, flinging a small parcel on the table; <you've got me. Take 'em and go to hell.>

<Eventually—why not now?> murmured his lordship. <Thanks frightfully. Man of peace myself, you know—hate unpleasantness and all that.> He scrutinised his booty carefully, running the stones expertly between his fingers. Over the portrait he pursed up his lips. <Yes,> he murmured, <that \textit{would} have made a row.> He replaced the wrapping and slipped the parcel into his pocket.

<Well, good night, Melville—and thanks for a pleasant game.>

<I say, Biggs,> said Wimsey, when he had returned to the card-room. <You've had a lot of experience. What tactics d'you think one's justified in usin' with a blackmailer?>

<Ah!> said the \textsc{k.c.} <There you've put your finger on Society's sore place, where the Law is helpless. Speaking as a man, I'd say nothing could be too bad for the brute. It's a crime crueller and infinitely worse in its results than murder. As a lawyer, I can only say that I have consistently refused to defend a blackmailer or to prosecute any poor devil who does away with his tormentor.>

<H'm,> replied Wimsey. <What do you say, Colonel?>

<A man like that's a filthy pest,> said the little warrior stoutly. <Shootin's too good for him. I knew a man—close personal friend, in fact—hounded to death—blew his brains out—one of the best. Don't like to talk about it.>

<I want to show you something,> said Wimsey.

He picked up the pack which still lay scattered on the table, and shuffled it together.

<Catch hold of these, Colonel, and lay 'em out face downwards. That's right. First of all you cut 'em at the twentieth card—you'll see the seven of diamonds at the bottom. Correct? Now I'll call 'em. Ten of hearts, ace of spades, three of clubs, five of clubs, king of diamonds, nine, jack, two of hearts. Right? I could pick 'em all out, you see, except the ace of hearts, and that's here.>

He leaned forward and produced it dexterously from Sir Impey's breast-pocket.

<I learnt it from a man who shared my dug-out near Ypres,> he said. <You needn't mention to-night's business, you two. There are crimes which the Law cannot reach.>