%!TeX root=../viewsbodytop.tex
\addchap{The Abominable History of the Man with Copper Fingers}

\lettrine[lines=4]{T}{he} Egotists' Club is one of the most genial places in London. It is a place to which you may go when you want to tell that odd dream you had last night, or to announce what a good dentist you have discovered. You can write letters there if you like, and have the temperament of a Jane Austen, for there is no silence room, and it would be a breach of club manners to appear busy or absorbed when another member addresses you. You must not mention golf or fish, however, and, if the Hon. Freddy Arbuthnot's motion is carried at the next committee meeting (and opinion so far appears very favourable), you will not be allowed to mention wireless either. As Lord~Peter Wimsey said when the matter was mooted the other day in the smoking-room, those are things you can talk about anywhere. Otherwise the club is not specially exclusive. Nobody is ineligible \textit{per se}, except strong, silent men. Nominees are, however, required to pass certain tests, whose nature is sufficiently indicated by the fact that a certain distinguished explorer came to grief through accepting, and smoking, a powerful Trichinopoly cigar as an accompaniment to a '63 port. On the other hand, dear old Sir Roger Bunt (the coster millionaire who won the \textsterling 20,000 ballot offered by the \textit{Sunday Shriek}, and used it to found his immense catering business in the Midlands) was highly commended and unanimously elected after declaring frankly that beer and a pipe were all he really cared for in that way. As Lord~Peter said again: <Nobody minds coarseness but one must draw the line at cruelty.>

On this particular evening, Masterman (the cubist poet) had brought a guest with him, a man named Varden. Varden had started life as a professional athlete, but a strained heart had obliged him to cut short a brilliant career, and turn his handsome face and remarkably beautiful body to account in the service of the cinema screen. He had come to London from Los Angeles to stimulate publicity for his great new film, \textit{Marathon}, and turned out to be quite a pleasant, unspoiled person—greatly to the relief of the club, since Masterman's guests were apt to be something of a toss-up.

There were only eight men, including Varden, in the brown room that evening. This, with its panelled walls, shaded lamps, and heavy blue curtains was perhaps the cosiest and pleasantest of the small smoking-rooms, of which the club possessed half a dozen or so. The conversation had begun quite casually by Armstrong's relating a curious little incident which he had witnessed that afternoon at the Temple Station, and Bayes had gone on to say that that was nothing to the really very odd thing which had happened to him, personally, in a thick fog one night in the Euston Road.

Masterman said that the more secluded London squares teemed with subjects for a writer, and instanced his own singular encounter with a weeping woman and a dead monkey, and then Judson took up the tale and narrated how, in a lonely suburb, late at night, he had come upon the dead body of a woman stretched on the pavement with a knife in her side and a policeman standing motionless near by. He had asked if he could do anything, but the policeman had only said, <I wouldn't interfere if I was you, sir; she deserved what she got.> Judson said he had not been able to get the incident out of his mind, and then Pettifer told them of a queer case in his own medical practice, when a totally unknown man had led him to a house in Bloomsbury where there was a woman suffering from strychnine poisoning. This man had helped him in the most intelligent manner all night, and, when the patient was out of danger, had walked straight out of the house and never reappeared; the odd thing being that, when he (Pettifer) questioned the woman, she answered in great surprise that she had never seen the man in her life and had taken him to be Pettifer's assistant.

<That reminds me,> said Varden, <of something still stranger that happened to me once in New York—I've never been able to make out whether it was a madman or a practical joke, or whether I really had a very narrow shave.>

This sounded promising, and the guest was urged to go on with his story.

<Well, it really started ages ago,> said the actor, <seven years it must have been—just before America came into the war. I was twenty-five at the time, and had been in the film business a little over two years. There was a man called Eric P\@. Loder, pretty well known in New York at that period, who would have been a very fine sculptor if he hadn't had more money than was good for him, or so I understood from the people who go in for that kind of thing. He used to exhibit a good deal and had a lot of one-man shows of his stuff to which the highbrow people went—he did a good many bronzes, I believe. Perhaps you know about him, Masterman?>

<I've never seen any of his things,> said the poet, <but I remember some photographs in \textit{The Art of To-Morrow}. Clever, but rather overripe. Didn't he go in for a lot of that chryselephantine stuff? Just to show he could afford to pay for the materials, I suppose.>

<Yes, that sounds very like him.>

<Of course—and he did a very slick and very ugly realistic group called Lucina, and had the impudence to have it cast in solid gold and stood in his front hall.>

<Oh, that thing! Yes—simply beastly I thought it, but then I never could see anything artistic in the idea. Realism, I suppose you'd call it. I like a picture or a statue to make you feel good, or what's it there for? Still, there was something very attractive about Loder.>

<How did you come across him?>

<Oh, yes. Well, he saw me in that little picture of mine, \textit{Apollo Comes to New York}—perhaps you remember it. It was my first star part. About a statue that's brought to life—one of the old gods, you know—and how he gets on in a modern city. Dear old Reubenssohn produced it. Now, there was a man who could put a thing through with consummate artistry. You couldn't find an atom of offence from beginning to end, it was all so tasteful, though in the first part one didn't have anything to wear except a sort of scarf—taken from the classical statue, you know.>

<The Belvedere?>

<I dare say. Well. Loder wrote to me, and said as a sculptor he was interested in me, because I was a good shape and so on, and would I come and pay him a visit in New York when I was free. So I found out about Loder, and decided it would be good publicity, and when my contract was up, and I had a bit of time to fill in, I went up east and called on him. He was very decent to me, and asked me to stay a few weeks with him while I was looking around.

He had a magnificent great house about five miles out of the city, crammed full of pictures and antiques and so on. He was somewhere between thirty-five and forty, I should think, dark and smooth, and very quick and lively in his movements. He talked very well; seemed to have been everywhere and have seen everything and not to have any too good an opinion of anybody. You could sit and listen to him for hours; he'd got anecdotes about everybody, from the Pope to old Phineas E\@. Groot of the Chicago Ring. The only kind of story I didn't care about hearing from him was the improper sort. Not that I don't enjoy an after-dinner story—no, sir, I wouldn't like you to think I was a prig—but he'd tell it with his eye upon you as if he suspected you of having something to do with it. I've known women do that, and I've seen men do it to women and seen the women squirm, but he was the only man that's ever given me that feeling. Still, apart from that, Loder was the most fascinating fellow I've ever known. And, as I say, his house surely was beautiful, and he kept a first-class table.

He liked to have everything of the best. There was his mistress, Maria Morano. I don't think I've ever seen anything to touch her, and when you work for the screen you're apt to have a pretty exacting standard of female beauty. She was one of those big, slow, beautifully moving creatures, very placid, with a slow, wide smile. We don't grow them in the States. She'd come from the South—had been a cabaret dancer he said, and she didn't contradict him. He was very proud of her, and she seemed to be devoted to him in her own fashion. He'd show her off in the studio with nothing on but a fig-leaf or so—stand her up beside one of the figures he was always doing of her, and compare them point by point. There was literally only one half inch of her, it seemed, that wasn't absolutely perfect from the sculptor's point of view—the second toe of her left foot was shorter than the big toe. He used to correct it, of course, in the statues. She'd listen to it all with a good-natured smile, sort of vaguely flattered, you know. Though I think the poor girl sometimes got tired of being gloated over that way. She'd sometimes hunt me out and confide to me that what she had always hoped for was to run a restaurant of her own, with a cabaret show and a great many cooks with white aprons, and lots of polished electric cookers. <And then I would marry,> she'd say, <and have four sons and one daughter,> and she told me all the names she had chosen for the family. I thought it was rather pathetic. Loder came in at the end of one of these conversations. He had a sort of a grin on, so I dare say he'd overheard. I don't suppose he attached much importance to it, which shows that he never really understood the girl. I don't think he ever imagined any woman would chuck up the sort of life he'd accustomed her to, and if he was a bit possessive in his manner, at least he never gave her a rival. For all his talk and his ugly statues, she'd got him, and she knew it.

I stayed there getting on for a month altogether, having a thundering good time. On two occasions Loder had an art spasm, and shut himself up in his studio to work and wouldn't let anybody in for several days on end. He was rather given to that sort of stunt, and when it was over we would have a party, and all Loder's friends and hangers-on would come to have a look at the work of art. He was doing a figure of some nymph or goddess, I fancy, to be cast in silver, and Maria used to go along and sit for him. Apart from those times, he went about everywhere, and we saw all there was to be seen.

I was fairly annoyed, I admit, when it came to an end. War was declared, and I'd made up my mind to join up when that happened. My heart put me out of the running for trench service, but I counted on getting some sort of a job, with perseverance, so I packed up and went off.

I wouldn't have believed Loder would have been so genuinely sorry to say good-bye to me. He said over and over again that we'd meet again soon. However, I did get a job with the hospital people, and was sent over to Europe, and it wasn't till 1920 that I saw Loder again.

He'd written to me before, but I'd had two big pictures to make in '19, and it couldn't be done. However, in '20 I found myself back in New York, doing publicity for \textit{The Passion Streak}, and got a note from Loder begging me to stay with him, and saying he wanted me to sit for him. Well, that was advertisement that he'd pay for himself, you know, so I agreed. I had accepted an engagement to go out with Mystofilms Ltd. in \textit{Jake of Dead Man's Bush}—the dwarfmen picture, you know, taken on the spot among the Australian bushmen. I wired them that I would join them at Sydney the third week in April, and took my bags out to Loder's.

Loder greeted me very cordially, though I thought he looked older than when I last saw him. He had certainly grown more nervous in his manner. He was—how shall I describe it?—more \textit{intense}—more real, in a way. He brought out his pet cynicisms as if he thoroughly meant them, and more and more with that air of getting at you personally. I used to think his disbelief in everything was a kind of artistic pose, but I began to feel I had done him an injustice. He was really unhappy, I could see that quite well, and soon I discovered the reason. As we were driving out in the car I asked after Maria.

<She has left me,> he said.

Well, now, you know, that really surprised me. Honestly, I hadn't thought the girl had that much initiative. <Why,> I said, <has she gone and set up in that restaurant of her own she wanted so much?>

<Oh! she talked to you about restaurants, did she?> said Loder. <I suppose you are one of the men that women tell things to. No. She made a fool of herself. She's gone.>

I didn't quite know what to say. He was so obviously hurt in his vanity, you know, as well as in his feelings. I muttered the usual things, and added that it must be a great loss to his work as well as in other ways. He said it was.

I asked him when it had happened and whether he'd finished the nymph he was working on before I left. He said, <Oh, yes, he'd finished that and done another—something pretty original, which I should like.>

Well, we got to the house and dined, and Loder told me he was going to Europe shortly, a few days after I left myself, in fact. The nymph stood in the dining-room, in a special niche let into the wall. It really was a beautiful thing, not so showy as most of Loder's work, and a wonderful likeness of Maria. Loder put me opposite it, so that I could see it during dinner, and, really, I could hardly take my eyes off it. He seemed very proud of it, and kept on telling me over and over again how glad he was that I liked it. It struck me that he was falling into a trick of repeating himself.

We went into the smoking-room after dinner. He'd had it rearranged, and the first thing that caught one's eye was a big settee drawn before the fire. It stood about a couple of feet from the ground, and consisted of a base made like a Roman couch, with cushions and a highish back, all made of oak with a silver inlay, and on top of this, forming the actual seat one sat on, if you follow me, there was a great silver figure of a nude woman, fully life-size, lying with her head back and her arms extended along the sides of the couch. A few big loose cushions made it possible to use the thing as an actual settee, though I must say it never was really comfortable to sit on respectably. As a stage prop. for registering dissipation it would have been excellent, but to see Loder sprawling over it by his own fireside gave me a kind of shock. He seemed very much attached to it, though.

<I told you,> he said, <that it was something original.>

Then I looked more closely at it, and saw that the figure actually was Maria's, though the face was rather sketchily done, if you understand what I mean. I suppose he thought a bolder treatment more suited to a piece of furniture.

But I did begin to think Loder a trifle degenerate when I saw that couch. And in the fortnight that followed I grew more and more uncomfortable with him. That personal manner of his grew more marked every day, and sometimes, while I was giving him sittings, he would sit there and tell one of the most beastly things, with his eyes fixed on one in the nastiest way, just to see how one would take it. Upon my word, though he certainly did me uncommonly well, I began to feel I'd be more at ease among the bushmen.

Well, now I come to the odd thing.>

Everybody sat up and listened a little more eagerly.

<It was the evening before I had to leave New York,> went on Varden. <I was sitting\longdash>

Here somebody opened the door of the brown room, to be greeted by a warning sign from Bayes. The intruder sank obscurely into a large chair and mixed himself a whisky with extreme care not to disturb the speaker.

<I was sitting in the smoking-room,> continued Varden, <waiting for Loder to come in. I had the house to myself, for Loder had given the servants leave to go to some show or lecture or other, and he himself was getting his things together for his European trip and had had to keep an appointment with his man of business. I must have been very nearly asleep, because it was dusk when I came to with a start and saw a young man quite close to me.

He wasn't at all like a housebreaker, and still less like a ghost. He was, I might almost say, exceptionally ordinary-looking. He was dressed in a grey English suit, with a fawn overcoat on his arm, and his soft hat and stick in his hand. He had sleek, pale hair, and one of those rather stupid faces, with a long nose and a monocle. I stared at him, for I knew the front door was locked, but before I could get my wits together he spoke. He had a curious, hesitating, husky voice and a strong English accent. He said, surprisingly:

<Are you Mr~Varden?>

<You have the advantage of me,> I said.

He said, <Please excuse my butting in; I know it looks like bad manners, but you'd better clear out of this place very quickly, don't you know.>

<What the hell do you mean?> I said.

He said, <I don't mean it in any impertinent way, but you must realise that Loder's never forgiven you, and I'm afraid he means to make you into a hat-stand or an electric-light fitting, or something of that sort.>

My God! I can tell you I felt queer. It was such a quiet voice, and his manners were perfect, and yet the words were quite meaningless! I remembered that madmen are supposed to be extra strong, and edged towards the bell—and then it came over me with rather a chill that I was alone in the house.

<How did you get here?> I asked, putting a bold face on it.

<I'm afraid I picked the lock,> he said, as casually as though he were apologising for not having a card about him. <I couldn't be sure Loder hadn't come back. But I do really think you had better get out as quickly as possible.>

<See here,> I said, <who are you and what the hell are you driving at? What do you mean about Loder never forgiving me? Forgiving me what?>

<Why,> he said, <about—you \textit{will} pardon me prancing in on your private affairs, won't you—about Maria Morano.>

<\textit{What} about her, in the devil's name?> I cried. <What do you know about her, anyway? She went off while I was at the war. What's it to do with me?>

<Oh!> said the very odd young man, <I beg your pardon. Perhaps I have been relying too much on Loder's judgment. Damned foolish; but the possibility of his being mistaken did not occur to me. He fancies you were Maria Morano's lover when you were here last time.>

<Maria's lover?> I said. <Preposterous! She went off with her man, whoever he was. He must know she didn't go with me.>

<Maria never left the house,> said the young man, <and if you don't get out of it this moment, I won't answer for \textit{your} ever leaving, either.>

<In God's name,> I cried, exasperated, <what do you mean?>

The man turned and threw the blue cushions off the foot of the silver couch.

<Have you ever examined the toes of this?> he asked.

<Not particularly,> I said, more and more astonished. <Why should I\@?>

<Did you ever know Loder make any figure of her but this with that short toe on the left foot?> he went on.

Well, I did take a look at it then, and saw it was as he said—the left foot had a short second toe.

<So it is,> I said, <but, after all, why not?>

<Why not, indeed?> said the young man. <Wouldn't you like to see why, of all the figures Loder made of Maria Morano, this is the only one that has the feet of the living woman?>

He picked up the poker.

<Look!> he said.

With a lot more strength than I should have expected from him, he brought the head of the poker down with a heavy crack on the silver couch. It struck one of the arms of the figure neatly at the elbow-joint, smashing a jagged hole in the silver. He wrenched at the arm and brought it away. It was hollow, and, as I am alive, I tell you there was a long, dry arm-bone inside it!>

Varden paused, and put away a good mouthful of whisky.

<Well?> cried several breathless voices.

<Well,> said Varden, <I'm not ashamed to say I went out of that house like an old buck-rabbit that hears the man with the gun. There was a car standing just outside, and the driver opened the door. I tumbled in, and then it came over me that the whole thing might be a trap, and I tumbled out again and ran till I reached the trolley-cars. But I found my bags at the station next day, duly registered for Vancouver.

When I pulled myself together I did rather wonder what Loder was thinking about my disappearance, but I could no more have gone back into that horrible house than I could have taken poison. I left for Vancouver next morning, and from that day to this I never saw either of those men again. I've still not the faintest idea who the fair man was, or what became of him, but I heard in a round-about way that Loder was dead—in some kind of an accident, I fancy.>

There was a pause. Then:

<It's a damned good story, Mr~Varden,> said Armstrong—he was a dabbler in various kinds of handiwork, and was, indeed, chiefly responsible for Mr~Arbuthnot's motion to ban wireless—<but are you suggesting there was a complete skeleton inside that silver casting? Do you mean Loder put it into the core of the mould when the casting was done? It would be awfully difficult and dangerous—the slightest accident would have put him at the mercy of his workmen. And that statue must have been considerably over life-size to allow of the skeleton being well covered.>

<Mr~Varden has unintentionally misled you, Armstrong,> said a quiet, husky voice suddenly from the shadow behind Varden's chair. <The figure was not silver, but electro-plated on a copper base deposited direct on the body. The lady was Sheffield-plated, in fact. I fancy the soft parts of her must have been digested away with pepsin, or some preparation of the kind, after the process was complete, but I can't be positive about that.>

<Hullo, Wimsey,> said Armstrong, <was that you came in just now? And why this confident pronouncement?>

The effect of Wimsey's voice on Varden had been extraordinary. He had leapt to his feet, and turned the lamp so as to light up Wimsey's face.

<Good evening, Mr~Varden,> said Lord~Peter. <I'm delighted to meet you again and to apologise for my unceremonious behaviour on the occasion of our last encounter.>

Varden took the proffered hand, but was speechless.

<D'you mean to say, you mad mystery-monger, that \textit{you} were Varden's Great Unknown?> demanded Bayes. <Ah, well,> he added rudely, <we might have guessed it from his vivid description.>

<Well, since you're here,> said Smith-Hartington, the \textit{Morning Yell} man, <I think you ought to come across with the rest of the story.>

<Was it just a joke?> asked Judson.

<Of course not,> interrupted Pettifer, before Lord~Peter had time to reply. <Why should it be? Wimsey's seen enough queer things not to have to waste his time inventing them.>

<That's true enough,> said Bayes. <Comes of having deductive powers and all that sort of thing, and always sticking one's nose into things that are better not investigated.>

<That's all very well, Bayes,> said his lordship, <but if I hadn't just mentioned the matter to Mr~Varden that evening, where would he be?>

<Ah, where? That's exactly what we want to know,> demanded Smith-Hartington. <Come on, Wimsey, no shirking; we must have the tale.>

<And the whole tale,> added Pettifer.

<And nothing but the tale,> said Armstrong, dexterously whisking away the whisky-bottle and the cigars from under Lord~Peter's nose. <Get on with it, old son. Not a smoke do you smoke and not a sup do you sip till Burd Ellen is set free.>

<Brute!> said his lordship plaintively. <As a matter of fact,> he went on, with a change of tone, <it's not really a story I want to get about. It might land me in a very unpleasant sort of position—manslaughter probably, and murder possibly.>

<Gosh!> said Bayes.

<That's all right,> said Armstrong, <nobody's going to talk. We can't afford to lose you from the club, you know. Smith-Hartington will have to control his passion for copy, that's all.>

Pledges of discretion having been given all round, Lord~Peter settled himself back and began his tale

\divider

<The curious case of Eric P\@. Loder affords one more instance of the strange manner in which some power beyond our puny human wills arranges the affairs of men. Call it Providence—call it Destiny\longdash>

<We'll call it off,> said Bayes; <you can leave out that part.>

Lord~Peter groaned and began again.

<Well, the first thing that made me feel a bit inquisitive about Loder was a casual remark by a man at the Emigration Office in New York where I happened to go about that silly affair of Mrs~Bilt's. He said, <What on earth is Eric Loder going to do in Australia? I should have thought Europe was more in his line.>

<Australia?> I said, <you're wandering, dear old thing. He told me the other day he was off to Italy in three weeks' time.>

<Italy, nothing,> he said, <he was all over our place to-day, asking about how you got to Sydney and what were the necessary formalities, and so on.>

<Oh,> I said, <I suppose he's going by the Pacific route, and calling at Sydney on his way.> But I wondered why he hadn't said so when I'd met him the day before. He had distinctly talked about sailing for Europe and doing Paris before he went on to Rome.

I felt so darned inquisitive that I went and called on Loder two nights later.

He seemed quite pleased to see me, and was full of his forthcoming trip. I asked him again about his route, and he told me quite distinctly he was going via Paris.

Well, that was that, and it wasn't really any of my business, and we chatted about other things. He told me that Mr~Varden was coming to stay with him before he went, and that he hoped to get him to pose for a figure before he left. He said he'd never seen a man so perfectly formed. <I meant to get him to do it before,> he said, <but war broke out, and he went and joined the army before I had time to start.>

He was lolling on that beastly couch of his at the time, and, happening to look round at him, I caught such a nasty sort of glitter in his eye that it gave me quite a turn. He was stroking the figure over the neck and grinning at it.

<None of your efforts in Sheffield-plate, I hope,> said I\@.

<Well,> he said, <I thought of making a kind of companion to this, \textit{The Sleeping Athlete}, you know, or something of that sort.>

<You'd much better cast it,> I said. <Why did you put the stuff on so thick? It destroys the fine detail.>

That annoyed him. He never liked to hear any objection made to that work of art.

<This was experimental,> he said. <I mean the next to be a real masterpiece. You'll see.>

We'd got to about that point when the butler came in to ask should he make up a bed for me, as it was such a bad night. We hadn't noticed the weather particularly, though it had looked a bit threatening when I started from New York. However, we now looked out, and saw that it was coming down in sheets and torrents. It wouldn't have mattered, only that I'd only brought a little open racing car and no overcoat, and certainly the prospect of five miles in that downpour wasn't altogether attractive. Loder urged me to stay, and I said I would.

I was feeling a bit fagged, so I went to bed right off. Loder said he wanted to do a bit of work in the studio first, and I saw him depart along the corridor.

You won't allow me to mention Providence, so I'll only say it was a very remarkable thing that I should have woken up at two in the morning to find myself lying in a pool of water. The man had stuck a hot-water bottle into the bed, because it hadn't been used just lately, and the beastly thing had gone and unstoppered itself. I lay awake for ten minutes in the deeps of damp misery before I had sufficient strength of mind to investigate. Then I found it was hopeless—sheets, blankets, mattress, all soaked. I looked at the arm-chair, and then I had a brilliant idea. I remembered there was a lovely great divan in the studio, with a big skin rug and a pile of cushions. Why not finish the night there? I took the little electric torch which always goes about with me, and started off.

The studio was empty, so I supposed Loder had finished and trotted off to roost. The divan was there, all right, with a screen drawn partly across it, so I rolled myself up under the rug and prepared to snooze off.

I was just getting beautifully sleepy again when I heard footsteps, not in the passage, but apparently on the other side of the room. I was surprised, because I didn't know there was any way out in that direction. I lay low, and presently I saw a streak of light appear from the cupboard where Loder kept his tools and things. The streak widened, and Loder emerged, carrying an electric torch. He closed the cupboard door very gently after him, and padded across the studio. He stopped before the easel and uncovered it; I could see him through a crack in the screen. He stood for some minutes gazing at a sketch on the easel, and then gave one of the nastiest gurgly laughs I've ever had the pleasure of hearing. If I'd ever seriously thought of announcing my unauthorised presence, I abandoned all idea of it then. Presently he covered the easel again, and went out by the door at which I had come in.

I waited till I was sure he had gone, and then got up—uncommonly quietly, I may say. I tiptoed over to the easel to see what the fascinating work of art was. I saw at once it was the design for the figure of \textit{The Sleeping Athlete}, and as I looked at it I felt a sort of horrid conviction stealing over me. It was an idea which seemed to begin in my stomach, and work its way up to the roots of my hair.

My family say I'm too inquisitive. I can only say that wild horses wouldn't have kept me from investigating that cupboard. With the feeling that something absolutely vile might hop out at me—I was a bit wrought up, and it was a rotten time of night—I put a heroic hand on the door knob.

To my astonishment, the thing wasn't even locked. It opened at once, to show a range of perfectly innocent and orderly shelves, which couldn't possibly have held Loder.

My blood was up, you know, by this time, so I hunted round for the spring-lock which I knew must exist, and found it without much difficulty. The back of the cupboard swung noiselessly inwards, and I found myself at the top of a narrow flight of stairs.

I had the sense to stop and see that the door could be opened from the inside before I went any farther, and I also selected a good stout pestle which I found on the shelves as a weapon in case of accident. Then I closed the door and tripped with elf-like lightness down that jolly old staircase.

There was another door at the bottom, but it didn't take me long to fathom the secret of that. Feeling frightfully excited, I threw it boldly open, with the pestle ready for action.

However, the room seemed to be empty. My torch caught the gleam of something liquid, and then I found the wall-switch.

I saw a biggish square room, fitted up as a workshop. On the right-hand wall was a big switchboard, with a bench beneath it. From the middle of the ceiling hung a great flood-light, illuminating a glass vat, fully seven feet long by about three wide. I turned on the flood-light, and looked down into the vat. It was filled with a dark brown liquid which I recognised as the usual compound of cyanide and copper-sulphate which they use for copper-plating.

The rods hung over it with their hooks all empty, but there was a packing-case half-opened at one side of the room, and, pulling the covering aside, I could see rows of copper anodes—enough of them to put a plating over a quarter of an inch thick on a life-size figure. There was a smaller case, still nailed up, which from its weight and appearance I guessed to contain the silver for the rest of the process. There was something else I was looking for, and I soon found it—a considerable quantity of prepared graphite and a big jar of varnish.

Of course, there was no evidence, really, of anything being on the cross. There was no reason why Loder shouldn't make a plaster cast and Sheffield-plate it if he had a fancy for that kind of thing. But then I found something that \textit{couldn't} have come there legitimately.

On the bench was an oval slab of copper about an inch and a half long—Loder's night's work, I guessed. It was an electrotype of the American Consular seal, the thing they stamp on your passport photograph to keep you from hiking it off and substituting the picture of your friend Mr~Jiggs, who would like to get out of the country because he is so popular with Scotland Yard.

I sat down on Loder's stool, and worked out that pretty little plot in all its details. I could see it all turned on three things. First of all, I must find out if Varden was proposing to make tracks shortly for Australia, because, if he wasn't, it threw all my beautiful theories out. And, secondly, it would help matters greatly if he happened to have dark hair like Loder's, as he has, you see—near enough, anyway, to fit the description on a passport. I'd only seen him in that Apollo Belvedere thing, with a fair wig on. But I knew if I hung about I should see him presently when he came to stay with Loder. And, thirdly, of course, I had to discover if Loder was likely to have any grounds for a grudge against Varden.

Well, I figured out I'd stayed down in that room about as long as was healthy. Loder might come back at any moment, and I didn't forget that a vatful of copper sulphate and cyanide of potassium would be a highly handy means of getting rid of a too-inquisitive guest. And I can't say I had any great fancy for figuring as part of Loder's domestic furniture. I've always hated things made in the shape of things—volumes of Dickens that turn out to be a biscuit-tin, and dodges like that; and, though I take no overwhelming interest in my own funeral, I should like it to be in good taste. I went so far as to wipe away any finger-marks I might have left behind me, and then I went back to the studio and rearranged that divan. I didn't feel Loder would care to think I'd been down there.

There was just one other thing I felt inquisitive about. I tiptoed back through the hall and into the smoking-room. The silver couch glimmered in the light of the torch. I felt I disliked it fifty times more than ever before. However, I pulled myself together and took a careful look at the feet of the figure. I'd heard all about that second toe of Maria Morano's.

I passed the rest of the night in the arm-chair after all.

What with Mrs~Bilt's job and one thing and another, and the enquiries I had to make, I had to put off my interference in Loder's little game till rather late. I found out that Varden had been staying with Loder a few months before the beautiful Maria Morano had vanished. I'm afraid I was rather stupid about that, Mr~Varden. I thought perhaps there \textit{had} been something.>

<Don't apologise,> said Varden, with a little laugh. <Cinema actors are notoriously immoral.>

<Why rub it in?> said Wimsey, a trifle hurt. <I apologise. Anyway, it came to the same thing as far as Loder was concerned. Then there was one bit of evidence I had to get to be absolutely certain. Electro-plating—especially such a ticklish job as the one I had in mind—wasn't a job that could be finished in a night; on the other hand, it seemed necessary that Mr~Varden should be seen alive in New York up to the day he was scheduled to depart. It was also clear that Loder meant to be able to prove that a Mr~Varden had left New York all right, according to plan, and had actually arrived in Sydney. Accordingly, a false Mr~Varden was to depart with Varden's papers and Varden's passport, furnished with a new photograph duly stamped with the Consular stamp, and to disappear quietly at Sydney and be retransformed into Mr~Eric Loder, travelling with a perfectly regular passport of his own. Well, then, in that case, obviously a cablegram would have to be sent off to Mystofilms Ltd., warning them to expect Varden by a later boat than he had arranged. I handed over this part of the job to my man, Bunter, who is uncommonly capable. The devoted fellow shadowed Loder faithfully for getting on for three weeks, and at length, the very day before Mr~Varden was due to depart, the cablegram was sent from an office in Broadway, where, by a happy providence (once more) they supply extremely hard pencils.>

<By Jove!> cried Varden, <I remember now being told something about a cablegram when I got out, but I never connected it with Loder. I thought it was just some stupidity of the Western Electric people.>

<Quite so. Well, as soon as I'd got that, I popped along to Loder's with a picklock in one pocket and an automatic in the other. The good Bunter went with me, and, if I didn't return by a certain time, had orders to telephone for the police. So you see everything was pretty well covered. Bunter was the chauffeur who was waiting for you, Mr~Varden, but you turned suspicious—I don't blame you altogether—so all we could do was to forward your luggage along to the train.

On the way out we met the Loder servants \textit{en route} for New York in a car, which showed us that we were on the right track, and also that I was going to have a fairly simple job of it.

You've heard all about my interview with Mr~Varden. I really don't think I could improve upon his account. When I'd seen him and his traps safely off the premises, I made for the studio. It was empty, so I opened the secret door, and, as I expected, saw a line of light under the workshop door at the far end of the passage.>

<So Loder was there all the time?>

<Of course he was. I took my little pop-gun tight in my fist and opened the door very gently. Loder was standing between the tank and the switchboard, very busy indeed—so busy he didn't hear me come in. His hands were black with graphite, a big heap of which was spread on a sheet on the floor, and he was engaged with a long, springy coil of copper wire, running to the output of the transformer. The big packing-case had been opened, and all the hooks were occupied.

<Loder!> I said.

He turned on me with a face like nothing human. <Wimsey!> he shouted, <what the hell are you doing here?>

<I have come,> I said, <to tell you that I know how the apple gets into the dumpling.> And I showed him the automatic.

He gave a great yell and dashed at the switchboard, turning out the light, so that I could not see to aim. I heard him leap at me—and then there came in the darkness a crash and a splash—and a shriek such as I never heard—not in five years of war—and never want to hear again.

I groped forward for the switchboard. Of course, I turned on everything before I could lay my hand on the light, but I got it at last—a great white glare from the flood-light over the vat.

He lay there, still twitching faintly. Cyanide, you see, is about the swiftest and painfullest thing out. Before I could move to do anything, I knew he was dead—poisoned and drowned and dead. The coil of wire that had tripped him had gone into the vat with him. Without thinking, I touched it, and got a shock that pretty well staggered me. Then I realised that I must have turned on the current when I was hunting for the light. I looked into the vat again. As he fell, his dying hands had clutched at the wire. The coils were tight round his fingers, and the current was methodically depositing a film of copper all over his hands, which were blackened with the graphite.

I had just sense enough to realise that Loder was dead, and that it might be a nasty sort of look-out for me if the thing came out, for I'd certainly gone along to threaten him with a pistol.

I searched about till I found some solder and an iron. Then I went upstairs and called in Bunter, who had done his ten miles in record time. We went into the smoking-room and soldered the arm of that cursed figure into place again, as well as we could, and then we took everything back into the workshop. We cleaned off every finger-print and removed every trace of our presence. We left the light and the switchboard as they were, and returned to New York by an extremely round-about route. The only thing we brought away with us was the facsimile of the Consular seal, and that we threw into the river.

Loder was found by the butler next morning. We read in the papers how he had fallen into the vat when engaged on some experiments in electro-plating. The ghastly fact was commented upon that the dead man's hands were thickly coppered over. They couldn't get it off without irreverent violence, so he was buried like that.

That's all. Please, Armstrong, may I have my whisky-and-soda now?>

<What happened to the couch?> enquired Smith-Hartington presently.

<I bought it in at the sale of Loder's things,> said Wimsey, <and got hold of a dear old Catholic priest I knew, to whom I told the whole story under strict vow of secrecy. He was a very sensible and feeling old bird; so one moonlight night Bunter and I carried the thing out in the car to his own little church, some miles out of the city, and gave it Christian burial in a corner of the graveyard. It seemed the best thing to do.>