%!TeX root=../viewsbodytop.tex
\addchap{The Bibulous Business of a Matter of Taste}

\lettrine[lines=4,ante=‘]{H}{alte-là!}... Attention!... F—e!'

\zz
The young man in the grey suit pushed his way through the protesting porters and leapt nimbly for the footboard of the guard's van as the Paris-Evreux express steamed out of the Invalides. The guard, with an eye to a tip, fielded him adroitly from among the detaining hands.

<It is happy for monsieur that he is so agile,> he remarked. <Monsieur is in a hurry?>

<Somewhat. Thank you. I can get through by the corridor?>

<But certainly. The \textit{premières} are two coaches away, beyond the luggage-van.>

The young man rewarded his rescuer, and made his way forward, mopping his face. As he passed the piled-up luggage, something caught his eye, and he stopped to investigate. It was a suit-case, nearly new, of expensive-looking leather, labelled conspicuously:

\vspace{-0.2cm}

\begin{center}
\textsc{Lord~Peter Wimsey},\\
Hôtel Saumon d'Or,\\
Verneuil-sur-Eure
\end{center}

\vspace{-0.2cm}
\noindent and bore witness to its itinerary thus:
\vspace{-0.2cm}

\begin{center}
\textsc{London—Paris}\\
(Waterloo) (Gare St~Lazare)\\
via Southampton-Havre\\
~\\
\textsc{Paris—Verneuil}\\
(Ch. de Fer de l'Ouest)
\end{center}

The young man whistled, and sat down on a trunk to think it out.

Somewhere there had been a leakage, and they were on his trail. Nor did they care who knew it. There were hundreds of people in London and Paris who would know the name of Wimsey, not counting the police of both countries. In addition to belonging to one of the oldest ducal families in England, Lord~Peter had made himself conspicuous by his meddling with crime detection. A label like this was a gratuitous advertisement.

But the amazing thing was that the pursuers were not troubling to hide themselves from the pursued. That argued very great confidence. That he should have got into the guard's van was, of course, an accident, but, even so, he might have seen it on the platform, or anywhere.

An accident? It occurred to him—not for the first time, but definitely now, and without doubt—that it was indeed an accident for them that he was here. The series of maddening delays that had held him up between London and the Invalides presented itself to him with an air of pre-arrangement. The preposterous accusation, for instance, of the woman who had accosted him in Piccadilly, and the slow process of extricating himself at Marlborough Street. It was easy to hold a man up on some trumped-up charge till an important plan had matured. Then there was the lavatory door at Waterloo, which had so ludicrously locked itself upon him. Being athletic, he had climbed over the partition, to find the attendant mysteriously absent. And, in Paris, was it by chance that he had had a deaf taxi-driver, who mistook the direction <Quai d'Orléans> for <Gare de Lyon,> and drove a mile and a half in the wrong direction before the shouts of his fare attracted his attention? They were clever, the pursuers, and circumspect. They had accurate information; they would delay him, but without taking any overt step; they knew that, if only they could keep time on their side, they needed no other ally.

Did they know he was on the train? If not, he still kept the advantage, for they would travel in a false security, thinking him to be left, raging and helpless, in the Invalides. He decided to make a cautious reconnaissance.

The first step was to change his grey suit for another of inconspicuous navy-blue cloth, which he had in his small black bag. This he did in the privacy of the toilet, substituting for his grey soft hat a large travelling-cap, which pulled well down over his eyes.

There was little difficulty in locating the man he was in search of. He found him seated in the inner corner of a first-class compartment, facing the engine, so that the watcher could approach unseen from behind. On the rack was a handsome dressing-case, with the initials \textsc{p.d.b.w.} The young man was familiar with Wimsey's narrow, beaky face, flat yellow hair, and insolent dropped eyelids. He smiled a little grimly.

<He is confident,> he thought, <and has regrettably made the mistake of underrating the enemy. Good! This is where I retire into a \textit{seconde} and keep my eyes open. The next act of this melodrama will take place, I fancy, at Dreux.>

\divider
It is a rule on the Chemin de Fer de l'Ouest that all Paris-Evreux trains, whether of Grande Vitesse or what Lord~Peter Wimsey preferred to call Grande Paresse, shall halt for an interminable period at Dreux. The young man (now in navy-blue) watched his quarry safely into the refreshment-room, and slipped unobtrusively out of the station. In a quarter of an hour he was back—this time in a heavy motoring-coat, helmet, and goggles, at the wheel of a powerful hired Peugeot. Coming quietly on to the platform, he took up his station behind the wall of the \textit{lampisterie}, whence he could keep an eye on the train and the buffet door. After fifteen minutes his patience was rewarded by the sight of his man again boarding the express, dressing-case in hand. The porters slammed the doors, crying: <Next stop Verneuil!> The engine panted and groaned; the long train of grey-green carriages clanked slowly away. The motorist drew a breath of satisfaction, and, hurrying past the barrier, started up the car. He knew that he had a good eighty miles an hour under his bonnet, and there is no speed-limit in France.

\divider
Mon Souci, the seat of that eccentric and eremitical genius the Comte de Rueil, is situated three kilometres from Verneuil. It is a sorrowful and decayed château, desolate at the termination of its neglected avenue of pines. The mournful state of a nobility without an allegiance surrounds it. The stone nymphs droop greenly over their dry and mouldering fountains. An occasional peasant creaks with a single wagon-load of wood along the ill-forested glades. It has the atmosphere of sunset at all hours of the day. The woodwork is dry and gaping for lack of paint. Through the jalousies one sees the prim \textit{salon}, with its beautiful and faded furniture. Even the last of its ill-dressed, ill-favoured women has withered away from Mon Souci, with her inbred, exaggerated features and her long white gloves. But at the rear of the château a chimney smokes incessantly. It is the furnace of the laboratory, the only living and modern thing among the old and dying; the only place tended and loved, petted and spoiled, heir to the long solicitude which counts of a more light-hearted day had given to stable and kennel, portrait-gallery and ballroom. And below, in the cool cellar, lie row upon row the dusty bottles, each an enchanted glass coffin in which the Sleeping Beauty of the vine grows ever more ravishing in sleep.

As the Peugeot came to a standstill in the courtyard, the driver observed with considerable surprise that he was not the count's only visitor. An immense super-Renault, like a \textit{merveilleuse} of the Directoire, all bonnet and no body, had been drawn so ostentatiously across the entrance as to embarrass the approach of any new-comer. Its glittering panels were embellished with a coat of arms, and the count's elderly servant was at that moment staggering beneath the weight of two large and elaborate suit-cases, bearing in silver letters that could be read a mile away the legend: <Lord~Peter Wimsey.>

The Peugeot driver gazed with astonishment at this display, and grinned sardonically. <Lord~Peter seems rather ubiquitous in this country,> he observed to himself. Then, taking pen and paper from his bag, he busied himself with a little letter-writing. By the time that the suit-cases had been carried in, and the Renault had purred its smooth way to the outbuildings, the document was complete and enclosed in an envelope addressed to the Comte de Rueil. <The hoist with his own petard touch,> said the young man, and, stepping up to the door, presented the envelope to the manservant.

<I am the bearer of a letter of introduction to monsieur le comte,> he said. <Will you have the obligingness to present it to him? My name is Bredon—Death Bredon.>

The man bowed, and begged him to enter.

<If monsieur will have the goodness to seat himself in the hall for a few moments. Monsieur le comte is engaged with another gentleman, but I will lose no time in making monsieur's arrival known.>

The young man sat down and waited. The windows of the hall looked out upon the entrance, and it was not long before the château's sleep was disturbed by the hooting of yet another motor-horn. A station taxi-cab came noisily up the avenue. The man from the first-class carriage and the luggage labelled \textsc{p.d.b.w.} were deposited upon the doorstep. Lord~Peter Wimsey dismissed the driver and rang the bell.

<Now,> said Mr~Bredon, <the fun is going to begin.> He effaced himself as far as possible in the shadow of a tall \textit{armoire normande}.

<Good evening,> said the new-comer to the manservant, in admirable French, <I am Lord~Peter Wimsey. I arrive upon the invitation of Monsieur le comte de Rueil. Monsieur le comte is at liberty?>

<Milord Peter Wimsey? Pardon, monsieur, but I do not understand. Milord de Wimsey is already arrived and is with monsieur le comte at this moment.>

<You surprise me,> said the other, with complete imperturbability, <for certainly no one but myself has any right to that name. It seems as though some person more ingenious than honest has had the bright idea of impersonating me.>

The servant was clearly at a loss.

<Perhaps,> he suggested, <monsieur can show his \textit{papiers d'identité}.>

<Although it is somewhat unusual to produce one's credentials on the doorstep when paying a private visit,> replied his lordship, with unaltered good humour, <I have not the slightest objection. Here is my passport, here is a \textit{permis de séjour} granted to me in Paris, here my visiting-card, and here a quantity of correspondence addressed to me at the Hôtel Meurice, Paris, at my flat in Piccadilly, London, at the Marlborough Club, London, and at my brother's house at King's Denver. Is that sufficiently in order?>

The servant perused the documents carefully, appearing particularly impressed by the \textit{permis de séjour}.

<It appears there is some mistake,> he murmured dubiously; <if monsieur will follow me, I will acquaint monsieur le comte.>

They disappeared through the folding doors at the back of the hall, and Bredon was left alone.

<Quite a little boom in Richmonds to-day,> he observed, <each of us more unscrupulous than the last. The occasion obviously calls for a refined subtlety of method.>

After what he judged to be a hectic ten minutes in the count's library, the servant reappeared, searching for him.

<Monsieur le comte's compliments, and would monsieur step this way?>

Bredon entered the room with a jaunty step. He had created for himself the mastery of this situation. The count, a thin, elderly man, his fingers deeply stained with chemicals, sat, with a perturbed expression, at his desk. In two arm-chairs sat the two Wimseys. Bredon noted that, while the Wimsey he had seen in the train (whom he mentally named Peter I) retained his unruffled smile, Peter II (he of the Renault) had the flushed and indignant air of an Englishman affronted. The two men were superficially alike—both fair, lean, and long-nosed, with the nondescript, inelastic face which predominates in any assembly of well bred Anglo-Saxons.

<Mr~Bredon,> said the count, <I am charmed to have the pleasure of making your acquaintance, and regret that I must at once call upon you for a service as singular as it is important. You have presented to me a letter of introduction from your cousin, Lord~Peter Wimsey. Will you now be good enough to inform me which of these gentlemen he is?>

Bredon let his glance pass slowly from the one claimant to the other, meditating what answer would best serve his own ends. One, at any rate, of the men in this room was a formidable intellect, trained in the detection of imposture.

<Well?> said Peter II. <Are you going to acknowledge me, Bredon?>

Peter I extracted a cigarette from a silver case. <Your confederate does not seem very well up in his part,> he remarked, with a quiet smile at Peter II.

<Monsieur le comte,> said Bredon, <I regret extremely that I cannot assist you in the matter. My acquaintance with my cousin, like your own, has been made and maintained entirely through correspondence on a subject of common interest. My profession,> he added, <has made me unpopular with my family.>

There was a very slight sigh of relief somewhere. The false Wimsey—whichever he was—had gained a respite. Bredon smiled.

<An excellent move, Mr~Bredon,> said Peter I, <but it will hardly explain—Allow me.> He took the letter from the count's hesitating hand. <It will hardly explain the fact that the ink of this letter of recommendation, dated three weeks ago, is even now scarcely dry—though I congratulate you on the very plausible imitation of my handwriting.>

<If \textit{you} can forge my handwriting,> said Peter II, <so can this Mr~Bredon.> He read the letter aloud over his double's shoulder.

<<Monsieur le comte—I have the honour to present to you my friend and cousin, Mr~Death Bredon, who, I understand, is to be travelling in your part of France next month. He is very anxious to view your interesting library. Although a journalist by profession, he really knows something about books.> I am delighted to learn for the first time that I have such a cousin. An interviewer's trick, I fancy, monsieur le comte. Fleet Street appears well informed about our family names. Possibly it is equally well informed about the object of my visit to Mon Souci?>

<If,> said Bredon boldly, <you refer to the acquisition of the de Rueil formula for poison gas for the British Government, I can answer for my own knowledge, though possibly the rest of Fleet Street is less completely enlightened.> He weighed his words carefully now, warned by his slip. The sharp eyes and detective ability of Peter I alarmed him far more than the caustic tongue of Peter II.

The count uttered an exclamation of dismay.

<Gentlemen,> he said, <one thing is obvious—that there has been somewhere a disastrous leakage of information. Which of you is the Lord~Peter Wimsey to whom I should entrust the formula I do not know. Both of you are supplied with papers of identity; both appear completely instructed in this matter; both of your handwritings correspond with the letters I have previously received from Lord~Peter, and both of you have offered me the sum agreed upon in Bank of England notes. In addition, this third gentleman arrives endowed with an equal facility in handwritings, an introductory letter surrounded by most suspicious circumstances, and a degree of acquaintance with this whole matter which alarms me. I can see but one solution. All of you must remain here at the château while I send to England for some elucidation of this mystery. To the genuine Lord~Peter I offer my apologies, and assure him that I will endeavour to make his stay as agreeable as possible. Will this satisfy you? It will? I am delighted to hear it. My servants will show you to your bedrooms, and dinner will be at half-past seven.>

\divider
<It is delightful to think,> said Mr~Bredon, as he fingered his glass and passed it before his nostrils with the air of a connoisseur, <that whichever of these gentlemen has the right to the name which he assumes is assured to-night of a truly Olympian satisfaction.> His impudence had returned to him, and he challenged the company with an air. <Your cellars, monsieur le comte, are as well known among men endowed with a palate as your talents among men of science. No eloquence could say more.>

The two Lord~Peters murmured assent.

<I am the more pleased by your commendation,> said the count, <that it suggests to me a little test which, with your kind co-operation, will, I think, assist us very much in determining which of you gentlemen is Lord~Peter Wimsey and which his talented impersonator. Is it not matter of common notoriety that Lord~Peter has a palate for wine almost unequalled in Europe?>

<You flatter me, monsieur le comte,> said Peter II modestly.

<I wouldn't like to say unequalled,> said Peter I, chiming in like a well-trained duet; <let's call it fair to middling. Less liable to misconstruction and all that.>

<Your lordship does yourself an injustice,> said Bredon, addressing both men with impartial deference. <The bet which you won from Mr~Frederick Arbuthnot at the Egotists' Club, when he challenged you to name the vintage years of seventeen wines blindfold, received its due prominence in the \textit{Evening Wire}.>

<I was in extra form that night,> said Peter I.

<A fluke,> laughed Peter II.

<The test I propose, gentlemen, is on similar lines,> pursued the count, <though somewhat less strenuous. There are six courses ordered for dinner to-night. With each we will drink a different wine, which my butler shall bring in with the label concealed. You shall each in turn give me your opinion upon the vintage. By this means we shall perhaps arrive at something, since the most brilliant forger—of whom I gather I have at least two at my table to-night—can scarcely forge a palate for wine. If too hazardous a mixture of wines should produce a temporary incommodity in the morning, you will, I feel sure, suffer it gladly for this once in the cause of truth.>

The two Wimseys bowed.

<\textit{In vino veritas},> said Mr~Bredon, with a laugh. He at least was well seasoned, and foresaw opportunities for himself.

<Accident, and my butler, having placed you at my right hand, monsieur,> went on the count, addressing Peter I, <I will ask you to begin by pronouncing, as accurately as may be, upon the wine which you have just drunk.>

<That is scarcely a searching ordeal,> said the other, with a smile. <I can say definitely that it is a very pleasant and well-matured Chablis Moutonne; and, since ten years is an excellent age for a Chablis—a real Chablis—I should vote for 1916, which was perhaps the best of the war vintages in that district.>

<Have you anything to add to that opinion, monsieur?> enquired the count, deferentially, of Peter II.

<I wouldn't like to be dogmatic to a year or so,> said that gentleman critically, <but if I must commit myself, don't you know, I should say 1915—decidedly 1915.>

The count bowed, and turned to Bredon.

<Perhaps you, too, monsieur, would be interested to give an opinion,> he suggested, with the exquisite courtesy always shown to the plain man in the society of experts.

<I'd rather not set a standard which I might not be able to live up to,> replied Bredon, a little maliciously. <I know that it is 1915, for I happened to see the label.>

Peter II looked a little disconcerted.

<We will arrange matters better in future,> said the count. <Pardon me.> He stepped apart for a few moments' conference with the butler, who presently advanced to remove the oysters and bring in the soup.

The next candidate for attention arrived swathed to the lip in damask.

<It is your turn to speak first, monsieur,> said the count to Peter II. <Permit me to offer you an olive to cleanse the palate. No haste, I beg. Even for the most excellent political ends, good wine must not be used with disrespect.>

The rebuke was not unnecessary, for, after a preliminary sip, Peter II had taken a deep draught of the heady white richness. Under Peter I's quizzical eye he wilted quite visibly.

<It is—it is Sauterne,> he began, and stopped. Then, gathering encouragement from Bredon's smile, he said, with more aplomb, <Château Yquem, 1911—ah! the queen of white wines, sir, as what's-his-name says.> He drained his glass defiantly.

The count's face was a study as he slowly detached his fascinated gaze from Peter II to fix it on Peter I.

<If I had to be impersonated by somebody,> murmured the latter gently, <it would have been more flattering to have had it undertaken by a person to whom all white wines were \textit{not} alike. Well, now, sir, this admirable vintage is, of course, a Montrachet of—let me see>—he rolled the wine delicately upon his tongue—<of 1911. And a very attractive wine it is, though, with all due deference to yourself, monsieur le comte, I feel that it is perhaps slightly too sweet to occupy its present place in the menu. True, with this excellent \textit{consommé marmite}, a sweetish wine is not altogether out of place, but, in my own humble opinion, it would have shown to better advantage with the \textit{confitures}.>

<There, now,> said Bredon innocently, <it just shows how one may be misled. Had not I had the advantage of Lord~Peter's expert opinion—for certainly nobody who could mistake Montrachet for Sauterne has any claim to the name of Wimsey—I should have pronounced this to be, not the Montrachet-Aîné, but the Chevalier-Montrachet of the same year, which is a trifle sweeter. But no doubt, as your lordship says, drinking it with the soup has caused it to appear sweeter to me than it actually is.>

The count looked sharply at him, but made no comment.

<Have another olive,> said Peter I kindly. <You can't judge wine if your mind is on other flavours.>

<Thanks frightfully,> said Bredon. <And that reminds me\longdash> He launched into a rather pointless story about olives, which lasted out the soup and bridged the interval to the entrance of an exquisitely cooked sole.

The count's eye followed the pale amber wine rather thoughtfully as it trilled into the glasses. Bredon raised his in the approved manner to his nostrils, and his face flushed a little. With the first sip he turned excitedly to his host.

<Good God, sir\longdash> he began.

The lifted hand cautioned him to silence.

Peter I sipped, inhaled, sipped again, and his brows clouded. Peter II had by this time apparently abandoned his pretensions. He drank thirstily, with a beaming smile and a lessening hold upon reality.

<Eh bien, monsieur?> enquired the count gently.

<This,> said Peter I, <is certainly hock, and the noblest hock I have ever tasted, but I must admit that for the moment I cannot precisely place it.>

<No?> said Bredon. His voice was like bean-honey now, sweet and harsh together. <Nor the other gentleman? And yet I fancy I could place it within a couple of miles, though it is a wine I had hardly looked to find in a French cellar at this time. It is hock, as your lordship says, and at that it is Johannisberger. Not the plebeian cousin, but the \textit{echter} Schloss Johannisberger from the castle vineyard itself. Your lordship must have missed it (to your great loss) during the war years. My father laid some down the year before he died, but it appears that the ducal cellars at Denver were less well furnished.>

<I must set about remedying the omission,> said the remaining Peter, with determination.

The \textit{poulet} was served to the accompaniment of an argument over the Lafitte, his lordship placing it at 1878, Bredon maintaining it to be a relic of the glorious 'seventy-fives, slightly over-matured, but both agreeing as to its great age and noble pedigree.

As to the Clos-Vougeôt, on the other hand, there was complete agreement; after a tentative suggestion of 1915, it was pronounced finally by Peter I to belong to the equally admirable though slightly lighter 1911 crop. The \textit{pré-salé} was removed amid general applause, and the dessert was brought in.

<Is it necessary,> asked Peter I, with a slight smile in the direction of Peter II—now happily murmuring, <Damn good wine, damn good dinner, damn good show>—<is it necessary to prolong this farce any further?>

<Your lordship will not, surely, refuse to proceed with the discussion?> cried the count.

<The point is sufficiently made, I fancy.>

<But no one will surely ever refuse to discuss wine,> said Bredon, <least of all your lordship, who is so great an authority.>

<Not on this,> said the other. <Frankly, it is a wine I do not care about. It is sweet and coarse, qualities that would damn any wine in the eyes—the mouth, rather—of a connoisseur. Did your excellent father have this laid down also, Mr~Bredon?>

Bredon shook his head.

<No,> he said, <no. Genuine Imperial Tokay is beyond the opportunities of Grub Street, I fear. Though I agree with you that it is horribly overrated—with all due deference to yourself, monsieur le comte.>

<In that case,> said the count, <we will pass at once to the liqueur. I admit that I had thought of puzzling these gentlemen with the local product, but, since one competitor seems to have scratched, it shall be brandy—the only fitting close to a good wine-list.>

In a slightly embarrassing silence the huge, round-bellied balloon glasses were set upon the table, and the few precious drops poured gently into each and set lightly swinging to release the bouquet.

<This,> said Peter I, charmed again into amiability, <is, indeed, a wonderful old French brandy. Half a century old, I suppose.>

<Your lordship's praise lacks warmth,> replied Bredon. <This is \textit{the} brandy—the brandy of brandies—the superb—the incomparable—the true Napoleon. It should be honoured like the emperor it is.>

He rose to his feet, his napkin in his hand.

<Sir,> said the count, turning to him, <I have on my right a most admirable judge of wine, but you are unique.> He motioned to Pierre, who solemnly brought forward the empty bottles, unswathed now, from the humble Chablis to the stately Napoleon, with the imperial seal blown in the glass. <Every time you have been correct as to growth and year. There cannot be six men in the world with such a palate as yours, and I thought that but one of them was an Englishman. Will you not favour us, this time, with your real name?>

<It doesn't matter what his name is,> said Peter I. He rose. <Put up your hands, all of you. Count, the formula!>

Bredon's hands came up with a jerk, still clutching the napkin. The white folds spurted flame as his shot struck the other's revolver cleanly between trigger and barrel, exploding the charge, to the extreme detriment of the glass chandelier. Peter I stood shaking his paralysed hand and cursing.

Bredon kept him covered while he cocked a wary eye at Peter II, who, his rosy visions scattered by the report, seemed struggling back to aggressiveness.

<Since the entertainment appears to be taking a lively turn,> observed Bredon, <perhaps you would be so good, count, as to search these gentlemen for further firearms. Thank you. Now, why should we not all sit down again and pass the bottle round?>

<You—\textit{you} are\longdash> growled Peter I.

<Oh, my name is Bredon all right,> said the young man cheerfully. <I loathe aliases. Like another fellow's clothes, you know—never seem quite to fit. Peter Death Bredon Wimsey—a bit lengthy and all that, but handy when taken in instalments. I've got a passport and all those things, too, but I didn't offer them, as their reputation here seems a little blown upon, so to speak. As regards the formula, I think I'd better give you my personal cheque for it—all sorts of people seem able to go about flourishing Bank of England notes. Personally, I think all this secret diplomacy work is a mistake, but that's the War Office's pigeon. I suppose we all brought similar credentials. Yes, I thought so. Some bright person seems to have sold himself very successfully in two places at once. But you two must have been having a lively time, each thinking the other was me.>

<My lord,> said the count heavily, <these two men are, or were, Englishmen, I suppose. I do not care to know what Governments have purchased their treachery. But where they stand, I, alas! stand too. To our venal and corrupt Republic I, as a Royalist, acknowledge no allegiance. But it is in my heart that I have agreed to sell my country to England because of my poverty. Go back to your War Office and say I will not give you the formula. If war should come between our countries—which may God avert!—I will be found on the side of France. That, my lord, is my last word.>

Wimsey bowed.

<Sir,> said he, <it appears that my mission has, after all, failed. I am glad of it. This trafficking in destruction is a dirty kind of business after all. Let us shut the door upon these two, who are neither flesh nor fowl, and finish the brandy in the library.>