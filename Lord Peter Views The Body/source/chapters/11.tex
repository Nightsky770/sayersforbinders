%!TeX root=../viewsbodytop.tex
\addchap{The Unsolved Puzzle of the Man With No Face}

\lettrine[lines=4,ante=‘]{A}{nd} what would \textit{you} say, sir,' said the stout man, »to this here business of the bloke what's been found down on the beach at East Felpham?«

\zz
The rush of travellers after the Bank Holiday had caused an overflow of third-class passengers into the firsts, and the stout man was anxious to seem at ease in his surroundings. The youngish gentleman whom he addressed had obviously paid full fare for a seclusion which he was fated to forgo. He took the matter amiably enough, however, and replied in a courteous tone:

»I'm afraid I haven't read more than the headlines. Murdered, I suppose, wasn't he?«

»It's murder, right enough,« said the stout man, with relish. »Cut about he was, something shocking.«

»More like as if a wild beast had done it,« chimed in the thin, elderly man opposite. »No face at all he hadn't got, by what my paper says. It'll be one of these maniacs, I shouldn't be surprised, what goes about killing children.«

»I wish you wouldn't talk about such things,« said his wife, with a shudder. »I lays awake at nights thinking what might 'appen to Lizzie's girls, till my head feels regular in a fever, and I has such a sinking in my inside I has to get up and eat biscuits. They didn't ought to put such dreadful things in the papers.«

»It's better they should, ma'am,« said the stout man, »then we're warned, so to speak, and can take our measures accordingly. Now, from what I can make out, this unfortunate gentleman had gone bathing all by himself in a lonely spot. Now, quite apart from cramps, as is a thing that might 'appen to the best of us, that's a very foolish thing to do.«

»Just what I'm always telling my husband,« said the young wife. The young husband frowned and fidgeted. »Well, dear, it really isn't safe, and you with your heart not strong\longdash« Her hand sought his under the newspaper. He drew away, self-consciously, saying, »That'll do, Kitty.«

»The way I look at it is this,« pursued the stout man. »Here we've been and had a war, what has left 'undreds o' men in what you might call a state of unstable ekilibrium. They've seen all their friends blown up or shot to pieces. They've been through five years of 'orrors and bloodshed, and it's given 'em what you might call a twist in the mind towards 'orrors. They may seem to forget it and go along as peaceable as anybody to all outward appearance, but it's all artificial, if you get my meaning. Then, one day something 'appens to upset them—they 'as words with the wife, or the weather's extra hot, as it is to-day—and something goes pop inside their brains and makes raving monsters of them. It's all in the books. I do a good bit of reading myself of an evening, being a bachelor without encumbrances.«

»That's all very true,« said a prim little man, looking up from his magazine, »very true indeed—too true. But do you think it applies in the present case? I've studied the literature of crime a good deal—I may say I make it my hobby—and it's my opinion there's more in this than meets the eye. If you will compare this murder with some of the most mysterious crimes of late years—crimes which, mind you, have never been solved, and, in my opinion, never will be—what do you find?« He paused and looked round. »You will find many features in common with this case. But especially you will find that the face—and the face only, mark you—has been disfigured, as though to prevent recognition. As though to blot out the victim's personality from the world. And you will find that, in spite of the most thorough investigation, the criminal is never discovered. Now what does all that point to? To organisation. Organisation. To an immensely powerful influence at work behind the scenes. In this very magazine that I'm reading now«—he tapped the page impressively\longdash»there's an account—not a faked-up story, but an account extracted from the annals of the police—of the organisation of one of these secret societies, which mark down men against whom they bear a grudge, and destroy them. And, when they do this, they disfigure their faces with the mark of the Secret Society, and they cover up the track of the assassin so completely—having money and resources at their disposal—that nobody is ever able to get at them.«

»I've read of such things, of course,« admitted the stout man, »but I thought as they mostly belonged to the medeevial days. They had a thing like that in Italy once. What did they call it now? A Gomorrah, was it? Are there any Gomorrahs nowadays?«

»You spoke a true word, sir, when you said Italy,« replied the prim man. »The Italian mind is made for intrigue. There's the Fascisti. That's come to the surface now, of course, but it started by being a secret society. And, if you were to look below the surface, you would be amazed at the way in which that country is honeycombed with hidden organisations of all sorts. Don't you agree with me, sir?« he added, addressing the first-class passenger.

»Ah!« said the stout man, »no doubt this gentleman has been in Italy and knows all about it. Should you say this murder was the work of a Gomorrah, sir?«

»I hope not, I'm sure,« said the first-class passenger. »I mean, it rather destroys the interest, don't you think? I like a nice, quiet, domestic murder myself, with the millionaire found dead in the library. The minute I open a detective story and find a Camorra in it, my interest seems to dry up and turn to dust and ashes—a sort of Sodom and Camorra, as you might say.«

»I agree with you there,« said the young husband, »from what you might call the artistic standpoint. But in this particular case I think there may be something to be said for this gentleman's point of view.«

»Well,« admitted the first-class passenger, »not having read the details\longdash«

»The details are clear enough,« said the prim man. »This poor creature was found lying dead on the beach at East Felpham early this morning, with his face cut about in the most dreadful manner. He had nothing on him but his bathing-dress\longdash«

»Stop a minute. Who was he, to begin with?«

»They haven't identified him yet. His clothes had been taken\longdash«

»That looks more like robbery, doesn't it?« suggested Kitty.

»If it was just robbery,« retorted the prim man, »why should his face have been cut up in that way? No—the clothes were taken away, as I said, to prevent identification. That's what these societies always try to do.«

»Was he stabbed?« demanded the first-class passenger.

»No,« said the stout man. »He wasn't. He was strangled.«

»Not a characteristically Italian method of killing,« observed the first-class passenger.

»No more it is,« said the stout man. The prim man seemed a little disconcerted.

»And if he went down there to bathe,« said the thin, elderly man, »how did he get there? Surely somebody must have missed him before now, if he was staying at Felpham. It's a busy spot for visitors in the holiday season.«

»No,« said the stout man, »not East Felpham. You're thinking of West Felpham, where the yacht-club is. East Felpham is one of the loneliest spots on the coast. There's no house near except a little pub all by itself at the end of a long road, and after that you have to go through three fields to get to the sea. There's no real road, only a cart-track, but you can take a car through. I've been there.«

»He came in a car,« said the prim man. »They found the track of the wheels. But it had been driven away again.«

»It looks as though the two men had come there together,« suggested Kitty.

»I think they did,« said the prim man. »The victim was probably gagged and bound and taken along in the car to the place, and then he was taken out and strangled and\longdash«

»But why should they have troubled to put on his bathing-dress?« said the first-class passenger.

»Because,« said the prim man, »as I said, they didn't want to leave any clothes to reveal his identity.«

»Quite; but why not leave him naked? A bathing-dress seems to indicate an almost excessive regard for decorum, under the circumstances.«

»Yes, yes,« said the stout man impatiently, »but you 'aven't read the paper carefully. The two men couldn't have come there in company, and for why? There was only one set of footprints found, and they belonged to the murdered man.«

He looked round triumphantly.

»Only one set of footprints, eh?« said the first-class passenger quickly. »This looks interesting. Are you sure?«

»It says so in the paper. A single set of footprints, it says, made by bare feet, which by a careful comparison 'ave been shown to be those of the murdered man, lead from the position occupied by the car to the place where the body was found. What do you make of that?«

»Why,« said the first-class passenger, »that tells one quite a lot, don't you know. It gives one a sort of a bird's eye view of the place, and it tells one the time of the murder, besides castin' quite a good bit of light on the character and circumstances of the murderer—or murderers.«

»How do you make that out, sir?« demanded the elderly man.

»Well, to begin with—though I've never been near the place, there is obviously a sandy beach from which one can bathe.«

»That's right,« said the stout man.

»There is also, I fancy, in the neighbourhood, a spur of rock running out into the sea, quite possibly with a handy diving-pool. It must run out pretty far; at any rate, one can bathe there before it is high water on the beach.«

»I don't know how you know that, sir, but it's a fact. There's rocks and a bathing-pool, exactly as you describe, about a hundred yards farther along. Many's the time I've had a dip off the end of them.«

»And the rocks run right back inland, where they are covered with short grass.«

»That's right.«

»The murder took place shortly before high tide, I fancy, and the body lay just about at high-tide mark.«

»Why so?«

»Well, you say there were footsteps leading right up to the body. That means that the water hadn't been up beyond the body. But there were no other marks. Therefore the murderer's footprints must have been washed away by the tide. The only explanation is that the two men were standing together just below the tide-mark. The murderer came up out of the sea. He attacked the other man—maybe he forced him back a little on his own tracks—and there he killed him. Then the water came up and washed out any marks the murderer may have left. One can imagine him squatting there, wondering if the sea was going to come up high enough.«

»Ow!« said Kitty, »you make me creep all over.«

»Now, as to these marks on the face,« pursued the first-class passenger. »The murderer, according to the idea I get of the thing, was already in the sea when the victim came along. You see the idea?«

»I get you,« said the stout man. »You think as he went in off them rocks what we was speaking of, and came up through the water, and that's why there weren't no footprints.«

»Exactly. And since the water is deep round those rocks, as you say, he was presumably in a bathing-dress too.«

»Looks like it.«

»Quite so. Well, now—what was the face-slashing done with? People don't usually take knives out with them when they go for a morning dip.«

»That's a puzzle,« said the stout man.

»Not altogether. Let's say, either the murderer had a knife with him or he had not. If he had\longdash«

»If he had,« put in the prim man eagerly, »he must have laid wait for the deceased on purpose. And, to my mind, that bears out my idea of a deep and cunning plot.«

»Yes. But, if he was waiting there with the knife, why didn't he stab the man and have done with it? Why strangle him, when he had a perfectly good weapon there to hand? No—I think he came unprovided, and, when he saw his enemy there, he made for him with his hands in the characteristic British way.«

»But the slashing?«

»Well, I think that when he had got his man down, dead before him, he was filled with a pretty grim sort of fury and wanted to do more damage. He caught up something that was lying near him on the sand—it might be a bit of old iron, or even one of those sharp shells you sometimes see about, or a bit of glass—and he went for him with that in a desperate rage of jealousy or hatred.«

»Dreadful, dreadful!« said the elderly woman.

»Of course, one can only guess in the dark, not having seen the wounds. It's quite possible that the murderer dropped his knife in the struggle and had to do the actual killing with his hands, picking the knife up afterwards. If the wounds were clean knife-wounds, that is probably what happened, and the murder was premeditated. But if they were rough, jagged gashes, made by an impromptu weapon, then I should say it was a chance encounter, and that the murderer was either mad or\longdash«

»Or?«

»Or had suddenly come upon somebody whom he hated very much.«

»What do you think happened afterwards?«

»That's pretty clear. The murderer, having waited, as I said, to see that all his footprints were cleaned up by the tide, waded or swam back to the rock where he had left his clothes, taking the weapon with him. The sea would wash away any blood from his bathing-dress or body. He then climbed out upon the rocks, walked, with bare feet, so as to leave no tracks on any seaweed or anything, to the short grass of the shore, dressed, went along to the murdered man's car, and drove it away.«

»Why did he do that?«

»Yes, why? He may have wanted to get somewhere in a hurry. Or he may have been afraid that if the murdered man were identified too soon it would cast suspicion on him. Or it may have been a mixture of motives. The point is, where did he come from? How did he come to be bathing at that remote spot, early in the morning? He didn't get there by car, or there would be a second car to be accounted for. He may have been camping near the spot; but it would have taken him a long time to strike camp and pack all his belongings into the car, and he might have been seen. I am rather inclined to think he had bicycled there, and that he hoisted the bicycle into the back of the car and took it away with him.«

»But, in that case, why take the car?«

»Because he had been down at East Felpham longer than he expected, and he was afraid of being late. Either he had to get back to breakfast at some house, where his absence would be noticed, or else he lived some distance off, and had only just time enough for the journey home. I think, though, he had to be back to breakfast.«

»Why?«

»Because, if it was merely a question of making up time on the road, all he had to do was to put himself and his bicycle on the train for part of the way. No; I fancy he was staying in a smallish hotel somewhere. Not a large hotel, because there nobody would notice whether he came in or not. And not, I think, in lodgings, or somebody would have mentioned before now that they had had a lodger who went bathing at East Felpham. Either he lives in the neighbourhood, in which case he should be easy to trace, or was staying with friends who have an interest in concealing his movements. Or else—which I think is more likely—he was in a smallish hotel, where he would be missed from the breakfast-table, but where his favourite bathing-place was not matter of common knowledge.«

»That seems feasible,« said the stout man.

»In any case,« went on the first-class passenger, »he must have been staying within easy bicycling distance of East Felpham, so it shouldn't be too hard to trace him. And then there is the car.«

»Yes. Where is the car, on your theory?« demanded the prim man, who obviously still had hankerings after the Camorra theory.

»In a garage, waiting to be called for,« said the first-class passenger promptly.

»Where?« persisted the prim man.

»Oh! somewhere on the other side of wherever it was the murderer was staying. If you have a particular reason for not wanting it to be known that you were in a certain place at a specified time, it's not a bad idea to come back from the opposite direction. I rather think I should look for the car at West Felpham, and the hotel in the nearest town on the main road beyond where the two roads to East and West Felpham join. When you've found the car, you've found the name of the victim, naturally. As for the murderer, you will have to look for an active man, a good swimmer and ardent bicyclist—probably not very well off, since he cannot afford to have a car—who has been taking a holiday in the neighbourhood of the Felphams, and who has a good reason for disliking the victim, whoever he may be.«

»Well, I never,« said the elderly woman admiringly. »How beautiful you do put it all together. Like Sherlock Holmes, I do declare.«

»It's a very pretty theory,« said the prim man, »but, all the same, you'll find it's a secret society. Mark my words. Dear me! We're just running in. Only twenty minutes late. I call that very good for holiday-time. Will you excuse me? My bag is just under your feet.«

There was an eighth person in the compartment, who had remained throughout the conversation apparently buried in a newspaper. As the passengers decanted themselves upon the platform, this man touched the first-class passenger upon the arm.

»Excuse me, sir,« he said. »That was a very interesting suggestion of yours. My name is Winterbottom, and I am investigating this case. Do you mind giving me your name? I might wish to communicate with you later on.«

»Certainly,« said the first-class passenger. »Always delighted to have a finger in any pie, don't you know. Here is my card. Look me up any time you like.«

Detective-Inspector Winterbottom took the card and read the name:
\begin{center}
Lord~Peter Wimsey,\\
110A Piccadilly.
\end{center}

The \textit{Evening Views} vendor outside Piccadilly Tube Station arranged his placard with some care. It looked very well, he thought.

\begin{center}\scshape
Man With\\
No Face\\
Identified
\end{center}

It was, in his opinion, considerably more striking than that displayed by a rival organ, which announced, unimaginatively:
\begin{center}\scshape
Beach Murder\\
Victim\\
Identified
\end{center}

A youngish gentleman in a grey suit who emerged at that moment from the Criterion Bar appeared to think so too, for he exchanged a copper for the \textit{Evening Views}, and at once plunged into its perusal with such concentrated interest that he bumped into a hurried man outside the station and had to apologise.

The \textit{Evening Views}, grateful to murderer and victim alike for providing so useful a sensation in the dead days after the Bank Holiday, had torn Messrs. Negretti \& Zambra's rocketing thermometrical statistics from the »banner« position which they had occupied in the lunch edition, and substituted:
\begin{center}\scshape
Faceless Victim of Beach Outrage Identified

Murder Of Prominent\\
Publicity Artist\\

\pagebreak[1]

Police Clues
\end{center}

\begin{quotation}
The body of a middle-aged man who was discovered, attired only in a bathing-costume and with his face horribly disfigured by some jagged instrument, on the beach at East Felpham last Monday morning, has been identified as that of Mr~Coreggio Plant, studio manager of Messrs. Crichton Ltd., the well-known publicity experts of Holborn.

Mr~Plant, who was forty-five years of age and a bachelor, was spending his annual holiday in making a motoring tour along the West Coast. He had no companion with him and had left no address for the forwarding of letters, so that, without the smart work of Detective-Inspector Winterbottom of the Westshire police, his disappearance might not in the ordinary way have been noticed until he became due to return to his place of business in three weeks' time. The murderer had no doubt counted on this, and had removed the motor-car, containing the belongings of his victim, in the hope of covering up all traces of this dastardly outrage so as to gain time for escape.

A rigorous search for the missing car, however, eventuated in its discovery in a garage at West Felpham, where it had been left for decarbonisation and repairs to the magneto. Mr~Spiller, the garage proprietor, himself saw the man who left the car, and has furnished a description of him to the police. He is said to be a small, dark man of foreign appearance. The police hold a clue to his identity, and an arrest is confidently expected in the near future.

Mr~Plant was for fifteen years in the employment of Messrs. Crichton, being appointed Studio Manager in the latter years of the war. He was greatly liked by all his colleagues, and his skill in the lay-out and designing of advertisements did much to justify the truth of Messrs. Crichton's well-known slogan: »Crichton's for Admirable Advertising.«

The funeral of the victim will take place to-morrow at Golders Green Cemetery.

(Pictures on Back Page.)
\end{quotation}

Lord~Peter Wimsey turned to the back page. The portrait of the victim did not detain him long; it was one of those characterless studio photographs which establish nothing except that the sitter has a tolerable set of features. He noted that Mr~Plant had been thin rather than fat, commercial in appearance rather than artistic, and that the photographer had chosen to show him serious rather than smiling. A picture of East Felpham beach, marked with a cross where the body was found, seemed to arouse in him rather more than a casual interest. He studied it intently for some time, making little surprised noises. There was no obvious reason why he should have been surprised, for the photograph bore out in every detail the deductions he had made in the train. There was the curved line of sand, with a long spur of rock stretching out behind it into deep water, and running back till it mingled with the short, dry turf. Nevertheless, he looked at it for several minutes with close attention, before folding the newspaper and hailing a taxi; and when he was in the taxi he unfolded the paper and looked at it again.

»Your lordship having been kind enough,« said Inspector Winterbottom, emptying his glass rather too rapidly for true connoisseurship, »to suggest I should look you up in Town, I made bold to give you a call in passing. Thank you, I won't say no. Well, as you've seen in the papers by now, we found that car all right.«

Wimsey expressed his gratification at this result.

»And very much obliged I was to your lordship for the hint,« went on the Inspector generously, »not but what I wouldn't say but I should have come to the same conclusion myself, given a little more time. And, what's more, we're on the track of the man.«

»I see he's supposed to be foreign-looking. Don't say he's going to turn out to be a Camorrist after all!«

»No, my lord.« The Inspector winked. »Our friend in the corner had got his magazine stories a bit on the brain, if you ask me. And \textit{you} were a bit out too, my lord, with your bicyclist idea.«

»Was I\@? That's a blow.«

»Well, my lord, these here theories \textit{sound} all right, but half the time they're too fine-spun altogether. Go for the facts—that's our motto in the Force—facts and motive, and you won't go far wrong.«

»Oh! you've discovered the motive, then?«

The Inspector winked again.

»There's not many motives for doing a man in,« said he. »Women or money—or women \textit{and} money—it mostly comes down to one or the other. This fellow Plant went in for being a bit of a lad, you see. He kept a little cottage down Felpham way, with a nice little skirt to furnish it and keep the love-nest warm for him—see?«

»Oh! I thought he was doing a motor-tour.«

»Motor-tour your foot!« said the Inspector, with more energy than politeness. »That's what the old [epithet] told 'em at the office. Handy reason, don't you see, for leaving no address behind him. No, no. There was a lady in it all right. I've seen her. A very taking piece too, if you like 'em skinny, which I don't. I prefer 'em better upholstered myself.«

»That chair is really more comfortable with a cushion,« put in Wimsey, with anxious solicitude. »Allow me.«

»Thanks, my lord, thanks. I'm doing very well. It seems that this woman—by the way, we're speaking in confidence, you understand. I don't want this to go further till I've got my man under lock and key.«

Wimsey promised discretion.

»That's all right, my lord, that's all right. I know I can rely on you. Well, the long and the short is, this young woman had another fancy man—a sort of an Italiano, whom she'd chucked for Plant, and this same dago got wind of the business and came down to East Felpham on the Sunday night, looking for her. He's one of these professional partners in a Palais de Danse up Cricklewood way, and that's where the girl comes from, too. I suppose she thought Plant was a cut above him. Anyway, down he comes, and busts in upon them Sunday night when they were having a bit of supper—and that's when the row started.«

»Didn't you know about this cottage and the goings-on there?«

»Well, you know, there's such a lot of these week-enders nowadays. We can't keep tabs on all of them, so long as they behave themselves and don't make a disturbance. The woman's been there—so they tell me—since last June, with him coming down Saturday to Monday; but it's a lonely spot, and the constable didn't take much notice. He came in the evenings, so there wasn't anybody much to recognise him, except the old girl who did the slops and things, and she's half-blind. And of course, when they found him, he hadn't any face to recognise. It'd be thought he'd just gone off in the ordinary way. I dare say the dago fellow reckoned on that. As I was saying, there was a big row, and the dago was kicked out. He must have lain wait for Plant down by the bathing-place, and done him in.«

»By strangling?«

»Well, he \textit{was} strangled.«

»Was his face cut up with a knife, then?«

»Well, no—I don't think it was a knife. More like a broken bottle, I should say, if you ask me. There's plenty of them come in with the tide.«

»But then we're brought back to our old problem. If this Italian was lying in wait to murder Plant, why didn't he take a weapon with him, instead of trusting to the chance of his hands and a broken bottle?«

The Inspector shook his head.

»Flighty,« he said. »All these foreigners are flighty. No headpiece. But there's our man and there's our motive, plain as a pikestaff. You don't want more.«

»And where is the Italian fellow now?«

»Run away. That's pretty good proof of guilt in itself. But we'll have him before long. That's what I've come to Town about. He can't get out of the country. I've had an all-stations call sent out to stop him. The dance-hall people were able to supply us with a photo and a good description. I'm expecting a report in now any minute. In fact, I'd best be getting along. Thank you very much for your hospitality, my lord.«

»The pleasure is mine,« said Wimsey, ringing the bell to have the visitor shown out. »I have enjoyed our little chat immensely.«

Sauntering into the Falstaff at twelve o'clock the following morning, Wimsey, as he had expected, found Salcombe Hardy supporting his rather plump contours against the bar. The reporter greeted his arrival with a heartiness amounting almost to enthusiasm, and called for two large Scotches immediately. When the usual skirmish as to who should pay had been honourably settled by the prompt disposal of the drinks and the standing of two more, Wimsey pulled from his pocket the copy of last night's \textit{Evening Views}.

»I wish you'd ask the people over at your place to get hold of a decent print of this for me,« he said, indicating the picture of East Felpham beach.

Salcome Hardy gazed limpid enquiry at him from eyes like drowned violets.

»See here, you old sleuth,« he said, »does this mean you've got a theory about the thing? I'm wanting a story badly. Must keep up the excitement, you know. The police don't seem to have got any further since last night.«

»No; I'm interested in this from another point of view altogether. I did have a theory—of sorts—but it seems it's all wrong. Bally old Homer nodding, I suppose. But I'd like a copy of the thing.«

»I'll get Warren to get you one when we come back. I'm just taking him down with me to Crichton's. We're going to have a look at a picture. I say, I wish you'd come too. Tell me what to say about the damned thing.«

»Good God! I don't know anything about commercial art.«

»'Tisn't commercial art. It's supposed to be a portrait of this blighter Plant. Done by one of the chaps in his studio or something. Kid who told me about it says it's clever. I don't know. Don't suppose she knows, either. You go in for being artistic, don't you?«

»I wish you wouldn't use such filthy expressions, Sally. Artistic! Who is this girl?«

»Typist in the copy department.«

»Oh, Sally!«

»Nothing of that sort. I've never met her. Name's Gladys Twitterton. I'm sure that's beastly enough to put anybody off. Rang us up last night and told us there was a bloke there who'd done old Plant in oils and was it any use to us? Drummer thought it might be worth looking into. Make a change from that everlasting syndicated photograph.«

»I see. If you haven't got an exclusive story, an exclusive picture's better than nothing. The girl seems to have her wits about her. Friend of the artist's?«

»No—said he'd probably be frightfully annoyed at her having told me. But I can wangle that. Only I wish you'd come and have a look at it. Tell me whether I ought to say it's an unknown masterpiece or merely a striking likeness.«

»How the devil can I say if it's a striking likeness of a bloke I've never seen?«

»I'll say it's that, in any case. But I want to know if it's well painted.«

»Curse it, Sally, what's it matter whether it is or not? I've got other things to do. Who's the artist, by the way? Anybody one's ever heard of?«

»Dunno. I've got the name here somewhere.« Sally rooted in his hip-pocket and produced a mass of dirty correspondence, its angles blunted by constant attrition. »Some comic name like Buggle or Snagtooth—wait a bit—here it is. Crowder. Thomas Crowder. I knew it was something out of the way.«

»Singularly like Buggle or Snagtooth. All right, Sally, I'll make a martyr of myself. Lead me to it.«

»We'll have another quick one. Here's Warren. This is Lord~Peter Wimsey. This is on me.«

»On me,« corrected the photographer, a jaded young man with a disillusioned manner. »Three large White Labels, please. Well, here's all the best. Are you fit, Sally? Because we'd better make tracks. I've got to be up at Golders Green by two for the funeral.«

Mr~Crowder of Crichton's appeared to have had the news broken to him already by Miss Twitterton, for he received the embassy in a spirit of gloomy acquiescence.

»The directors won't like it,« he said, »but they've had to put up with such a lot that I suppose one irregularity more or less won't give 'em apoplexy.« He had a small, anxious, yellow face like a monkey. Wimsey put him down as being in his late thirties. He noticed his fine, capable hands, one of which was disfigured by a strip of sticking-plaster.

»Damaged yourself?« said Wimsey pleasantly, as they made their way upstairs to the studio. »Mustn't make a practice of that, what? An artist's hands are his livelihood—except, of course, for Armless Wonders and people of that kind! Awkward job, painting with your toes.«

»Oh, it's nothing much,« said Crowder, »but it's best to keep the paint out of surface scratches. There's such a thing as lead-poisoning. Well, here's this dud portrait, such as it is. I don't mind telling you that it didn't please the sitter. In fact, he wouldn't have it at any price.«

»Not flattering enough?« asked Hardy.

»As you say.« The painter pulled out a four by three canvas from its hiding-place behind a stack of poster cartoons, and heaved it up on to the easel.

»Oh!« said Hardy, a little surprised. Not that there was any reason for surprise as far as the painting itself was concerned. It was a straight-forward handling enough; the skill and originality of the brush-work being of the kind that interests the painter without shocking the ignorant.

»Oh!« said Hardy. »Was he really like that?«

He moved closer to the canvas, peering into it as he might have peered into the face of the living man, hoping to get something out of him. Under this microscopic scrutiny, the portrait, as is the way of portraits, dislimned, and became no more than a conglomeration of painted spots and streaks. He made the discovery that, to the painter's eye, the human face is full of green and purple patches.

He moved back again, and altered the form of his question:

»So that's what he was like, was he?«

He pulled out the photograph of Plant from his pocket, and compared it with the portrait. The portrait seemed to sneer at his surprise.

»Of course, they touch these things up at these fashionable photographers,« he said. »Anyway, that's not my business. This thing will make a jolly good eye-catcher, don't you think so, Wimsey? Wonder if they'd give us a two-column spread on the front page? Well, Warren, you'd better get down to it.«

The photographer, bleakly unmoved by artistic or journalistic considerations, took silent charge of the canvas, mentally resolving it into a question of pan-chromatic plates and coloured screens. Crowder gave him a hand in shifting the easel into a better light. Two or three people from other departments, passing through the studio on their lawful occasions, stopped, and lingered in the neighbourhood of the disturbance, as though it were a street accident. A melancholy, grey-haired man, temporary head of the studio, vice Coreggio Plant, deceased, took Crowder aside, with a muttered apology, to give him some instructions about adapting a whole quad to an eleven-inch treble. Hardy turned to Lord~Peter.

»It's damned ugly,« he said. »Is it good?«

»Brilliant,« said Wimsey. »You can go all out. Say what you like about it.«

»Oh, splendid! Could we discover one of our neglected British masters?«

»Yes; why not? You'll probably make the man the fashion and ruin him as an artist, but that's his pigeon.«

»But, I say—do you think it's a good likeness? He's made him look a most sinister sort of fellow. After all, Plant thought it was so bad he wouldn't have it.«

»The more fool he. Ever heard of the portrait of a certain statesman that was so revealing of his inner emptiness that he hurriedly bought it up and hid it to prevent people like you from getting hold of it?«

Crowder came back.

»I say,« said Wimsey, »whom does that picture belong to? You? Or the heirs of the deceased, or what?«

»I suppose it's back on my hands,« said the painter. »Plant—well, he more or less commissioned it, you see, but\longdash«

»How more or less?«

»Well, he kept on hinting, don't you know, that he would like me to do him, and, as he was my boss, I thought I'd better. No price actually mentioned. When he saw it, he didn't like it, and told me to alter it.«

»But you didn't.«

»Oh—well, I put it aside and said I'd see what I could do with it. I thought he'd perhaps forget about it.«

»I see. Then presumably it's yours to dispose of.«

»I should think so. Why?«

»You have a very individual technique, haven't you?« pursued Wimsey. »Do you exhibit much?«

»Here and there. I've never had a show in London.«

»I fancy I once saw a couple of small sea-scapes of yours somewhere. Manchester, was it? or Liverpool? I wasn't sure of your name, but I recognised the technique immediately.«

»I dare say. I did send a few things to Manchester about two years ago.«

»Yes—I felt sure I couldn't be mistaken. I want to buy the portrait. Here's my card, by the way. I'm not a journalist; I collect things.«

Crowder looked from the card to Wimsey and from Wimsey to the card, a little reluctantly.

»If you want to exhibit it, of course,« said Lord~Peter, »I should be delighted to leave it with you as long as you liked.«

»Oh, it's not that,« said Crowder. »The fact is, I'm not altogether keen on the thing. I should like to—that is to say, it's not really finished.«

»My dear man, it's a bally masterpiece.«

»Oh, the painting's all right. But it's not altogether satisfactory as a likeness.«

»What the devil does the likeness matter? I don't know what the late Plant looked like and I don't care. As I look at the thing it's a damn fine bit of brush-work, and if you tinker about with it you'll spoil it. You know that as well as I do. What's biting you? It isn't the price, is it? You know I shan't boggle about that. I can afford my modest pleasures, even in these thin and piping times. You don't want me to have it? Come now—what's the real reason?«

»There's no reason at all why you shouldn't have it if you really want it, I suppose,« said the painter, still a little sullenly. »If it's really the painting that interests you.«

»What do you suppose it is? The notoriety? I can have all I want of \textit{that} commodity, you know, for the asking—or even without asking. Well, anyhow, think it over, and when you've decided, send me a line and name your price.«

Crowder nodded without speaking, and the photographer having by this time finished his job, the party took their leave.

As they left the building, they became involved in the stream of Crichton's staff going out to lunch. A girl, who seemed to have been loitering in a semi-intentional way in the lower hall, caught them as the lift descended.

»Are you the \textit{Evening Views} people? Did you get your picture all right?«

»Miss Twitterton?« said Hardy interrogatively. »Yes, rather—thank you so much for giving us the tip. You'll see it on the front page this evening.«

»Oh! that's splendid! I'm frightfully thrilled. It has made an excitement here—all this business. Do they know anything yet about who murdered Mr~Plant? Or am I being horribly indiscreet?«

»We're expecting news of an arrest any minute now,« said Hardy. »As a matter of fact, I shall have to buzz back to the office as fast as I can, to sit with one ear glued to the telephone. You will excuse me, won't you? And, look here—will you let me come round another day, when things aren't so busy, and take you out to lunch?«

»Of course. I should love to.« Miss Twitterton giggled. »I do so want to hear about all the murder cases.«

»Then here's the man to tell you about them, Miss Twitterton,« said Hardy, with mischief in his eye. »Allow me to introduce Lord~Peter Wimsey.«

Miss Twitterton offered her hand in an ecstasy of excitement which almost robbed her of speech.

»How do you do?« said Wimsey. »As this blighter is in such a hurry to get back to his gossip-shop, what do you say to having a spot of lunch with me?«

»Well, really\longdash« began Miss Twitterton.

»He's all right,« said Hardy; »he won't lure you into any gilded dens of infamy. If you look at him, you will see he has a kind, innocent face.«

»I'm sure I never thought of such a thing,« said Miss Twitterton. »But you know—really—I've only got my old things on. It's no good wearing anything decent in this dusty old place.«

»Oh, nonsense!« said Wimsey. »You couldn't possibly look nicer. It isn't the frock that matters—it's the person who wears it. \textit{That's} all right, then. See you later, Sally! Taxi! Where shall we go? What time do you have to be back, by the way?«

»Two o'clock,« said Miss Twitterton regretfully.

»Then we'll make the Savoy do,« said Wimsey; »it's reasonably handy.«

Miss Twitterton hopped into the waiting taxi with a little squeak of agitation.

»Did you see Mr~Crichton?« she said. »He went by just as we were talking. However, I dare say he doesn't really know me by sight. I hope not—or he'll think I'm getting too grand to need a salary.» She rooted in her hand-bag. «I'm sure my face is getting all shiny with excitement. What a silly taxi. It hasn't got a mirror—and I've bust mine.«

Wimsey solemnly produced a small looking-glass from his pocket.

»How wonderfully competent of you!« exclaimed Miss Twitterton. »I'm afraid, Lord~Peter, you are used to taking girls about.«

»Moderately so,« said Wimsey. He did not think it necessary to mention that the last time he had used that mirror it had been to examine the back teeth of a murdered man.

»Of course,« said Miss Twitterton, »they had to say he was popular with his colleagues. Haven't you noticed that murdered people are always well dressed and popular?«

»They have to be,« said Wimsey. »It makes it more mysterious and pathetic. Just as girls who disappear are always bright and home-loving and have no men friends.«

»Silly, isn't it?« said Miss Twitterton, with her mouth full of roast duck and green peas. »I should think everybody was only too glad to get rid of Plant—nasty, rude creature. So mean, too, always taking credit for other people's work. All those poor things in the studio, with all the spirit squashed out of them. I always say, Lord~Peter, you can tell if a head of a department's fitted for his job by noticing the atmosphere of the place as you go into it. Take the copy-room, now. We're all as cheerful and friendly as you like, though I must say the language that goes on there is something awful, but these writing fellows are like that, and they don't mean anything by it. But then, Mr~Ormerod is a real gentleman—that's our copy-chief, you know—and he makes them all take an interest in the work, for all they grumble about the cheese-bills and the department-store bilge they have to turn out. But it's quite different in the studio. A sort of dead-and-alive feeling about it, if you understand what I mean. We girls notice things like that more than some of the high-up people think. Of course, I'm very sensitive to these feelings—almost psychic, I've been told.«

Lord~Peter said there was nobody like a woman for sizing up character at a glance. Women, he thought, were remarkably intuitive.

»That's a fact,« said Miss Twitterton. »I've often said, if I could have a few frank words with Mr~Crichton, I could tell him a thing or two. There are wheels within wheels beneath the surface of a place like this that these brass-hats have no idea of.«

Lord~Peter said he felt sure of it.

»The way Mr~Plant treated people he thought were beneath him,« went on Miss Twitterton, »I'm sure it was enough to make your blood boil. I'm sure, if Mr~Ormerod sent me with a message to him, I was glad to get out of the room again. Humiliating, it was, the way he'd speak to you. I don't care if he's dead or not; being dead doesn't make a person's past behaviour any better, Lord~Peter. It wasn't so much the rude things he said. There's Mr~Birkett, for example; \textit{he's} rude enough, but nobody minds him. He's just like a big, blundering puppy—rather a lamb, really. It was Mr~Plant's nasty sneering way we all hated so. And he was always running people down.«

»How about this portrait?« asked Wimsey. »Was it like him at all?«

»It was a lot too like him,« said Miss Twitterton emphatically. »That's why he hated it so. He didn't like Crowder, either. But, of course, he knew he could paint, and he made him do it, because he thought he'd be getting a valuable thing cheap. And Crowder couldn't very well refuse, or Plant would have got him sacked.«

»I shouldn't have thought that would have mattered much to a man of Crowder's ability.«

»Poor Mr~Crowder! I don't think he's ever had much luck. Good artists don't always seem able to sell their pictures. And I know he wanted to get married—otherwise he'd never have taken up this commercial work. He's told me a good bit about himself. I don't know why—but I'm one of the people men seem to tell things to.«

Lord~Peter filled Miss Twitterton's glass.

»Oh, please! No, really! Not a drop more! I'm talking a lot too much as it is. I don't know what Mr~Ormerod will say when I go in to take his letters. I shall be writing down all kinds of funny things. Ooh! I really must be getting back. Just look at the time!«

»It's not really late. Have a black coffee—just as a corrective.« Wimsey smiled. »You haven't been talking at all too much. I've enjoyed your picture of office life enormously. You have a very vivid way of putting things, you know. I see now why Mr~Plant was not altogether a popular character.«

»Not in the office, anyway—whatever he may have been elsewhere,« said Miss Twitterton darkly.

»Oh?«

»Oh! he was a one,« said Miss Twitterton. »He certainly was a one. Some friends of mine met him one evening up in the West End, and they came back with some nice stories. It was quite a joke in the office—old Plant and his rosebuds, you know. Mr~Cowley—he's \textit{the} Cowley, you know, who rides in the motor-cycle races—he always said he knew what to think of Mr~Plant and his motor-tours. That time Mr~Plant pretended he'd gone touring in Wales, Mr~Cowley was asking him about the roads, and he didn't know a thing about them. Because Mr~Cowley really had been touring there, and he knew quite well Mr~Plant hadn't been where he said he had; and, as a matter of fact, Mr~Cowley knew he'd been staying the whole time in a hotel at Aberystwyth, in very attractive company.«

Miss Twitterton finished her coffee and slapped the cup down defiantly.

»And now I really \textit{must} run away, or I shall be most dreadfully late. And thank you ever so much.«

»Hullo!« said Inspector Winterbottom, »you've bought that portrait, then?«

»Yes,« said Wimsey. »It's a fine bit of work.« He gazed thoughtfully at the canvas. »Sit down, inspector; I want to tell you a story.«

»And I want to tell \textit{you} a story,« replied the inspector.

»Let's have yours first,« said Wimsey, with an air of flattering eagerness.

»No, no, my lord. You take precedence. Go ahead.«

He snuggled down with a chuckle into his arm-chair.

»Well!« said Wimsey. »Mine's a sort of a fairy-story. And, mind you, I haven't verified it.«

»Go ahead, my lord, go ahead.«

»Once upon a time\longdash« said Wimsey, sighing.

»That's the good old-fashioned way to begin a fairy-story,« said Inspector Winterbottom.

»Once upon a time,« repeated Wimsey, »there was a painter. He was a good painter, but the bad fairy of Financial Success had not been asked to his christening—what?«

»That's often the way with painters,« agreed the inspector.

»So he had to take up a job as a commercial artist, because nobody would buy his pictures and, like so many people in fairy-tales, he wanted to marry a goose-girl.«

»There's many people want to do the same,« said the inspector.

»The head of his department,« went on Wimsey, »was a man with a mean, sneering soul. He wasn't even really good at his job, but he had been pushed into authority during the war, when better men went to the Front. Mind you, I'm rather sorry for the man. He suffered from an inferiority complex«—the inspector snorted—»and he thought the only way to keep his end up was to keep other people's end down. So he became a little tin tyrant and a bully. He took all the credit for the work of the men under his charge, and he sneered and harassed them till they got inferiority complexes even worse than his own.«

»I've known that sort,« said the inspector, »and the marvel to me is how they get away with it.«

»Just so,« said Wimsey. »Well, I dare say this man would have gone on getting away with it all right, if he hadn't thought of getting this painter to paint his portrait.«

»Damn silly thing to do,« said the inspector. »It was only making the painter-fellow conceited with himself.«

»True. But, you see, this tin tyrant person had a fascinating female in tow, and he wanted the portrait for the lady. He thought that, by making the painter do it, he would get a good portrait at starvation price. But unhappily he'd forgotten that, however much an artist will put up with in the ordinary way, he is bound to be sincere with his art. That's the one thing a genuine artist won't muck about with.«

»I dare say,« said the inspector. »I don't know much about artists.«

»Well, you can take it from me. So the painter painted the portrait as he saw it, and he put the man's whole creeping, sneering, paltry soul on the canvas for everybody to see.«

Inspector Winterbottom stared at the portrait, and the portrait sneered back at him.

»It's not what you'd call a flattering picture, certainly,« he admitted.

»Now, when a painter paints a portrait of anybody,« went on Wimsey, »that person's face is never the same to him again. It's like—what shall I say? Well, it's like the way a gunner, say, looks at a landscape where he happens to be posted. He doesn't see it as a landscape. He doesn't see it as a thing of magic beauty, full of sweeping lines and lovely colour. He sees it as so much cover, so many landmarks to aim by, so many gun-emplacements. And when the war is over and he goes back to it, he will still see it as cover and landmarks and gun-emplacements. It isn't a landscape any more. It's a war map.«

»I know that,« said Inspector Winterbottom. »I was a gunner myself.«

»A painter gets just the same feeling of deadly familiarity with every line of a face he's once painted,« pursued Wimsey. »And, if it's a face he hates, he hates it with a new and more irritable hatred. It's like a defective barrel-organ, everlastingly grinding out the same old maddening tune, and making the same damned awful wrong note every time the barrel goes round.«

»Lord! how you can talk!« ejaculated the inspector.

»That was the way the painter felt about this man's hateful face. All day and every day he had to see it. He couldn't get away because he was tied to his job, you see.«

»He ought to have cut loose,« said the inspector. »It's no good going on like that, trying to work with uncongenial people.«

»Well, anyway, he said to himself, he could escape for a bit during his holidays. There was a beautiful little quiet spot he knew on the West Coast, where nobody ever came. He'd been there before and painted it. Oh! by the way, that reminds me—I've got another picture to show you.«

He went to a bureau and extracted a small panel in oils from a drawer.

»I saw that two years ago at a show in Manchester, and I happened to remember the name of the dealer who bought it.«

Inspector Winterbottom gaped at the panel.

»But that's East Felpham!« he exclaimed.

»Yes. It's only signed \textsc{t.c.}, but the technique is rather unmistakable, don't you think?«

The inspector knew little about technique, but initials he understood. He looked from the portrait to the panel and back at Lord~Peter.

»The painter\longdash«

»Crowder?«

»If it's all the same to you, I'd rather go on calling him the painter. He packed up his traps on his push-bike carrier, and took his tormented nerves down to this beloved and secret spot for a quiet week-end. He stayed at a quiet little hotel in the neighbourhood, and each morning he cycled off to this lovely little beach to bathe. He never told anybody at the hotel where he went, because it was \textit{his} place, and he didn't want other people to find it out.«

Inspector Winterbottom set the panel down on the table, and helped himself to whisky.

»One morning—it happened to be the Monday morning«—Wimsey's voice became slower and more reluctant—»he went down as usual. The tide was not yet fully in, but he ran out over the rocks to where he knew there was a deep bathing-pool. He plunged in and swam about, and let the small noise of his jangling troubles be swallowed up in the innumerable laughter of the sea.«

»Eh?«

»[Greek: kumatôn anêrithmon gelasma]—quotation from the classics. Some people say it means the dimpled surface of the waves in the sunlight—but how could Prometheus, bound upon his rock, have seen it? Surely it was the chuckle of the incoming tide among the stones that came up to his ears on the lonely peak where the vulture fretted at his heart. I remember arguing about it with old Philpotts in class, and getting rapped over the knuckles for contradicting him. I didn't know at the time that he was engaged in producing a translation on his own account, or doubtless I should have contradicted him more rudely and been told to take my trousers down. Dear old Philpotts!«

»I don't know anything about that,« said the inspector.

»I beg your pardon. Shocking way I have of wandering. The painter—well! he swam round the end of the rocks, for the tide was nearly in by that time; and, as he came up from the sea, he saw a man standing on the beach—that beloved beach, remember, which he thought was his own sacred haven of peace. He came wading towards it, cursing the Bank Holiday rabble who must needs swarm about everywhere with their cigarette-packets and their kodaks and their gramophones—and then he saw that it was a face he knew. He knew every hated line in it, on that clear sunny morning. And, early as it was, the heat was coming up over the sea like a haze.«

»It was a hot week-end,« said the Inspector.

»And then the man hailed him, in his smug, mincing voice. »Hullo!« he said, »you here? How did you find my little bathing-place?« And that was too much for the painter. He felt as if his last sanctuary had been invaded. He leapt at the lean throat—it's rather a stringy one, you may notice, with a prominent Adam's apple—an irritating throat. The water chuckled round their feet as they swayed to and fro. He felt his thumbs sink into the flesh he had painted. He saw, and laughed to see, the hateful familiarity of the features change and swell into an unrecognisable purple. He watched the sunken eyes bulge out and the thin mouth distort itself as the blackened tongue thrust through it—I am not unnerving you, I hope?«

The inspector laughed.

»Not a bit. It's wonderful, the way you describe things. You ought to write a book.«

\begin{quote}
»I sing but as the throstle sings,\\
Amid the branches dwelling,«
\end{quote}
\indent replied his lordship negligently, and went on without further comment.

»The painter throttled him. He flung him back on the sand. He looked at him, and his heart crowed within him. He stretched out his hand, and found a broken bottle, with a good jagged edge. He went to work with a will, stamping and tearing away every trace of the face he knew and loathed. He blotted it out and destroyed it utterly.

He sat beside the thing he had made. He began to be frightened. They had staggered back beyond the edge of the water, and there were the marks of his feet on the sand. He had blood on his face and on his bathing-suit, and he had cut his hand with the bottle. But the blessed sea was still coming in. He watched it pass over the bloodstains and the footprints and wipe the story of his madness away. He remembered that this man had gone from his place, leaving no address behind him. He went back, step by step, into the water, and, as it came up to his breast, he saw the red stains smoke away like a faint mist in the brown-blueness of the tide. He went—wading and swimming and plunging his face and arms deep in the water, looking back from time to time to see what he had left behind him. I think that when he got back to the point and drew himself out, clean and cool, upon the rocks, he remembered that he ought to have taken the body back with him and let the tide carry it away, but it was too late. He was clean, and he could not bear to go back for the thing. Besides, he was late, and they would wonder at the hotel if he was not back in time for breakfast. He ran lightly over the bare rocks and the grass that showed no footprint. He dressed himself, taking care to leave no trace of his presence. He took the car, which would have told a story. He put his bicycle in the back seat, under the rugs, and he went—but you know as well as I do where he went.«

Lord~Peter got up with an impatient movement, and went over to the picture, rubbing his thumb meditatively over the texture of the painting.

»You may say, if he hated the face so much, why didn't he destroy the picture? He couldn't. It was the best thing he'd ever done. He took a hundred guineas for it. It was cheap at a hundred guineas. But then—I think he was afraid to refuse me. My name is rather well known. It was a sort of blackmail, I suppose. But I wanted that picture.«

Inspector Winterbottom laughed again.

»Did you take any steps, my lord, to find out if Crowder has really been staying at East Felpham?«

»No.« Wimsey swung round abruptly. »I have taken no steps at all. That's your business. I have told you the story, and, on my soul, I'd rather have stood by and said nothing.«

»You needn't worry.« The inspector laughed for the third time. »It's a good story, my lord, and you told it well. But you're right when you say it's a fairy-story. We've found this Italian fellow—Franceso, he called himself, and he's the man all right.«

»How do you know? Has he confessed?«

»Practically. He's dead. Killed himself. He left a letter to the woman, begging her forgiveness, and saying that when he saw her with Plant he felt murder come into his heart. »I have revenged myself,« he says, »on him who dared to love you.« I suppose he got the wind up when he saw we were after him—I wish these newspapers wouldn't be always putting these criminals on their guard—so he did away with himself to cheat the gallows. I may say it's been a disappointment to me.«

»It must have been,« said Wimsey. »Very unsatisfactory, of course. But I'm glad my story turned out to be only a fairy-tale after all. You're not going?«

»Got to get back to my duty,« said the inspector, heaving himself to his feet. »Very pleased to have met you, my lord. And I mean what I say—you ought to take to literature.«

Wimsey remained after he had gone, still looking at the portrait.

»»What is Truth?« said jesting Pilate. No wonder, since it is so completely unbelievable.... I could prove it ... if I liked ... but the man had a villainous face, and there are few good painters in the world.«