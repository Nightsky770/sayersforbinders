%!TeX root=../viewsbodytop.tex
\addchap{The Entertaining Episode of the Article in Question}
	
\lettrine[lines=4]{T}{he} unprofessional detective career of Lord Peter Wimsey was regulated (though the word has no particular propriety in this connection) by a persistent and undignified inquisitiveness. The habit of asking silly questions—natural, though irritating, in the immature male—remained with him long after his immaculate man, Bunter, had become attached to his service to shave the bristles from his chin and see to the due purchase and housing of Napoleon brandies and Villar y Villar cigars. At the age of thirty-two his sister Mary christened him Elephant's Child. It was his idiotic enquiries (before his brother, the Duke of Denver, who grew scarlet with mortification) as to what the Woolsack was really stuffed with that led the then Lord Chancellor idly to investigate the article in question, and to discover, tucked deep within its recesses, that famous diamond necklace of the Marchioness of Writtle, which had disappeared on the day Parliament was opened and been safely secreted by one of the cleaners. It was by a continual and personal badgering of the Chief Engineer at 2LO on the question of »Why is Oscillation and How is it Done?« that his lordship incidentally unmasked the great Ploffsky gang of Anarchist conspirators, who were accustomed to converse in code by a methodical system of howls, superimposed (to the great annoyance of listeners in British and European stations) upon the London wave-length and duly relayed by 5XX over a radius of some five or six hundred miles. He annoyed persons of more leisure than decorum by suddenly taking into his head to descend to the Underground by way of the stairs, though the only exciting things he ever actually found there were the bloodstained boots of the Sloane Square murderer; on the other hand, when the drains were taken up at Glegg's Folly, it was by hanging about and hindering the plumbers at their job that he accidentally made the discovery which hanged that detestable poisoner, William Girdlestone Chitty.

Accordingly, it was with no surprise at all that the reliable Bunter, one April morning, received the announcement of an abrupt change of plan.

They had arrived at the Gare St Lazare in good time to register the luggage. Their three months' trip to Italy had been purely for enjoyment, and had been followed by a pleasant fortnight in Paris. They were now intending to pay a short visit to the Duc de Sainte-Croix in Rouen on their way back to England. Lord Peter paced the Salle des Pas Perdus for some time, buying an illustrated paper or two and eyeing the crowd. He bent an appreciative eye on a slim, shingled creature with the face of a Paris \textit{gamin}, but was forced to admit to himself that her ankles were a trifle on the thick side; he assisted an elderly lady who was explaining to the bookstall clerk that she wanted a map of Paris and not a \textit{carte postale}, consumed a quick cognac at one of the little green tables at the far end, and then decided he had better go down and see how Bunter was getting on.

In half an hour Bunter and his porter had worked themselves up to the second place in the enormous queue—for, as usual, one of the weighing-machines was out of order. In front of them stood an agitated little group—the young woman Lord Peter had noticed in the Salle des Pas Perdus, a sallow-faced man of about thirty, their porter, and the registration official, who was peering eagerly through his little \textit{guichet}.

\begin{french}»Mais je te répète que je ne les ai pas,« \end{french} said the sallow man heatedly. \begin{french}»Voyons, voyons. C'est bien toi qui les as pris, n'est-ce-pas? Eh bien, alors, comment veux-tu que je les aie, moi?«

»Mais non, mais non, je te les ai bien donnés là-haut, avant d'aller chercher les journaux.«

»Je t'assure que non. Enfin, c'est évident! J'ai cherché partout, que diable! Tu ne m'as rien donné, du tout, du tout.«

»Mais puisque je t'ai dit d'aller faire enrégistrer les bagages! Ne faut-il pas que je t'aie bien remis les billets? Me prends-tu pour un imbécile? Va! On n'est pas dépourvu de sens! Mais regarde l'heure! Le train part à 11 h. 20 m. Cherche un peu, au moins.«

»Mais puisque j'ai cherché partout—le gilet, rien! Le jacquet rien, rien! Le pardessus—rien! rien! rien! C'est toi\longdash«
\end{french}

Here the porter, urged by the frantic cries and stamping of the queue, and the repeated insults of Lord Peter's porter, flung himself into the discussion.
\begin{french}»P't-être qu' m'sieur a bouté les billets dans son pantalon,«\end{french} he suggested.

»Triple idiot!« snapped the traveller, \begin{french}»je vous le demande—est-ce qu'on a jamais entendu parler de mettre des billets dans son pantalon? Jamais\longdash«\end{french}

The French porter is a Republican, and, moreover, extremely ill-paid. The large tolerance of his English colleague is not for him.

»Ah!« said he, dropping two heavy bags and looking round for moral support. \begin{french}»Vous dîtes? En voila du joli! Allons, mon p'tit, ce n'est pas parce qu'on porte un faux col qu'on a le droit d'insulter les gens.«\end{french}

The discussion might have become a full-blown row, had not the young man suddenly discovered the missing tickets—incidentally, they were in his trousers-pocket after all—and continued the registration of his luggage, to the undisguised satisfaction of the crowd.

»Bunter,« said his lordship, who had turned his back on the group and was lighting a cigarette, »I am going to change the tickets. We shall go straight on to London. Have you got that snapshot affair of yours with you?«

»Yes, my lord.«

»The one you can work from your pocket without anyone noticing?«

»Yes, my lord.«

»Get me a picture of those two.«

»Yes, my lord.«

»I will see to the luggage. Wire to the Duc that I am unexpectedly called home.«

»Very good, my lord.«

Lord Peter did not allude to the matter again till Bunter was putting his trousers in the press in their cabin on board the \textit{Normannia}. Beyond ascertaining that the young man and woman who had aroused his curiosity were on the boat as second-class passengers, he had sedulously avoided contact with them.

»Did you get that photograph?«

»I hope so, my lord. As your lordship knows, the aim from the breast-pocket tends to be unreliable. I have made three attempts, and trust that one at least may prove to be not unsuccessful.«

»How soon can you develop them?«

»At once, if your lordship pleases. I have all the materials in my suit case.«

»What fun!« said Lord Peter, eagerly tying himself into a pair of mauve silk pyjamas. »May I hold the bottles and things?«

Mr Bunter poured 3 ounces of water into an 8-ounce measure, and handed his master a glass rod and a minute packet.

»If your lordship would be so good as to stir the contents of the white packet slowly into the water,« he said, bolting the door, »and, when dissolved, add the contents of the blue packet.«

»Just like a Seidlitz powder,« said his lordship happily. »Does it fizz?«

»Not much, my lord,« replied the expert, shaking a quantity of hypo crystals into the hand-basin.

»That's a pity,« said Lord Peter. »I say, Bunter, it's no end of a bore to dissolve.«

»Yes, my lord,« returned Bunter sedately. »I have always found that part of the process exceptionally tedious, my lord.«

Lord Peter jabbed viciously with the glass rod.

»Just you wait,« he said, in a vindictive tone, »till we get to Waterloo.«

\noindent\hfil\rule{0.5\textwidth}{.4pt}\hfil 

Three days later Lord Peter Wimsey sat in his book-lined sitting-room at 110A Piccadilly. The tall bunches of daffodils on the table smiled in the spring sunshine, and nodded to the breeze which danced in from the open window. The door opened, and his lordship glanced up from a handsome edition of the \textit{Contes de la Fontaine}, whose handsome hand-coloured Fragonard plates he was examining with the aid of a lens.

»Morning, Bunter. Anything doing?«

»I have ascertained, my lord, that the young person in question has entered the service of the elder Duchess of Medway. Her name is Célestine Berger.«

»You are less accurate than usual, Bunter. Nobody off the stage is called Célestine. You should say »under the name of Célestine Berger.« And the man?«

»He is domiciled at this address in Guilford Street, Bloomsbury, my lord.«

»Excellent, my Bunter. Now give me \textit{Who's Who}. Was it a very tiresome job?«

»Not exceptionally so, my lord.«

»One of these days I suppose I shall give you something to do which you \textit{will} jib at,« said his lordship, »and you will leave me and I shall cut my throat. Thanks. Run away and play. I shall lunch at the club.«

The book which Bunter had handed his employer indeed bore the words \textit{Who's Who} engrossed upon its cover, but it was to be found in no public library and in no bookseller's shop. It was a bulky manuscript, closely filled, in part with the small print-like handwriting of Mr Bunter, in part with Lord Peter's neat and altogether illegible hand. It contained biographies of the most unexpected people, and the most unexpected facts about the most obvious people. Lord Peter turned to a very long entry under the name of the Dowager Duchess of Medway. It appeared to make satisfactory reading, for after a time he smiled, closed the book, and went to the telephone.

»Yes—this is the Duchess of Medway. Who is it?«

The deep, harsh old voice pleased Lord Peter. He could see the imperious face and upright figure of what had been the most famous beauty in the London of the `sixties.

»It's Peter Wimsey, duchess.«

»Indeed, and how do you do, young man? Back from your Continental jaunting?«

»Just home—and longing to lay my devotion at the feet of the most fascinating lady in England.«

»God bless my soul, child, what do you want?« demanded the duchess. »Boys like you don't flatter an old woman for nothing.«

»I want to tell you my sins, duchess.«

»You should have lived in the great days,« said the voice appreciatively. »Your talents are wasted on the young fry.«

»That is why I want to talk to you, duchess.«

»Well, my dear, if you've committed any sins worth hearing I shall enjoy your visit.«

»You are as exquisite in kindness as in charm. I am coming this afternoon.«

»I will be at home to you and to no one else. There.«

»Dear lady, I kiss your hands,« said Lord Peter, and he heard a deep chuckle as the duchess rang off.

\noindent\hfil\rule{0.5\textwidth}{.4pt}\hfil 

»You may say what you like, duchess,« said Lord Peter from his reverential position on the fender-stool, »but you are the youngest grandmother in London, not excepting my own mother.«

»Dear Honoria is the merest child,« said the duchess. »I have twenty years more experience of life, and have arrived at the age when we boast of them. I have every intention of being a great-grandmother before I die. Sylvia is being married in a fortnight's time, to that stupid son of Attenbury's.«

»Abcock?«

»Yes. He keeps the worst hunters I ever saw, and doesn't know still champagne from sauterne. But Sylvia is stupid, too, poor child, so I dare say they will get on charmingly. In my day one had to have either brains or beauty to get on—preferably both. Nowadays nothing seems to be required but a total lack of figure. But all the sense went out of society with the House of Lords' veto. I except you, Peter. You have talents. It is a pity you do not employ them in politics.«

»Dear lady, God forbid.«

»Perhaps you are right, as things are. There were giants in my day. Dear Dizzy. I remember so well, when his wife died, how hard we all tried to get him—Medway had died the year before—but he was wrapped up in that stupid Bradford woman, who had never even read a line of one of his books, and couldn't have understood `em if she had. And now we have Abcock standing for Midhurst, and married to Sylvia!«

»You haven't invited me to the wedding, duchess dear. I'm so hurt,« sighed his lordship.

»Bless you, child, \textit{I} didn't send out the invitations, but I suppose your brother and that tiresome wife of his will be there. You must come, of course, if you want to. I had no idea you had a passion for weddings.«

»Hadn't you?« said Peter. »I have a passion for this one. I want to see Lady Sylvia wearing white satin and the family lace and diamonds, and to sentimentalise over the days when my fox-terrier bit the stuffing out of her doll.«

»Very well, my dear, you shall. Come early and give me your support. As for the diamonds, if it weren't a family tradition, Sylvia shouldn't wear them. She has the impudence to complain of them.«

»I thought they were some of the finest in existence.«

»So they are. But she says the settings are ugly and old-fashioned, and she doesn't like diamonds, and they won't go with her dress. Such nonsense. Whoever heard of a girl not liking diamonds? She wants to be something romantic and moonshiny in pearls. I have no patience with her.«

»I'll promise to admire them,« said Peter—»use the privilege of early acquaintance and tell her she's an ass and so on. I'd love to have a view of them. When do they come out of cold storage?«

»Mr Whitehead will bring them up from the Bank the night before,« said the duchess, »and they'll go into the safe in my room. Come round at twelve o'clock and you shall have a private view of them.«

»That would be delightful. Mind they don't disappear in the night, won't you?«

»Oh, my dear, the house is going to be over-run with policemen. Such a nuisance. I suppose it can't be helped.«

»Oh, I think it's a good thing,« said Peter. »I have rather an unwholesome weakness for policemen.«

\noindent\hfil\rule{0.5\textwidth}{.4pt}\hfil 

On the morning of the wedding-day, Lord Peter emerged from Bunter's hands a marvel of sleek brilliance. His primrose-coloured hair was so exquisite a work of art that to eclipse it with his glossy hat was like shutting up the sun in a shrine of polished jet; his spats, light trousers, and exquisitely polished shoes formed a tone-symphony in monochrome. It was only by the most impassioned pleading that he persuaded his tyrant to allow him to place two small photographs and a thin, foreign letter in his breast-pocket. Mr Bunter, likewise immaculately attired, stepped into the taxi after him. At noon precisely they were deposited beneath the striped awning which adorned the door of the Duchess of Medway's house in Park Lane. Bunter promptly disappeared in the direction of the back entrance, while his lordship mounted the steps and asked to see the dowager.

The majority of the guests had not yet arrived, but the house was full of agitated people, flitting hither and thither, with flowers and prayer-books, while a clatter of dishes and cutlery from the dining-room proclaimed the laying of a sumptuous breakfast. Lord Peter was shown into the morning-room while the footman went to announce him, and here he found a very close friend and devoted colleague, Detective-Inspector Parker, mounting guard in plain clothes over a costly collection of white elephants. Lord Peter greeted him with an affectionate hand-grip.

»All serene so far?« he enquired.

»Perfectly \textsc{o.k.}«

»You got my note?«

»Sure thing. I've got three of our men shadowing your friend in Guilford Street. The girl is very much in evidence here. Does the old lady's wig and that sort of thing. Bit of a coming-on disposition, isn't she?«

»You surprise me,« said Lord Peter. »No«—as his friend grinned sardonically—»you really do. Not seriously? That would throw all my calculations out.«

»Oh, no! Saucy with her eyes and her tongue, that's all.«

»Do her job well?«

»I've heard no complaints. What put you on to this?«

»Pure accident. Of course I may be quite mistaken.«

»Did you receive any information from Paris?«

»I wish you wouldn't use that phrase,« said Lord Peter peevishly. »It's so of the Yard—yardy. One of these days it'll give you away.«

»Sorry,« said Parker. »Second nature, I suppose.«

»Those are the things to beware of,« returned his lordship, with an earnestness that seemed a little out of place. »One can keep guard on everything but just those second-nature tricks.« He moved across to the window, which overlooked the tradesmen's entrance. »Hullo!« he said, »here's our bird.«

Parker joined him, and saw the neat, shingled head of the French girl from the Gare St Lazare, topped by a neat black bandeau and bow. A man with a basket full of white narcissi had rung the bell, and appeared to be trying to make a sale. Parker gently opened the window, and they heard Célestine say with a marked French accent, »No, nossing to-day, sank you.« The man insisted in the monotonous whine of his type, thrusting a big bunch of the white flowers upon her, but she pushed them back into the basket with an angry exclamation and flirted away, tossing her head and slapping the door smartly to. The man moved off muttering. As he did so a thin, unhealthy-looking lounger in a check cap detached himself from a lamp-post opposite and mouched along the street after him, at the same time casting a glance up at the window. Mr Parker looked at Lord Peter, nodded, and made a slight sign with his hand. At once the man in the check cap removed his cigarette from his mouth, extinguished it, and, tucking the stub behind his ear, moved off without a second glance.

»Very interesting,« said Lord Peter, when both were out of sight. »Hark!«

There was a sound of running feet overhead—a cry—and a general commotion. The two men dashed to the door as the bride, rushing frantically downstairs with her bevy of bridesmaids after her, proclaimed in a hysterical shriek: »The diamonds! They're stolen! They're gone!«

Instantly the house was in an uproar. The servants and the caterers' men crowded into the hall; the bride's father burst out from his room in a magnificent white waistcoat and no coat; the Duchess of Medway descended upon Mr Parker, demanding that something should be done; while the butler, who never to the day of his death got over the disgrace, ran out of the pantry with a corkscrew in one hand and a priceless bottle of crusted port in the other, which he shook with all the vehemence of a town-crier ringing a bell. The only dignified entry was made by the dowager duchess, who came down like a ship in sail, dragging Célestine with her, and admonishing her not to be so silly.

»Be quiet, girl,« said the dowager. »Anyone would think you were going to be murdered.«

»Allow me, your grace,« said Mr Bunter, appearing suddenly from nowhere in his usual unperturbed manner, and taking the agitated Célestine firmly by the arm. »Young woman, calm yourself.«

»But what is to be \textit{done}?« cried the bride's mother. »How did it happen?«

It was at this moment that Detective-Inspector Parker took the floor. It was the most impressive and dramatic moment in his whole career. His magnificent calm rebuked the clamorous nobility surrounding him.

»Your grace,« he said, »there is no cause for alarm. Our measures have been taken. We have the criminals and the gems, thanks to Lord Peter Wimsey, from whom we received inf\longdash«

»Charles!« said Lord Peter in an awful voice.

»Warning of the attempt. One of our men is just bringing in the male criminal at the front door, taken red-handed with your grace's diamonds in his possession.« (All gazed round, and perceived indeed the check-capped lounger and a uniformed constable entering with the flower-seller between them.) »The female criminal, who picked the lock of your grace's safe, is—here! No, you don't,« he added, as Célestine, amid a torrent of apache language which nobody, fortunately, had French enough to understand, attempted to whip out a revolver from the bosom of her demure black dress. »Célestine Berger,« he continued, pocketing the weapon, »I arrest you in the name of the law, and I warn you that anything you say will be taken down and used as evidence against you.«

»Heaven help us,« said Lord Peter; »the roof would fly off the court. And you've got the name wrong, Charles. Ladies and gentlemen, allow me to introduce to you Jacques Lerouge, known as Sans-culotte—the youngest and cleverest thief, safe-breaker, and female impersonator that ever occupied a dossier in the Palais de Justice.«

There was a gasp. Jacques Sans-culotte gave vent to a low oath and cocked a \textit{gamin} grimace at Peter.

\begin{french}»C'est parfait,«\end{french} said he; »toutes mes félicitations, milord, what you call a fair cop, hein? And now I know him,« he added, grinning at Bunter, »the so-patient Englishman who stand behind us in the queue at St Lazare. But tell me, please, how you know me, that I may correct it, \textit{next time}.«

»I have mentioned to you before, Charles,« said Lord Peter, »the unwisdom of falling into habits of speech. They give you away. Now, in France, every male child is brought up to use masculine adjectives about himself. He says: »Que je suis beau!« But a little girl has it rammed home to her that she is female; she must say: »Que je suis belle!« It must make it beastly hard to be a female impersonator. When I am at a station and I hear an excited young woman say to her companion, »Me prends-tu pour \textit{un} imbécile«—the masculine article arouses curiosity. And that's that!« he concluded briskly. »The rest was merely a matter of getting Bunter to take a photograph and communicating with our friends of the Sureté and Scotland Yard.«

Jacques Sans-culotte bowed again.

»Once more I congratulate milord. He is the only Englishman I have ever met who is capable of appreciating our beautiful language. I will pay great attention in future to the article in question.«

With an awful look, the Dowager Duchess of Medway advanced upon Lord Peter.

»Peter,« she said, »do you mean to say you \textit{knew} about this, and that for the last three weeks you have allowed me to be dressed and undressed and put to bed by a \textit{young man}?«

His lordship had the grace to blush.

»Duchess,« he said humbly, »on my honour I didn't know absolutely for certain till this morning. And the police were so anxious to have these people caught red-handed. What can I do to show my penitence? Shall I cut the privileged beast in pieces?«

The grim old mouth relaxed a little.

»After all,« said the dowager duchess, with the delightful consciousness that she was going to shock her daughter-in-law, »there are very few women of my age who could make the same boast. It seems that we die as we have lived, my dear.«

For indeed the Dowager Duchess of Medway had been notable in her day.