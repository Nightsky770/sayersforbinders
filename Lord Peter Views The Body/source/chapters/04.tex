%!TeX root=../viewsbodytop.tex
\addchap{The Fantastic Horror of the Cat in the Bag}
	
	
\lettrine[lines=4]{T}{he} Great North Road wound away like a flat, steel-grey ribbon. Up it, with the sun and wind behind them, two black specks moved swiftly. To the yokel in charge of the hay-wagon they were only two of »they dratted motor-cyclists,« as they barked and zoomed past him in rapid succession. A little farther on, a family man, driving delicately with a two-seater side-car, grinned as the sharp rattle of the o.h.v. Norton was succeeded by the feline shriek of an angry Scott Flying-Squirrel. He, too, in bachelor days, had taken a side in that perennial feud. He sighed regretfully as he watched the racing machines dwindle away northwards.

At that abominable and unexpected S-bend across the bridge above Hatfield, the Norton man, in the pride of his heart, turned to wave a defiant hand at his pursuer. In that second, the enormous bulk of a loaded charabanc loomed down upon him from the bridgehead. He wrenched himself away from it in a fierce wobble, and the Scott, cornering melodramatically, with left and right foot-rests alternately skimming the tarmac, gained a few triumphant yards. The Norton leapt forward with wide-open throttle. A party of children, seized with sudden panic, rushed helter-skelter across the road. The Scott lurched through them in drunken swerves. The road was clear, and the chase settled down once more.

It is not known why motorists, who sing the joys of the open road, spend so much petrol every week-end grinding their way to Southend and Brighton and Margate, in the stench of each other's exhausts, one hand on the horn and one foot on the brake, their eyes starting from their orbits in the nerve-racking search for cops, corners, blind turnings, and cross-road suicides. They ride in a baffled fury, hating each other. They arrive with shattered nerves and fight for parking places. They return, blinded by the headlights of fresh arrivals, whom they hate even worse than they hate each other. And all the time the Great North Road winds away like a long, flat, steel-grey ribbon—a surface like a race-track, without traps, without hedges, without side-roads, and without traffic. True, it leads to nowhere in particular; but, after all, one pub is very much like another.

The tarmac reeled away, mile after mile. The sharp turn to the right at Baldock, the involute intricacies of Biggleswade, with its multiplication of sign-posts, gave temporary check, but brought the pursuer no nearer. Through Tempsford at full speed, with bellowing horn and exhaust, then, screaming like a hurricane past the \textsc{r.a.c.} post where the road forks in from Bedford. The Norton rider again glanced back; the Scott rider again sounded his horn ferociously. Flat as a chessboard, dyke and field revolved about the horizon.

The constable at Eaton Socon was by no means an anti-motor fiend. In fact, he had just alighted from his push-bike to pass the time of day with the \textsc{a.a.} man on point duty at the cross-roads. But he was just and God-fearing. The sight of two maniacs careering at seventy miles an hour into his protectorate was more than he could be expected to countenance—the more, that the local magistrate happened to be passing at that very moment in a pony-trap. He advanced to the middle of the road, spreading his arms in a majestic manner. The Norton rider looked, saw the road beyond complicated by the pony-trap and a traction-engine, and resigned himself to the inevitable. He flung the throttle-lever back, stamped on his squealing brakes, and skidded to a standstill. The Scott, having had notice, came up mincingly, with a voice like a pleased kitten.

»Now, then,« said the constable, in a tone of reproof, »ain't you got no more sense than to come drivin' into the town at a `undred mile an hour. This ain't Brooklands, you know. I never see anything like it. `Ave to take your names and numbers, if \textit{you} please. You'll bear witness, Mr Nadgett, as they was doin' over eighty.«

The \textsc{a.a.} man, after a swift glance over the two sets of handle-bars to assure himself that the black sheep were not of his flock, said, with an air of impartial accuracy, »About sixty-six and a half, I should say, if you was to ask me in court.«

»Look here, you blighter,« said the Scott man indignantly to the Norton man, »why the hell couldn't you stop when you heard me hoot? I've been chasing you with your beastly bag nearly thirty miles. Why can't you look after your own rotten luggage?«

He indicated a small, stout bag, tied with string to his own carrier.

»That?« said the Norton man, with scorn. »What do you mean? It's not mine. Never saw it in my life.«

This bare-faced denial threatened to render the Scott rider speechless.

»Of all the\longdash« he gasped. »Why, you crimson idiot, I saw it fall off, just the other side of Hatfield. I yelled and blew like fury. I suppose that overhead gear of yours makes so much noise you can't hear anything else. I take the trouble to pick the thing up, and go after you, and all you do is to race off like a lunatic and run me into a cop. Fat lot of thanks one gets for trying to be decent to fools on the road.«

»That ain't neither here nor there,« said the policeman. »Your licence, please, sir.«

»Here you are,« said the Scott man, ferociously flapping out his pocket-book. »My name's Walters, and it's the last time I'll try to do anybody a good turn, you can lay your shirt.«

»Walters,« said the constable, entering the particulars laboriously in his notebook, »and Simpkins. You'll `ave your summonses in doo course. It'll be for about a week `ence, on Monday or thereabouts, I shouldn't wonder.«

»Another forty bob gone west,« growled Mr Simpkins, toying with his throttle. »Oh, well, can't be helped, I suppose.«

»Forty bob?« snorted the constable. »What do \textit{you} think? Furious driving to the common danger, that's wot it is. You'll be lucky to get off with five quid apiece.«

»Oh, blast!« said the other, stamping furiously on the kick-starter. The engine roared into life, but Mr Walters dexterously swung his machine across the Norton's path.

»Oh, no, you don't,« he said viciously. »You jolly well take your bleeding bag, and no nonsense. I tell you, I \textit{saw} it fall off.«

»Now, no language,« began the constable, when he suddenly became aware that the \textsc{a.a.} man was staring in a very odd manner at the bag and making signs to him.

»`Ullo,« he demanded, »wot's the matter with the—bleedin' bag, did you say? `Ere, I'd like to `ave a look at that `ere bag, sir, if you don't mind.«

»It's nothing to do with me,« said Mr Walters, handing it over. »I saw it fall off and\longdash« His voice died away in his throat, and his eyes became fixed upon one corner of the bag, where something damp and horrible was seeping darkly through.

»Did you notice this `ere corner when you picked it up?« asked the constable. He prodded it gingerly and looked at his fingers.

»I don't know—no—not particularly,« stammered Walters. »I didn't notice anything. I—I expect it burst when it hit the road.«

The constable probed the split seam in silence, and then turned hurriedly round to wave away a couple of young women who had stopped to stare. The \textsc{a.a.} man peered curiously, and then started back with a sensation of sickness.

»Ow, Gawd!« he gasped. »It's curly—it's a woman's.«

»It's not me,« screamed Simpkins. »I swear to heaven it's not mine. This man's trying to put it across me.«

»Me?« gasped Walters. »Me? Why, you filthy, murdering brute, I tell you I saw it fall off your carrier. No wonder you blinded off when you saw me coming. Arrest him, constable. Take him away to prison\longdash«

»Hullo, officer!« said a voice behind them. »What's all the excitement? You haven't seen a motor-cyclist go by with a little bag on his carrier, I suppose?«

A big open car with an unnaturally long bonnet had slipped up to them, silent as an owl. The whole agitated party with one accord turned upon the driver.

»Would this be it, sir?«

The motorist pushed off his goggles, disclosing a long, narrow nose and a pair of rather cynical-looking grey eyes.

»It looks rather\longdash« he began; and then, catching sight of the horrid relic protruding from one corner, »In God's name,« he enquired, »what's that?«

»That's what we'd like to know, sir,« said the constable grimly.

»H'm,« said the motorist, »I seem to have chosen an uncommonly suitable moment for enquirin' after my bag. Tactless. To say now that it is not my bag is simple, though in no way convincing. As a matter of fact, it is not mine, and I may say that, if it had been, I should not have been at any pains to pursue it.«

The constable scratched his head.

»Both these gentlemen\longdash« he began.

The two cyclists burst into simultaneous and heated disclaimers. By this time a small crowd had collected, which the \textsc{a.a.} scout helpfully tried to shoo away.

»You'll all `ave to come with me to the station,« said the harassed constable. »Can't stand `ere `oldin' up the traffic. No tricks, now. You wheel them bikes, and I'll come in the car with you, sir.«

»But supposing I was to let her rip and kidnap you,« said the motorist, with a grin. »Where'd you be? Here,« he added, turning to the \textsc{a.a.} man, »can you handle this outfit?«

»You bet,« said the scout, his eye running lovingly over the long sweep of the exhaust and the rakish lines of the car.

»Right. Hop in. Now, officer, you can toddle along with the other suspects and keep an eye on them. Wonderful head I've got for detail. By the way, that foot-brake's on the fierce side. Don't bully it, or you'll surprise yourself.«

The lock of the bag was forced at the police-station in the midst of an excitement unparalleled in the calm annals of Eaton Socon, and the dreadful contents laid reverently upon a table. Beyond a quantity of cheese-cloth in which they had been wrapped, there was nothing to supply any clue to the mystery.

»Now,« said the superintendent, »what do you gentlemen know about this?«

»Nothing whatever,« said Mr Simpkins, with a ghastly countenance, »except that this man tried to palm it off on me.«

»I saw it fall off this man's carrier just the other side of Hatfield,« repeated Mr Walters firmly, »and I rode after him for thirty miles trying to stop him. That's all I know about it, and I wish to God I'd never touched the beastly thing.«

»Nor do I know anything about it personally,« said the car-owner, »but I fancy I know what it is.«

»What's that?« asked the superintendent sharply.

»I rather imagine it's the head of the Finsbury Park murder—though, mind you, that's only a guess.«

»That's just what I've been thinking myself,« agreed the superintendent, glancing at a daily paper which lay on his desk, its headlines lurid with the details of that very horrid crime, »and, if so, you are to be congratulated, constable, on a very important capture.«

»Thank you, sir,« said the gratified officer, saluting.

»Now I'd better take all your statements,« said the superintendent. »No, no; I'll hear the constable first. Yes, Briggs?«

The constable, the \textsc{a.a.} man, and the two motor-cyclists having given their versions of the story, the superintendent turned to the motorist.

»And what have you got to say about it?« he enquired. »First of all, your name and address.«

The other produced a card, which the superintendent copied out and returned to him respectfully.

»A bag of mine, containing some valuable jewellery, was stolen from my car yesterday, in Piccadilly,« began the motorist. »It is very much like this, but has a cipher lock. I made enquiries through Scotland Yard, and was informed to-day that a bag of precisely similar appearance had been cloak-roomed yesterday afternoon at Paddington, main line. I hurried round there, and was told by the clerk that just before the police warning came through the bag had been claimed by a man in motor-cycling kit. A porter said he saw the man leave the station, and a loiterer observed him riding off on a motor-bicycle. That was about an hour before. It seemed pretty hopeless, as, of course, nobody had noticed even the make of the bike, let alone the number. Fortunately, however, there was a smart little girl. The smart little girl had been dawdling round outside the station, and had heard a motor-cyclist ask a taxi-driver the quickest route to Finchley. I left the police hunting for the taxi-driver, and started off, and in Finchley I found an intelligent boy-scout. He had seen a motor-cyclist with a bag on the carrier, and had waved and shouted to him that the strap was loose. The cyclist had got off and tightened the strap, and gone straight on up the road towards Chipping Barnet. The boy hadn't been near enough to identify the machine—the only thing he knew for certain was that it wasn't a Douglas, his brother having one of that sort. At Barnet I got an odd little story of a man in a motor-coat who had staggered into a pub with a ghastly white face and drunk two double brandies and gone out and ridden off furiously. Number?—of course not. The barmaid told me. \textit{She} didn't notice the number. After that it was a tale of furious driving all along the road. After Hatfield, I got the story of a road-race. And here we are.«

»It seems to me, my lord,« said the superintendent, »that the furious driving can't have been all on one side.«

»I admit it,« said the other, »though I do plead in extenuation that I spared the women and children and hit up the miles in the wide, open spaces. The point at the moment is\longdash«

»Well, my lord,« said the superintendent, »I've got your story, and, if it's all right, it can be verified by enquiry at Paddington and Finchley and so on. Now, as for these two gentlemen\longdash«

»It's perfectly obvious,« broke in Mr Walters, »the bag dropped off this man's carrier, and, when he saw me coming after him with it, he thought it was a good opportunity to saddle me with the cursed thing. Nothing could be clearer.«

»It's a lie,« said Mr Simpkins. »Here's this fellow has got hold of the bag—I don't say how, but I can guess—and he has the bright idea of shoving the blame on me. It's easy enough to \textit{say} a thing's fallen off a man's carrier. Where's the proof? Where's the strap? If his story's true, you'd find the broken strap on my `bus. The bag \textit{was} on \textit{his} machine—tied on, tight.«

»Yes, with string,« retorted the other. »If I'd gone and murdered someone and run off with their head, do you think I'd be such an ass as to tie it on with a bit of twopenny twine? The strap's worked loose and fallen off on the road somewhere; that's what's happened to that.«

»Well, look here,« said the man addressed as »my lord,« »I've got an idea for what it's worth. Suppose, superintendent, you turn out as many of your men as you think adequate to keep an eye on three desperate criminals, and we all tool down to Hatfield together. I can take two in my `bus at a pinch, and no doubt you have a police car. If this thing \textit{did} fall off the carrier, somebody beside Mr Walters may have seen it fall.«

»They didn't,« said Mr Simpkins.

»There wasn't a soul,« said Mr Walters, »but how do \textit{you} know there wasn't, eh? I thought you didn't know anything about it.«

»I mean, it didn't fall off, so nobody \textit{could} have seen it,« gasped the other.

»Well, my lord,« said the superintendent, »I'm inclined to accept your suggestion, as it gives us a chance of enquiring into your story at the same time. Mind you, I'm not saying I doubt it, you being who you are. I've read about some of your detective work, my lord, and very smart I considered it. But, still, it wouldn't be my duty not to get corroborative evidence if possible.«

»Good egg! Quite right,« said his lordship. »Forward the light brigade. We can do it easily in—that is to say, at the legal rate of progress it needn't take us much over an hour and a half.«

\noindent\hfil\rule{0.5\textwidth}{.4pt}\hfil 

About three-quarters of an hour later, the racing car and the police car loped quietly side by side into Hatfield. Henceforward, the four-seater, in which Walters and Simpkins sat glaring at each other, took the lead, and presently Walters waved his hand and both cars came to a stop.

»It was just about here, as near as I can remember, that it fell off,« he said. »Of course, there's no trace of it now.«

»You're quite sure as there wasn't a strap fell off with it?« suggested the superintendent, »because, you see, there must `a' been something holding it on.«

»Of course there wasn't a strap,« said Simpkins, white with passion. »You haven't any business to ask him leading questions like that.«

»Wait a minute,« said Walters slowly. »No, there was no strap. But I've got a sort of a recollection of seeing something on the road about a quarter of a mile farther up.«

»It's a lie!« screamed Simpkins. »He's inventing it.«

»Just about where we passed that man with the side-car a minute or two ago,« said his lordship. »I told you we ought to have stopped and asked if we could help him, superintendent. Courtesy of the road, you know, and all that.«

»He couldn't have told us anything,« said the superintendent. »He'd probably only just stopped.«

»I'm not so sure,« said the other. »Didn't you notice what he was doing? Oh, dear, dear, where were your eyes? Hullo! here he comes.«

He sprang out into the road and waved to the rider, who, seeing four policemen, thought it better to pull up.

»Excuse me,« said his lordship. »Thought we'd just like to stop you and ask if you were all right, and all that sort of thing, you know. Wanted to stop in passing, throttle jammed open, couldn't shut the confounded thing. Little trouble, what?«

»Oh, yes, perfectly all right, thanks, except that I would be glad if you could spare a gallon of petrol. Tank came adrift. Beastly nuisance. Had a bit of a struggle. Happily, Providence placed a broken strap in my way and I've fixed it. Split a bit, though, where that bolt came off. Lucky not to have an explosion, but there's a special cherub for motor-cyclists.«

»Strap, eh?« said the superintendent. »Afraid I'll have to trouble you to let me have a look at that.«

»What?« said the other. »And just as I've got the damned thing fixed? What the—? All right, dear, all right«—to his passenger. »Is it something serious, officer?«

»Afraid so, sir. Sorry to trouble you.«

»Hi!« yelled one of the policemen, neatly fielding Mr Simpkins as he was taking a dive over the back of the car. »No use doin' that. You're for it, my lad.«

»No doubt about it,« said the superintendent triumphantly, snatching at the strap which the side-car rider held out to him. »Here's his name on it, »J. Simpkins,« written on in ink as large as life. Very much obliged to \textit{you}, sir, I'm sure. You've helped us effect a very important capture.«

»No! \textit{Who} is it?« cried the girl in the side-car. »How frightfully thrilling! Is it a murder?«

»Look in your paper to-morrow, miss,« said the superintendent, »and you may see something. Here, Briggs, better put the handcuffs on him.«

»And how about my tank?« said the man mournfully. »It's all right for you to be excited, Babs, but you'll have to get out and help push.«

»Oh, no,« said his lordship. »Here's a strap. A \textit{much} nicer strap. A really superior strap. And petrol. \textit{And} a pocket-flask. Everything a young man ought to know. And, when you're in town, mind you both look me up. Lord Peter Wimsey, 110A Piccadilly. Delighted to see you any time. Chin, chin!«

»Cheerio!« said the other, wiping his lips and much mollified. »Only too charmed to be of use. Remember it in my favour, officer, next time you catch me speeding.«

»Very fortunate we spotted him,« said the superintendent complacently, as they continued their way into Hatfield. »Quite providential, as you might say.«

\noindent\hfil\rule{0.5\textwidth}{.4pt}\hfil 

»I'll come across with it,« said the wretched Simpkins, sitting hand-cuffed in the Hatfield police-station. »I swear to God I know nothing whatever about it—about the murder, I mean. There's a man I know who has a jewellery business in Birmingham. I don't know him very well. In fact, I only met him at Southend last Easter, and we got pally. His name's Owen—Thomas Owen. He wrote me yesterday and said he'd accidentally left a bag in the cloakroom at Paddington and asked if I'd take it out—he enclosed the ticket—and bring it up next time I came that way. I'm in transport service, you see—you've got my card—and I'm always up and down the country. As it happened, I was just going up in that direction with this Norton, so I fetched the thing out at lunch-time and started off with it. I didn't notice the date on the cloakroom ticket. I know there wasn't anything to pay on it, so it can't have been there long. Well, it all went just as you said up to Finchley, and there that boy told me my strap was loose and I went to tighten it up. And then I noticed that the corner of the bag was split, and it was damp—and—well, I saw what you saw. That sort of turned me over, and I lost my head. The only thing I could think of was to get rid of it, quick. I remembered there were a lot of lonely stretches on the Great North Road, so I cut the strap nearly through—that was when I stopped for that drink at Barnet—and then, when I thought there wasn't anybody in sight, I just reached back and gave it a tug, and it went—strap and all; I hadn't put it through the slots. It fell off, just like a great weight dropping off my mind. I suppose Walters must just have come round into sight as it fell. I had to slow down a mile or two farther on for some sheep going into a field, and then I heard him hooting at me—and—oh, my God!«

He groaned, and buried his head in his hands.

»I see,« said the Eaton Socon superintendent. »Well, that's your statement. Now, about this Thomas Owen\longdash«

»Oh,« cried Lord Peter Wimsey, »never mind Thomas Owen. He's not the man you want. You can't suppose that a bloke who'd committed a murder would want a fellow tailin' after him to Birmingham with the head. It stands to reason that was intended to stay in Paddington cloakroom till the ingenious perpetrator had skipped, or till it was unrecognisable, or both. Which, by the way, is where we'll find those family heirlooms of mine, which your engaging friend Mr Owen lifted out of my car. Now, Mr Simpkins, just pull yourself together and tell us who was standing next to you at the cloakroom when you took out that bag. Try hard to remember, because this jolly little island is no place for him, and he'll be taking the next boat while we stand talking.«

»I can't remember,« moaned Simpkins. »I didn't notice. My head's all in a whirl.«

»Never mind. Go back. Think quietly. Make a picture of yourself getting off your machine—leaning it up against something\longdash«

»No, I put it on the stand.«

»Good! That's the way. Now, think—you're taking the cloakroom ticket out of your pocket and going up—trying to attract the man's attention.«

»I couldn't at first. There was an old lady trying to cloakroom a canary, and a very bustling man in a hurry with some golf-clubs. He was quite rude to a quiet little man with a—by Jove! yes, a hand-bag like that one. Yes, that's it. The timid man had had it on the counter quite a long time, and the big man pushed him aside. I don't know what happened, quite, because mine was handed out to me just then. The big man pushed his luggage in front of both of us and I had to reach over it—and I suppose—yes, I must have taken the wrong one. Good God! Do you mean to say that that timid little insignificant-looking man was a murderer?«

»Lots of `em like that,« put in the Hatfield superintendent. »But what was he like—come!«

»He was only about five foot five, and he wore a soft hat and a long, dust-coloured coat. He was very ordinary, with rather weak, prominent eyes, I think, but I'm not sure I should know him again. Oh, wait a minute! I do remember one thing. He had an odd scar—crescent-shaped—under his left eye.«

»That settles it,« said Lord Peter. »I thought as much. Did you recognise the—the face when we took it out, superintendent? No? I did. It was Dahlia Dallmeyer, the actress, who is supposed to have sailed for America last week. And the short man with the crescent-shaped scar is her husband, Philip Storey. Sordid tale and all that. She ruined him, treated him like dirt, and was unfaithful to him, but it looks as though he had had the last word in the argument. And now, I imagine, the Law will have the last word with him. Get busy on the wires, superintendent, and you might ring up the Paddington people and tell `em to let me have my bag, before Mr Thomas Owen tumbles to it that there's been a slight mistake.«

»Well, anyhow,« said Mr Walters, extending a magnanimous hand to the abashed Mr Simpkins, »it was a top-hole race—well worth a summons. We must have a return match one of these days.«

\noindent\hfil\rule{0.5\textwidth}{.4pt}\hfil 

Early the following morning a little, insignificant-looking man stepped aboard the trans-Atlantic liner \textit{Volucria}. At the head of the gangway two men blundered into him. The younger of the two, who carried a small bag, was turning to apologise, when a light of recognition flashed across his face.

»Why, if it isn't Mr Storey!« he exclaimed loudly. »Where are you off to? I haven't seen you for an age.«

»I'm afraid,« said Philip Storey, »I haven't the pleasure\longdash«

»Cut it out,« said the other, laughing. »I'd know that scar of yours anywhere. Going out to the States?«

»Well, yes,« said the other, seeing that his acquaintance's boisterous manner was attracting attention. »I beg your pardon. It's Lord Peter Wimsey, isn't it? Yes. I'm joining the wife out there.«

»And how is she?« enquired Wimsey, steering the way into the bar and sitting down at a table. »Left last week didn't she? I saw it in the papers.«

»Yes. She's just cabled me to join her. We're—er—taking a holiday in—er—the lakes. Very pleasant there in summer.«

»Cabled you, did she? And so here we are on the same boat. Odd how things turn out, what? I only got my sailing orders at the last minute. Chasing criminals—my hobby, you know.«

»Oh, really?« Mr Storey licked his lips.

»Yes. This is Detective-Inspector Parker of Scotland Yard—great pal of mine. Yes. Very unpleasant matter, annoying and all that. Bag that ought to have been reposin' peacefully at Paddington Station turns up at Eaton Socon. No business there, what?«

He smacked the bag on the table so violently that the lock sprang open.

Storey leapt to his feet with a shriek, flinging his arms across the opening of the bag as though to hide its contents.

»How did you get that?« he screamed. »Eaton Socon? It—I never\longdash«

»It's mine,« said Wimsey quietly, as the wretched man sank back, realising that he had betrayed himself. »Some jewellery of my mother's. What did you think it was?«

Detective Parker touched his charge gently on the shoulder.

»You needn't answer that,« he said. »I arrest you, Philip Storey, for the murder of your wife. Anything that you say may be used against you.«