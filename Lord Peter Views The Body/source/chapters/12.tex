%!TeX root=../viewsbodytop.tex
\addchap{The Adventurous Exploit of the Cave of Ali Baba}

\lettrine[lines=4]{I}{n} the front room of a grim and narrow house in Lambeth a man sat eating kippers and glancing through the \textit{Morning Post}. He was smallish and spare, with brown hair rather too regularly waved and a strong, brown beard, cut to a point. His double-breasted suit of navy-blue and his socks, tie, and handkerchief, all scrupulously matched, were a trifle more point-device than the best taste approves, and his boots were slightly too bright a brown. He did not look a gentleman, not even a gentleman's gentleman, yet there was something about his appearance which suggested that he was accustomed to the manner of life in good families. The breakfast-table, which he had set with his own hands, was arrayed with the attention to detail which is exacted of good-class servants. His action, as he walked over to a little side-table and carved himself a plate of ham, was the action of a superior butler; yet he was not old enough to be a retired butler; a footman, perhaps, who had come into a legacy.

He finished the ham with good appetite, and, as he sipped his coffee, read through attentively a paragraph which he had already noticed and put aside for consideration.

\begin{newspaper}{Lord~Peter Wimsey's Will}[Bequest To Valet\\\textsterling 10,000 To Charities]

The will of Lord~Peter Wimsey, who was killed last December while shooting big game in Tanganyika, was proved yesterday at \textsterling 500,000. A sum of \textsterling 10,000 was left to various charities, including [here followed a list of bequests]. To his valet, Mervyn Bunter, was left an annuity of \textsterling 500 and the lease of the testator's flat in Piccadilly. [Then followed a number of personal bequests.] The remainder of the estate, including the valuable collection of books and pictures at 110A Piccadilly, was left to the testator's mother, the Dowager Duchess of Denver.

Lord~Peter Wimsey was thirty-seven at the time of his death. He was the younger brother of the present Duke of Denver, who is the wealthiest peer in the United Kingdom. Lord~Peter was distinguished as a criminologist and took an active part in the solution of several famous mysteries. He was a well-known book-collector and man-about-town.
\end{newspaper}

The man gave a sigh of relief.

<No doubt about that,> he said aloud. <People don't give their money away if they're going to come back again. The blighter's dead and buried right enough. I'm free.>

He finished his coffee, cleared the table, and washed up the crockery, took his bowler hat from the hall-stand, and went out.

A bus took him to Bermondsey. He alighted, and plunged into a network of gloomy streets, arriving after a quarter of an hour's walk at a seedy-looking public-house in a low quarter. He entered and called for a double whisky.

The house had only just opened, but a number of customers, who had apparently been waiting on the doorstep for this desirable event, were already clustered about the bar. The man who might have been a footman reached for his glass, and in doing so jostled the elbow of a flash person in a check suit and regrettable tie.

<Here!> expostulated the flash person, <what d'yer mean by it? We don't want your sort here. Get out!>

He emphasised his remarks with a few highly coloured words, and a violent push in the chest.

<Bar's free to everybody, isn't it?> said the other, returning the shove with interest.

<Now then!> said the barmaid, <none o' that. The gentleman didn't do it intentional, Mr~Jukes.>

<Didn't he?> said Mr~Jukes. <Well, I \textit{did}.>

<And you ought to be ashamed of yourself,> retorted the young lady, with a toss of the head. <I'll have no quarrelling in my bar—not this time in the morning.>

<It was quite an accident,> said the man from Lambeth. <I'm not one to make a disturbance, having always been used to the best houses. But if any gentleman \textit{wants} to make trouble\longdash>

<All right, all right,> said Mr~Jukes, more pacifically. <I'm not keen to give you a new face. Not but what any alteration wouldn't be for the better. Mind your manners another time, that's all. What'll you have?>

<No, no,> protested the other, <this one must be on me. Sorry I pushed you. I didn't mean it. But I didn't like to be taken up so short.>

<Say no more about it,> said Mr~Jukes generously. <I'm standing this. Another double whisky, miss, and one of the usual. Come over here where there isn't so much of a crowd, or you'll be getting yourself into trouble again.>

He led the way to a small table in the corner of the room.

<That's all right,> said Mr~Jukes. <Very nicely done. I don't think there's any danger here, but you can't be too careful. Now, what about it, Rogers? Have you made up your mind to come in with us?>

<Yes,> said Rogers, with a glance over his shoulder, <yes, I have. That is, mind you, if everything seems all right. I'm not looking for trouble, and I don't want to get let in for any dangerous games. I don't mind giving you information, but it's understood as I take no active part in whatever goes on. Is that straight?>

<You wouldn't be allowed to take an active part if you wanted to,> said Mr~Jukes. <Why, you poor fish, Number One wouldn't have anybody but experts on his jobs. All you have to do is to let us know where the stuff is and how to get it. The Society does the rest. It's some organisation, I can tell you. You won't even know who's doing it, or how it's done. You won't know anybody, and nobody will know you—except Number One, of course. He knows everybody.>

<And you,> said Rogers.

<And me, of course. But I shall be transferred to another district. We shan't meet again after to-day, except at the general meetings, and then we shall all be masked.>

<Go on!> said Rogers incredulously.

<Fact. You'll be taken to Number One—he'll see you, but you won't see him. Then, if he thinks you're any good, you'll be put on the roll, and after that you'll be told where to make your reports to. There is a divisional meeting called once a fortnight, and every three months there's a general meeting and share-out. Each member is called up by number and has his whack handed over to him. That's all.>

<Well, but suppose two members are put on the same job together?>

<If it's a daylight job, they'll be so disguised their mothers wouldn't know 'em. But it's mostly night work.>

<I see. But, look here—what's to prevent somebody following me home and giving me away to the police?>

<Nothing, of course. Only I wouldn't advise him to try it, that's all. The last man who had that bright idea was fished out of the river down Rotherhithe way, before he had time to get his precious report in. Number One knows everybody, you see.>

<Oh!—and who is this Number One?>

<There's lots of people would give a good bit to know that.>

<Does nobody know?>

<Nobody. He's a fair marvel, is Number One. He's a gentleman, I can tell you that, and a pretty high-up one, from his ways. \textit{And} he's got eyes all round his head. \textit{And} he's got an arm as long as from here to Australia. \textit{But} nobody knows anything about him, unless it's Number Two, and I'm not even sure about her.>

<There are women in it, then?>

<You can bet your boots there are. You can't do a job without 'em nowadays. But that needn't worry you. The women are safe enough. They don't want to come to a sticky end, no more than you and me.

But, look here, Jukes—how about the money? It's a big risk to take. Is it worth it?>

<Worth it?> Jukes leant across the little marble-topped table and whispered.

<Coo!> gasped Rogers. <And how much of that would I get, now?>

<You'd share and share alike with the rest, whether you'd been in that particular job or not. There's fifty members, and you'd get one-fiftieth, same as Number One and same as me.>

<Really? No kidding?>

<See that wet, see that dry!> Jukes laughed. <Say, can you beat it? There's never been anything like it. It's the biggest thing ever been known. He's a great man, is Number One.>

<And do you pull off many jobs?>

<Many? Listen. You remember the Carruthers necklace, and the Gorleston Bank robbery? And the Faversham burglary? And the big Rubens that disappeared from the National Gallery? And the Frensham pearls? All done by the Society. And never one of them cleared up.>

Rogers licked his lips.

<But now, look here,> he said cautiously. <Supposing I was a spy, as you might say, and supposing I was to go straight off and tell the police about what you've been saying?>

<Ah!> said Jukes, <suppose you did, eh? Well, supposing something nasty didn't happen to you on the way there—which I wouldn't answer for, mind\longdash>

<Do you mean to say you've got me watched?>

<You can bet your sweet life we have. Yes. Well, \textit{supposing} nothing happened on the way there, and you was to bring the slops to this pub, looking for yours truly\longdash>

<Yes?>

<You wouldn't find me, that's all. I should have gone to Number Five.>

<Who's Number Five?>

<Ah! I don't know. But he's the man that makes you a new face while you wait. Plastic surgery, they call it. And new finger-prints. New everything. We go in for up-to-date methods in our show.>

Rogers whistled.

<Well, how about it?> asked Jukes, eyeing his acquaintance over the rim of his tumbler.

<Look here—you've told me a lot of things. Shall I be safe if I say 'no'?>

<Oh, yes—if you behave yourself and don't make trouble for us.>

<H'm, I see. And if I say 'yes'?>

<Then you'll be a rich man in less than no time, with money in your pocket to live like a gentleman. And nothing to do for it, except to tell us what you know about the houses you've been to when you were in service. It's money for jam if you act straight by the Society.>

Rogers was silent, thinking it over.

<I'll do it!> he said at last.

<Good for you. Miss! The same again, please. Here's to it, Rogers! I knew you were one of the right sort the minute I set eyes on you. Here's to money for jam, and take care of Number One! Talking of Number One, you'd better come round and see him to-night. No time like the present.>

<Right you are. Where'll I come to? Here?>

<Nix. No more of this little pub for us. It's a pity, because it's nice and comfortable, but it can't be helped. Now, what you've got to do is this. At ten o'clock to-night exactly, you walk north across Lambeth Bridge.> (Rogers winced at this intimation that his abode was known), <and you'll see a yellow taxi standing there, with the driver doing something to his engine. You'll say to him, <Is your bus fit to go?> and he'll say, <Depends where you want to go to.> And you'll say, <Take me to Number One, London.> There's a shop called that, by the way, but he won't take you there. You won't know where he \textit{is} taking you, because the taxi-windows will be covered up, but you mustn't mind that. It's the rule for the first visit. Afterwards, when you're regularly one of us, you'll be told the name of the place. And when you get there, do as you're told and speak the truth, because, if you don't, Number One will deal with you. See?>

<I see.>

<Are you game? You're not afraid?>

<Of course I'm not afraid.>

<Good man! Well, we'd better be moving now. And I'll say good-bye, because we shan't see each other again. Good-bye—and good luck!>

<Good-bye.>

They passed through the swing-doors, and out into the mean and dirty street.

The two years subsequent to the enrolment of the ex-footman Rogers in a crook society were marked by a number of startling and successful raids on the houses of distinguished people. There was the theft of the great diamond tiara from the Dowager Duchess of Denver; the burglary at the flat formerly occupied by the late Lord~Peter Wimsey, resulting in the disappearance of \textsterling 7,000 worth of silver and gold plate; the burglary at the country mansion of Theodore Winthrop, the millionaire—which, incidentally, exposed that thriving gentleman as a confirmed Society blackmailer and caused a reverberating scandal in Mayfair; and the snatching of the famous eight-string necklace of pearls from the Marchioness of Dinglewood during the singing of the Jewel Song in \textit{Faust} at Covent Garden. It is true that the pearls turned out to be imitation, the original string having been pawned by the noble lady under circumstances highly painful to the Marquis, but the coup was nevertheless a sensational one.

On a Saturday afternoon in January, Rogers was sitting in his room in Lambeth, when a slight noise at the front door caught his ear. He sprang up almost before it had ceased, dashed through the small hall-way, and flung the door open. The street was deserted. Nevertheless, as he turned back to the sitting-room, he saw an envelope lying on the hat-stand. It was addressed briefly to <Number Twenty-one.> Accustomed by this time to the somewhat dramatic methods used by the Society to deliver its correspondence, he merely shrugged his shoulders, and opened the note.

It was written in cipher, and, when transcribed, ran thus:

<Number Twenty-one,—An Extraordinary General Meeting will be held to-night at the house of Number One at 11.30. You will be absent at your peril. The word is Finality.>

Rogers stood for a little time considering this. Then he made his way to a room at the back of the house, in which there was a tall safe, built into the wall. He manipulated the combination and walked into the safe, which ran back for some distance, forming, indeed, a small strong-room. He pulled out a drawer marked <Correspondence,> and added the paper he had just received to the contents.

After a few moments he emerged, re-set the lock to a new combination, and returned to the sitting-room.

<Finality,> he said. <Yes—I think so.> He stretched out his hand to the telephone—then appeared to alter his mind.

He went upstairs to an attic, and thence climbed into a loft close under the roof. Crawling among the rafters, he made his way into the farthest corner; then carefully pressed a knot on the timber-work. A concealed trap-door swung open. He crept through it, and found himself in the corresponding loft of the next house. A soft cooing noise greeted him as he entered. Under the skylight stood three cages, each containing a carrier pigeon.

He glanced cautiously out of the skylight, which looked out upon a high blank wall at the back of some factory or other. There was nobody in the dim little courtyard, and no window within sight. He drew his head in again, and, taking a small fragment of thin paper from his pocket-book, wrote a few letters and numbers upon it. Going to the nearest cage, he took out the pigeon and attached the message to its wing. Then he carefully set the bird on the window-ledge. It hesitated a moment, shifted its pink feet a few times, lifted its wings, and was gone. He saw it tower up into the already darkening sky over the factory roof and vanish into the distance.

He glanced at his watch and returned downstairs. An hour later he released the second pigeon, and in another hour the third. Then he sat down to wait.

At half-past nine he went up to the attic again. It was dark, but a few frosty stars were shining, and a cold air blew through the open window. Something pale gleamed faintly on the floor. He picked it up—it was warm and feathery. The answer had come.

He ruffled the soft plumes and found the paper. Before reading it, he fed the pigeon and put it into one of the cages. As he was about to fasten the door, he checked himself.

<If anything happens to me,> he said, <there's no need for you to starve to death, my child.>

He pushed the window a little wider open and went downstairs again. The paper in his hand bore only the two letters, <\textsc{o.k.}> It seemed to have been written hurriedly, for there was a long smear of ink in the upper left-hand corner. He noted this with a smile, put the paper in the fire, and, going out into the kitchen, prepared and ate a hearty meal of eggs and corned beef from a new tin. He ate it without bread, though there was a loaf on the shelf near at hand, and washed it down with water from the tap, which he let run for some time before venturing to drink it. Even then he carefully wiped the tap, both inside and outside, before drinking.

When he had finished, he took a revolver from a locked drawer, inspecting the mechanism with attention to see that it was in working order, and loaded it with new cartridges from an unbroken packet. Then he sat down to wait again.

At a quarter before eleven, he rose and went out into the street. He walked briskly, keeping well away from the wall, till he came out into a well-lighted thoroughfare. Here he took a bus, securing the corner seat next the conductor, from which he could see everybody who got on and off. A succession of buses eventually brought him to a respectable residential quarter of Hampstead. Here he alighted and, still keeping well away from the walls, made his way up to the Heath.

The night was moonless, but not altogether black, and, as he crossed a deserted part of the Heath, he observed one or two other dark forms closing in upon him from various directions. He paused in the shelter of a large tree, and adjusted to his face a black velvet mask, which covered him from brow to chin. At its base the number 21 was clearly embroidered in white thread.

At length a slight dip in the ground disclosed one of those agreeable villas which stand, somewhat isolated, among the rural surroundings of the Heath. One of the windows was lighted. As he made his way to the door, other dark figures, masked like himself, pressed forward and surrounded him. He counted six of them.

The foremost man knocked on the door of the solitary house. After a moment, it was opened slightly. The man advanced his head to the opening; there was a murmur, and the door opened wide. The man stepped in, and the door was shut.

When three of the men had entered, Rogers found himself to be the next in turn. He knocked, three times loudly, then twice faintly. The door opened to the extent of two or three inches, and an ear was presented to the chink. Rogers whispered <Finality.> The ear was withdrawn, the door opened, and he passed in.

Without any further word of greeting, Number Twenty-one passed into a small room on the left, which was furnished like an office, with a desk, a safe, and a couple of chairs. At the desk sat a massive man in evening dress, with a ledger before him. The new arrival shut the door carefully after him; it clicked to, on a spring-lock. Advancing to the desk, he announced, <Number Twenty-one, sir,> and stood respectfully waiting. The big man looked up, showing the number 1 startlingly white on his velvet mask. His eyes, of a curious hard blue, scanned Rogers attentively. At a sign from him, Rogers removed his mask. Having verified his identity with care, the President said, <Very well, Number Twenty-one,> and made an entry in the ledger. The voice was hard and metallic, like his eyes. The close scrutiny from behind the immovable black mask seemed to make Rogers uneasy; he shifted his feet, and his eyes fell. Number One made a sign of dismissal, and Rogers, with a faint sigh as though of relief, replaced his mask and left the room. As he came out, the next comer passed in in his place.

The room in which the Society met was a large one, made by knocking the two largest of the first-floor rooms into one. It was furnished in the standardised taste of twentieth-century suburbia and brilliantly lighted. A gramophone in one corner blared out a jazz tune, to which about ten couples of masked men and women were dancing, some in evening dress and others in tweeds and jumpers.

In one corner of the room was an American bar. Rogers went up and asked the masked man in charge for a double whisky. He consumed it slowly, leaning on the bar. The room filled. Presently somebody moved across to the gramophone and stopped it. He looked round. Number One had appeared on the threshold. A tall woman in black stood beside him. The mask, embroidered with a white 2, covered hair and face completely; only her fine bearing and her white arms and bosom and the dark eyes shining through the eye-slits proclaimed her a woman of power and physical attraction.

<Ladies and gentlemen.> Number One was standing at the upper end of the room. The woman sat beside him; her eyes were cast down and betrayed nothing, but her hands were clenched on the arms of the chair and her whole figure seemed tensely aware.

<Ladies and gentlemen. Our numbers are two short to-night.> The masks moved; eyes were turned, seeking and counting. <I need not inform you of the disastrous failure of our plan for securing the plans of the Court-Windlesham helicopter. Our courageous and devoted comrades, Number Fifteen and Number Forty-eight, were betrayed and taken by the police.>

An uneasy murmur rose among the company.

<It may have occurred to some of you that even the well-known steadfastness of these comrades might give way under examination. There is no cause for alarm. The usual orders have been issued, and I have this evening received the report that their tongues have been effectually silenced. You will, I am sure, be glad to know that these two brave men have been spared the ordeal of so great a temptation to dishonour, and that they will not be called upon to face a public trial and the rigours of a long imprisonment.>

A hiss of intaken breath moved across the assembled members like the wind over a barley-field.

<Their dependants will be discreetly compensated in the usual manner. I call upon Numbers Twelve and Thirty-four to undertake this agreeable task. They will attend me in my office for their instructions after the meeting. Will the Numbers I have named kindly signify that they are able and willing to perform this duty?>

Two hands were raised in salute. The President continued, looking at his watch:

<Ladies and gentlemen, please take your partners for the next dance.>

The gramophone struck up again. Rogers turned to a girl near him in a red dress. She nodded, and they slipped into the movement of a fox-trot. The couples gyrated solemnly and in silence. Their shadows were flung against the blinds as they turned and stepped to and fro.

<What has happened?> breathed the girl in a whisper, scarcely moving her lips. <I'm frightened, aren't you? I feel as if something awful was going to happen.>

<It does take one a bit short, the President's way of doing things,> agreed Rogers, <but it's safer like that.>

<Those poor men\longdash>

A dancer, turning and following on their heels, touched Rogers on the shoulder.

<No talking, please,> he said. His eyes gleamed sternly; he twirled his partner into the middle of the crowd and was gone. The girl shuddered.

The gramophone stopped. There was a burst of clapping. The dancers again clustered before the President's seat.

<Ladies and gentlemen. You may wonder why this extraordinary meeting has been called. The reason is a serious one. The failure of our recent attempt was no accident. The police were not on the premises that night by chance. We have a traitor among us.>

Partners who had been standing close together fell distrustfully apart. Each member seemed to shrink, as a snail shrinks from the touch of a finger.

<You will remember the disappointing outcome of the Dinglewood affair,> went on the President, in his harsh voice. <You may recall other smaller matters which have not turned out satisfactorily. All these troubles have been traced to their origin. I am happy to say that our minds can now be easy. The offender has been discovered and will be removed. There will be no more mistakes. The misguided member who introduced the traitor to our Society will be placed in a position where his lack of caution will have no further ill-effects. There is no cause for alarm.>

Every eye roved about the company, searching for the traitor and his unfortunate sponsor. Somewhere beneath the black masks a face must have turned white; somewhere under the stifling velvet there must have been a brow sweating, not with the heat of the dance. But the masks hid everything.

<Ladies and gentlemen, please take your partners for the next dance.>

The gramophone struck into an old and half-forgotten tune: <There ain't nobody loves me.> The girl in red was claimed by a tall mask in evening dress. A hand laid on Roger's arm made him start. A small, plump woman in a green jumper slipped a cold hand into his. The dance went on.

When it stopped, amid the usual applause, everyone stood, detached, stiffened in expectation. The President's voice was raised again.

<Ladies and gentlemen, please behave naturally. This is a dance, not a public meeting.>

Rogers led his partner to a chair and fetched her an ice. As he stooped over her, he noticed the hurried rise and fall of her bosom.

<Ladies and gentlemen.> The endless interval was over. <You will no doubt wish to be immediately relieved from suspense. I will name the persons involved. Number Thirty-seven!>

A man sprang up with a fearful, strangled cry.

<Silence!>

The wretch choked and gasped.

<I never—I swear I never—I'm innocent.>

<Silence. You have failed in discretion. You will be dealt with. If you have anything to say in defence of your folly, I will hear it later. Sit down.>

Number Thirty-seven sank down upon a chair. He pushed his handkerchief under the mask to wipe his face. Two tall men closed in upon him. The rest fell back, feeling the recoil of humanity from one stricken by mortal disease.

The gramophone struck up.

<Ladies and gentlemen, I will now name the traitor. Number Twenty-one, stand forward.>

Rogers stepped forward. The concentrated fear and loathing of forty-eight pairs of eyes burned upon him. The miserable Jukes set up a fresh wail.

<Oh, my God! Oh, my God!>

<Silence! Number Twenty-one, take off your mask.>

The traitor pulled the thick covering from his face. The intense hatred of the eyes devoured him.

<Number Thirty-seven, this man was introduced here by you, under the name of Joseph Rogers, formerly second footman in the service of the Duke of Denver, dismissed for pilfering. Did you take steps to verify that statement?>

<I did—I did! As God's my witness, it was all straight. I had him identified by two of the servants. I made enquiries. The tale was straight—I'll swear it was.>

The President consulted a paper before him, then he looked at his watch again.

<Ladies and gentlemen, please take your partners....>

Number Twenty-one, his arms twisted behind him and bound, and his wrists hand-cuffed, stood motionless, while the dance of doom circled about him. The clapping, as it ended, sounded like the clapping of the men and women who sat, thirsty-lipped, beneath the guillotine.

<Number Twenty-one, your name has been given as Joseph Rogers, footman, dismissed for theft. Is that your real name?>

<No.>

<What is your name?>

<Peter Death Bredon Wimsey.>

<We thought you were dead.>

<Naturally. You were intended to think so.>

<What has become of the genuine Joseph Rogers?>

<He died abroad. I took his place. I may say that no real blame attaches to your people for not having realised who I was. I not only took Roger's place; I \textit{was} Rogers. Even when I was alone, I walked like Rogers, I sat like Rogers, I read Rogers's books, and wore Rogers's clothes. In the end, I almost thought Rogers's thoughts. The only way to keep up a successful impersonation is never to relax.>

<I see. The robbery of your own flat was arranged?>

<Obviously.>

<The robbery of the Dowager Duchess, your mother, was connived at by you?>

<It was. It was a very ugly tiara—no real loss to anybody with decent taste. May I smoke, by the way?>

<You may not. Ladies and gentlemen....>

The dance was like the mechanical jigging of puppets. Limbs jerked, feet faltered. The prisoner watched with an air of critical detachment.

<Numbers Fifteen, Twenty-two and Forty-nine. You have watched the prisoner. Has he made any attempts to communicate with anybody?>

<None.> Number Twenty-two was the spokesman. <His letters and parcels have been opened, his telephone tapped, and his movements followed. His water-pipes have been under observation for Morse signals.>

<You are sure of what you say?>

<Absolutely.>

<Prisoner, have you been alone in this adventure? Speak the truth, or things will be made somewhat more unpleasant for you than they might otherwise be.>

<I have been alone. I have taken no unnecessary risks.>

<It may be so. It will, however, be as well that steps should be taken to silence the man at Scotland Yard—what is his name?—Parker. Also the prisoner's manservant, Mervyn Bunter, and possibly also his mother and sister. The brother is a stupid oaf, and not, I think, likely to have been taken into the prisoner's confidence. A precautionary watch will, I think, meet the necessities of his case.>

The prisoner appeared, for the first time, to be moved.

<Sir, I assure you that my mother and sister know nothing which could possibly bring danger on the Society.>

<You should have thought of their situation earlier. Ladies and gentlemen, please take\longdash>

<No—no!> Flesh and blood could endure the mockery no longer. <No! Finish with him. Get it over. Break up the meeting. It's dangerous. The police\longdash>

<Silence!>

The President glanced round at the crowd. It had a dangerous look about it. He gave way.

<Very well. Take the prisoner away and silence him. He will receive Number 4 treatment. And be sure you explain it to him carefully first.>

<Ah!>

The eyes expressed a wolfish satisfaction. Strong hands gripped Wimsey's arms.

<One moment—for God's sake let me die decently.>

<You should have thought this over earlier. Take him away. Ladies and gentlemen, be satisfied—he will not die quickly.>

<Stop! Wait!> cried Wimsey desperately. <I have something to say. I don't ask for life—only for a quick death. I—I have something to sell.>

<To sell?>

<Yes.>

<We make no bargains with traitors.>

<No—but listen! Do you think I have not thought of this? I am not so mad. I have left a letter.>

<Ah! now it is coming. A letter. To whom?>

<To the police. If I do not return to-morrow\longdash>

<Well?>

<The letter will be opened.>

<Sir,> broke in Number Fifteen. <This is bluff. The prisoner has not sent any letter. He has been strictly watched for many months.>

<Ah! but listen. I left the letter before I came to Lambeth.>

<Then it can contain no information of value.>

<Oh, but it does.>

<What?>

<The combination of my safe.>

<Indeed? Has this man's safe been searched?>

<Yes, sir.>

<What did it contain?>

<No information of importance, sir. An outline of our organisation—the name of this house—nothing that cannot be altered and covered before morning.>

Wimsey smiled.

<Did you investigate the inner compartment of the safe?>

There was a pause.

<You hear what he says,> snapped the President sharply. <Did you find this inner compartment?>

<There was no inner compartment, sir. He is trying to bluff.>

<I hate to contradict you,> said Wimsey, with an effort at his ordinary pleasant tone, <but I really think you must have overlooked the inner compartment.>

<Well,> said the President, <and what do you say is in this inner compartment, if it does exist?>

<The names of every member of this Society, with their addresses, photographs, and finger-prints.>

<What?>

The eyes round him now were ugly with fear. Wimsey kept his face steadily turned towards the President.

<How do you say you have contrived to get this information?>

<Well, I have been doing a little detective work on my own, you know.>

<But you have been watched.>

<True. The finger-prints of my watchers adorn the first page of the collection.>

<This statement can be proved?>

<Certainly. I will prove it. The name of Number Fifty, for example\longdash>

<Stop!>

A fierce muttering arose. The President silenced it with a gesture.

<If you mention names here, you will certainly have no hope of mercy. There is a fifth treatment—kept specially for people who mention names. Bring the prisoner to my office. Keep the dance going.>

The President took an automatic from his hip-pocket and faced his tightly fettered prisoner across the desk.

<Now speak!> he said.

<I should put that thing away, if I were you,> said Wimsey contemptuously. <It would be a much pleasanter form of death than treatment Number 5, and I might be tempted to ask for it.>

<Ingenious,> said the President, <but a little too ingenious. Now, be quick; tell me what you know.>

<Will you spare me if I tell you?>

<I make no promises. Be quick.>

Wimsey shrugged his bound and aching shoulders.

<Certainly. I will tell you what I know. Stop me when you have heard enough.>

He leaned forward and spoke low. Overhead the noise of the gramophone and the shuffling of feet bore witness that the dance was going on. Stray passers-by crossing the Heath noted that the people in the lonely house were making a night of it again.

<Well,> said Wimsey, <am I to go on?>

From beneath the mask the President's voice sounded as though he were grimly smiling.

<My lord,> he said, <your story fills me with regret that you are not, in fact, a member of our Society. Wit, courage, and industry are valuable to an association like ours. I fear I cannot persuade you? No—I supposed not.>

He touched a bell on his desk.

<Ask the members kindly to proceed to the supper-room,> he said to the mask who entered.

The <supper-room> was on the ground-floor, shuttered and curtained. Down its centre ran a long, bare table, with chairs set about it.

<A Barmecide feast, I see,> said Wimsey pleasantly. It was the first time he had seen this room. At the far end, a trap-door in the floor gaped ominously.

The President took the head of the table.

<Ladies and gentlemen,> he began, as usual—and the foolish courtesy had never sounded so sinister—<I will not conceal from you the seriousness of the situation. The prisoner has recited to me more than twenty names and addresses which were thought to be unknown, except to their owners and to me. There has been great carelessness>—his voice rang harshly—<which will have to be looked into. Finger-prints have been obtained—he has shown me the photographs of some of them. How our investigators came to overlook the inner door of this safe is a matter which calls for enquiry.>

<Don't blame them,> put in Wimsey. <It was meant to be overlooked, you know. I made it like that on purpose.>

The President went on, without seeming to notice the interruption.

<The prisoner informs me that the book with the names and addresses is to be found in this inner compartment, together with certain letters and papers stolen from the houses of members, and numerous objects bearing authentic finger-prints. I believe him to be telling the truth. He offers the combination of the safe in exchange for a quick death. I think the offer should be accepted. What is your opinion, ladies and gentlemen?>

<The combination is known already,> said Number Twenty-two.

<Imbecile! This man has told us, and has proved to me, that he is Lord~Peter Wimsey. Do you think he will have forgotten to alter the combination? And then there is the secret of the inner door. If he disappears to-night and the police enter his house\longdash>

<I say,> said a woman's rich voice, <that the promise should be given and the information used—and quickly. Time is getting short.>

A murmur of agreement went round the table.

<You hear,> said the President, addressing Wimsey. <The Society offers you the privilege of a quick death in return for the combination of the safe and the secret of the inner door.>

<I have your word for it?>

<You have.>

<Thank you. And my mother and sister?>

<If you in your turn will give us your word—you are a man of honour—that these women know nothing that could harm us, they shall be spared.>

<Thank you, sir. You may rest assured, upon my honour, that they know nothing. I should not think of burdening any woman with such dangerous secrets—particularly those who are dear to me.>

<Very well. It is agreed—yes?>

The murmur of assent was given, though with less readiness than before.

<Then I am willing to give you the information you want. The word of the combination is \textsc{unreliability}.>

<And the inner door?>

<In anticipation of the visit of the police, the inner door—which might have presented difficulties—is open.>

<Good! You understand that if the police interfere with our messenger\longdash>

<That would not help me, would it?>

<It is a risk,> said the President thoughtfully, <but a risk which I think we must take. Carry the prisoner down to the cellar. He can amuse himself by contemplating apparatus Number 5. In the meantime, Numbers Twelve and Forty-six\longdash>

<No, no!>

A sullen mutter of dissent arose and swelled threateningly.

<No,> said a tall man with a voice like treacle. <No—why should any members be put in possession of this evidence? We have found one traitor among us to-night and more than one fool. How are we to know that Numbers Twelve and Forty-six are not fools and traitors also?>

The two men turned savagely upon the speaker, but a girl's voice struck into the discussion, high and agitated.

<Hear, hear! That's right, I say. How about us? We ain't going to have our names read by somebody we don't know nothing about. I've had enough of this. They might sell the 'ole lot of us to the narks.>

<I agree,> said another member. <Nobody ought to be trusted, nobody at all.>

The President shrugged his shoulders.

<Then what, ladies and gentlemen, do you suggest?>

There was a pause. Then the same girl shrilled out again:

<I say Mr~President oughter go himself. He's the only one as knows all the names. It won't be no cop to him. Why should we take all the risk and trouble and him sit at home and collar the money? Let him go himself, that's what I say.>

A long rustle of approbation went round the table.

<I second that motion,> said a stout man who wore a bunch of gold seals at his fob. Wimsey smiled as he looked at the seals; it was that trifling vanity which had led him directly to the name and address of the stout man, and he felt a certain affection for the trinkets on that account.

The President looked round.

<It is the wish of the meeting, then, that I should go?> he said, in an ominous voice.

Forty-five hands were raised in approbation. Only the woman known as Number Two remained motionless and silent, her strong white hands clenched on the arm of the chair.

The President rolled his eyes slowly round the threatening ring till they rested upon her.

<Am I to take it that this vote is unanimous?> he enquired.

The woman raised her head.

<Don't go,> she gasped faintly.

<You hear,> said the President, in a faintly derisive tone. <This lady says, don't go.>

<I submit that what Number Two says is neither here nor there,> said the man with the treacly voice. <Our own ladies might not like us to be going, if they were in madam's privileged position.> His voice was an insult.

<Hear, hear!> cried another man. <This is a democratic society, this is. We don't want no privileged classes.>

<Very well,> said the President. <You hear, Number Two. The feeling of the meeting is against you. Have you any reasons to put forward in favour of your opinion?>

<A hundred. The President is the head and soul of our Society. If anything should happen to him—where should we be? You>—she swept the company magnificently with her eyes—<you have all blundered. We have your carelessness to thank for all this. Do you think we should be safe for five minutes if the President were not here to repair your follies?>

<Something in that,> said a man who had not hitherto spoken.

<Pardon my suggesting,> said Wimsey maliciously, <that, as the lady appears to be in a position peculiarly favourable for the reception of the President's confidences, the contents of my modest volume will probably be no news to her. Why should not Number Two go herself?>

<Because I say she must not,> said the President sternly, checking the quick reply that rose to his companion's lips. <If it is the will of the meeting, I will go. Give me the key of the house.>

One of the men extracted it from Wimsey's jacket-pocket and handed it over.

<Is the house watched?> he demanded of Wimsey.

<No.>

<That is the truth?>

<It is the truth.>

The President turned at the door.

<If I have not returned in two hours' time,> he said, <act for the best to save yourselves, and do what you like with the prisoner. Number Two will give orders in my absence.>

He left the room. Number Two rose from her seat with a gesture of command.

<Ladies and gentlemen. Supper is now considered over. Start the dancing again.>

Down in the cellar the time passed slowly, in the contemplation of apparatus Number 5. The miserable Jukes, alternately wailing and raving, at length shrieked himself into exhaustion. The four members guarding the prisoners whispered together from time to time.

<An hour and a half since the President left,> said one.

Wimsey glanced up. Then he returned to his examination of the room. There were many curious things in it, which he wanted to memorise.

Presently the trap-door was flung open. <Bring him up!> cried a voice. Wimsey rose immediately, and his face was rather pale.

The members of the gang were again seated round the table. Number Two occupied the President's chair, and her eyes fastened on Wimsey's face with a tigerish fury, but when she spoke it was with a self-control which roused his admiration.

<The President has been two hours gone,> she said. <What has happened to him? Traitor twice over—what has happened to him?>

<How should I know?> said Wimsey. <Perhaps he has looked after Number One and gone while the going was good!>

She sprang up with a little cry of rage, and came close to him.

<Beast! liar!> she said, and struck him on the mouth. <You know he would never do that. He is faithful to his friends. What have you done with him? Speak—or I will make you speak. You two, there—bring the irons. He \textit{shall} speak!>

<I can only form a guess, madame,> replied Wimsey, <and I shall not guess any the better for being stimulated with hot irons, like Pantaloon at the circus. Calm yourself, and I will tell you what I think. I think—indeed, I greatly fear—that Monsieur le Président in his hurry to examine the interesting exhibits in my safe may, quite inadvertently, no doubt, have let the door of the inner compartment close behind him. In which case\longdash>

He raised his eyebrows, his shoulders being too sore for shrugging, and gazed at her with a limpid and innocent regret.

<What do you mean?>

Wimsey glanced round the circle.

<I think,> he said, <I had better begin from the beginning by explaining to you the mechanism of my safe. It is rather a nice safe,> he added plaintively. <I invented the idea myself—not the principle of its working, of course; that is a matter for scientists—but just the idea of the thing.

The combination I gave you is perfectly correct as far as it goes. It is a three-alphabet thirteen-letter lock by Bunn \& Fishett—a very good one of its kind. It opens the outer door, leading into the ordinary strong-room, where I keep my cash and my Froth Blower's cuff-links and all that. But there is an inner compartment with two doors, which open in quite a different manner. The outermost of these two inner doors is merely a thin steel skin, painted to look like the back of the safe and fitting closely, so as not to betray any join. It lies in the same plane as the wall of the room, you understand, so that if you were to measure the outside and the inside of the safe you would discover no discrepancy. It opens outwards with an ordinary key, and, as I truly assured the President, it was left open when I quitted my flat.>

<Do you think,> said the woman sneeringly, <that the President is so simple as to be caught in a so obvious trap? He will have wedged open that inner door undoubtedly.>

<Undoubtedly, madame. But the sole purpose of that outer inner door, if I may so express myself, is to appear to be the only inner door. But hidden behind the hinge of that door is another door, a sliding panel, set so closely in the thickness of the wall that you would hardly see it unless you knew it was there. This door was also left open. Our revered Number One had nothing to do but to walk straight through into the inner compartment of the safe, which, by the way, is built into the chimney of the old basement kitchen, which runs up the house at that point. I hope I make myself clear?>

<Yes, yes—get on. Make your story short.>

Wimsey bowed, and, speaking with even greater deliberation than ever, resumed:

<Now, this interesting list of the Society's activities, which I have had the honour of compiling, is written in a very large book—bigger, even, than Monsieur le Président's ledger which he uses downstairs. (I trust, by the way, madame, that you have borne in mind the necessity of putting that ledger in a safe place. Apart from the risk of investigation by some officious policeman, it would be inadvisable that any junior member of the Society should get hold of it. The feeling of the meeting would, I fancy, be opposed to such an occurrence.)>

<It is secure,> she answered hastily. <\textit{Mon dieu!} get on with your story.>

<Thank you—you have relieved my mind. Very good. This big book lies on a steel shelf at the back of the inner compartment. Just a moment. I have not described this inner compartment to you. It is six feet high, three feet wide, and three feet deep. One can stand up in it quite comfortably, unless one is very tall. It suits me nicely—as you may see, I am not more than five feet eight and a half. The President has the advantage of me in height; he might be a little cramped, but there would be room for him to squat if he grew tired of standing. By the way, I don't know if you know it, but you have tied me up rather tightly.>

<I would have you tied till your bones were locked together. Beat him, you! He is trying to gain time.>

<If you beat me,> said Wimsey, <I'm damned if I'll speak at all. Control yourself, madame; it does not do to move hastily when your king is in check.>

<Get on!> she cried again, stamping with rage.

<Where was I\@? Ah! the inner compartment. As I say, it is a little snug—the more so that it is not ventilated in any way. Did I mention that the book lay on a steel shelf?>

<You did.>

<Yes. The steel shelf is balanced on a very delicate concealed spring. When the weight of the book—a heavy one, as I said—is lifted, the shelf rises almost imperceptibly. In rising it makes an electrical contact. Imagine to yourself, madame; our revered President steps in—propping the false door open behind him—he sees the book—quickly he snatches it up. To make sure that it is the right one, he opens it—he studies the pages. He looks about for the other objects I have mentioned, which bear the marks of finger-prints. And silently, but very, very quickly—you can imagine it, can you not?—the secret panel, released by the rising of the shelf, leaps across like a panther behind him. Rather a trite simile, but apt, don't you think?>

<My God! oh, my God!> Her hand went up as though to tear the choking mask from her face. <You—you devil—devil! What is the word that opens the inner door? Quick! I will have it torn out of you—the word!>

<It is not a hard word to remember, madame—though it has been forgotten before now. Do you recollect, when you were a child, being told the tale of <Ali Baba and the Forty Thieves>? When I had that door made, my mind reverted, with rather a pretty touch of sentimentality, in my opinion, to the happy hours of my childhood. The words that open the door are—<Open Sesame>.>

<Ah! How long can a man live in this devil's trap of yours?>

<Oh,> said Wimsey cheerfully, <I should think he might hold out a few hours if he kept cool and didn't use up the available oxygen by shouting and hammering. If we went there at once, I dare say we should find him fairly all right.>

<I shall go myself. Take this man and—do your worst with him. Don't finish him till I come back. I want to see him die!>

<One moment,> said Wimsey, unmoved by this amiable wish. <I think you had better take me with you.>

<Why—why?>

<Because, you see, I'm the only person who can open the door.>

<But you have given me the word. Was that a lie?>

<No—the word's all right. But, you see, it's one of these new-style electric doors. In fact, it's really the very latest thing in doors. I'm rather proud of it. It opens to the words <Open Sesame> all right—\textit{but to my voice only}.>

<Your voice? I will choke your voice with my own hands. What do you mean—your voice only?>

<Just what I say. Don't clutch my throat like that, or you may alter my voice so that the door won't recognise it. That's better. It's apt to be rather pernickety about voices. It got stuck up for a week once, when I had a cold and could only implore it in a hoarse whisper. Even in the ordinary way, I sometimes have to try several times before I hit on the exact right intonation.>

She turned and appealed to a short, thick-set man standing beside her.

<Is this true? Is it possible?>

<Perfectly, ma'am, I'm afraid,> said the man civilly. From his voice Wimsey took him to be a superior workman of some kind—probably an engineer.

<Is it an electrical device? Do you understand it?>

<Yes, ma'am. It will have a microphone arrangement somewhere, which converts the sound into a series of vibrations controlling an electric needle. When the needle has traced the correct pattern, the circuit is completed and the door opens. The same thing can be done by light vibrations equally easily.>

<Couldn't you open it with tools?>

<In time, yes, ma'am. But only by smashing the mechanism, which is probably well protected.>

<You may take that for granted,> interjected Wimsey reassuringly.

She put her hands to her head.

<I'm afraid we're done in,> said the engineer, with a kind of respect in his tone for a good job of work.

<No—wait! Somebody must know—the workmen who made this thing?>

<In Germany,> said Wimsey briefly.

<Or—yes, yes, I have it—a gramophone. This—this—\textit{he}—shall be made to say the word for us. Quick—how can it be done?>

<Not possible, ma'am. Where should we get the apparatus at half-past three on a Sunday morning? The poor gentleman would be dead long before\longdash>

There was a silence, during which the sounds of the wakening day came through the shuttered windows. A motor-horn sounded distantly.

<I give in,> she said. <We must let him go. Take the ropes off him. You will free him, won't you?> she went on, turning piteously to Wimsey. <Devil as you are, you are not such a devil as that! You will go straight back and save him!>

<Let him go, nothing!> broke in one of the men. <He doesn't go to peach to the police, my lady, don't you think it. The President's done in, that's all, and we'd all better make tracks while we can. It's all up, boys. Chuck this fellow down the cellar and fasten him in, so he can't make a row and wake the place up. I'm going to destroy the ledgers. You can see it done if you don't trust me. And you, Thirty, you know where the switch is. Give us a quarter of an hour to clear, and then you can blow the place to glory.>

<No! You can't go—you can't leave him to die—your President—your leader—my—I won't let it happen. Set this devil free. Help me, one of you, with the ropes\longdash>

<None of that, now,> said the man who had spoken before. He caught her by the wrists, and she twisted, shrieking, in his arms, biting and struggling to get free.

<Think, think,> said the man with the treacly voice. <It's getting on to morning. It'll be light in an hour or two. The police may be here any minute.>

<The police!> She seemed to control herself by a violent effort. <Yes, yes, you are right. We must not imperil the safety of all for the sake of one man. \textit{He} himself would not wish it. That is so. We will put this carrion in the cellar where it cannot harm us, and depart, every one to his own place, while there is time.>

<And the other prisoner?>

<He? Poor fool—he can do no harm. He knows nothing. Let him go,> she answered contemptuously.

In a few minutes' time Wimsey found himself bundled unceremoniously into the depths of the cellar. He was a little puzzled. That they should refuse to let him go, even at the price of Number One's life, he could understand. He had taken the risk with his eyes open. But that they should leave him as a witness against them seemed incredible.

The men who had taken him down strapped his ankles together and departed, switching the lights out as they went.

<Hi! Kamerad!> said Wimsey. <It's a bit lonely sitting here. You might leave the light on.>

<It's all right, my friend,> was the reply. <You will not be in the dark long. They have set the time-fuse.>

The other man laughed with rich enjoyment, and they went out together. So that was it. He was to be blown up with the house. In that case the President would certainly be dead before he was extricated. This worried Wimsey; he would rather have been able to bring the big crook to justice. After all, Scotland Yard had been waiting six years to break up this gang.

He waited, straining his ears. It seemed to him that he heard footsteps over his head. The gang had all crept out by this time....

There was certainly a creak. The trap-door had opened; he felt, rather than heard, somebody creeping into the cellar.

<Hush!> said a voice in his ear. Soft hands passed over his face, and went fumbling about his body. There came the cold touch of steel on his wrists. The ropes slackened and dropped off. A key clicked in the handcuffs. The strap about his ankles was unbuckled.

<Quick! quick! they have set the time-switch. The house is mined. Follow me as fast as you can. I stole back—I said I had left my jewellery. It was true. I left it on purpose. \textit{He} must be saved—only you can do it. Make haste!>

Wimsey, staggering with pain, as the blood rushed back into his bound and numbed arms, crawled after her into the room above. A moment, and she had flung back the shutters and thrown the window open.

<Now go! Release him! You promise?>

<I promise. And I warn you, madame, that this house is surrounded. When my safe-door closed it gave a signal which sent my servant to Scotland Yard. Your friends are all taken\longdash>

<Ah! But you go—never mind me—quick! The time is almost up.>

<Come away from this!>

He caught her by the arm, and they went running and stumbling across the little garden. An electric torch shone suddenly in the bushes.

<That you, Parker?> cried Wimsey. <Get your fellows away. Quick! the house is going up in a minute.>

The garden seemed suddenly full of shouting, hurrying men. Wimsey, floundering in the darkness, was brought up violently against the wall. He made a leap at the coping, caught it, and hoisted himself up. His hands groped for the woman; he swung her up beside him. They jumped; everyone was jumping; the woman caught her foot and fell with a gasping cry. Wimsey tried to stop himself, tripped over a stone, and came down headlong. Then, with a flash and a roar, the night went up in fire.

Wimsey picked himself painfully out from among the débris of the garden wall. A faint moaning near him proclaimed that his companion was still alive. A lantern was turned suddenly upon them.

<Here you are!> said a cheerful voice. <Are you all right, old thing? Good lord! what a hairy monster!>

<All right,> said Wimsey. <Only a bit winded. Is the lady safe? H'm—arm broken, apparently—otherwise sound. What's happened?>

<About half a dozen of 'em got blown up; the rest we've bagged.> Wimsey became aware of a circle of dark forms in the wintry dawn. <Good Lord, what a day! What a come-back for a public character! You old stinker—to let us go on for two years thinking you were dead! I bought a bit of black for an arm-band. I did, really. Did anybody know, besides Bunter?>

<Only my mother and sister. I put it in a secret trust—you know, the thing you send to executors and people. We shall have an awful time with the lawyers, I'm afraid, proving I'm me. Hullo! Is that friend Sugg?>

<Yes, my lord,> said Inspector Sugg, grinning and nearly weeping with excitement. <Damned glad to see your lordship again. Fine piece of work, your lordship. They're all wanting to shake hands with you, sir.>

<Oh, Lord! I wish I could get washed and shaved first. Awfully glad to see you all again, after two years' exile in Lambeth. Been a good little show, hasn't it?>

<Is he safe?>

Wimsey started at the agonised cry.

<Good Lord!> he cried. <I forgot the gentleman in the safe. Here, fetch a car, quickly. I've got the great big top Moriarty of the whole bunch quietly asphyxiating at home. Here—hop in, and put the lady in too. I promised we'd get back and save him—though> (he finished the sentence in Parker's ear) <there may be murder charges too, and I wouldn't give much for his chance at the Old Bailey. Whack her up. He can't last much longer shut up there. He's the bloke you've been wanting, the man at the back of the Morrison case and the Hope-Wilmington case, and hundreds of others.>

The cold morning had turned the streets grey when they drew up before the door of the house in Lambeth. Wimsey took the woman by the arm and helped her out. The mask was off now, and showed her face, haggard and desperate, and white with fear and pain.

<Russian, eh?> whispered Parker in Wimsey's ear.

<Something of the sort. Damn! the front door's blown shut, and the blighter's got the key with him in the safe. Hop through the window, will you?>

Parker bundled obligingly in, and in a few seconds threw open the door to them. The house seemed very still. Wimsey led the way to the back room, where the strong-room stood. The outer door and the second door stood propped open with chairs. The inner door faced them like a blank green wall.

<Only hope he hasn't upset the adjustment with thumping at it,> muttered Wimsey. The anxious hand on his arm clutched feverishly. He pulled himself together, forcing his tone to one of cheerful commonplace.

<Come on, old thing,> he said, addressing himself conversationally to the door. <Show us your paces. Open Sesame, confound you. Open Sesame!>

The green door slid suddenly away into the wall. The woman sprang forward and caught in her arms the humped and senseless thing that rolled out from the safe. Its clothes were torn to ribbons, and its battered hands dripped blood.

<It's all right,> said Wimsey, <it's all right! He'll live—to stand his trial.>