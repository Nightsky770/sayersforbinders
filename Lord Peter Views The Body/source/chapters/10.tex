%!TeX root=../viewsbodytop.tex
\addchap{The Piscatorial Farce of the Stolen Stomach}

\lettrine[lines=4,ante=‘]{W}{hat} in the world,' said Lord~Peter Wimsey, »is that?«

\zz
Thomas Macpherson disengaged the tall jar from its final swathings of paper and straw and set it tenderly upright beside the coffee-pot.

»That,« he said, »is Great-Uncle Joseph's legacy.«

»And who is Great-Uncle Joseph?«

»He was my mother's uncle. Name of Ferguson. Eccentric old boy. I was rather a favourite of his.«

»It looks like it. Was that all he left you?«

»Imph'm. He said a good digestion was the most precious thing a man could have.«

»Well, he was right there. Is this his? Was it a good one?«

»Good enough. He lived to be ninety-five, and never had a day's illness.«

Wimsey looked at the jar with increased respect.

»What did he die of?«

»Chucked himself out of a sixth-story window. He had a stroke, and the doctors told him—or he guessed for himself—that it was the beginning of the end. He left a letter. Said he had never been ill in his life and wasn't going to begin now. They brought it in temporary insanity, of course, but I think he was thoroughly sensible.«

»I should say so. What was he when he was functioning?«

»He used to be in business—something to do with ship-building, I believe, but he retired long ago. He was what the papers call a recluse. Lived all by himself in a little top flat in Glasgow, and saw nobody. Used to go off by himself for days at a time, nobody knew where or why. I used to look him up about once a year and take him a bottle of whisky.«

»Had he any money?«

»Nobody knew. He ought to have had—he was a rich man when he retired. But, when we came to look into it, it turned out he only had a balance of about five hundred pounds in the Glasgow Bank. Apparently he drew out almost everything he had about twenty years ago. There were one or two big bank failures round about that time, and they thought he must have got the wind up. But what he did with it, goodness only knows.«

»Kept it in an old stocking, I expect.«

»I should think Cousin Robert devoutly hopes so.«

»Cousin Robert?«

»He's the residuary legatee. Distant connection of mine, and the only remaining Ferguson. He was awfully wild when he found he'd only got five hundred. He's rather a bright lad, is Robert, and a few thousands would have come in handy.«

»I see. Well, how about a bit of brekker? You might stick Great-Uncle Joseph out of the way somewhere. I don't care about the looks of him.«

»I thought you were rather partial to anatomical specimens.«

»So I am, but not on the breakfast-table. 'A place for everything and everything in its place,' as my grandmother used to say. Besides, it would give Maggie a shock if she saw it.«

Macpherson laughed, and transferred the jar to a cupboard.

»Maggie's shock-proof. I brought a few odd bones and things with me, by way of a holiday task. I'm getting near my final, you know. She'll just think this is another of them. Ring the bell, old man, would you? We'll see what the trout's like.«

The door opened to admit the housekeeper, with a dish of grilled trout and a plate of fried scones.

»These look good, Maggie,« said Wimsey, drawing his chair up and sniffing appreciatively.

»Aye, sir, they're gude, but they're awfu' wee fish.«

»Don't grumble at them,« said Macpherson. »They're the sole result of a day's purgatory up on Loch Whyneon. What with the sun fit to roast you and an east wind, I'm pretty well flayed alive. I very nearly didn't shave at all this morning.« He passed a reminiscent hand over his red and excoriated face. »Ugh! It's a stiff pull up that hill, and the boat was going wallop, wallop all the time, like being in the Bay of Biscay.«

»Damnable, I should think. But there's a change coming. The glass is going back. We'll be having some rain before we're many days older.«

»Time, too,« said Macpherson. »The burns are nearly dry, and there's not much water in the Fleet.« He glanced out of the window to where the little river ran tinkling and skinkling over the stones at the bottom of the garden. »If only we get a few days' rain now, there'll be some grand fishing.«

»It \textit{would} come just as I've got to go, naturally,« remarked Wimsey.

»Yes; can't you stay a bit longer? I want to have a try for some sea-trout.«

»Sorry, old man, can't be done. I must be in Town on Wednesday. Never mind. I've had a fine time in the fresh air and got in some good rounds of golf.«

»You must come up another time. I'm here for a month—getting my strength up for the exams and all that. If you can't get away before I go, we'll put it off till August and have a shot at the grouse. The cottage is always at your service, you know, Wimsey.«

»Many thanks. I may get my business over quicker than I think, and, if I do, I'll turn up here again. When did you say your great-uncle died?«

Macpherson stared at him.

»Some time in April, as far as I can remember. Why?«

»Oh, nothing—I just wondered. You were a favourite of his, didn't you say?«

»In a sense. I think the old boy liked my remembering him from time to time. Old people are pleased by little attentions, you know.«

»M'm. Well, it's a queer world. What did you say his name was?«

»Ferguson—Joseph Alexander Ferguson, to be exact. You seem extraordinarily interested in Great-Uncle Joseph.«

»I thought, while I was about it, I might look up a man I know in the ship-building line, and see if he knows anything about where the money went to.«

»If you can do that, Cousin Robert will give you a medal. But, if you really want to exercise your detective powers on the problem, you'd better have a hunt through the flat in Glasgow.«

»Yes—what is the address, by the way?«

Macpherson told him the address.

»I'll make a note of it, and, if anything occurs to me, I'll communicate with Cousin Robert. Where does he hang out?«

»Oh, he's in London, in a solicitor's office. Crosbie \& Plump, somewhere in Bloomsbury. Robert was studying for the Scottish Bar, you know, but he made rather a mess of things, so they pushed him off among the Sassenachs. His father died a couple of years ago—he was a Writer to the Signet in Edinburgh—and I fancy Robert has rather gone to the bow-wows since then. Got among a cheerful crowd down there, don't you know, and wasted his substance somewhat.«

»Terrible! Scotsmen shouldn't be allowed to leave home. What are you going to do with Great-Uncle?«

»Oh, I don't know. Keep him for a bit, I think. I liked the old fellow, and I don't want to throw him away. He'll look rather well in my consulting-room, don't you think, when I'm qualified and set up my brass plate. I'll say he was presented by a grateful patient on whom I performed a marvellous operation.«

»That's a good idea. Stomach-grafting. Miracle of surgery never before attempted. He'll bring sufferers to your door in flocks.«

»Good old Great-Uncle—he may be worth a fortune to me after all.«

»So he may. I don't suppose you've got such a thing as a photograph of him, have you?«

»A photograph?« Macpherson stared again. »Great-Uncle seems to be becoming a passion with you. I don't suppose the old man had a photograph taken these thirty years. There was one done then—when he retired from business. I expect Robert's got that.«

»Och aye,« said Wimsey, in the language of the country.

Wimsey left Scotland that evening, and drove down through the night towards London, thinking hard as he went. He handled the wheel mechanically, swerving now and again to avoid the green eyes of rabbits as they bolted from the roadside to squat fascinated in the glare of his head-lamps. He was accustomed to say that his brain worked better when his immediate attention was occupied by the incidents of the road.

Monday morning found him in town with his business finished and his thinking done. A consultation with his ship-building friend had put him in possession of some facts about Great-Uncle Joseph's money, together with a copy of Great-Uncle Joseph's photograph, supplied by the London representative of the Glasgow firm to which he had belonged. It appeared that old Ferguson had been a man of mark in his day. The portrait showed a fine, dour old face, long-lipped and high in the cheek-bones—one of those faces which alter little in a lifetime. Wimsey looked at the photograph with satisfaction as he slipped it into his pocket and made a bee-line for Somerset House.

Here he wandered timidly about the wills department, till a uniformed official took pity on him and enquired what he wanted.

»Oh, thank you,« said Wimsey effusively, »thank you so much. Always feel nervous in these places. All these big desks and things, don't you know, so awe-inspiring and business-like. Yes, I just wanted to have a squint at a will. I'm told you can see anybody's will for a shilling. Is that really so?«

»Yes, sir, certainly. Anybody's will in particular, sir?«

»Oh, yes, of course—how silly of me. Yes. Curious, isn't it, that when you're dead any stranger can come and snoop round your private affairs—see how much you cut up for and who your lady friends were, and all that. Yes. Not at all nice. Horrid lack of privacy, what?«

The attendant laughed.

»I expect it's all one when you're dead, sir.«

»That's awfully true. Yes, naturally, you're dead by then and it doesn't matter. May be a bit trying for your relations, of course, to learn what a bad boy you've been. Great fun annoyin' one's relations. Always do it myself. Now, what were we sayin'? Ah! yes—the will. (I'm always so absent-minded.) Whose will, you said? Well, it's an old Scots gentleman called Joseph Alexander Ferguson that died at Glasgow—you know Glasgow, where the accent's so strong that even Scotsmen faint when they hear it—in April, this last April as ever was. If it's not troubling you too much, may I have a bob's-worth of Joseph Alexander Ferguson?«

The attendant assured him that he might, adding the caution that he must memorise the contents of the will and not on any account take notes. Thus warned, Wimsey was conducted into a retired corner, where in a short time the will was placed before him.

It was a commendably brief document, written in holograph, and was dated the previous January. After the usual preamble and the bequest of a few small sums and articles of personal ornament to friends, it proceeded somewhat as follows:

»And I direct that, after my death, the alimentary organs be removed entire with their contents from my body, commencing with the œsophagus and ending with the anal canal, and that they be properly secured at both ends with a suitable ligature, and be enclosed in a proper preservative medium in a glass vessel and given to my great-nephew Thomas Macpherson of the Stone Cottage, Gatehouse-of-the-Fleet, in Kirkcudbrightshire, now studying medicine in Aberdeen. And I bequeath him these my alimentary organs with their contents for his study and edification, they having served me for ninety-five years without failure or defect, because I wish him to understand that no riches in the world are comparable to the riches of a good digestion. And I desire of him that he will, in the exercise of his medical profession, use his best endeavours to preserve to his patients the blessing of good digestion unimpaired, not needlessly filling their stomachs with drugs out of concern for his own pocket, but exhorting them to a sober and temperate life agreeably to the design of Almighty Providence.«

After this remarkable passage, the document went on to make Robert Ferguson residuary legatee without particular specification of any property, and to appoint a firm of lawyers in Glasgow executors of the will.

Wimsey considered the bequest for some time. From the phraseology he concluded that old Mr~Ferguson had drawn up his own will without legal aid, and he was glad of it, for its wording thus afforded a valuable clue to the testator's mood and intention. He mentally noted three points: the »alimentary organs with their contents« were mentioned twice over, with a certain emphasis; they were to be ligatured top and bottom; and the legacy was accompanied by the expression of a wish that the legatee should not allow his financial necessities to interfere with the conscientious exercise of his professional duties. Wimsey chuckled. He felt he rather liked Great-Uncle Joseph.

He got up, collected his hat, gloves, and stick, and advanced with the will in his hand to return it to the attendant. The latter was engaged in conversation with a young man, who seemed to be expostulating about something.

»I'm sorry, sir,« said the attendant, »but I don't suppose the other gentleman will be very long. Ah!« He turned and saw Wimsey. »Here is the gentleman.«

The young man, whose reddish hair, long nose, and slightly sodden eyes gave him the appearance of a dissipated fox, greeted Wimsey with a disagreeable stare.

»What's up? Want me?« asked his lordship airily.

»Yes, sir. Very curious thing, sir; here's a gentleman enquiring for that very same document as you've been studying, sir. I've been in this department fifteen years, and I don't know as I ever remember such a thing happening before.«

»No,« said Wimsey, »I don't suppose there's much of a run on any of your lines as a rule.«

»It's a very curious thing indeed,« said the stranger, with marked displeasure in his voice.

»Member of the family?« suggested Wimsey.

»I \textit{am} a member of the family,« said the foxy-faced man. »May I ask whether \textit{you} have any connection with us?«

»By all means,« replied Wimsey graciously.

»I don't believe it. I don't know you.«

»No, no—I meant you might ask, by all means.«

The young man positively showed his teeth.

»Do you mind telling me who you are, anyhow, and why you're so damned inquisitive about my great-uncle's will?«

Wimsey extracted a card from his case and presented it with a smile. Mr~Robert Ferguson changed colour.

»If you would like a reference as to my respectability,« went on Wimsey affably, »Mr~Thomas Macpherson will, I am sure, be happy to tell you about me. I am inquisitive,« said his lordship—»a student of humanity. Your cousin mentioned to me the curious clause relating to your esteemed great-uncle's—er—stomach and appurtenances. Curious clauses are a passion with me. I came to look it up and add it to my collection of curious wills. I am engaged in writing a book on the subject—\textit{Clauses and Consequences}. My publishers tell me it should enjoy a ready sale. I regret that my random jottings should have encroached upon your doubtless far more serious studies. I wish you a very good morning.«

As he beamed his way out, Wimsey, who had quick ears, heard the attendant informing the indignant Mr~Ferguson that he was »a very funny gentleman—not quite all there, sir.« It seemed that his criminological fame had not penetrated to the quiet recesses of Somerset House. »But,« said Wimsey to himself, »I am sadly afraid that Cousin Robert has been given food for thought.«

Under the spur of this alarming idea, Wimsey wasted no time, but took a taxi down to Hatton Garden, to call upon a friend of his. This gentleman, rather curly in the nose and fleshy about the eyelids, nevertheless came under Mr~Chesterton's definition of a nice Jew, for his name was neither Montagu nor McDonald, but Nathan Abrahams, and he greeted Lord~Peter with a hospitality amounting to enthusiasm.

»So pleased to see you. Sit down and have a drink. You have come at last to select the diamonds for the future Lady Peter, eh?«

»Not yet,« said Wimsey.

»No? That's too bad. You should make haste and settle down. It is time you became a family man. Years ago we arranged I should have the privilege of decking the bride for the happy day. That is a promise, you know. I think of it when the fine stones pass through my hands. I say, 'That would be the very thing for my friend Lord~Peter.' But I hear nothing, and I sell them to stupid Americans who think only of the price and not of the beauty.«

»Time enough to think of the diamonds when I've found the lady.«

Mr~Abrahams threw up his hands.

»Oh, yes! And then everything will be done in a hurry! »Quick, Mr~Abrahams! I have fallen in love yesterday and I am being married to-morrow.« But it may take months—years—to find and match perfect stones. It can't be done between to-day and to-morrow. Your bride will be married in something ready-made from the jeweller's.«

»If three days are enough to choose a wife,« said Wimsey, laughing, »one day should surely be enough for a necklace.«

»That is the way with Christians,« replied the diamond-merchant resignedly. »You are so casual. You do not think of the future. Three days to choose a wife! No wonder the divorce-courts are busy. My son Moses is being married next week. It has been arranged in the family these ten years. Rachel Goldstein, it is. A good girl, and her father is in a very good position. We are all very pleased, I can tell you. Moses is a good son, a very good son, and I am taking him into partnership.«

»I congratulate you,« said Wimsey heartily. »I hope they will be very happy.«

»Thank you, Lord~Peter. They will be happy, I am sure. Rachel is a sweet girl and very fond of children. And she is pretty, too. Prettiness is not everything, but it is an advantage for a young man in these days. It is easier for him to behave well to a pretty wife.«

»True,« said Wimsey. »I will bear it in mind when my time comes. To the health of the happy pair, and may you soon be an ancestor. Talking of ancestors, I've got an old bird here that you may be able to tell me something about.«

»Ah, yes! Always delighted to help you in any way, Lord~Peter.«

»This photograph was taken some thirty years ago, but you may possibly recognise it.«

Mr~Abrahams put on a pair of horn-rimmed spectacles, and examined the portrait of Great-Uncle Joseph with serious attention.

»Oh, yes, I know him quite well. What do you want to know about him, eh?« He shot a swift and cautious glance at Wimsey.

»Nothing to his disadvantage. He's dead, anyhow. I thought it just possible he had been buying precious stones lately.«

»It is not exactly business to give information about a customer,« said Mr~Abrahams.

»I'll tell you what I want it for,« said Wimsey. He lightly sketched the career of Great-Uncle Joseph, and went on: »You see, I looked at it this way. When a man gets a distrust of banks, what does he do with his money? He puts it into property of some kind. It may be land, it may be houses—but that means rent, and more money to put into banks. He is more likely to keep it in gold or notes, or to put it into precious stones. Gold and notes are comparatively bulky; stones are small. Circumstances in this case led me to think he might have chosen stones. Unless we can discover what he did with the money, there will be a great loss to his heirs.«

»I see. Well, if it is as you say, there is no harm in telling you. I know you to be an honourable man, and I will break my rule for you. This gentleman, Mr~Wallace\longdash«

»Wallace, did he call himself?«

»That was not his name? They are funny, these secretive old gentlemen. But that is nothing unusual. Often, when they buy stones, they are afraid of being robbed, so they give another name. Yes, yes. Well, this Mr~Wallace used to come to see me from time to time, and I had instructions to find diamonds for him. He was looking for twelve big stones, all matching perfectly and of superb quality. It took a long time to find them, you know.«

»Of course.«

»Yes. I supplied him with seven altogether, over a period of twenty years or so. And other dealers supplied him also. He is well known in this street. I found the last one for him—let me see—in last December, I think. A beautiful stone—beautiful! He paid seven thousand pounds for it.«

»Some stone. If they were all as good as that, the collection must be worth something.«

»Worth anything. It is difficult to tell how much. As you know, the twelve stones, all matched together, would be worth far more than the sum of the twelve separate prices paid for the individual diamonds.«

»Naturally they would. Do you mind telling me how he was accustomed to pay for them?«

»In Bank of England notes—always—cash on the nail. He insisted on discount for cash,« added Mr~Abrahams, with a chuckle.

»He was a Scotsman,« replied Wimsey. »Well, that's clear enough. He had a safe-deposit somewhere, no doubt. And, having collected the stones, he made his will. That's clear as daylight, too.«

»But what has become of the stones?« enquired Mr~Abrahams, with professional anxiety.

»I think I know that too,« said Wimsey. »I'm enormously obliged to you, and so, I fancy, will his heir be.«

»If they should come into the market again\longdash« suggested Mr~Abrahams.

»I'll see you have the handling of them,« said Wimsey promptly.

»That is kind of you,« said Mr~Abrahams. »Business is business. Always delighted to oblige you. Beautiful stones—beautiful. If you thought of being the purchaser, I would charge you a special commission, as my friend.«

»Thank you,« said Wimsey, »but as yet I have no occasion for diamonds, you know.«

»Pity, pity,« said Mr~Abrahams. »Well, very glad to have been of service to you. You are not interested in rubies? No? Because I have something very pretty here.«

He thrust his hand casually into a pocket, and brought out a little pool of crimson fire like a miniature sunset.

»Look nice in a ring, now, wouldn't it?« said Mr~Abrahams. »An engagement ring, eh?«

Wimsey laughed, and made his escape.

He was strongly tempted to return to Scotland and attend personally to the matter of Great-Uncle Joseph, but the thought of an important book sale next day deterred him. There was a manuscript of Catullus which he was passionately anxious to secure, and he never entrusted his interests to dealers. He contented himself with sending a wire to Thomas Macpherson:

»\textsc{Advise opening up Great-uncle Joseph immediately.}«

The girl at the post-office repeated the message aloud and rather doubtfully. »Quite right,« said Wimsey, and dismissed the affair from his mind.

He had great fun at the sale next day. He found a ring of dealers in possession, happily engaged in conducting a knock-out. Having lain low for an hour in a retired position behind a large piece of statuary, he emerged, just as the hammer was falling upon the Catullus for a price representing the tenth part of its value, with an overbid so large, prompt, and sonorous that the ring gasped with a sense of outrage. Skrymes—a dealer who had sworn an eternal enmity to Wimsey, on account of a previous little encounter over a Justinian—pulled himself together and offered a fifty-pound advance. Wimsey promptly doubled his bid. Skrymes overbid him fifty again. Wimsey instantly jumped another hundred, in the tone of a man prepared to go on till Doomsday. Skrymes scowled and was silent. Somebody raised it fifty more; Wimsey made it guineas and the hammer fell. Encouraged by this success, Wimsey, feeling that his hand was in, romped happily into the bidding for the next lot, a \textit{Hypnerotomachia} which he already possessed, and for which he felt no desire whatever. Skrymes, annoyed by his defeat, set his teeth, determining that, if Wimsey was in the bidding mood, he should pay through the nose for his rashness. Wimsey, entering into the spirit of the thing, skied the bidding with enthusiasm. The dealers, knowing his reputation as a collector, and fancying that there must be some special excellence about the book that they had failed to observe, joined in whole-heartedly, and the fun became fast and furious. Eventually they all dropped out again, leaving Skrymes and Wimsey in together. At which point Wimsey, observing a note of hesitation in the dealer's voice, neatly extricated himself and left Mr~Skrymes with the baby. After this disaster, the ring became sulky and demoralised and refused to bid at all, and a timid little outsider, suddenly flinging himself into the arena, became the owner of a fine fourteenth-century missal at bargain price. Crimson with excitement and surprise, he paid for his purchase and ran out of the room like a rabbit, hugging the missal as though he expected to have it snatched from him. Wimsey thereupon set himself seriously to acquire a few fine early printed books, and, having accomplished this, retired, covered with laurels and hatred.

After this delightful and satisfying day, he felt vaguely hurt at receiving no ecstatic telegram from Macpherson. He refused to imagine that his deductions had been wrong, and supposed rather that the rapture of Macpherson was too great to be confined to telegraphic expression and would come next day by post. However, at eleven next morning the telegram arrived. It said:

»\textsc{Just got your wire what does it mean great-uncle stolen last night burglar escaped please write fully.}«

Wimsey committed himself to a brief comment in language usually confined to the soldiery. Robert had undoubtedly got Great-Uncle Joseph, and, even if they could trace the burglary to him, the legacy was by this time gone for ever. He had never felt so furiously helpless. He even cursed the Catullus, which had kept him from going north and dealing with the matter personally.

While he was meditating what to do, a second telegram was brought in. It ran:

»\textsc{Great-uncle's bottle found broken in fleet dropped by burglar in flight contents gone what next.}«

Wimsey pondered this.

»Of course,« he said, »if the thief simply emptied the bottle and put Great-Uncle in his pocket, we're done. Or if he's simply emptied Great-Uncle and put the contents in his pocket, we're done. But 'dropped in flight' sounds rather as though Great-Uncle had gone overboard lock, stock, and barrel. Why can't the fool of a Scotsman put a few more details into his wires? It'd only cost him a penny or two. I suppose I'd better go up myself. Meanwhile a little healthy occupation won't hurt him.«

He took a telegraph form from the desk and despatched a further message:

»\textsc{Was great-uncle in bottle when dropped if so drag river if not pursue burglar probably Robert Ferguson spare no pains starting for Scotland to-night hope arrive early to-morrow urgent important put your back into it will explain.}«

The night express decanted Lord~Peter Wimsey at Dumfries early the following morning, and a hired car deposited him at the Stone Cottage in time for breakfast. The door was opened to him by Maggie, who greeted him with hearty cordiality:

»Come awa' in, sir. All's ready for ye, and Mr~Macpherson will be back in a few minutes, I'm thinkin'. Ye'll be tired with your long journey, and hungry, maybe? Aye. Will ye tak' a bit parritch to your eggs and bacon? There's nae troot the day, though yesterday was a gran' day for the fush. Mr~Macpherson has been up and doun, up and doun the river wi' my Jock, lookin' for ane of his specimens, as he ca's them, that was dropped by the thief that cam' in. I dinna ken what the thing may be—my Jock says it's like a calf's pluck to look at, by what Mr~Macpherson tells him.«

»Dear me!« said Wimsey. »And how did the burglary happen, Maggie?«

»Indeed, sir, it was a vera' remarkable circumstance. Mr~Macpherson was awa' all day Monday and Tuesday, up at the big loch by the viaduct, fishin'. There was a big rain Saturday and Sunday, ye may remember, and Mr~Macpherson says, »There'll be grand fishin' the morn, Jock,« says he. »We'll go up to the viaduct if it stops rainin' and we'll spend the nicht at the keeper's lodge.« So on Monday it stoppit rainin' and was a grand warm, soft day, so aff they went together. There was a telegram come for him Tuesday mornin', and I set it up on the mantelpiece, where he'd see it when he cam' in, but it's been in my mind since that maybe that telegram had something to do wi' the burglary.«

»I wouldn't say but you might be right, Maggie,« replied Wimsey gravely.

»Aye, sir, that wadna surprise me.« Maggie set down a generous dish of eggs and bacon before the guest and took up her tale again.

»Well, I was sittin' in my kitchen the Tuesday nicht, waitin' for Mr~Macpherson and Jock to come hame, and sair I pitied them, the puir souls, for the rain was peltin' down again, and the nicht was sae dark I was afraid they micht ha' tummelt into a bog-pool. Weel, I was listenin' for the sound o' the door-sneck when I heard something movin' in the front room. The door wasna lockit, ye ken, because Mr~Macpherson was expectit back. So I up from my chair and I thocht they had mebbe came in and I not heard them. I waited a meenute to set the kettle on the fire, and then I heard a crackin' sound. So I cam' out and I called, »Is't you, Mr~Macpherson?« And there was nae answer, only anither big crackin' noise, so I ran forrit, and a man cam' quickly oot o' the front room, brushin' past me an' puttin' me aside wi' his hand, so, and oot o' the front door like a flash o' lightnin'. So, wi' that, I let oot a skelloch, an' Jock's voice answered me fra' the gairden gate. »Och!« I says, »Jock! here's a burrglar been i' the hoose!« An' I heerd him runnin' across the gairden, doun tae the river, tramplin' doun a' the young kail and the stra'berry beds, the blackguard!«

Wimsey expressed his sympathy.

»Aye, that was a bad business. An' the next thing, there was Mr~Macpherson and Jock helter-skelter after him. If Davie Murray's cattle had brokken in, they couldna ha' done mair deevastation. An' then there was a big splashin' an' crashin', an', after a bit, back comes Mr~Macpherson an' he says, »He's jumpit intil the Fleet,« he says, »an' he's awa'. What has he taken?« he says. »I dinna ken,« says I, »for it all happened sae quickly I couldna see onything.« »Come awa' ben,« says he, »an' we'll see what's missin'.« So we lookit high and low, an' all we could find was the cupboard door in the front room broken open, and naething taken but this bottle wi' the specimen.«

»Aha!« said Wimsey.

»Ah! an' they baith went oot tegither wi' lichts, but naething could they see of the thief. Sae Mr~Macpherson comes back, and »I'm gaun to ma bed,« says he, »for I'm that tired I can dae nae mair the nicht,« says he. »Oh!« I said, »I daurna gae tae bed; I'm frichtened.« An' Jock said, »Hoots, wumman, dinna fash yersel'. There'll be nae mair burglars the nicht, wi' the fricht we've gied 'em.« So we lockit up a' the doors an' windies an' gaed to oor beds, but I couldna sleep a wink.«

»Very natural,« said Wimsey.

»It wasna till the next mornin',« said Maggie, »that Mr~Macpherson opened yon telegram. Eh! but he was in a taking. An' then the telegrams startit. Back an' forrit, back an' forrit atween the hoose an' the post-office. An' then they fund the bits o' the bottle that the specimen was in, stuck between twa stanes i' the river. And aff goes Mr~Macpherson an' Jock wi' their waders on an' a couple o' gaffs, huntin' in a' the pools an' under the stanes to find the specimen. An' they're still at it.«

At this point three heavy thumps sounded on the ceiling.

»Gude save us!« ejaculated Maggie, »I was forgettin' the puir gentleman.«

»What gentleman?« enquired Wimsey.

»Him that was feshed oot o' the Fleet,« replied Maggie. »Excuse me juist a moment, sir.«

She fled swiftly upstairs. Wimsey poured himself out a third cup of coffee and lit a pipe.

Presently a thought occurred to him. He finished the coffee—not being a man to deprive himself of his pleasures—and walked quietly upstairs in Maggie's wake. Facing him stood a bedroom door, half open—the room which he had occupied during his stay at the cottage. He pushed it open. In the bed lay a red-headed gentleman, whose long, foxy countenance was in no way beautified by a white bandage, tilted rakishly across the left temple. A breakfast-tray stood on a table by the bed. Wimsey stepped forward with extended hand.

»Good morning, Mr~Ferguson,« said he. »This is an unexpected pleasure.«

»Good morning,« said Mr~Ferguson snappishly.

»I had no idea, when we last met,« pursued Wimsey, advancing to the bed and sitting down upon it, »that you were thinking of visiting my friend Macpherson.«

»Get off my leg,« growled the invalid. »I've broken my kneecap.«

»What a nuisance! Frightfully painful, isn't it? And they say it takes years to get right—if it ever does get right. Is it what they call a Potts fracture? I don't know who Potts was, but it sounds impressive. How did you get it? Fishing?«

»Yes. A slip in that damned river.«

»Beastly. Sort of thing that might happen to anybody. A keen fisher, Mr~Ferguson?«

»So-so.«

»So am I, when I get the opportunity. What kind of fly do you fancy for this part of the country? I rather like a Greenaway's Gadget myself. Ever tried it?«

»No,« said Mr~Ferguson briefly.

»Some people find a Pink Sisket better, so they tell me. Do you use one? Have you got your fly-book here?«

»Yes—no,« said Mr~Ferguson. »I dropped it.«

»Pity. But do give me your opinion of the Pink Sisket.«

»Not so bad,« said Mr~Ferguson. »I've sometimes caught trout with it.«

»You surprise me,« said Wimsey, not unnaturally, since he had invented the Pink Sisket on the spur of the moment, and had hardly expected his improvisation to pass muster. »Well, I suppose this unlucky accident has put a stop to your sport for the season. Damned bad luck. Otherwise, you might have helped us to have a go at the Patriarch.«

»What's that? A trout?«

»Yes—a frightfully wily old fish. Lurks about in the Fleet. You never know where to find him. Any moment he may turn up in some pool or other. I'm going out with Mac to try for him to-day. He's a jewel of a fellow. We've nicknamed him Great-Uncle Joseph. Hi! don't joggle about like that—you'll hurt that knee of yours. Is there anything I can get for you?«

He grinned amiably, and turned to answer a shout from the stairs.

»Hullo! Wimsey! is that you?«

»It is. How's sport?«

Macpherson came up the stairs four steps at a time, and met Wimsey on the landing as he emerged from the bedroom.

»I say, d'you know who that is? It's Robert.«

»I know. I saw him in town. Never mind him. Have you found Great-Uncle?«

»No, we haven't. What's all this mystery about? And what's Robert doing here? What did you mean by saying he was the burglar? And why is Great-Uncle Joseph so important?«

»One thing at a time. Let's find the old boy first. What have you been doing?«

»Well, when I got your extraordinary messages I thought, of course, you were off your rocker.« (Wimsey groaned with impatience.) »But then I considered what a funny thing it was that somebody should have thought Great-Uncle worth stealing, and thought there might be some sense in what you said, after all.« (»Dashed good of you,« said Wimsey.) »So I went out and poked about a bit, you know. Not that I think there's the faintest chance of finding anything, with the river coming down like this. Well, I hadn't got very far—by the way, I took Jock with me. I'm sure he thinks I'm mad, too. Not that he says anything; these people here never commit themselves\longdash«

»Confound Jock! Get on with it.«

»Oh—well, before we'd got very far, we saw a fellow wading about in the river with a rod and a creel. I didn't pay much attention, because, you see, I was wondering what you—Yes. Well! Jock noticed him and said to me, »Yon's a queer kind of fisherman, I'm thinkin'.« So I had a look, and there he was, staggering about among the stones with his fly floating away down the stream in front of him; and he was peering into all the pools he came to, and poking about with a gaff. So I hailed him, and he turned round, and then he put the gaff away in a bit of a hurry and started to reel in his line. He made an awful mess of it,« added Macpherson appreciatively.

»I can believe it,« said Wimsey. »A man who admits to catching trout with a Pink Sisket would make a mess of anything.«

»A pink what?«

»Never mind. I only meant that Robert was no fisher. Get on.«

»Well, he got the line hooked round something, and he was pulling and hauling, you know, and splashing about, and then it came out all of a sudden, and he waved it all over the place and got my hat. That made me pretty wild, and I made after him, and he looked round again, and I yelled out, »Good God, it's Robert!« And he dropped his rod and took to his heels. And of course he slipped on the stones and came down an awful crack. We rushed forward and scooped him up and brought him home. He's got a nasty bang on the head and a fractured patella. Very interesting. I should have liked to have a shot at setting it myself, but it wouldn't do, you know, so I sent for Strachan. He's a good man.«

»You've had extraordinary luck about this business so far,« said Wimsey. »Now the only thing left is to find Great-Uncle. How far down have you got?«

»Not very far. You see, what with getting Robert home and setting his knee and so on, we couldn't do much yesterday.«

»Damn Robert! Great-Uncle may be away out to sea by this time. Let's get down to it.«

He took up a gaff from the umbrella-stand (»Robert's,« interjected Macpherson), and led the way out. The little river was foaming down in a brown spate, rattling stones and small boulders along in its passage. Every hole, every eddy might be a lurking-place for Great-Uncle Joseph. Wimsey peered irresolutely here and there—then turned suddenly to Jock.

»Where's the nearest spit of land where things usually get washed up?« he demanded.

»Eh, well! there's the Battery Pool, about a mile doon the river. Ye'll whiles find things washed up there. Aye. Imph'm. There's a pool and a bit sand, where the river mak's a bend. Ye'll mebbe find it there, I'm thinkin'. Mebbe no. I couldna say.«

»Let's have a look, anyway.«

Macpherson, to whom the prospect of searching the stream in detail appeared rather a dreary one, brightened a little at this.

»That's a good idea. If we take the car down to just above Gatehouse, we've only got two fields to cross.«

The car was still at the door; the hired driver was enjoying the hospitality of the cottage. They pried him loose from Maggie's scones and slipped down the road to Gatehouse.

»Those gulls seem rather active about something,« said Wimsey, as they crossed the second field. The white wings swooped backwards and forwards in narrowing circles over the yellow shoal. Raucous cries rose on the wind. Wimsey pointed silently with his hand. A long, unseemly object, like a drab purse, lay on the shore. The gulls, indignant, rose higher, squawking at the intruders. Wimsey ran forward, stooped, rose again with the long bag dangling from his fingers.

»Great-Uncle Joseph, I presume,« he said, and raised his hat with old-fashioned courtesy.

»The gulls have had a wee peck at it here and there,« said Jock. »It'll be tough for them. Aye. They havena done so vera much with it.«

»Aren't you going to open it?« said Macpherson impatiently.

»Not here,« said Wimsey. »We might lose something.« He dropped it into Jock's creel. »We'll take it home first and show it to Robert.«

Robert greeted them with ill-disguised irritation.

»We've been fishing,« said Wimsey cheerfully. »Look at our bonny wee fush.« He weighed the catch in his hand. »What's inside this wee fush, Mr~Ferguson?«

»I haven't the faintest idea,« said Robert.

»Then why did you go fishing for it?« asked Wimsey pleasantly. »Have you got a surgical knife there, Mac?«

»Yes—here. Hurry up.«

»I'll leave it to you. Be careful. I should begin with the stomach.«

Macpherson laid Great-Uncle Joseph on the table, and slit him open with a practised hand.

»Gude be gracious to us!« cried Maggie, peering over his shoulder. »What'll that be?«

Wimsey inserted a delicate finger and thumb into the cavities of Uncle Joseph. »One—two—three\longdash« The stones glittered like fire as he laid them on the table. »Seven—eight—nine. That seems to be all. Try a little farther down, Mac.«

Speechless with astonishment, Mr~Macpherson dissected his legacy.

»Ten—eleven,« said Wimsey. »I'm afraid the sea-gulls have got number twelve. I'm sorry, Mac.«

»But how did they get there?« demanded Robert foolishly.

»Simple as shelling peas. Great-Uncle Joseph makes his will, swallows his diamonds\longdash«

»He must ha' been a grand man for a pill,« said Maggie, with respect.

»—and jumps out of the window. It was as clear as crystal to anybody who read the will. He told you, Mac, that the stomach was given you to study.«

Robert Ferguson gave a deep groan.

»I knew there was something in it,« he said. »That's why I went to look up the will. And when I saw \textit{you} there, I knew I was right. (Curse this leg of mine!) But I never imagined for a moment\longdash«

His eyes appraised the diamonds greedily.

»And what will the value of these same stones be?« enquired Jock.

»About seven thousand pounds apiece, taken separately. More than that, taken together.«

»The old man was mad,« said Robert angrily. »I shall dispute the will.«

»I think not,« said Wimsey. »There's such an offence as entering and stealing, you know.«

»My God!« said Macpherson, handling the diamonds like a man in a dream. »My God!«

»Seven thousan' pund,« said Jock. »Did I unnerstan' ye richtly to say that one o' they gulls is gaun aboot noo wi' seven thousan' punds' worth o' diamonds in his wame? Ech! it's just awfu' to think of. Guid day to you, sirs. I'll be gaun round to Jimmy McTaggart to ask will he lend me the loan o' a gun.«

